\documentclass[reqno,centertags,12pt]{amsart}

\usepackage{amsmath}
\usepackage{amscd}
\usepackage{stackrel}
\usepackage{amssymb}
\usepackage{amsthm}
\usepackage{bbm}
\usepackage{latexsym}
\usepackage{mathrsfs}
\usepackage{verbatim}
\usepackage{tikz-cd}
\usepackage{mathtools}
\usepackage[hidelinks]{hyperref}


%\usepackage{refcheck}

\usepackage[shortlabels]{enumitem}

\usepackage{xcolor}

% For figure
\usepackage{tikz}
\pgfdeclarelayer{nodelayer}
\pgfdeclarelayer{edgelayer}
\pgfsetlayers{edgelayer,nodelayer,main}
\tikzstyle{arrow}=[draw=black,arrows=-latex]

\textheight 21cm \topmargin 0cm \leftmargin 0cm \marginparwidth 0mm
\textwidth 16.6cm \hsize \textwidth \advance \hsize by
-\marginparwidth \oddsidemargin -4mm \evensidemargin \oddsidemargin




%%%%%%%%%%%%% fonts/sets %%%%%%%%%%%%%%%%%%%%%%%

\newtheorem{theorem}{Theorem}[section]
\newtheorem*{t1}{Theorem 1}
\newtheorem{proposition}[theorem]{Proposition}
\newtheorem{lemma}[theorem]{Lemma}
\newtheorem{corollary}[theorem]{Corollary}
\theoremstyle{definition}
\newtheorem{definition}[theorem]{Definition}
\newtheorem{example}[theorem]{Example}
\newtheorem{conjecture}[theorem]{Conjecture}
\newtheorem{xca}[theorem]{Exercise}
%\theoremstyle{remark}
%\newtheorem{remark}[theorem]{Remark}
\newtheorem*{remark}{Remark}

%%%%%%%%%%%%%%  Rowan's unspaced list %%%%%%%%%%%%%%%%

\newcounter{smalllist}
\newenvironment{SL}{\begin{list}{{\rm\roman{smalllist})}}{%
\diffetlength{\topsep}{0mm}\diffetlength{\parsep}{0mm}\diffetlength{\itemsep}{0mm}%
\diffetlength{\labelwidth}{2em}\diffetlength{\leftmargin}{2em}\usecounter{smalllist}%
}}{\end{list}}

%%%%%%%%%%%%%%% operators %%%%%%%%%%%%%%%%%%%%%%

\DeclareMathOperator{\real}{Re} 
\DeclareMathOperator{\ima}{Im}
\DeclareMathOperator{\diam}{diam}
\DeclareMathOperator*{\slim}{s-lim}
\DeclareMathOperator*{\wlim}{w-lim}
\DeclareMathOperator*{\simlim}{\sim}
\DeclareMathOperator*{\eqlim}{=}
\DeclareMathOperator*{\arrow}{\rightarrow}
\DeclareMathOperator*{\dist}{dist} 
\DeclareMathOperator*{\divg}{div}
\DeclareMathOperator*{\Lip}{Lip} 
\DeclareMathOperator*{\sgn}{sgn} 
\DeclareMathOperator*{\ches}{chess}
\DeclareMathOperator*{\ch}{ch}
\allowdisplaybreaks
\numberwithin{equation}{section}

% Absolute value notation
\newcommand{\abs}[1]{\left\lvert#1\right\rvert}

% Norm notation
\newcommand{\norm}[1]{\left\|#1\right\|}

\newcommand{\set}[1]{\left\{ #1 \right\}}
\newcommand{\setbc}[2]{\left\{ #1\colon#2 \right\}}
\newcommand{\seq}[1]{\left( #1 \right)}

%\renewcommand{\qedsymbol}{}

%%%%%%%%%%%%%%%%%%  abbreviations %%%%%%%%%%%%%%%%

\newcommand{\dott}{\,\cdot\,}
\newcommand{\no}{\nonumber}
\newcommand{\lb}{\label}
\newcommand{\f}{\frac}
\newcommand{\ul}{\underline}
\newcommand{\ol}{\overline}
\newcommand{\ti}{\tilde  }
\newcommand{\wti}{\widetilde  }
\newcommand{\bi}{\bibitem}
\newcommand{\hatt}{\widehat}
%\newcommand{%\marginlabel}[1]{\mbox{}%\marginpar{\raggedleft\hspace{0pt}#1}}

\newcommand{\Oh}{O}
\newcommand{\oh}{o}
\newcommand{\tr}{\text{\rm{Tr}}}
\newcommand{\loc}{\text{\rm{loc}}}
\newcommand{\spec}{\text{\rm{spec}}}
\newcommand{\rank}{\text{\rm{rank}}}
\newcommand{\dom}{\text{\rm{dom}}}
\newcommand{\ess}{\text{\rm{ess}}}
\newcommand{\ac}{\text{\rm{ac}}}
\newcommand{\singc}{\text{\rm{sc}}}
\newcommand{\sing}{\text{\rm{sing}}}
\newcommand{\pp}{\text{\rm{pp}}}
\newcommand{\supp}{\text{\rm{supp}}}
\newcommand{\AC}{\text{\rm{AC}}}

\newcommand{\beq}{\begin{equation}}
\newcommand{\eeq}{\end{equation}}
\newcommand{\eq}{equation}
\newcommand{\bal}{\begin{align}}
\newcommand{\eal}{\end{align}}
\newcommand{\bals}{\begin{align*}}
\newcommand{\eals}{\end{align*}}

%%%%%%%%%%%%%% fonts/sets %%%%%%%%%%%%%%%%%%%%%%%

\newcommand{\calA}{{\mathcal A}}
\newcommand{\bbN}{{\mathbb{N}}}
\newcommand{\bbR}{{\mathbb{R}}}
\newcommand{\bbD}{{\mathbb{D}}}
\newcommand{\bbP}{{\mathbb{P}}}
\newcommand{\bbE}{{\mathbb{E}}}
\newcommand{\bbZ}{{\mathbb{Z}}}
\newcommand{\bbC}{{\mathbb{C}}}
\newcommand{\bbQ}{{\mathbb{Q}}}
\newcommand{\bbT}{{\mathbb{T}}}
\newcommand{\bbS}{{\mathbb{S}}}

\newcommand{\calE}{{\mathcal E}}
\newcommand{\calS}{{\mathcal S}}
\newcommand{\calT}{{\mathcal T}}
\newcommand{\calM}{{\mathcal M}}
\newcommand{\calN}{{\mathcal N}}
\newcommand{\calB}{{\mathcal B}}
\newcommand{\calI}{{\mathcal I}}
\newcommand{\calL}{{\mathcal L}}
\newcommand{\calC}{{\mathcal C}}
\newcommand{\calF}{{\mathcal F}}
\newcommand{\calH}{{\mathcal H}}
\newcommand{\calK}{{\mathcal K}}
\newcommand{\calG}{{\mathcal G}}
\newcommand{\calZ}{{\mathcal Z}}
\newcommand{\calU}{{\mathcal U}}

\newcommand{\eps}{\varepsilon}
\newcommand{\del}{\delta}
\newcommand{\tht}{\theta}
\newcommand{\ka}{\kappa}
\newcommand{\al}{\alpha}
\newcommand{\be}{\beta}
\newcommand{\ga}{\gamma}
\newcommand{\laa}{\lambda}
\newcommand{\partt}{\tfrac{\partial}{\partial t}}
\newcommand{\lan}{\langle}
\newcommand{\ran}{\rangle}
\newcommand{\til}{\tilde}
\newcommand{\tilth}{\til\tht}
\newcommand{\tilT}{\til T}
\newcommand{\tildel}{\til\del}
\newcommand{\ffi}{\varphi}

%\newcommand{\ce}{c^*_e}
\newcommand{\diff}{D}
\newcommand{\diffe}{\diff_e}
\newcommand{\Om}{\Omega}
\newcommand{\tilOm}{{\tilde\Omega}}
\newcommand{\aaa }{a}



\newcommand{\izero}{\iota}


%%%%%%%%%%%%%%%%%%%%%%%%%%%%%%%%%%%%%%%%%%%%%
%%%%%%%%%%%%%%%%%%%% end of  definitions %%%%%%%%%%%%%%%
%%%%%%%%%%%%%%%%%%%%%%%%%%%%%%%%%%%%%%%%%%%%%


\begin{document}
\title[TBD]
{TBD}

\author{Junekey Jeon and Andrej Zlato\v{s}}

\address{\noindent Department of Mathematics \\ University of
California San Diego \\ La Jolla, CA 92093 \newline Email: \tt
zlatos@ucsd.edu,
j6jeon@ucsd.edu}


\begin{abstract}
    This is the abstract.
\end{abstract}

\maketitle

\section{Introduction}

We are concerned with the PDE in two dimensions
\begin{equation}\label{1.1}
    \partial_{t}\omega + u\cdot\nabla\omega = 0
\end{equation}
with
\begin{equation}\label{1.2}
    u \coloneqq -\nabla^{\perp}(-\Delta)^{-1+\alpha}\omega
\end{equation}
and $\alpha\in\left(0,\frac{1}{2}\right)$, where
$(x_{1},x_{2})^{\perp}\coloneqq(-x_{2},x_{1})$ and
$\nabla^{\perp}\coloneqq(-\partial_{x_{2}},\partial_{x_{1}})$.

Given $\omega\in L^{1}(\bbR^{2})\cap L^{\infty}(\bbR^{2})$,
the velocity field $u(\omega)$ generated by $\omega$ as in \eqref{1.2} is given as
\[
    u(\omega;x) \coloneqq \int_{\bbR^{2}}
    \nabla^{\perp}K(x - y)\omega(y)\,dy,
\]
where kernel $K\colon \bbR^{2}\to(0,\infty]$ is defined as
\[
    K(x)\coloneqq \frac{c_{\alpha}}{2\alpha\abs{x}^{2\alpha}}
\]
for some $c_{\alpha}>0$. Since $\omega$ is in $L^{1}(\bbR^{2})\cap L^{\infty}(\bbR^{2})$,
it can be easily seen that $u(\omega)$ is
a well-defined $(1-2\alpha)$-H\"{o}lder continuous function $\bbR^{2}\to\bbR^{2}$.

More generally, when $\omega$ is a finite signed Borel measure on $\bbR^{2}$,
\eqref{1.2} yields
\[
    u(\omega;x) \coloneqq \int_{\bbR^{2}}
    \nabla^{\perp}K(x - y)\,d\omega(y)
\]
whenever the integral converges absolutely. (The reason for considering this general case
is merely because of convenience in certain aspects of developing the theory, and we will
not be concerned with the well-posedness of \eqref{1.1}--\eqref{1.2} in this general setting.)
When $\omega$ is an $L^{1}$ function, we identify it with the finite signed Borel measure
it defines through integration with respect to the Lebesgue measure, so that
the interpretations of $u(\omega)$ in both ways are consistent.
Note that for any measure-preserving homeomorphism $\Phi\colon\bbR^{2}\to\bbR^{2}$,
this correspondence between $L^{1}$ functions and finite signed Borel measures
identifies the function $\omega\circ\Phi^{-1}$ with the pushforward measure $\Phi_{*}\omega$.

Our first result is local well-posedness of \eqref{1.1}--\eqref{1.2}
within a class of $\omega\in L^{1}(\bbR^{2})\cap L^{\infty}(\bbR^{2})$
admitting a decomposition of the form
\begin{equation}\label{1.3}
    \omega(x) = \int_{\mathcal{L}}\mathbbm{1}_{\Omega^{\lambda}}(x)\,d\theta(\lambda),
\end{equation}
where $\mathcal{L}$ is a measurable space (whose $\sigma$-algebra is not explicitly written),
$\theta$ is a $\sigma$-finite signed measure on $\mathcal{L}$, $\Omega$ is a set in
the product $\sigma$-algebra of $\bbR^{2}\times\mathcal{L}$,
and $\Omega^{\lambda}\subseteq\bbR^{2}$ is the $\lambda$-section of $\Omega$
for each $\lambda\in\mathcal{L}$.

A natural choice of $(\Omega,\theta)$ is that each $\Omega^{\lambda}$
is a super-level set of $\omega^{+}$ and $\omega^{-}$, and
$\theta^{+}$, $\theta^{-}$ are respectively the uniform measures on
$\left(0,\sup\omega\right]$ and
$\left[\inf\omega,0\right)$, so that \eqref{1.3}
is the standard layer cake representation.
In this reason, we call the pair $(\Omega,\theta)$, or simply $\Omega$,
a \emph{generalized layer cake representation} of $\omega$.\smallskip

\textit{Remark}. The measurable space $\mathcal{L}$ is implicit from $\Omega$ and $\theta$,
so we will suppress it in the notation. From now on, $\mathcal{L}$ always
denotes the measurable space on which $\Omega$ and $\theta$ are defined.\smallskip

It turns out that each $\Omega^{\lambda}$ being a super-level set of $\omega$
is not necessary for the well-posedness theory we develop.
Furthermore, this abstract setting allows a simple setup for
studying $H^{2}$ regularity of level sets, which our second result is mainly about,
that encompasses situations where some level sets of $\omega$
may consist of multiple (or even infinitely many) disjoint curves.
In these reasons, we state and prove our well-posedness result in terms of
an arbitrary generalized layer cake representation, rather than
the standard layer cake representation.

In addition to merely having a decomposition \eqref{1.3},
we impose a regularity condition
\begin{equation}\label{1.4}
    L_{\tht}(\Omega)\coloneqq \sup_{x\in\bbR^{2}}\int_{\mathcal{L}}
    \frac{d|\theta|(\lambda)}{d(x,\partial\Omega^{\lambda})^{2\alpha}} < \infty
\end{equation}
where $\abs{\theta}$ is the total variation of $\theta$. 
%(This quantity also depends on $\mathcal L$ and $\tht$, but since these will be time-independent for any solution to \eqref{1.1}, we will suppress them in the notation.)
Note that a standard measure theory argument shows joint measurability of
\[
    d(x,\partial\Omega^{\lambda}) = \max\{
        d(x,\Omega^{\lambda}),
        d(x,\bbR^{2}\setminus\Omega^{\lambda})
    \}
\]
in $(x,\lambda)$, so $L_{\tht}(\Omega)$ is always well-defined
(with the convention $\frac{1}{0} = \infty$ and $\infty\cdot 0 = 0$).
The quantity $L_{\tht}(\Omega)$ arises in the context of estimating the growth of
$H^{2}$ norm of a boundary curve $\partial\Omega^{\lambda}$ (Lemma~\ref{L3.7}),
which was the motivation for us to study the well-posedness problem
under the condition \eqref{1.4}.

It turns out that any $\omega$ admitting $\Omega$ with \eqref{1.4} must be
$2\alpha$-H\"{o}lder continuous. In fact, this is the best possible
$L^{\infty}$-type regularity condition \eqref{1.4} implies because
there exists $\omega$ satisfying \eqref{1.4} whose modulus of continuity $\rho(\delta)$
is in between constant multiples of $\min\left\{\delta^{2\alpha},1\right\}$.
In the below (Lemma~\ref{L2.1}) we show that \eqref{1.4} implies
Lipschitz continuity of $u(\omega)$, which does not follow from mere
$2\alpha$-H\"{o}lder continuity in general. This is one of the key components that lead to
our well-posedness result.

On the other hand, investigating the distributional derivative of
$u(\omega)$ shows that the most general condition on $\omega$
\emph{in terms of $\rho$} that ensures Lipschitz continuity of $u(\omega)$ is
\begin{equation}\label{1.5}
    \int_{0}^{1}\frac{\min\set{\rho(\delta),1}}{\delta^{1+2\alpha}}\,d\delta < \infty.
\end{equation}
It can be shown that \eqref{1.5} in fact implies \eqref{1.4} when $(\Omega,\theta)$
is the standard layer cake representation.
Since \eqref{1.5} does not hold for
$\rho(\delta) = \min\set{\delta^{2\alpha},1}$, we see that \eqref{1.4}
is strictly more general than any assumption on the modulus of continuity of $\omega$
from which Lipschitz continuity of $u(\omega)$ is guaranteed.

Since any $\omega$ in the class of functions we consider always generates
Lipschitz velocity field, it is natural to consider
the following notion of solutions to \eqref{1.1}--\eqref{1.2}.

\begin{definition}
    Let $\omega^{0}\in L^{1}(\bbR^{2})\cap L^{\infty}(\bbR^{2})$.
    A \emph{Lagrangian solution}
    to \eqref{1.1}--\eqref{1.2} on a time interval $I\owns 0$ with
    the initial data $\omega^{0}$ is a function
    $\omega\colon I\to L^{1}(\bbR^{2})\cap L^{\infty}(\bbR^{2})$ given as
    $\omega^{t}\coloneqq \omega\circ(\Phi^{t})^{-1}$, where
    $\Phi\in C\left(I;C(\bbR^{2};\bbR^{2})\right)$
    (with respect to the extended metric $d(F,G) \coloneqq \norm{F - G}_{L^{\infty}}$)
    satisfies the initial value problem
    \begin{equation}\label{1.6}
    %\left\{
        \begin{aligned}
           \partial_{t}\Phi^{t} &= u(\omega^{t}) \circ \Phi^{t}, \\
            \Phi^{0} &= {\rm Id},
        \end{aligned}
    %\right.
    \end{equation}
    %for all $x\in\bbR^{2}$
    where the time derivative is one-sided at any end-point of $I$,
    and each $\Phi^{t}$ is a measure-preserving homeomorphism.
    We call $\Phi$ the \emph{flow map} associdated to $\omega$.
\end{definition}

\textit{Remark}. It is easy to show that any Lagrangian solution $\omega$ is
a weak solution to \eqref{1.1}--\eqref{1.2} in the sense that
\[
    \frac{d}{dt}\int_{\bbR^{2}}\varphi(x)\,\omega^{t}(x)\,dx
    = \int_{\bbR^{2}}\left(u(\omega^{t};x)\cdot\nabla\varphi(x)\right)
    \omega^{t}(x)\,dx
\]
holds for all $\varphi\in C^{1}(\bbR^{2})$.\bigskip

Given a measure-preserving homeomorphism
$\Phi\colon\bbR^{2}\to\bbR^{2}$, it can be easily seen that
$\Phi_{*}\Omega\coloneqq\left\{(\Phi(x),\lambda)\in\bbR^{2}\times\mathcal{L}\colon
(x,\lambda)\in\Omega\right\}$ is a generalized layer cake representation of
$\omega\circ\Phi^{-1}$ and
\[
    \omega(\Phi^{-1}(x)) = \int_{\mathcal{L}}
    \mathbbm{1}_{\Phi(\Omega^{\lambda})}(x)\,d\theta(\lambda)
\]
for $x\in\bbR^{2}$. Also,
\begin{equation}\label{1.7}
    \begin{aligned}
        L_{\tht}(\Phi_{*}\Omega)
        &= \sup_{x\in\bbR^{2}}\int_{\mathcal{L}}\frac{d|\theta|(\lambda)}
        {d(x,\partial\Phi(\Omega^{\lambda}))^{2\alpha}}
        = \sup_{x\in\bbR^{2}}\int_{\mathcal{L}}\frac{d|\theta|(\lambda)}
        {d(x,\Phi(\partial\Omega^{\lambda}))^{2\alpha}} \\
        &= \sup_{x\in\bbR^{2}}\int_{\mathcal{L}}\frac{d|\theta|(\lambda)}
        {d(\Phi(x),\Phi(\partial\Omega^{\lambda}))^{2\alpha}}
        \leq \norm{\Phi^{-1}}_{\dot{C}^{0,1}}^{2\alpha} L_{\tht}(\Omega).
    \end{aligned}
\end{equation}

Now we state our first main result.

\begin{theorem}\label{T1.2}
    Let $\omega^{0}\in L^{1}(\bbR^{2})\cap L^{\infty}(\bbR^{2})$ admit
    a generalized layer cake representation $(\Omega,\theta)$ such that
    $L_{\tht}(\Omega)<\infty$. Then there is an open interval $I\ni 0$ and a Lagrangian solution
    $\omega\colon I\to L^{1}(\bbR^{2})\cap L^{\infty}(\bbR^{2})$ to \eqref{1.1}--\eqref{1.2}
    with initial data $\omega^{0}$ and
    the associated flow map $\Phi\in C\left(I;C(\bbR^{2};\bbR^{2})\right)$ such that
    $\sup_{t\in J}L_{\tht}(\Phi_{*}^{t}\Omega) < \infty$ for any compact interval $J\subseteq I$.
    This solution is unique and independent of the choice of $(\Omega,\theta)$, and for any compact interval $J\subseteq I$ we have  $\sup_{t\in J}\max\set{\norm{\Phi^{t}}_{\dot{C}^{0,1}},    \norm{(\Phi^{t})^{-1}}_{\dot{C}^{0,1}}}<\infty$.
  If we let $I$ be the maximal interval as above and  $T$ is an endpoint of $I$, then either $|T|=\infty$  or
    $\lim_{t\to T}L_{\tht}(\Phi_{*}^{t}\Omega) = \infty$.
\end{theorem}

Our second result shows that in the setting of Theorem~\ref{T1.2},
if each $\partial\Omega^{\lambda}$ is an $H^{2}$ curve
and certain additional assumptions are satisfied, then these properties must be retained
by the solution provided by Theorem~\ref{T1.2}, at least for a short time.
To state it precisely, let us give the following definitions.
We refer to \cite{JeoZlaTouching} for notions related to the space of closed plane curves
($\operatorname{CC}(\bbR^{2})$, $\operatorname{PSC}(\bbR^{2})$,
$\operatorname{im}(\,\cdot\,)$, $\ell(\,\cdot\,)$, $\norm{\,\cdot\,}_{\dot{H}^{2}}$
and $\Delta(\,\cdot\,,\,\cdot\,)$).

\begin{definition}
    Let $\omega\in L^{1}(\bbR^{2})\cap L^{\infty}(\bbR^{2})$.
    Then a generalized layer cake representation $(\Omega,\theta)$
    of $\omega$ is said to be \emph{composed of simple closed curves} if for each
    $\lambda\in\mathcal{L}$, $\Omega^{\lambda}$ is a bounded open set and
    $\partial\Omega^{\lambda} = \operatorname{im}(z^{\lambda})$
    for some $z^{\lambda}\in\operatorname{PSC}(\bbR^{2})$.
    In this case, we define
    \begin{equation*}
        Q(\Omega) \coloneqq \sup_{\lambda\in\mathcal{L}}
        \ell(z^{\lambda})\norm{z^{\lambda}}_{\dot{H}^{2}}^{2},\quad
        R_{\tht}(\Omega) \coloneqq \sup_{\lambda\in\mathcal{L}}
        \ell(z^{\lambda})^{1/2}\overline{\int_{\mathcal{L}}}\frac{d|\theta|(\lambda')}
        {\ell(z^{\lambda'})^{1/2}\Delta(z^{\lambda},z^{\lambda'})^{2\alpha}}.
    \end{equation*}
\end{definition}

Here, $\overline{\int_{\mathcal{L}}}f(\lambda)\,d|\theta|(\lambda)$
for any function $f\colon\mathcal{L}\to[0,\infty]$ is the \emph{upper Lebesgue integral}
of $f$; that is,
\[
    \overline{\int_{\mathcal{L}}}f(\lambda)\,d|\theta|(\lambda)
    \coloneqq \inf_{g}\int_{\mathcal{L}}g(\lambda)\,d|\theta|(\lambda)
\]
where $g\colon\mathcal{L}\to[0,\infty]$ ranges over all
measurable functions bounded below by $f$.

For our second result, we will impose in addition to $L_{\tht}(\Omega)<\infty$
that $\Omega$ is composed of simple closed curves and
$Q(\Omega), R_{\tht}(\Omega) < \infty$. The condition $Q(\Omega)<\infty$ ensures
a form of \emph{scaling-invariant} uniform $H^{2}$ regularity of $z^{\lambda}$'s.
(Recall that for $a>0$ and $\gamma\in\operatorname{CC}(\bbR^{2})$,
$\norm{a\gamma}_{\dot{H}^{2}}^{2} = \frac{1}{a}\norm{\gamma}_{\dot{H}^{2}}^{2}$
while $\ell(a\gamma) = a\ell(\gamma)$.)
The condition $R_{\tht}(\Omega)<\infty$ on the other hand controls how densely
$z^{\lambda}$'s of \emph{different scales} can be packed together.
More specifically, it prevents too many small $z^{\lambda}$'s to be placed
near a large $z^{\lambda}$. Interpreting $z^{\lambda}$'s as level sets of $\omega$,
this in effect rules out too sharp ``pinched tops/bottoms''.
(More elaboration of why this is needed?)

Next we state our second main result.

\begin{theorem}\label{T1.4}
    Let $\omega^{0}\in L^{1}(\bbR^{2})\cap L^{\infty}(\bbR^{2})$ admit a
    generalized layer cake representation $(\Omega,\theta)$
    composed of simple closed curves such that $L_{\tht}(\Omega), Q(\Omega), R_{\tht}(\Omega) < \infty$.
    Let $\omega\colon I\to L^{1}(\bbR^{2})\cap L^{\infty}(\bbR^{2})$ be
    the Lagrangian solution to \eqref{1.1}--\eqref{1.2}
%    with the initial data $\omega^{0}$ and the associated flow map
%    $\Phi\in C\left(I;C(\bbR^{2};\bbR^{2})\right)$ provided by 
from Theorem~\ref{T1.2}.
    Then $\Phi_{*}^{t}\Omega$ is composed of simple closed curves for each $t\in I$,
    $\sup_{t\in J}R_{\tht}(\Phi_{*}^{t}\Omega) < \infty$ for each compact interval
    $J\subseteq I$, and the set
    \[
        \set{t\in I \colon Q(\Phi_{*}^{t}\Omega) < \infty}
    \]
    is  open.
\end{theorem}

Consider some smooth even $\chi\colon\bbR\to\bbR$
such that $0\leq\chi\leq 1$, $\chi\equiv 1$ on $\bbR\setminus(-1,1)$, and
$0\notin\operatorname{supp}\chi$. For each $\eps>0$, let
$K_{\eps}(x) \coloneqq \chi\left(\frac{\abs{x}}{\eps}\right)K(x)$.
Note that for any $n\in\bbZ_{\geq 0}$, there is $C_{\alpha,n}$
that only depends on $\alpha,n$ such that the norm of the $n$-linear form
$D^{n}K_{\eps}(x)$ is bounded by
$\frac{C_{\alpha,n}}{\max\{\abs{x},\eps\}^{n+2\alpha}}$.
For any finite signed Borel measure $\omega$ on $\bbR^{2}$
we now define the mollified velocity field
\[
    u_{\eps}(\omega;x)\coloneqq
    \int_{\bbR^{2}}\nabla^{\perp}K_{\eps}(x-y)\,d\omega(y)
\]
for $x\in\bbR^{2}$. Since $\nabla^{\perp}K_{\eps}$ is a smooth function
whose all derivatives vanish at infinity,
this integral is always well-defined and $u_{\eps}(\omega)$ is a smooth function such that
\begin{equation}\label{1.8}
    \norm{D^{k}(u_{\eps}(\omega))}_{L^{\infty}} \leq
    \norm{D^{k}(\nabla^{\perp}K_{\eps})}_{L^{\infty}}\norm{\omega}_{\mathrm{TV}}
\end{equation}
for all $k\in\bbZ_{\geq 0}$.


\section{Proof of Theorem~\ref{T1.2}}

We start with some estimates on the velocity field $u(\omega)$
in terms of $L_{\tht}(\Omega)$, where $(\Omega,\theta)$ is
a generalized layer cake representation of $\omega$.
%Unless specified otherwise, 
All constants $C_{\alpha}$ below
can change from one inequality to another, but they always only depend  on $\alpha$.

\begin{lemma}\label{L2.1}
    There is $C_{\alpha}$ such that for any
    $\omega\in L^{1}(\bbR^{2})\cap L^{\infty}(\bbR^{2})$
    with a generalized layer cake representation $(\Omega,\theta)$, and for any
    $\eps>0$ and $x\in\bbR^{2}$, we have
    \[
        \abs{D(u_{\eps}(\omega))(x)} \leq
        C_{\alpha}\int_{\mathcal{L}}\frac{d|\theta|(\lambda)}
        {d(x,\partial\Omega^{\lambda})^{2\alpha}}.
    \]
    Therefore,
    \[
        \norm{u_\eps(\omega)}_{\dot{C}^{0,1}} \leq C_{\alpha}L_{\tht}(\Omega)
        \qquad\text{and} \qquad \norm{u(\omega)}_{\dot{C}^{0,1}} \leq C_{\alpha}L_{\tht}(\Omega).
    \]
\end{lemma}

\begin{proof}
    For each $\eps>0$ and $x,h\in\bbR^{2}$, oddness of $\nabla^{\perp}K_{\eps}$ shows that
    \begin{align*}
        u_{\eps}(\omega;x+h) - u_{\eps}(\omega;x)
        = \int_{\bbR^{2}}\left(
            \nabla^{\perp}K_{\eps}(x + h - y) - \nabla^{\perp}K_{\eps}(x - y)
        \right)
        (\omega(y) - \omega(x))\,dy.
    \end{align*}
    Replacing $h$ by $sh$ with $s\in\bbR$, and then taking $s\to 0$ yields
    \begin{align*}
        D(u_{\eps}(\omega))(x)h
        &= \int_{\bbR^{2}}
        D(\nabla^{\perp}K_{\eps})(x-y)h \,
        (\omega(y) - \omega(x))\,dy \\
        &= \int_{\bbR^{2}}\int_{\mathcal{L}}
        D(\nabla^{\perp}K_{\eps})(x-y)h
        \left(
            \mathbbm{1}_{\Omega^{\lambda}}(y)
            - \mathbbm{1}_{\Omega^{\lambda}}(x)
        \right)
        d\theta(\lambda)\,dy.
    \end{align*}
%    follows, because the difference between $u_{\eps}(\omega;x+h) - u_{\eps}(\omega;x)$
%    and the right-hand side of the first equality of the above is $O\left(\abs{h}^{2}\right)$.
    Note that $\mathbbm{1}_{\Omega^{\lambda}}(y)
    - \mathbbm{1}_{\Omega^{\lambda}}(x) \neq 0$ implies
    $\abs{x - y} \geq d(x,\partial\Omega^{\lambda})$, so
    \begin{align*}
        \abs{D(u_{\eps}(\omega))(x)}
        &\leq \int_{\mathcal{L}}\int_{\abs{x-y} \geq d(x,\partial\Omega^{\lambda})}
        \frac{C_{\alpha}}{\abs{x - y}^{2+2\alpha}}\,dy\,d|\theta|(\lambda)
        \leq \int_{\mathcal{L}}\frac{C_{\alpha}}
        {d(x,\partial\Omega^{\lambda})^{2\alpha}}d|\theta|(\lambda).
    \end{align*}
   This proves the first and second claims, and the third follows by taking 
   %the supremum over $x\in\bbR^{2}$ and letting 
   $\eps\to 0^{+}$.
\end{proof}

\begin{lemma}\label{L2.2}
    There is $C_{\alpha}$ such that for any
    $\omega\in L^{1}(\bbR^{2})\cap L^{\infty}(\bbR^{2})$ with
    generalized layer cake representations
    $(\Omega_{i},\theta_{i})$ and any measure-preserving homeomorphisms
    $\Phi_{i}\colon\bbR^{2}\to\bbR^{2}$ ($i=1,2$),
    \[
        \norm{u(\Phi_{1*}\omega) - u(\Phi_{2*}\omega)}_{L^{\infty}} \leq
        C_{\alpha} \left(
            L_{\theta_{1}}(\Phi_{1*}\Omega_{1}) + L_{\theta_{2}}(\Phi_{2*}\Omega_{2})
        \right)
        \norm{\Phi_{1} - \Phi_{2}}_{L^{\infty}}.
    \]
\end{lemma}

\begin{proof}
    Let    $\mathcal{L}_{i}$ the measurable space associated to $(\Omega_{i},\theta_{i})$ and
    $\omega_{i}\coloneqq \omega\circ\Phi_{i}^{-1}$ for $i=1,2$.  Let $d\coloneqq\norm{\Phi_{1} - \Phi_{2}}_{L^{\infty}}$ and fix any $x\in\bbR^{2}$.
    Then $u(\Phi_{1*}\omega;x) - u(\Phi_{2*}\omega;x)$ is the sum of
    \begin{align*}
        I_{1} &\coloneqq \int_{\abs{x - y}\leq 2d}
        \nabla^{\perp}K(x - y)(\omega_{1}(y) - \omega_{2}(y))\,dy, \\
        I_{2} &\coloneqq \int_{\abs{x - y} > 2d}
        \nabla^{\perp}K(x - y)(\omega_{1}(y) - \omega_{2}(y))\,dy.
    \end{align*}

    \textbf{Estimate for $I_{1}$.} By oddness of $\nabla^{\perp}K$, we have
    \[
        I_{1} = \int_{\abs{x - y}\leq 2d}\nabla^{\perp}K(x - y)
        (\omega_{1}(y) - \omega_{1}(x))\,dy
        - \int_{\abs{x - y}\leq 2d}\nabla^{\perp}K(x - y)
        (\omega_{2}(y) - \omega_{2}(x))\,dy.
    \]
Since
%    and by symmetry, it is enough to estimate
    \[
        I_{3} \coloneqq \int_{\abs{x - y} \leq 2d}
        \frac{\abs{\omega_{1}(y) - \omega_{1}(x)}}{\abs{x - y}^{1 + 2\alpha}}\,dy
        \leq \int_{\mathcal{L}_{1}}\int_{\abs{x - y}\leq 2d}
        \frac{\abs{\mathbbm{1}_{\Phi_{1}(\Omega_{1}^{\lambda})}(y)
        - \mathbbm{1}_{\Phi_{1}(\Omega_{1}^{\lambda})}(x)}}
        {\abs{x - y}^{1+2\alpha}}\,dy\,d|\theta_{1}|(\lambda)
    \]
    and (as in the proof of Lemma~\ref{L2.1})
    we have $\abs{x - y} \geq d(x,\partial\Phi_{1}(\Omega_{1}^{\lambda}))$
    whenever the  last integrand  is nonzero,
    we see that
    \begin{align*}
        I_{3} &\leq \int_{\mathcal{L}_{1}}\int_{\abs{x - y}\leq 2d}
        \frac{1}{\abs{x - y}\, d(x,\partial\Phi_{1}(\Omega_{1}^{\lambda}))^{2\alpha}}
        \,dy\,d|\theta_{1}|(\lambda)
        \leq 4\pi L_{\theta_{1}}(\Phi_{1*}\Omega_{1})d.
    \end{align*}
    The same argument for $\omega_{2}$ in place of $\omega_{1}$ now yields
    \begin{equation} \label{2.100}
        \abs{I_{1}}\leq C_{\alpha}\left(
            L_{\theta_{1}}(\Phi_{1*}\Omega_{1}) + L_{\theta_{2}}(\Phi_{2*}\Omega_{2})
        \right)d.
    \end{equation}

    \textbf{Estimate for $I_{2}$.} For each $R>2d$ let
    \[
        I_{2}^{R}\coloneqq \int_{2d<\abs{x-y}\leq R}
        \nabla^{\perp}K(x-y)(\omega_{1}(y) - \omega_{2}(y))\,dy,
    \]
    so that $I_{2} = \lim_{R\to\infty}I_{2}^{R}$. Fix $R>2d$, and then $\Phi_i$ being measure-preserving yields
    \begin{align*}
        I_{2}^{R} &= \int_{2d< \abs{x - y} \leq R}
        \nabla^{\perp}K(x - y)(\omega_{1}(y) - \omega_{1}(x))\,dy
        \\&\quad\quad\quad\quad\quad
        - \int_{2d < \abs{x - y} \leq R}
        \nabla^{\perp}K(x - y)(\omega_{2}(y) - \omega_{1}(x))\,dy \\
        &= \int_{2d < \abs{x - \Phi_{1}(y)} \leq R}\nabla^{\perp}K(x - \Phi_{1}(y))
        (\omega(y) - \omega_{1}(x))\,dy
        \\&\quad\quad\quad\quad\quad
        - \int_{2d < \abs{x - \Phi_{2}(y)} \leq R}\nabla^{\perp}K(x - \Phi_{2}(y))
        (\omega(y) - \omega_{1}(x))\,dy \\
        &= \int_{\abs{x - \Phi_{1}(y)},\abs{x - \Phi_{2}(y)}\in (2d,R]}
        \left[
            \nabla^{\perp}K(x - \Phi_{1}(y)) - \nabla^{\perp}K(x - \Phi_{2}(y))
        \right]
        (\omega(y) - \omega_{1}(x))\,dy
        \\&\quad\quad\quad\quad\quad
        + \int_{\abs{x - \Phi_{2}(y)} \leq 2d < \abs{x - \Phi_{1}(y)} \leq R}
        \nabla^{\perp}K(x - \Phi_{1}(y))(\omega(y) - \omega_{1}(x))\,dy
        \\&\quad\quad\quad\quad\quad
        - \int_{\abs{x - \Phi_{1}(y)} \leq 2d < \abs{x - \Phi_{2}(y)} \leq R}
        \nabla^{\perp}K(x - \Phi_{2}(y))(\omega(y) - \omega_{1}(x))\,dy
        \\&\quad\quad\quad\quad\quad
        + \int_{2d < \abs{x - \Phi_{1}(y)} \leq R < \abs{x - \Phi_{2}(y)}}
        \nabla^{\perp}K(x - \Phi_{1}(y))(\omega(y) - \omega_{1}(x))\,dy
        \\&\quad\quad\quad\quad\quad
        - \int_{2d < \abs{x - \Phi_{2}(y)} \leq R < \abs{x - \Phi_{1}(y)}}
        \nabla^{\perp}K(x - \Phi_{2}(y))(\omega(y) - \omega_{1}(x))\,dy.
    \end{align*}
    Let us denote the integrals on the right-hand side  $I_{4},I_{5},I_{6},I_{7},I_{8}$
     (in the order of appearance).

    To estimate $I_{4}$, note that for any $y$ in the domain of integration we have
    \[
        \min_{\eta\in[0,1]}
        \abs{x - (1-\eta)\Phi_{1}(y) - \eta\Phi_{2}(y)}
        \geq \abs{x - \Phi_{1}(y)} - d
        \geq \frac{\abs{x - \Phi_{1}(y)}}{2},
    \]
     so the mean value theorem shows that
    \[
        \abs{\nabla^{\perp}K(x - \Phi_{1}(y)) - \nabla^{\perp}K(x - \Phi_{2}(y))}
        \leq \frac{C_{\alpha}d}{\abs{x - \Phi_{1}(y)}^{2 + 2\alpha}}.
    \]
    The change of variables formula now yields
    \begin{align*}
        \abs{I_{4}} &\leq \int_{\abs{x - y} > 2d}
        \frac{C_{\alpha}d\abs{\omega_{1}(y) - \omega_{1}(x)}}
        {\abs{x - y}^{2 + 2\alpha}}\,dy \\
        &\leq \int_{\mathcal{L}_{1}}\int_{\abs{x - y} > 2d}
        \frac{C_{\alpha}d\abs{\mathbbm{1}_{\Phi_{1}(\Omega_{1}^{\lambda})}(y)
        - \mathbbm{1}_{\Phi_{1}(\Omega_{1}^{\lambda})}(x)}}
        {\abs{x - y}^{2 + 2\alpha}}\,dy\,d|\theta_{1}|(\lambda).
    \end{align*}
    Again, $\abs{x - y} \geq d(x,\partial\Phi_{1}(\Omega_{1}^{\lambda}))$
    holds whenever the last integrand  is nonzero, so
    \begin{align*}
        \abs{I_{4}} &\leq \int_{\mathcal{L}_{1}}
        \int_{\abs{x - y}\geq d(x,\partial\Phi_{1}(\Omega_{1}^{\lambda}))}
        \frac{C_{\alpha}d}{\abs{x - y}^{2+2\alpha}}\,dy\,d|\theta_{1}|(\lambda)
        \leq C_{\alpha}L_{\theta_{1}}(\Phi_{1*}\Omega_{1})d.
    \end{align*}

    For $I_{5}$, note that for any $y$ in the domain of integration we have
    \[
        \abs{x - \Phi_{1}(y)} \leq \abs{x - \Phi_{2}(y)} + d \leq 3d.
    \]
    By applying again  change of variables we obtain
    \[
        \abs{I_{5}} \leq \int_{\abs{x - y} \leq 3d}
        \frac{C_{\alpha}\abs{\omega_{1}(y) - \omega_{1}(x)}}{\abs{x - y}^{1 + 2\alpha}}\,dy,
    \]
    so the same argument as in the estimate for $I_{3}$
    shows $\abs{I_{5}} \leq C_{\alpha}L_{\theta_{1}}(\Phi_{1*}\Omega_{1})d$.
And clearly
    \[
        \abs{I_{6}} \leq \int_{\abs{x - \Phi_{1}(y)} \leq 2d}
        \frac{C_{\alpha}\abs{\omega(y) - \omega_{1}(x)}}
        {\abs{x - \Phi_{1}(y)}^{1+2\alpha}}\,dy
        = \int_{\abs{x - y}\leq 2d}\frac{C_{\alpha}\abs{\omega_{1}(y) - \omega_{1}(x)}}
        {\abs{x - y}^{1+2\alpha}}\,dy,
    \]
    so again $\abs{I_{6}} \leq C_{\alpha}L_{\theta_{1}}(\Phi_{1*}\Omega_{1})d$.
    % in the same way    as in the estimate of $I_{3}$.

    To estimate $I_{7}$, note that for any $y$ in the domain of integration we have
    \begin{align*}
        \abs{x - \Phi_{1}(y)} \geq \abs{x - \Phi_{2}(y)} - d > R - d,
    \end{align*}
    so the change of variables formula yields
    \begin{align*}
        \abs{I_{7}} \leq \int_{R - d < \abs{x - y} \leq R}
        \frac{C_{\alpha}\norm{\omega}_{L^{\infty}}}
        {\abs{x - y}^{1 + 2\alpha}}\,dy
       % = 2\pi C_{\alpha}\norm{\omega}_{L^{\infty}}
       % R^{1-2\alpha}\int_{1 - \frac{d}{R}}^{1}\frac{dr}{r^{2\alpha}}
        \leq \frac{ C_{\alpha}\norm{\omega}_{L^{\infty}}d}{R^{2\alpha}}
    \end{align*}
    %where the last inequality is 
    because ${R} >2d$.
    In the same way we also obtain $|I_{8}|\le \frac{C_{\alpha}\norm{\omega}_{L^{\infty}}d}{R^{2\alpha}}$.
    
    Collecting the above estimates and letting $R\to\infty$, we see that
    $\abs{I_{2}} \leq C_{\alpha}L_{\theta_{1}}(\Phi_{1*}\Omega_{1})d$.
    This and \eqref{2.100} now hold uniformly in $x\in\bbR^{2}$, finishing the proof.
\end{proof}

\begin{lemma}\label{L2.3}
    There is $C_{\alpha}$ such that for any
    $\omega\in L^{1}(\bbR^{2})\cap L^{\infty}(\bbR^{2})$ and $\eps>0$ we have
    \[
        \norm{u(\omega) - u_{\eps}(\omega)}_{L^{\infty}}
        \leq C_{\alpha}\norm{\omega}_{L^{\infty}}\eps^{1-2\alpha}.
    \]
\end{lemma}

\begin{proof}
    Since $\nabla^{\perp}K_{\eps}(x) = \nabla^{\perp}K(x)$
    when $\abs{x}\geq\eps$, for any $x\in\bbR^{2}$ we have
    \begin{align*}
        \abs{u(\omega;x) - u_{\eps}(\omega;x)}
        &\leq \int_{\abs{x - y}\leq \eps}
        \abs{\nabla^{\perp}K(x-y) - \nabla^{\perp}K_{\eps}(x-y)}
        \norm{\omega}_{L^{\infty}}\,dy \\
        &\leq \int_{\abs{x - y}\leq \eps}
        \frac{C_{\alpha}\norm{\omega}_{L^{\infty}}}{\abs{x - y}^{1 + 2\alpha}}\,dy
        = C_{\alpha}\norm{\omega}_{L^{\infty}}\eps^{1-2\alpha}.
    \end{align*}
\end{proof}

Now, fix any initial data $\omega^{0}\in L^{1}(\bbR^{2})\cap L^{\infty}(\bbR^{2})$
admitting a generalized layer cake representation
$(\Omega,\theta)$ with $L_{\tht}(\Omega) < \infty$.
Fix any $\eps>0$ and consider the ODE
\begin{equation}\label{2.1}
 %   \left\{
        \begin{aligned}
            \partial_{t}\Psi_{\eps}^{t}
            &= u_{\eps}((\mathrm{Id} + \Psi_{\eps}^{t})_{*}\omega^{0}) \circ ({\rm Id} + \Psi_{\eps}^{t}), \\
            \Psi_{\eps}^{0} &= 0
        \end{aligned}
%    \right
\end{equation}
with $\Psi_\eps^t \in BC(\bbR^{2};\bbR^{2})$ (the space of bounded continuous functions from $\bbR^{2}$ to $\bbR^{2}$).
That is, $\Phi_{\eps}^{t}\coloneqq \mathrm{Id} + \Psi_{\eps}^{t}$
is the flow map generated by the vector field
$u_{\eps}$ that is in turn generated by the measure $\Phi_{\eps *}^{t}\omega^{0}$. We will next show that \eqref{2.1} is globally well-posed in $BC(\bbR^{2};\bbR^{2})$.

\begin{lemma}\label{L2.4}
    For each $F\in BC(\bbR^{2};\bbR^{2})$, let
    \[
        \mathcal{F}(F) \coloneqq u_{\eps}((\mathrm{Id} + F)_{*}\omega^{0})\circ ({\rm Id} + F).
    \]
    Then $\mathcal{F}\colon    BC(\bbR^{2};\bbR^{2})\to BC(\bbR^{2};\bbR^{2})$
    is well-defined and  Lipschitz continuous.
\end{lemma}

\begin{proof}
    Clearly $\mathcal{F}(F)\in C(\bbR^{2}; \bbR^{2})$
    for any $F\in BC(\bbR^{2};\bbR^{2})$.
    For any $F_{1},F_{2}\in BC(\bbR^{2};\bbR^{2})$ and $x\in\bbR^{2}$ we see that $(\mathcal{F}(F_{1}) - \mathcal{F}(F_{2}))(x)$ equals
    \begin{equation*}
        \begin{aligned}
&\             \int_{\bbR^{2}} \nabla^{\perp}K_{\eps}(x + F_{1}(x) - y)
            \,d(\mathrm{Id} + F_{1})_{*}\omega^{0}(y)
            - \int_{\bbR^{2}} \nabla^{\perp}K_{\eps}(x + F_{2}(x) - y)
            \,d(\mathrm{Id} + F_{2})_{*}\omega^{0}(y)
            \\&\ 
            = \int_{\bbR^{2}}
            \left[
                \nabla^{\perp}K_{\eps}(x - y + F_{1}(x) - F_{1}(y))
                - \nabla^{\perp}K_{\eps}(x - y + F_{2}(x) - F_{2}(y))
            \right]
            \omega^{0}(y)\,dy,
        \end{aligned}
    \end{equation*}
    so
    \[
        \norm{\mathcal{F}(F_{1}) - \mathcal{F}(F_{2})}_{L^{\infty}}
        \leq 2\norm{D(\nabla^{\perp}K_{\eps})}_{L^{\infty}}\norm{\omega^{0}}_{L^{1}}
        \norm{F_{1} - F_{2}}_{L^{\infty}}.
    \]
    Since $\mathcal{F}(0) = u_{\eps}(\omega^{0})$ is bounded,
    both claims follow from this.
\end{proof}

Therefore \eqref{2.1} is globally well-posed, and we  let
$\Phi_{\eps}^{t}\coloneqq \mathrm{Id} + \Psi_{\eps}^{t}$ and
$\omega_{\eps}^{t}\coloneqq\Phi_{\eps*}^{t}\omega^{0}$, with the latter being for now only a finite signed Borel measure.
We will next show that $\Phi_{\eps}^{t}$ is in fact a measure-preserving homeomorphism, which will mean that $\omega_{\eps}^{t}$ is an $L^{1}\cap L^\infty$ function.

Clearly the ODE
\begin{equation}\label{2.2}
    \partial_{t}G^{t} = u_{\eps}(\omega_{\eps}^{t})\circ ({\rm Id} + G^{t})
\end{equation}
is globally well-posed in $BC(\bbR^{2};\bbR^{2})$ for any initial data at any initial time.
For each $t_{0},t_1\in\bbR$, let $\Theta_{\eps}^{t_{0},t}$ be the unique solution to \eqref{2.2}
with initial data $\Theta_{\eps}^{t_{0},t_{0}}: = 0$ at time $t=t_0$, and consider
$G^{t} \coloneqq \Theta_{\eps}^{t_{0},t_{1}} +
\Theta_{\eps}^{t_{1},t}\circ(\mathrm{Id} + \Theta_{\eps}^{t_{0},t_{1}})$.  Then
 $G^t$ solves \eqref{2.2} and 
$G^{t_{1}} = \Theta_{\eps}^{t_{0},t_{1}}$, so uniqueness of the solution
with the initial data $\Theta_{\eps}^{t_{0},t_{1}}$ at time $t=t_{1}$ shows that
\[
    \mathrm{Id} + \Theta_{\eps}^{t_{0},t}
    = \mathrm{Id} + G^{t}
    = (\mathrm{Id} + \Theta_{\eps}^{t_{1},t})
    \circ(\mathrm{Id} + \Theta_{\eps}^{t_{0},t_{1}})
\]
 for all $t\in\bbR$. Letting $t:=t_{0}$ shows that
$(\mathrm{Id} + \Theta_{\eps}^{t_{1},t_{0}})
\circ(\mathrm{Id} + \Theta_{\eps}^{t_{0},t_{1}}) = \mathrm{Id}$,
so we conclude that each $\mathrm{Id} + \Theta_{\eps}^{t_{0},t}$ is a homeomorphism.
Then so is $\Phi_{\eps}^{t} = \mathrm{Id} + \Theta_{\eps}^{0,t}$.
%and so $\omega_{\eps}^{t}\in L^1(\bbR^2)$.

Letting $BC^{1}(\bbR^{2};\bbR^{2})$ be the space of bounded $C^{1}$ functions from $\bbR^{2}$ to $\bbR^{2}$ with bounded first derivatives, we see that for any $F\in BC^{1}(\bbR^{2};\bbR^{2})$ and $x,h\in\bbR^{2}$ we have
\begin{align*}
    D(u_{\eps}(\omega_{\eps}^{t})\circ(\mathrm{Id} + F))(x)h
    = \int_{\bbR^{2}}D(\nabla^{\perp}K_{\eps})(x + F(x) - y)(h + DF(x)h)
    \,d\omega_{\eps}^{t}(y).
\end{align*}
Therefore
$F\mapsto u_{\eps}(\omega_{\eps}^{t})\circ(\mathrm{Id} + F)$
is locally Lipschitz on $BC^{1}(\bbR^{2};\bbR^{2})$, so
\eqref{2.2} is locally well-posed there.
But since for any $F\in BC^{1}(\bbR^{2};\bbR^{2})$ we have
\[
    \norm{D(u_{\eps}(\omega_{\eps}^{t})\circ(\mathrm{Id} + F))}_{L^{\infty}}
    \leq \norm{D(\nabla^{\perp}K_{\eps})}_{L^{\infty}}\norm{\omega^{0}}_{L^{1}}
    \norm{\mathrm{Id} + DF}_{L^{\infty}},
\]
a Gr\"{o}nwall-type argument shows that the $C^{1}$ norm of any solution to \eqref{2.2}
can grow no faster than exponentially.  Therefore \eqref{2.2} is even globally well-posed in
$BC^{1}(\bbR^{2};\bbR^{2})$, and so $\Theta_{\eps}^{t_{0},t}\in BC^{1}(\bbR^{2};\bbR^{2})$
for all $t\in\bbR$.  This and $\nabla \cdot u_{\eps}(\omega_{\eps}^{t}) \equiv 0$
now show that the map $\mathrm{Id} + \Theta_{\eps}^{t_{0},t}$ is measure-preserving.
Then $\omega_{\eps}^{t}=\omega^{0}\circ(\Phi_{\eps}^{t})^{-1}\in L^1(\bbR^2)\cap L^\infty(\bbR^2)$
and  $\Phi_{\eps*}^{t}\Omega$ is its generalized layer cake representation.
%\smallskip

%\textit{Remark}. 
Similarly, with $BC^{2}(\bbR^{2};\bbR^{2})$ the space of
bounded $C^{2}$ functions from $\bbR^{2}$ to $\bbR^{2}$ with bounded first and second derivatives,
for each $F\in BC^{2}(\bbR^{2};\bbR^{2})$ and $x,h_{1},h_{2}\in\bbR^{2}$ we have
\begin{align*}
    &D^{2}(u_{\eps}(\omega_{\eps}^{t})\circ(\mathrm{Id} + F))(x)(h_{1},h_{2})
    \\&\quad\quad = \int_{\bbR^{2}}D^{2}(\nabla^{\perp}K_{\eps})
    (x + F(x) - y)(h_{1} + DF(x)h_{1}, h_{2} + DF(x)h_{2})\,\omega_{\eps}^{t}(y)\,dy
    \\&\quad\quad\quad\quad\quad
    + \int_{\bbR^{2}}D(\nabla^{\perp}K_{\eps})(x + F(x) - y)
    \left(D^{2}F(x)(h_{1}, h_{2})\right)\omega_{\eps}^{t}(y)\,dy
\end{align*}
and
\begin{align*}
    \norm{D^{2}(u_{\eps}(\omega_{\eps}^{t})\circ(\mathrm{Id} + F))}_{L^{\infty}}
    &\leq \norm{D^{2}(\nabla^{\perp}K_{\eps})}_{L^{\infty}} \norm{\omega^{0}}_{L^{1}}
    \norm{\mathrm{Id} + DF}_{L^{\infty}}^{2}
    \\&\quad\quad\quad
    + \norm{D(\nabla^{\perp}K_{\eps})}_{L^{\infty}} \norm{\omega^{0}}_{L^{1}}
    \norm{D^{2}F}_{L^{\infty}}.
\end{align*}
Another Gr\"{o}nwall-type argument and the time-exponential bound on the $C^{1}$ norms of solutions to \eqref{2.2} 
now shows that \eqref{2.2} is globally well-posed in $BC^{2}(\bbR^{2};\bbR^{2})$, which we will use  in Section~\ref{S3}.
One can continue and inductively show that
\eqref{2.2} is globally well-posed in $BC^{k}(\bbR^{2};\bbR^{2})$ for all $k\in\bbN$ (then each $\Phi_{\eps}^{t}$ is a diffeomorphism), but we will not need this here.
%\smallskip

Next we derive an $\eps$-independent estimate on the growth of
$L_{\tht}(\Phi_{\eps*}^{t}\Omega)$.

\begin{lemma}\label{L2.5}
    $\norm{D^{k}(u_{\eps}(\omega_{\eps}^{t}))}_{L^{\infty}}$ is continuous in $t$
    for all $k\in\bbZ_{\geq 0}$, and
    \begin{align}
        &\abs{\Theta_{\eps}^{t_{0},t_{1}}(x) - \Theta_{\eps}^{t_{0},t_{1}}(y)
        - \Theta_{\eps}^{t_{0},t_{2}}(x) + \Theta_{\eps}^{t_{0},t_{2}}(y)}  \notag
        \\&\qquad\qquad
        \leq \left(\exp\left(\abs{
            \int_{t_{2}}^{t_{1}}\norm{u_{\eps}(\omega_{\eps}^{\tau})}_{\dot{C}^{0,1}}d\tau
        }\right) - 1\right)
        \abs{x + \Theta_{\eps}^{t_{0},t_{2}}(x) - y - \Theta_{\eps}^{t_{0},t_{2}}(y)}  \lb{2.101}
    \end{align}
    holds for all $x,y\in\bbR^{2}$ and $t_{0},t_{1},t_{2}\in\bbR$.
\end{lemma}

\begin{proof}
    Change of variables  yields
    \[
        \norm{D^{k}(u_{\eps}(\omega_{\eps}^{t_{1}}))
        - D^{k}(u_{\eps}(\omega_{\eps}^{t_{2}}))}_{L^{\infty}}
        \leq \norm{D^{k+1}(\nabla^{\perp}K_{\eps})}_{L^{\infty}}
        \norm{\omega^{0}}_{L^{1}}
        \norm{\Phi_{\eps}^{t_{1}} - \Phi_{\eps}^{t_{2}}}_{L^{\infty}}
    \]
    for any $(k,t_{1},t_{2})\in\bbZ_{\geq 0}\times\bbR^{2}$.  This shows
    the first claim, and in particular that $\norm{u_{\eps}(\omega_{\eps}^{t})}_{\dot{C}^{0,1}}$
    is continuous in $t$. 

    Next, letting $x':=x + \Theta_{\eps}^{t_{0},t_{2}}(x)$, we see that
    \[
        \Theta_{\eps}^{t_{0},t_{1}}(x) = x' + \Theta_{\eps}^{t_{2},t_{1}}(x') - x
        = \Theta_{\eps}^{t_{2},t_{1}}(x') + \Theta_{\eps}^{t_{0},t_{2}}(x).
    \]
    So with $y':=y + \Theta_{\eps}^{t_{0},t_{2}}(y)$, the left-hand side of \eqref{2.101}
    is just $|\Theta_{\eps}^{t_{2},t_{1}}(x')-\Theta_{\eps}^{t_{2},t_{1}}(y')|$,
    while the last factor is $|x'-y'|$. The result now follows from the definition of
    $\Theta_{\eps}^{t_{2},t}$.
%
%    We first note that the change of variables formula yields
%    \[
%        \norm{D^{k}(u_{\eps}(\omega_{\eps}^{t_{1}}))
%        - D^{k}(u_{\eps}(\omega_{\eps}^{t_{2}}))}_{L^{\infty}}
%        \leq \norm{D^{k+1}(\nabla^{\perp}K_{\eps})}_{L^{\infty}}
%        \norm{\omega^{0}}_{L^{1}}
%        \norm{\Phi_{\eps}^{t_{1}} - \Phi_{\eps}^{t_{2}}}_{L^{\infty}}
%    \]
%    for any $(k,t_{1},t_{2})\in\bbZ_{\geq 0}\times\bbR^{2}$, which shows
%    the first claim and in particular that $\norm{u_{\eps}(\omega_{\eps}^{t})}_{\dot{C}^{0,1}}$
%    is continuous in $t$. Then for any $x,y\in\bbR^{2}$ and $t_{0},t_{1},t_{2}\in\bbR$, we have
%    \begin{equation}\label{2.3}
%        \begin{aligned}
%            &\abs{\abs{x + \Theta_{\eps}^{t_{0},t_{1}}(x) - y - \Theta_{\eps}^{t_{0},t_{1}}(y)}
%            - \abs{x + \Theta_{\eps}^{t_{0},t_{2}}(x) - y - \Theta_{\eps}^{t_{0},t_{2}}(y)}}
%            \\&\qquad\qquad
%            \leq \abs{\Theta_{\eps}^{t_{0},t_{1}}(x) - \Theta_{\eps}^{t_{0},t_{1}}(y)
%            - \Theta_{\eps}^{t_{0},t_{2}}(x) + \Theta_{\eps}^{t_{0},t_{2}}(y)}
%            \\&\qquad\qquad
%            \leq \abs{\int_{t_{2}}^{t_{1}}
%            \abs{u_{\eps}(\omega_{\eps}^{\tau};\mathrm{Id} + \Theta_{\eps}^{t_{0},\tau}(x))
%            - u_{\eps}(\omega_{\eps}^{\tau};\mathrm{Id} + \Theta_{\eps}^{t_{0},\tau}(y))}\,d\tau}
%            \\&\qquad\qquad
%            \leq \abs{\int_{t_{2}}^{t_{1}}\norm{u_{\eps}(\omega_{\eps}^{\tau})}_{\dot{C}^{0,1}}
%            \abs{x + \Theta_{\eps}^{t_{0},\tau}(x) - y - \Theta_{\eps}^{t_{0},\tau}(y)}\,d\tau},
%        \end{aligned}
%    \end{equation}
%    so dividing by $\abs{t_{1} - t_{2}}$, letting $t_{1} \to t_{2}$
%    and then applying a Gr\"{o}nwall-type argument shows
%    \[
%        \abs{x + \Theta_{\eps}^{t_{0},t_{1}}(x) - y - \Theta_{\eps}^{t_{0},t_{1}}(y)}
%        \leq \exp\left(\abs{
%            \int_{t_{2}}^{t_{1}}\norm{u_{\eps}(\omega_{\eps}^{\tau})}_{\dot{C}^{0,1}}\,d\tau
%        }\right)
%        \abs{x + \Theta_{\eps}^{t_{0},t_{2}}(x) - y - \Theta_{\eps}^{t_{0},t_{2}}(y)}.
%    \]
%    Then replacing $(\tau,t_{1})$ by $(\tau',\tau)$ in the above inequality
%    and applying it to \eqref{2.3} shows the claim.
\end{proof}

\begin{proposition}\label{P2.6}
    $L_{\tht}(\Phi_{\eps*}^{t}\Omega)$ is continuous in $t$ and
    \[
        \max\set{
            \partial_{t}^{+}L_{\tht}(\Phi_{\eps*}^{t}\Omega),
            -\partial_{t-}L_{\tht}(\Phi_{\eps*}^{t}\Omega)
        }
        \leq \norm{u_{\eps}(\omega_{\eps}^{t})}_{\dot{C}^{0,1}}
        L_{\tht}(\Phi_{\eps*}^{t}\Omega).
    \]
\end{proposition}

\begin{proof}
    Fix any $t_{1},t_{2}\in\bbR$, $x\in\bbR^{2}$, $\lambda\in\mathcal{L}$, and $\eta>0$.
    Pick $y\in\partial\Omega^{\lambda}$ such that
    \[
      \abs{\Phi_{\eps}^{t_{1}}(x) - \Phi_{\eps}^{t_{1}}(y)} \le   
      d(\Phi_{\eps}^{t_{1}}(x), \partial\Phi_{\eps}^{t_{1}}(\Omega^{\lambda})) + \eta.
    \]
    Then Lemma~\ref{L2.5} and
    the inequality $\abs{\frac{1}{a^{2\alpha}} - \frac{1}{b^{2\alpha}}}
    \leq \frac{\abs{a - b}}{ab^{2\alpha}}$ for $a,b>0$ show that
    \begin{equation}\label{2.4}
        \begin{aligned}
            &\frac{1}{\left(
                d(\Phi_{\eps}^{t_{1}}(x),
                \partial\Phi_{\eps}^{t_{1}}(\Omega^{\lambda})) + 2\eta
            \right)^{2\alpha}}
            - \frac{1}{\left(
                d(\Phi_{\eps}^{t_{2}}(x),
                \partial\Phi_{\eps}^{t_{2}}(\Omega^{\lambda})) + \eta
            \right)^{2\alpha}}
            \\&\quad\quad\quad\quad
            \leq \frac{1}{\left(
                \abs{\Phi_{\eps}^{t_{1}}(x) - \Phi_{\eps}^{t_{1}}(y)} + \eta
            \right)^{2\alpha}}
            - \frac{1}{\left(
                \abs{\Phi_{\eps}^{t_{2}}(x) - \Phi_{\eps}^{t_{2}}(y)} + \eta
            \right)^{2\alpha}}
            \\&\quad\quad\quad\quad
            \leq \frac{\abs{\Phi_{\eps}^{t_{1}}(x) - \Phi_{\eps}^{t_{1}}(y)
            - \Phi_{\eps}^{t_{2}}(x) + \Phi_{\eps}^{t_{2}}(y)}}
            {\left(\abs{\Phi_{\eps}^{t_{1}}(x) - \Phi_{\eps}^{t_{1}}(y)} + \eta\right)
            \left(\abs{\Phi_{\eps}^{t_{2}}(x) - \Phi_{\eps}^{t_{2}}(y)} + \eta\right)^{2\alpha}}
            \\&\quad\quad\quad\quad
            \leq \frac{\exp\left(\abs{
                \int_{t_{2}}^{t_{1}}
                \norm{u_{\eps}(\omega_{\eps}^{\tau})}_{\dot{C}^{0,1}}\,d\tau
            }\right) - 1}
            {(d(\Phi_{\eps}^{t_{2}}(x),
            \partial\Phi_{\eps}^{t_{2}}(\Omega^{\lambda})) + \eta)^{2\alpha}},
        \end{aligned}
    \end{equation}
    so letting $\eta\to 0^{+}$, integrating over $\lambda$,
    and then taking supremum over $x\in\bbR^2$ shows
    \[
        L_{\tht}(\Phi_{\eps*}^{t_{1}}\Omega)
        \leq \exp\left(\abs{
            \int_{t_{2}}^{t_{1}}
            \norm{u_{\eps}(\omega_{\eps}^{\tau})}_{\dot{C}^{0,1}}\,d\tau
        }\right)L_{\tht}(\Phi_{\eps*}^{t_{2}}\Omega).
    \]
    Since $t_{1},t_{2}\in\bbR$ were arbitrary, both claims follow from this.
\end{proof}

From Lemma~\ref{L2.1}, Proposition~\ref{P2.6}, and  a Gr\"{o}nwall-type argument we now obtain the following result.

\begin{corollary}\label{C2.7}
    With $C_{\alpha}$ from Lemma~\ref{L2.1}, for all $t\in\bbR$ we have
    \[
        \max\set{
            \partial_{t}^{+}L_{\tht}(\Phi_{\eps*}^{t}\Omega),
            -\partial_{t-}L_{\tht}(\Phi_{\eps*}^{t}\Omega)
        }
        \leq C_{\alpha}L_{\tht}(\Phi_{\eps*}^{t}\Omega)^{2}.
    \]
    In particular, for all  $t\in(- \frac{1}{C_{\alpha}L_{\tht}(\Omega)},
    \frac{1}{C_{\alpha}L_{\tht}(\Omega)})$ we have
    \[
        L_{\tht}(\Phi_{\eps*}^{t}\Omega)
        \leq \frac{L_{\tht}(\Omega)}{1 - C_{\alpha}L_{\tht}(\Omega)\abs{t}}.
    \]
\end{corollary}

%\begin{proof}
%    The first claim follows from Lemma~\ref{L2.1} and     Proposition~\ref{P2.6}, and the second
%    follows from the first claim and a Gr\"{o}nwall-type argument.
%\end{proof}

Let $T_{0}\coloneqq \frac{1}{2C_{\alpha}L_{\tht}(\Omega)}$ with $C_{\alpha}$ from Lemma~\ref{L2.1}.

\begin{proposition}\label{P2.8}
    There is $\Psi:=\lim_{\eps\to 0} \Psi_{\eps}\in C\left([-T_{0},T_{0}];BC(\bbR^{2};\bbR^{2})\right)$,
%    converges uniformly to some $\Psi\in C\left([-T_{0},T_{0}];C_{b}(\bbR^{2};\bbR^{2})\right)$.
    and $\Phi^{t}\coloneqq\mathrm{Id} + \Psi^{t}$
    is a measure-preserving homeomorphism for each $t\in[-T_{0},T_{0}]$ that solves \eqref{1.6}. Moreover, for each $t\in[-T_{0},T_{0}]$ we have 
    \[
        L_{\tht}(\Phi_{*}^{t}\Omega) \leq
        \sup_{\eps>0}L_{\tht}(\Phi_{\eps*}^{t}\Omega) \leq 2L_{\tht}(\Omega)
    \]
    and
    \[
        \max\set{\norm{\Phi^{t}}_{\dot{C}^{0,1}}, \norm{(\Phi^{t})^{-1}}_{\dot{C}^{0,1}}}
        \leq \sup_{\eps>0}\max\set{
            \norm{\Phi_{\eps}^{t}}_{\dot{C}^{0,1}},
            \norm{(\Phi_{\eps}^{t})^{-1}}_{\dot{C}^{0,1}}
        }
        \leq e^{2C_{\alpha}L_{\tht}(\Omega)\abs{t}}.
    \]
\end{proposition}

\begin{proof}
    Corollary~\ref{C2.7} shows that
    \[
        M\coloneqq \sup_{\eps>0}\sup_{t\in[-T_{0},T_{0}]}L_{\tht}(\Phi_{\eps*}^{t}\Omega)
        \in [L_{\tht}(\Omega), 2L_{\tht}(\Omega)] .
    \]
    We may assume that $L_{\tht}(\Omega) > 0$ because otherwise $\omega^{0} \equiv 0$
    and the result follows trivially.
    Fix any $t_{0}\in[-T_{0},T_{0}]$ and pick any $t\in[-T_{0},T_{0}]$, $\eps>0$, and
    $\eps'\in(0,\eps)$.  Then Lemmas \ref{L2.1}, \ref{L2.2}, and \ref{L2.3} show that
    \begin{equation}\label{2.5}
        \begin{aligned}
            &\norm{u_{\eps}(\omega_{\eps}^{t})\circ
            (\operatorname{Id} + \Theta_{\eps}^{t_{0},t})
            - u_{\eps'}(\omega_{\eps'}^{t})\circ
            (\operatorname{Id} + \Theta_{\eps'}^{t_{0},t})}_{L^{\infty}}
            \\&\quad\quad\quad
            \leq \norm{u_{\eps}(\omega_{\eps}^{t})}_{\dot{C}^{0,1}}
            \norm{\Theta_{\eps}^{t_{0},t} - \Theta_{\eps'}^{t_{0},t}}_{L^{\infty}}
             + \norm{u_{\eps}(\omega_{\eps}^{t})
            - u(\omega_{\eps}^{t})}_{L^{\infty}}
            \\&\quad\quad\quad\quad\quad\quad\quad\quad\quad
           + \norm{u(\omega_{\eps}^{t})
            - u(\omega_{\eps'}^{t})}_{L^{\infty}}
            + \norm{u(\omega_{\eps'}^{t})
            - u_{\eps'}(\omega_{\eps'}^{t})}_{L^{\infty}}
            \\&\quad\quad\quad
            \leq C_{\alpha}M
            \norm{\Theta_{\eps}^{t_{0},t} - \Theta_{\eps'}^{t_{0},t}}_{L^{\infty}}
            + C_{\alpha}M
            \norm{\Phi_{\eps}^{t} - \Phi_{\eps'}^{t}}_{L^{\infty}}
            + C_{\alpha}\norm{\omega^{0}}_{L^{\infty}} \eps^{1-2\alpha}
        \end{aligned}
    \end{equation}
    where $C_{\alpha}$ (which we now fix for the rest of the proof)
    is two times the maximum of all the $C_{\alpha}$'s appearing in those lemmas.
    Integrating \eqref{2.5} between any $t_{1},t_{2}\in[-T_{0},T_{0}]$ yields
    \begin{equation}\label{2.6}
        \begin{aligned}
            &\norm{\Theta_{\eps}^{t_{0},t_{1}} - \Theta_{\eps'}^{t_{0},t_{1}}
            - \Theta_{\eps}^{t_{0},t_{2}} + \Theta_{\eps'}^{t_{0},t_{2}}}_{L^{\infty}}
            \\&\quad\quad
            \leq C_{\alpha}M\abs{\int_{t_{2}}^{t_{1}}
            \norm{\Theta_{\eps}^{t_{0},\tau} - \Theta_{\eps'}^{t_{0},\tau }}_{L^{\infty}}d\tau}
            + C_{\alpha}M\abs{\int_{t_{2}}^{t_{1}}
            \norm{\Phi_{\eps}^{\tau} - \Phi_{\eps'}^{\tau}}_{L^{\infty}}d\tau}
            \\&\quad\quad\quad\quad\quad\quad\quad\quad
            + C_{\alpha}\norm{\omega^{0}}_{L^{\infty}}\abs{t_{1} - t_{2}}\eps^{1-2\alpha}.
        \end{aligned}
    \end{equation}
    In particular, taking $t_{0} = 0$, dividing by $\abs{t_{1} - t_{2}}$,
    and letting $t_{1}\to t_{2}^{\pm}$ shows that
    \begin{align*}
        \max\set{
            \partial_{t}^{+}\norm{\Psi_{\eps}^{t} - \Psi_{\eps'}^{t}}_{L^{\infty}},
            -\partial_{t-}\norm{\Psi_{\eps}^{t} - \Psi_{\eps'}^{t}}_{L^{\infty}}
        }
        \leq 2C_{\alpha}M\norm{\Psi_{\eps}^{t} - \Psi_{\eps'}^{t}}_{L^{\infty}}
        + C_{\alpha}\norm{\omega^{0}}_{L^{\infty}}\eps^{1-2\alpha}
    \end{align*}
    for each $t\in[-T_{0},T_{0}]$, and then a Gr\"{o}nwall-type argument yields
    \[
        \norm{\Phi_{\eps}^{t} - \Phi_{\eps'}^{t}}_{L^{\infty}}
        = \norm{\Psi_{\eps}^{t} - \Psi_{\eps'}^{t}}_{L^{\infty}}
        \leq \frac{\norm{\omega^{0}}_{L^{\infty}}}{2M}
        (e^{2C_{\alpha}M\abs{t}} - 1)\eps^{1-2\alpha}.
    \]
    Applying this inequality to \eqref{2.6}, 
    %with $C_{\alpha}M$'s replaced by $2C_{\alpha}M$'s,
    dividing by $\abs{t_{1} - t_{2}}$ and then sending $t_{1}\to t_{2}^{\pm}$ shows that
    \begin{align*}
       \max & \left\{ \partial_{t}^{+}\norm{\Theta_{\eps}^{t_{0},t} - \Theta_{\eps'}^{t_{0},t}}_{L^{\infty}},
       -\partial_{t-}\norm{\Theta_{\eps}^{t_{0},t} - \Theta_{\eps'}^{t_{0},t}}_{L^{\infty}} \right\}
        \\ & \qquad\qquad\qquad\qquad\qquad \leq C_{\alpha}M\norm{\Theta_{\eps}^{t_{0},t} - \Theta_{\eps'}^{t_{0},t}}_{L^{\infty}}
        + C_{\alpha}\norm{\omega^{0}}_{L^{\infty}}
        e^{2C_{\alpha}M|t|}\eps^{1-2\alpha}
    \end{align*}
    for all $t\in[-T_0,T_0]$,
%     $t\geq 0$, and we can obtain a similar bound for
%    $-\partial_{t-}\norm{\Theta_{\eps}^{t_{0},t} - \Theta_{\eps'}^{t_{0},t}}_{L^{\infty}}$
%    and $t\leq 0$, 
    so a Gr\"{o}nwall-type argument yields
    \begin{equation*}
        \norm{\Theta_{\eps}^{t_{0},t} - \Theta_{\eps'}^{t_{0},t}}_{L^{\infty}}
        \leq \frac{\norm{\omega^{0}}_{L^{\infty}}e^{2C_{\alpha}MT_{0}}\eps^{1-2\alpha}}{M}
        \left(e^{C_{\alpha}M\abs{t - t_{0}}} - 1\right).
    \end{equation*}
    Therefore, $\Theta_{\eps}^{t_{0},\,\cdot\,}$ converges uniformly
    to some $\Theta^{t_{0},\,\cdot\,}\colon [-T_{0},T_{0}]\to BC(\bbR^{2};\bbR^{2})$
    as $\eps\to 0$. 
    
    Let $\Psi^{t}\coloneqq\Theta^{0,t}$.
    Since $(\mathrm{Id} + \Theta_{\eps}^{t_{0},t_{1}})
    \circ(\mathrm{Id}+\Theta_{\eps}^{t_{1},t_{0}}) = \mathrm{Id}$  for
    all $t_{0},t_{1}\in[-T_{0},T_{0}]$ and $\eps>0$, sending $\eps\to 0$ shows that
    $(\mathrm{Id} + \Theta^{t_{0},t_{1}})
    \circ(\mathrm{Id} + \Theta^{t_{1},t_{0}}) = \mathrm{Id}$.  In particular,
    $\Phi^{t}\coloneqq \mathrm{Id} + \Psi^{t}$ is a homeomorphism whose inverse is
    $\mathrm{Id} + \Theta^{t,0}$.
    Also, Lemma~\ref{L2.5} and the definition of $C_{\alpha}$ show that
    $\norm{\mathrm{Id} + \Theta_{\eps}^{t_{0},t}}_{\dot{C}^{0,1}}
    \leq e^{C_{\alpha}M\abs{t - t_{0}}}$ for all $t_{0},t\in[-T_{0},T_{0}]$ and $\eps>0$, thus
    $\max\set{\norm{\Phi^{t}}_{\dot{C}^{0,1}}, \norm{(\Phi^{t})^{-1}}_{\dot{C}^{0,1}}}
    \leq e^{C_{\alpha}M\abs{t}}$ holds for all $t\in[-T_{0},T_{0}]$.
    By Fatou's lemma we also have $L_{\tht}(\Phi_{*}^{t}\Omega)
    \leq \liminf_{\eps\to 0}L_{\tht}(\Phi_{\eps*}^{t}\Omega) \leq M$ for each $t\in[-T_{0},T_{0}]$.
   And since each $\Phi_{\eps}^{t}$ is measure-preserving,
    their uniform limit $\Phi^{t}$ is also such because for any open set $U\subseteq\bbR^{2}$ we have that
    $\mathbbm{1}_{U}\circ\Phi_{\eps}^{t}\to \mathbbm{1}_{U}\circ\Phi^{t}$ pointwise as $\eps\to 0$ .

    It remains to show that $\Phi^{t}$ satisfies \eqref{1.6}, that is,
    with $\omega^{t}\coloneqq\omega^{0}\circ (\Phi^{t})^{-1}$ we have
    \begin{equation}\label{2.7}
        \Phi^{t} = \mathrm{Id} + \int_{0}^{t}
        u(\omega^{\tau})\circ\Phi^{\tau}\,d\tau
    \end{equation}
    for each $t\in[-T_{0},T_{0}]$.
    Taking $t_{0} = 0$ and letting $\eps'\to 0^{+}$ in \eqref{2.5} yields
    \begin{align*}
        \norm{u_{\eps}(\omega_{\eps}^{t})\circ \Phi_{\eps}^{t}
        - u(\omega^{t})\circ \Phi^{t}}_{L^{\infty}}
        \leq 2C_{\alpha}M\norm{\Phi_{\eps}^{t} - \Phi^{t}}_{L^{\infty}}
        + C_{\alpha}\norm{\omega^{0}}_{L^{\infty}}\eps^{1-2\alpha},
    \end{align*}
    which shows that the right-hand side of
    \[
        \Phi_{\eps}^{t} = \mathrm{Id} + \int_{0}^{t}
        u(\omega_{\eps}^{\tau})\circ \Phi_{\eps}^{\tau}\,d\tau
    \]
    converges uniformly to the right-hand side of \eqref{2.7}
    as $\eps\to 0$. This now proves \eqref{1.6}.
\end{proof}

\begin{proposition}\label{P2.9}
    Let $(\Omega_{1},\theta_{1})$, $(\Omega_{2},\theta_{2})$ be
    generalized layer cake representations of $\omega^{0}$ and
    $\Phi_{1},\Phi_{2}\in C\left(I;C(\bbR^{2};\bbR^{2})\right)$ be solutions to \eqref{1.6}
    on a compact interval $I\ni 0$ that are  both measure-preserving homeomorphisms and
    $\sup_{t\in I}\max_{i\in\{1,2\}}L_{\theta_{i}}(\Phi_{i*}^{t}\Omega_{i}) <\infty$.
    Then $\Phi_{1} = \Phi_{2}$.
\end{proposition}

\begin{proof}
    Let
    \[
        M \coloneqq \sup_{t\in I}\max_{i=1,2}L_{\theta_{i}}(\Phi_{i*}^{t}\Omega_{i}).
    \]
    Then Lemmas \ref{L2.1} and \ref{L2.2} show that
    \begin{align*}
        \norm{u(\Phi_{1*}^{t}\omega^{0})\circ \Phi_{1}^{t}
        - u(\Phi_{2*}^{t}\omega^{0})\circ \Phi_{2}^{t}}_{L^{\infty}}
        &\leq \norm{u(\Phi_{1*}^{t}\omega^{0})}_{\dot{C}^{0,1}}
        \norm{\Phi_{1}^{t} - \Phi_{2}^{t}}_{L^{\infty}}
        + \norm{u(\Phi_{1*}^{t}\omega^{0}) - u(\Phi_{2*}^{t}\omega^{0})}_{L^{\infty}} \\
        &\leq C_{\alpha}M\norm{\Phi_{1}^{t} - \Phi_{2}^{t}}_{L^{\infty}}
    \end{align*}
    with some $C_{\alpha}$, which together with continuity of
    $\norm{\Phi_{1}^{t}-\Phi_{2}^{t}}_{L^{\infty}}$ in $t$ yields
    \[
        \max\set{
            \partial_{t}^{+}\norm{\Phi_{1}^{t} - \Phi_{2}^{t}}_{L^{\infty}},
            -\partial_{t-}\norm{\Phi_{1}^{t} - \Phi_{2}^{t}}_{L^{\infty}}
        }
        \leq C_{\alpha}M\norm{\Phi_{1}^{t} - \Phi_{2}^{t}}_{L^{\infty}}.
    \]
  A Gr\"{o}nwall-type argument  finishes the proof.
\end{proof}

Combining Propositions \ref{P2.8} and \ref{P2.9} with \eqref{1.7}, the latter showing that the time spans of maximal solutions for
any two generalized layer cake representations of $\omega^{0}$ must coincide (recall that
$\sup_{t\in J}\norm{(\Phi^{t})^{-1}}_{\dot{C}^{0,1}} < \infty$ for any compact interval $J\subseteq I$),
now yields  Theorem~\ref{T1.2}.



\section{Proof of Theorem~\ref{T1.4}}\label{S3}

Again, all constants $C_{\alpha}$ below
can change from one inequality to another, but they always only depend  on $\alpha$.
Suppose that the initial data $\omega^{0}\in L^{1}(\bbR^{2})\cap L^{\infty}(\bbR^{2})$
admits a generalized layer cake representation $(\Omega,\theta)$
composed of simple closed curves such that $L_{\tht}(\Omega), Q(\Omega), R_{\tht}(\Omega) < \infty$.
For each $\lambda\in\mathcal{L}$, let $z^{0,\lambda}\in\operatorname{PSC}(\bbR^{2})$ be
such that $\partial\Omega^{\lambda} = \operatorname{im}(z^{0,\lambda})$.

Let $\omega\colon I\to L^{1}(\bbR^{2})\cap L^{\infty}(\bbR^{2})$ be the
Lagrangian solution to \eqref{1.1}--\eqref{1.2} from Theorem~\ref{T1.2}, with  initial data $\omega^{0}$
and  flow map $\Phi\in C\left(I;C(\bbR^{2};\bbR^{2})\right)$. Then since  $\Phi^{t}$ is a homeomorphism for each $t\in I$, 
it follows that $\Phi_{*}^{t}\Omega$ is  composed of simple closed curves and we denote these  $z^{t,\lambda}:=\Phi^{t}\circ z^{0,\lambda} \in\operatorname{PSC}(\bbR^{2})$, where $\Phi^{t}\circ z^{0,\lambda}\in \operatorname{CC}(\bbR^{2})$ is the curve whose representative is $\Phi^{t}\circ\tilde{z}^{0,\lambda}$ whenever  $\tilde{z}^{0,\lambda}$ is a representative of $z^{0,\lambda}$ (since $\set{z^{t,\lambda}}_{t\in I}$ is clearly a connected subset of $\operatorname{CC}(\bbR^{2})$,  \cite[Lemma~B.4]{JeoZlaTouching} shows that each $z^{t,\lambda}$ is positively oriented).
% $\Phi^{t}\circ z^{0,\lambda}$
%(i.e., the equivalence class in $\operatorname{CC}(\bbR^{2})$
%of $\Phi^{t}\circ\tilde{z}^{0,\lambda}$ for any representative $\tilde{z}^{0,\lambda}$
%of $z^{0,\lambda}$) then .
%(Since $\set{z^{t,\lambda}}_{t\in I}$ is a connected subset of
%$\operatorname{CC}(\bbR^{2})$, \cite[Lemma~B.4]{JeoZlaTouching} shows that
%each $z^{t,\lambda}$ is positively oriented.)

Fix any $\eps>0$ and recall that $\Phi_{\eps}^{t}\coloneqq\mathrm{Id} + \Psi_{\eps}^{t}$,
where $\Psi_{\eps}^{t}$ is the solution to \eqref{2.1}.
For each $(t,\lambda)\in\bbR\times\mathcal{L}$ let $z_{\eps}^{t,\lambda} \coloneqq \Phi_{\eps}^{t}\circ z^{0,\lambda}$
and $\omega_{\eps}^{t}\coloneqq\omega^{0}\circ(\Phi_{\eps}^{t})^{-1}$, then fix any arclength parametrization of $z_{\eps}^{t,\lambda}$ (we denote it again $z_{\eps}^{t,\lambda}(\cdot)$) and for $s\in[0,\ell(z_{\eps}^{t,\lambda})]$ define (suppressing $\eps$ in most of this notation)
\begin{itemize}
    \item $\ell^{t,\lambda}\coloneqq \ell(z_{\eps}^{t,\lambda})$,

    \item $\mathbf{T}^{t,\lambda}(s) \coloneqq \partial_{s}z_{\eps}^{t,\lambda}(s)$,

    \item $\mathbf{N}^{t,\lambda}(s) \coloneqq \mathbf{T}^{t,\lambda}(s)^{\perp}$,

    \item $\kappa^{t,\lambda}(s) \coloneqq
    \partial_{s}^{2}z_{\eps}^{t,\lambda}(s)\cdot \mathbf{N}^{t,\lambda}(s)$,
    
        \item $\Delta^{t,\lambda,\lambda'} \coloneqq
    \Delta(z_{\eps}^{t,\lambda}, z_{\eps}^{t,\lambda'})$,

    \item $u_{\eps}^{t,\lambda}(s) \coloneqq
    u_{\eps}(\omega_{\eps}^{t}; z_{\eps}^{t,\lambda}(s))$.
\end{itemize}
Proposition \ref{P2.8} shows that $\lim_{\eps\to 0} z_\eps^{t,\lambda}= z^{t,\lambda}$ in $\operatorname{CC}(\bbR^{2})$, and
as noted in \cite[Section~4]{JeoZlaTouching}, 
\[
    \partial_{s}^{2}z_{\eps}^{t,\lambda}(s) = \partial_{s}\mathbf{T}^{t,\lambda}(s)
    = \kappa^{t,\lambda}(s)\mathbf{N}^{t,\lambda}(s)
    \qquad\textrm{and}\qquad
    \partial_{s}\mathbf{N}^{t,\lambda}(s) = -\kappa^{t,\lambda}(s)\mathbf{T}^{t,\lambda}(s)
\]
holds as well.
Then the argument in \cite[Lemma~4.1]{JeoZlaTouching} also applies here, and we obtain
\begin{equation}\label{3.2}
    \begin{aligned}
        \partial_{t}\norm{z_{\eps}^{t,\lambda}}_{\dot{H}^{2}}^{2}
        &= -3\int_{\ell^{t,\lambda}\bbT}\kappa^{t,\lambda}(s)^{2}
        \left(\partial_{s}u_{\eps}^{t,\lambda}(s) \cdot \mathbf{T}^{t,\lambda}(s)\right)ds
        \\&\qquad\qquad        
        + 2\int_{\ell^{t,\lambda}\bbT}\kappa^{t,\lambda}(s)
        \left(\partial_{s}^{2}u_{\eps}^{t,\lambda}(s) \cdot \mathbf{N}^{t,\lambda}(s)\right)ds.
    \end{aligned}
\end{equation}

In the proof of this we fix
%any $(t,\lambda)\in\bbR\times\mathcal{L}$ and 
some constant-speed parametrization
$\tilde{z}_{\eps}^{t,\lambda}\colon\bbT\to\bbR^{2}$ of $z_{\eps}^{t,\lambda}$, and for each $h\in\bbR$
let $\tilde{z}_{\eps}^{t+h,\lambda}\coloneqq
\Phi_{\eps}^{t+h}\circ(\Phi_{\eps}^{t})^{-1}\circ\tilde{z}_{\eps}^{t,\lambda}$. Since \eqref{2.2} is globally well-posed
in $BC^{2}(\bbR^{2};\bbR^{2})$ (see the paragraph before Lemma~\ref{L2.5}),
it easily follows that $\tilde{z}_{\eps}^{t,\lambda}\in H^{2}(\bbT;\bbR^{2})$ and
\begin{equation}\label{3.1}
    \tilde{z}_{\eps}^{t+h,\lambda}
    = \tilde{z}_{\eps}^{t,\lambda} + \int_{t}^{t+h}u_{\eps}(\omega_{\eps}^{\tau})
    \circ \tilde{z}_{\eps}^{\tau,\lambda}\,d\tau
\end{equation}
holds for all $h\in\bbR$ (in $H^{2}(\bbT;\bbR^{2})$).  This is also used to evaluate the time derivative of $\ell^{t,\lambda}$.

\begin{lemma}\label{L3.1}
    For any $(t,\lambda)\in\bbR\times\mathcal{L}$ we have
    \[
        \partial_{t}\ell^{t,\lambda}
        = \int_{\ell^{t,\lambda}\bbT}
        \partial_{s}u_{\eps}^{t,\lambda}(s)\cdot\mathbf{T}^{t,\lambda}(s)\,ds.
    \]
\end{lemma}

\begin{proof}
%%and the integral is taken with respect to the norm of $H^{2}(\bbT;\bbR^{2})$.
%    Let $(t,\lambda)\in\bbR\times\mathcal{L}$, fix a constant-speed parametrization
%    $\tilde{z}^{t,\lambda}\colon\bbT\to\bbR^{2}$ of $z^{t,\lambda}$
%    and define $\tilde{z}^{\tau,\lambda}$ for each $\tau\in\bbR$ as in \eqref{3.1}.
    Since $\abs{\partial_{\xi}\tilde{z}_{\eps}^{t,\lambda}(\xi)}
    = \ell^{t,\lambda}> 0$ for $\tilde{z}_{\eps}^{t,\lambda}$ as above, for any $(h,\xi)\in\bbR\times\bbT$ we get
    %, \eqref{3.1} shows
    \begin{equation*}
        \begin{aligned}
            &\abs{\partial_{\xi}\tilde{z}_{\eps}^{t+h,\lambda}(\xi)}
            - \abs{\partial_{\xi}\tilde{z}_{\eps}^{t,\lambda}(\xi)}
            \\&\quad\quad\quad
            = \frac{2\int_{t}^{t+h} \frac d{d\xi} u_{\eps}(\omega_{\eps}^{\tau};
            \tilde{z}_{\eps}^{\tau,\lambda}(\xi))
            \cdot \partial_{\xi}\tilde{z}_{\eps}^{t,\lambda}(\xi)\,d\tau}
            {\abs{\partial_{\xi}\tilde{z}_{\eps}^{t+h,\lambda}(\xi)}
            + \abs{\partial_{\xi}\tilde{z}_{\eps}^{t,\lambda}(\xi)}}
            + \frac{\abs{\int_{t}^{t+h} \frac d{d\xi} u_{\eps}(\omega_{\eps}^{\tau};
            \tilde{z}_{\eps}^{\tau,\lambda}(\xi))\,d\tau}^{2}}
            {\abs{\partial_{\xi}\tilde{z}_{\eps}^{t+h,\lambda}(\xi)}
            + \abs{\partial_{\xi}\tilde{z}_{\eps}^{t,\lambda}(\xi)}}.
        \end{aligned}
    \end{equation*}
    Since $\norm{D(u_{\eps}(\omega_{\eps}^{\tau}))}_{L^{\infty}}$ is continuous in $\tau$
    by Lemma~\ref{L2.5}, integrating in $\xi$, dividing by $h$,
    and then letting $h\to 0$ shows that
    \[
        \partial_{t}\ell^{t,\lambda}
        = \int_{\bbT}
        \frac d{d\xi} u_{\eps}(\omega_{\eps}^{t};\tilde{z}_{\eps}^{t,\lambda}(\xi))
        \cdot \frac{\partial_{\xi}\tilde{z}_{\eps}^{t,\lambda}(\xi)}
        {\abs{\partial_{\xi}\tilde{z}_{\eps}^{t,\lambda}(\xi)}}\,d\xi.
    \]
    The claim now follows by a change of variables.
\end{proof}

\begin{corollary}\label{C3.2}
    With $C_{\alpha}$ from Lemma~\ref{L2.1}, for any $(t,\lambda)\in I\times\mathcal{L}$ and all small enough $h\in\bbR$ we have
    \[
        e^{-3C_{\alpha}L_{\tht}(\Phi_{*}^{t}\Omega)\abs{h}}\ell(z^{t,\lambda})
        \leq \ell(z^{t+h,\lambda}) \leq
        e^{3C_{\alpha}L_{\tht}(\Phi_{*}^{t}\Omega)\abs{h}}\ell(z^{t,\lambda}).
    \]
%    holds for any small enough $h\in\bbR$.
\end{corollary}

\begin{proof}
    Lemmas \ref{L2.1}, \ref{L3.1} and Corollary~\ref{C2.7} show that for any  $\abs{t}\leq\frac{1}{2C_{\alpha}L_{\tht}(\Omega)}$ we have 
    \begin{equation}\label{3.3}
        \abs{\partial_{t}\ell^{t,\lambda}}
        \leq \norm{u_{\eps}(\omega_{\eps}^{t})}_{\dot{C}^{0,1}}
        \ell^{t,\lambda}
        \leq 2C_{\alpha}L_{\tht}(\Omega)\ell^{t,\lambda}
    \end{equation}
    (such $t$ must be in $I$ because $I$ is maximal).
    Since $\ell\colon\operatorname{CC}(\bbR^{2})\to[0,\infty]$ is lower semicontinuous
    (by definition) and $\lim_{\eps\to 0} z_\eps^{t,\lambda}= z^{t,\lambda}$ in $\operatorname{CC}(\bbR^{2})$, a Gr\"{o}nwall-type argument shows that
    \begin{equation}\label{3.4}
        \ell(z^{t,\lambda}) \leq e^{2C_{\alpha}L_{\tht}(\Omega)\abs{t}}\ell(z^{0,\lambda})
    \end{equation}
    holds when $\abs{t}\leq\frac{1}{2C_{\alpha}L_{\tht}(\Omega)}$.

    Next fix any $t_{0}\in I$ with $\abs{t_{0}}\leq\frac{1}{4C_{\alpha}L_{\tht}(\Omega)}$.
    Considering $\omega^{t_{0}}$ as the initial condition for \eqref{1.1}--\eqref{1.2}
    at time $ t_{0}$ and then repeating the proof of \eqref{3.4} shows that 
\[
        \ell(z^{t,\lambda}) \leq e^{2C_{\alpha}L_\tht(\Phi_{*}^{t_{0}}\Omega)\abs{t - t_{0}}}\ell(z^{t_0,\lambda})
\]
 %   \eqref{3.4} holds with $(z^{t_{0},\lambda},L_\tht(\Phi_{*}^{t_{0}}\Omega)\abs{t - t_{0}})$
  %  in place of $(z^{0,\lambda},L_\tht(\Omega)\abs{t})$
    whenever $\abs{t - t_{0}}\leq\frac{1}{2C_{\alpha}L_{\tht}(\Phi_{*}^{t_{0}}\Omega)}$.
    We do not yet know whether \eqref{3.1} holds in $H^{2}(\bbT;\bbR^{2})$ without all the $\eps$ because $\Phi^{t}$ is only Lipschitz,  but  all the involved curves are rectifiable and
    \eqref{3.1} without $\eps$ does hold in $C^{0,1}(\bbT;\bbR^{2})$.
    %, and that is enough  for further arguments.
    Since $L_{\tht}(\Phi_{*}^{t_{0}}\Omega) \leq \frac{4}{3}L_{\tht}(\Omega)$ by Corollary~\ref{C2.7},
    we see that
    %can substitute $0$ for $t$ which shows
    \[
        \ell(z^{0,\lambda}) \leq e^{3C_{\alpha}L_{\tht}(\Omega)\abs{t_{0}}}\ell(z^{t_{0},\lambda}),
    \]
and so
    \[
        e^{-3C_{\alpha}L_{\tht}(\Omega)\abs{t}}\ell(z^{0,\lambda})
        \leq \ell(z^{t,\lambda}) \leq
        e^{3C_{\alpha}L_{\tht}(\Omega)\abs{t}}\ell(z^{0,\lambda})
    \]
    holds whenever $\abs{t}\leq\frac{1}{4C_{\alpha}L_{\tht}(\Omega)}$.
    Applying this with $(\omega^t,t,h)$ in place of $(\omega^0,0,t)$,
    for any $t\in I$, now shows the claim.
\end{proof}

\begin{proposition}\label{P3.3}
    $R_{\tht}(\Phi_{*}^{t}\Omega)$ is continuous in $t$.   In particular,
    for any compact interval $J\subseteq I$ we have
    \[
        \sup_{t\in J}R_{\tht}(\Phi_{*}^{t}\Omega) < \infty.
    \]
\end{proposition}

\begin{proof}
    Take any compact interval $J\subseteq I$ containing $0$.
    With $T_{0}$ as in Proposition~\ref{P2.8},
    Lemmas~\ref{L2.1}, \ref{L2.5} and Corollary~\ref{C2.7} show that
    \begin{equation}\label{3.5}
        \abs{\Phi_{\eps}^{t_{1}}(x) - \Phi_{\eps}^{t_{1}}(y)
        - \Phi_{\eps}^{t_{2}}(x) + \Phi_{\eps}^{t_{2}}(y)}
        \leq (e^{2C_{\alpha}L_{\tht}(\Omega)\abs{t_{1}-t_{2}}} - 1)
        \abs{\Phi_{\eps}^{t_{1}}(x) - \Phi_{\eps}^{t_{1}}(y)}
    \end{equation}
    for all $\eps>0$, $t_{1},t_{2}\in[-T_{0},T_{0}]$ and $x,y\in\bbR^{2}$.
    Hence, taking $\eps\to 0$ shows that \eqref{3.5} continues to hold
    with $\Phi$ in place of $\Phi_{\eps}$. Then applying the same argument
    to $(\omega^{t},t)$ in place of $(\omega^{0},0)$ shows that
    each $t\in J$ has a neighborhood such that
    \begin{equation}\label{3.6}
        \abs{\Phi^{t_{1}}(x) - \Phi^{t_{1}}(y)
        - \Phi^{t_{2}}(x) + \Phi^{t_{2}}(y)}
        \leq (e^{2C_{\alpha}M\abs{t_{1}-t_{2}}} - 1)
        \abs{\Phi^{t_{1}}(x) - \Phi^{t_{1}}(y)}
    \end{equation}
    holds for any $t_{1},t_{2}\in J$ in that neighborhood, where
    \[
        M\coloneqq \sup_{t\in J}L_{\tht}(\Phi_{*}^t\Omega) < \infty.
    \]
 Since $J$ is an interval, it follows that \eqref{3.6} in fact holds for
    all $t_{1},t_{2}\in J$.

    Fix $t_{1},t_{2}\in J$, $\lambda,\lambda'\in\mathcal{L}$, and $\eta>0$.
    Then there are $x\in\operatorname{im}(z^{0,\lambda})$ and
    $y\in\operatorname{im}(z^{0,\lambda'})$ such that
    $\Delta(z^{t_{1},\lambda},z^{t_{1},\lambda'}) = \abs{\Phi^{t_{1}}(x) - \Phi^{t_{1}}(y)}$.
    A similar argument as in \eqref{2.4} now shows that
    \begin{align*}
        &\frac{1}{\left(
            \Delta(z^{t_{1},\lambda},z^{t_{1},\lambda'}) + \eta
        \right)^{2\alpha}}
        - \frac{1}{\left(
            \Delta(z^{t_{2},\lambda},z^{t_{2},\lambda'}) + \eta
        \right)^{2\alpha}}
        \\&\quad\quad\quad\quad
        \leq \frac{\abs{\Phi^{t_{1}}(x) - \Phi^{t_{1}}(y)
        - \Phi^{t_{2}}(x) + \Phi^{t_{2}}(y)}}
        {\left(\abs{\Phi^{t_{1}}(x) - \Phi^{t_{1}}(y)} + \eta\right)
        \left(\abs{\Phi^{t_{2}}(x) - \Phi^{t_{2}}(y)} + \eta\right)^{2\alpha}}
        \\&\quad\quad\quad\quad
        \leq \frac{e^{2C_{\alpha}M\abs{t_{1} - t_{2}}} - 1}
        {(\Delta(z^{t_{2},\lambda},z^{t_{2},\lambda'}) + \eta)^{2\alpha}},
    \end{align*}
    thus
    \begin{equation}\label{3.7}
        \frac{1}{\left(
            \Delta(z^{t_{1},\lambda},z^{t_{1},\lambda'}) + \eta
        \right)^{2\alpha}}
        \leq \frac{e^{2C_{\alpha}M\abs{t_{1} - t_{2}}}}
        {\Delta(z^{t_{2},\lambda},z^{t_{2},\lambda'})^{2\alpha}}.
    \end{equation}
    Corollary~\ref{C3.2} yields
    \[
        e^{-3C_{\alpha}M\abs{t_{1} - t_{2}}}\ell(z^{t_{2},\lambda})
        \leq \ell(z^{t_{1},\lambda}) \leq
        e^{3C_{\alpha}M\abs{t_{1} - t_{2}}}\ell(z^{t_{2},\lambda})
    \]
    for any $t_{1},t_{2}\in J$ because $J$ is an interval, and so we get
    \begin{equation}\label{3.8}
        \frac{\ell(z^{t_{1},\lambda})^{1/2}}
        {\ell(z^{t_{1},\lambda'})^{1/2}\left(
            \Delta(z^{t_{1},\lambda},z^{t_{1},\lambda'}) + \eta
        \right)^{2\alpha}}
        \leq \frac{e^{5C_{\alpha}M\abs{t_{1} - t_{2}}}\ell(z^{t_{2},\lambda})^{1/2}}
        {\ell(z^{t_{2},\lambda'})^{1/2}
        \Delta(z^{t_{2},\lambda},z^{t_{2},\lambda'})^{2\alpha}}.
    \end{equation}
    Letting $\eta\to 0^{+}$, taking the upper Lebesgue integral with respect to $\lambda'$,
    and then taking the supremum over $\lambda\in\calL$ shows that
    \begin{equation}\label{3.9}
        R_{\tht}(\Phi_{*}^{t_{1}}\Omega)
        \leq e^{5C_{\alpha}M\abs{t_{1} - t_{2}}}R_{\tht}(\Phi_{*}^{t_{2}}\Omega).
    \end{equation}
    Since $t_{1},t_{2}\in J$ were arbitrary, the claim follows.
\end{proof}

Next, we show that $Q(\Phi_{*}^{t}\Omega) < \infty$ for all small enough $t$.
Since $ \ell(\cdot)$ and $\norm{\cdot}_{\dot{H}^{2}}$ are both lower semicontinuous on $\operatorname{CC}(\bbR^{2})$ (the latter by \cite[Corollary~B.3]{JeoZlaTouching}), so is $\ell(\cdot)\norm{\cdot}_{\dot{H}^{2}}^{2}$ and so
it is enough to establish a uniform-in-$(\eps,\lambda)$ bound on
$\ell(z_{\eps}^{t,\lambda})\norm{z_{\eps}^{t,\lambda}}_{\dot{H}^{2}}^{2}$.
We do this by estimating the right-hand side of \eqref{3.2}
in Lemma~\ref{L3.7} below. Since the resulting estimate will involve
$R_{\tht}(\Phi_{\eps*}^{t}\Omega)$, we first extend a bound on $R_{\tht}(\Phi_{*}^{t}\Omega)$ from the proof of Proposition~\ref{P3.3} to $\eps>0$.

\begin{lemma}\label{L3.4}
    $R_{\tht}(\Phi_{\eps*}^{t}\Omega)$ is continuous in $t$, and for any $t\in[-T_{0},T_{0}]$ we have
    \begin{equation}\label{3.10}
        R_{\tht}(\Phi_{\eps*}^{t}\Omega) \leq e^{4C_{\alpha}L_{\tht}(\Omega)\abs{t}}R_{\tht}(\Omega)
        \leq 8R_{\tht}(\Omega),
    \end{equation}
  with    $C_{\alpha}$ from Lemma~\ref{L2.1} and $T_{0}$ from Proposition~\ref{P2.8}.
\end{lemma}

\begin{proof}
    We have \eqref{3.3} for $t\in[-T_{0},T_{0}]$, so a Gr\"{o}nwall-type argument shows that
    \[
        e^{-2C_{\alpha}L_{\tht}(\Omega)\abs{t_{1} - t_{2}}}\ell(z_{\eps}^{t_{2},\lambda})
        \leq \ell(z_{\eps}^{t_{1},\lambda}) \leq
        e^{2C_{\alpha}L_{\tht}(\Omega)\abs{t_{1} - t_{2}}}\ell(z_{\eps}^{t_{2},\lambda})
    \]
    for any $t_{1},t_{2}\in[-T_{0},T_{0}]$ and $\lambda\in\mathcal{L}$.
    Since \eqref{3.5} holds, a similar argument as in \eqref{3.7} shows
    \[
        \frac{1}{\left(
            \Delta(z_{\eps}^{t_{1},\lambda},z_{\eps}^{t_{1},\lambda'}) + \eta
        \right)^{2\alpha}}
        \leq \frac{e^{2C_{\alpha}L_{\tht}(\Omega)\abs{t_{1} - t_{2}}}}
        {\Delta(z_{\eps}^{t_{2},\lambda},z_{\eps}^{t_{2},\lambda'})^{2\alpha}}
    \]
    for any $t_{1},t_{2}\in[-T_{0},T_{0}]$, $\lambda,\lambda'\in\mathcal{L}$ and $\eta>0$.
    Hence we see that \eqref{3.8}, and thus also \eqref{3.9}, continue to hold with
    $(z_{\eps},4C_{\alpha}L_{\tht}(\Omega),\Phi_{\eps})$ in place of $(z,5C_{\alpha}M,\Phi)$.
    This now shows both claims 
    (the second inequality in \eqref{3.10} follows by the definition of $T_{0}$
    and $e^{2}\leq 8$).
\end{proof}

To estimate the right-hand side of \eqref{3.2}, we will use the following lemmas.

\begin{lemma}\label{L3.5}
    There is $C_{\alpha}$ such that for any $\beta\in(0,1]$, any $C^{1,\beta}$ closed curve
    $\gamma\colon\ell\bbT\to\bbR^{2}$ parametrized by arclength, and any $x\in\bbR^{2}$ we have
    \[
        \int_{\ell\bbT}\frac{ds}{\abs{x - \gamma(s)}^{1+2\alpha}}
        \leq C_{\alpha}\frac{\ell\norm{\gamma}_{\dot{C}^{1,\beta}}^{1/\beta}}
        {d(x,\operatorname{im}(\gamma))^{2\alpha}}.
    \]
\end{lemma}

\begin{proof}
    Let $d\coloneqq\frac{1}{4}\norm{\gamma}_{\dot{C}^{1,\beta}}^{-1/\beta}$
    and $\Delta\coloneqq d(x,\operatorname{im}(\gamma))$.
    Then \cite[Lemma~A.2]{JeoZlaTouching} shows that
    \begin{align*}
        \int_{\ell\bbT}\frac{ds}{\abs{x - \gamma(s)}^{1+2\alpha}}
        &\leq \frac{\ell}{4d}\left(
            \int_{\abs{s}\leq\Delta}\frac{ds}{\Delta^{1+2\alpha}}
            + \int_{\Delta\leq\abs{s}\leq 2d}\frac{ds}{\abs{s/2}^{1+2\alpha}}
        \right)
        + \frac{1}{\Delta^{2\alpha}}\int_{\ell\bbT}\frac{ds}{d} \\
        &\leq \frac{\ell}{2d\Delta^{2\alpha}}
        + \frac{\ell}{2^{1-2\alpha}\alpha d\Delta^{2\alpha}}
        + \frac{\ell}{d\Delta^{2\alpha}}
        = C_{\alpha}\frac{\ell\norm{\gamma}_{\dot{C}^{1,\beta}}^{1/\beta}}
        {\Delta^{2\alpha}}.
    \end{align*}
\end{proof}

\begin{lemma}\label{L3.6}
    For any $\omega\in L^{1}(\bbR^{2})\cap L^{\infty}(\bbR^{2})$ with
    a generalized layer cake representation $(\Omega,\theta)$
    such that $L_{\tht}(\Omega)<\infty$, and for any $\eps>0$ and $x,h_{1},h_{2}\in\bbR^{2}$, we have
    \begin{align*}
        D^{2}(u_{\eps}(\omega))(x)(h_{1},h_{2})
        &= \int_{\mathcal{L}}\int_{\bbR^{2}}
        D^{2}(\nabla^{\perp}K_{\eps})(x - y)(h_{1},h_{2})
        \left(\mathbbm{1}_{\Omega^{\lambda}}(y) - \mathbbm{1}_{\Omega^{\lambda}}(x)\right)
        dy\,d\theta(\lambda) \\
        &= \int_{\mathcal{L}}\int_{\Omega^{\lambda}}
        D^{2}(\nabla^{\perp}K_{\eps})(x - y)(h_{1},h_{2})
        \,dy\,d\theta(\lambda).
    \end{align*}
    Moreover, there is $C_{\alpha}$  such that
    \[
        \int_{\mathcal{L}}\int_{\bbR^{2}}
        \abs{D^{2}(\nabla^{\perp}K_{\eps})(x - y)}
        \abs{\mathbbm{1}_{\Omega^{\lambda}}(y) - \mathbbm{1}_{\Omega^{\lambda}}(x)}
        d|\theta|(\lambda)\,dy
        \leq \frac{C_{\alpha}}{\eps}
        \int_{\mathcal{L}}\frac{d|\theta|(\lambda)}{d(x,\partial\Omega^{\lambda})^{2\alpha}}.
    \]
\end{lemma}

\begin{proof}
    Oddness of $D^{2}(\nabla^{\perp}K_{\eps})$ shows that
    \begin{align*}
        D^{2}(u_{\eps}(\omega))(x)(h_{1},h_{2})
        &= \int_{\bbR^{2}}D^{2}(\nabla^{\perp}K_{\eps})(x - y)(h_{1},h_{2})
        (\omega(y) - \omega(x))\,dy \\
        &= \int_{\bbR^{2}}\int_{\mathcal{L}}
        D^{2}(\nabla^{\perp}K_{\eps})(x - y)(h_{1},h_{2})
        \left(\mathbbm{1}_{\Omega^{\lambda}}(y) - \mathbbm{1}_{\Omega^{\lambda}}(x)\right)
        d\theta(\lambda)\,dy.
    \end{align*}
    Then proceeding as in Lemma~\ref{L2.1} and using 
    $\abs{D^{2}(\nabla^{\perp}K_{\eps})(x - y)}
    \leq \frac{C_{\alpha}}{\eps\abs{x - y}^{2+2\alpha}}$
    in place of $\abs{D(\nabla^{\perp}K_{\eps})(x - y)}
    \leq \frac{C_{\alpha}}{\abs{x - y}^{2+2\alpha}}$ proves the second claim.
    Fubini's theorem now yields the first equality of the first claim,
    and the second one follows by oddness of $D^{2}(\nabla^{\perp}K_{\eps})$.
\end{proof}

\begin{lemma}\label{L3.7}
    There is $C_{\alpha}$ such that for each $(t,\lambda)\in\bbR\times\mathcal{L}$ and $\eps>0$ we have
    \begin{equation}\label{3.11}
        \abs{\partial_{t}\norm{z_{\eps}^{t,\lambda}}_{\dot{H}^{2}}^{2}}
        \leq C_{\alpha}(L_{\tht}(\Phi_{\eps*}^{t}\Omega) + R_{\tht}(\Phi_{\eps*}^{t}\Omega)) \,
        Q(\Phi_{\eps*}^{t}\Omega)\norm{z_{\eps}^{t,\lambda}}_{\dot{H}^{2}}^{2}
    \end{equation}
\end{lemma}

\begin{proof}
%    All constants $C_{\alpha}$ in this proof depend only on $\alpha$ and
    %can change from one inequality to another. 
    With $z_{\eps}^{t,\lambda}(\cdot)$ being
    the previously fixed arclength parametrization of $z_{\eps}^{t,\lambda}$, we have
    \begin{align*}
        \partial_{s}^{2}u_{\eps}^{t,\lambda}(s)
        &= D^{2}(u_{\eps}(\omega_{\eps}^{t}))(z_{\eps}^{t,\lambda}(s))
        (\mathbf{T}^{t,\lambda}(s), \mathbf{T}^{t,\lambda}(s))
        + \kappa^{t,\lambda}(s)D(u_{\eps}(\omega_{\eps}^{t}))
        (z_{\eps}^{t,\lambda}(s))(\mathbf{N}^{t,\lambda}(s)).
    \end{align*}
    Hence \eqref{3.2} yields
    \begin{equation}\label{3.12}
        \begin{aligned}
            \partial_{t}\norm{z_{\eps}^{t,\lambda}}_{\dot{H}^{2}}^{2} &=
            \int_{\ell^{t,\lambda}\bbT}\kappa^{t,\lambda}(s)^{2}
            \left(
                2D(u_{\eps}(\omega_{\eps}^{t}))(z_{\eps}^{t,\lambda}(s))
                (\mathbf{N}^{t,\lambda}(s))
                \cdot \mathbf{N}^{t,\lambda}(s)
                - 3\,\partial_{s}u_{\eps}^{t,\lambda}(s)\cdot\mathbf{T}^{t,\lambda}(s)
            \right)ds
            \\&\quad\quad
            + 2\int_{\ell^{t,\lambda}\bbT}\kappa^{t,\lambda}(s)
            \left(
                D^{2}(u_{\eps}(\omega_{\eps}^{t}))(z_{\eps}^{t,\lambda}(s))
                (\mathbf{T}^{t,\lambda}(s),\mathbf{T}^{t,\lambda}(s))
                \cdot \mathbf{N}^{t,\lambda}(s)
            \right)ds.
        \end{aligned}
    \end{equation}
    Lemma~\ref{L2.1} shows that the absolute value of the first integral is bounded by
    \[
        5\norm{u_{\eps}(\omega_{\eps}^{t})}_{\dot{C}^{0,1}}
        \norm{z_{\eps}^{t,\lambda}}_{\dot{H}^{2}}^{2}
        \leq C_{\alpha}L_{\tht}(\Phi_{\eps*}^{t}\Omega)\norm{z_{\eps}^{t,\lambda}}_{\dot{H}^{2}}^{2}
        \leq C_{\alpha}L_{\tht}(\Phi_{\eps*}^{t}\Omega)
        Q(\Phi_{\eps*}^{t}\Omega)\norm{z_{\eps}^{t,\lambda}}_{\dot{H}^{2}}^{2},
    \]
    where the second inequality holds by
    $Q(\Phi_{\eps*}^{t}\Omega)\geq 4$, which in turn follows from \cite[Lemma~A.1]{JeoZlaTouching}.
    Hence, it remains to estimate the second integral, which we denote $G_{1}$.
    We will suppress $t$ from the notation for the sake of simplicity
    because  it will be fixed in the arguments below.
    
    Since $L_{\tht}(\Phi_{\eps*}\Omega)<\infty$, Lemma~\ref{L3.6} and Green's theorem show that
    \begin{align*}
        G_{1} &= -\int_{\mathcal{L}}\int_{\ell^{\lambda}\bbT\times\ell^{\lambda'}\bbT}
        \kappa^{\lambda}(s)D^{2}K_{\eps}(z_{\eps}^{\lambda}(s) - z_{\eps}^{\lambda'}(s'))
        (\mathbf{T}^{\lambda}(s),\mathbf{T}^{\lambda}(s))
        (\mathbf{T}^{\lambda'}(s')\cdot \mathbf{N}^{\lambda}(s))
        \,ds'\,ds\,d\theta(\lambda').
    \end{align*}
    From
    \[
        \mathbf{T}^{\lambda}(s) =
        (\mathbf{T}^{\lambda'}(s')\cdot\mathbf{T}^{\lambda}(s))\mathbf{T}^{\lambda'}(s')
        + (\mathbf{N}^{\lambda'}(s')\cdot\mathbf{T}^{\lambda}(s))\mathbf{N}^{\lambda'}(s')
    \]
    and $\mathbf{N}^{\lambda'}(s')\cdot\mathbf{T}^{\lambda}(s)
    = -\mathbf{T}^{\lambda'}(s')\cdot\mathbf{N}^{\lambda}(s)$
    it now follows that $\abs{G_{1}}$ is bounded by the sum of 
    \begin{align*}
        G_{2} &\coloneqq \overline{\int_{\mathcal{L}}}
        \bigg|\int_{\ell^{\lambda}\bbT\times\ell^{\lambda'}\bbT}
        \kappa^{\lambda}(s)
        D^{2}K_{\eps}(z_{\eps}^{\lambda}(s) - z_{\eps}^{\lambda'}(s'))
        (\mathbf{T}^{\lambda}(s), \mathbf{T}^{\lambda'}(s'))
        \\&\qquad\qquad\qquad\qquad
        (\mathbf{T}^{\lambda'}(s') \cdot \mathbf{N}^{\lambda}(s))
        (\mathbf{T}^{\lambda'}(s') \cdot \mathbf{T}^{\lambda}(s))
        \,ds'\,ds\bigg|\,d|\theta|(\lambda'), \\
        G_{3} &\coloneqq \overline{\int_{\mathcal{L}}}
        \bigg|\int_{\ell^{\lambda}\bbT\times\ell^{\lambda'}\bbT}
        \kappa^{\lambda}(s)
        D^{2}K_{\eps}(z_{\eps}^{\lambda}(s) - z_{\eps}^{\lambda'}(s'))
        (\mathbf{T}^{\lambda}(s), \mathbf{N}^{\lambda'}(s'))
        (\mathbf{T}^{\lambda'}(s') \cdot \mathbf{N}^{\lambda}(s))^{2}
        \,ds'\,ds\bigg|\,d|\theta|(\lambda'),
    \end{align*}
    which we estimate separately next.

    \textbf{Estimate for $G_{2}$.} Since
    \begin{align*}
        &\frac{\partial}{\partial s'}\left(
            DK_{\eps}(z_{\eps}^{\lambda}(s) - z_{\eps}^{\lambda'}(s'))(\mathbf{T}^{\lambda}(s))
            (\mathbf{T}^{\lambda'}(s') \cdot \mathbf{N}^{\lambda}(s))
            (\mathbf{T}^{\lambda'}(s') \cdot \mathbf{T}^{\lambda}(s))
        \right)
        \\&\quad\quad
        = -D^{2}K_{\eps}(z_{\eps}^{\lambda}(s) - z_{\eps}^{\lambda'}(s'))
        (\mathbf{T}^{\lambda}(s), \mathbf{T}^{\lambda'}(s'))
        (\mathbf{T}^{\lambda'}(s') \cdot \mathbf{N}^{\lambda}(s))
        (\mathbf{T}^{\lambda'}(s') \cdot \mathbf{T}^{\lambda}(s))
        \\&\quad\quad\quad\ 
        + \kappa^{\lambda'}(s')DK_{\eps}(z_{\eps}^{\lambda}(s) - z_{\eps}^{\lambda'}(s'))
        \left(
            (\mathbf{T}^{\lambda'}(s') \cdot \mathbf{T}^{\lambda}(s))^{2}
            - (\mathbf{T}^{\lambda'}(s') \cdot \mathbf{N}^{\lambda}(s))^{2}
        \right),
    \end{align*}
    we see that
    \begin{align*}
       G_{2} \leq C_{\alpha}\overline{\int_{\mathcal{L}}}
        \int_{\ell^{\lambda}\bbT\times\ell^{\lambda'}\bbT}
        \frac{\abs{\kappa^{\lambda}(s)\kappa^{\lambda'}(s')}}
        {\abs{z_{\eps}^{\lambda}(s) - z_{\eps}^{\lambda'}(s')}^{1+2\alpha}}
        \,ds'\,ds\,d|\theta|(\lambda').
    \end{align*}
    By the Schwarz inequality, the inner integral
    of the right-hand side is bounded by
    \begin{align*}
        \left(
            \int_{\ell^{\lambda}\bbT}\kappa^{\lambda}(s)^{2}
            \int_{\ell^{\lambda'}\bbT}\frac{ds'}
            {\abs{z_{\eps}^{\lambda}(s) - z_{\eps}^{\lambda'}(s')}^{1+2\alpha}}
            \,ds
        \right)^{1/2}
        \left(
            \int_{\ell^{\lambda'}\bbT}\kappa^{\lambda'}(s')^{2}
            \int_{\ell^{\lambda}\bbT}\frac{ds}
            {\abs{z_{\eps}^{\lambda}(s) - z_{\eps}^{\lambda'}(s')}^{1+2\alpha}}
            \,ds'
        \right)^{1/2},
    \end{align*}
    and Lemma~\ref{L3.5} with $\beta=\frac 12$ shows that this is bounded by
    \begin{align*}
        C_{\alpha}\left(
            \norm{z_{\eps}^{\lambda}}_{\dot{H}^{2}}^{2}
            \frac{\ell^{\lambda'}\norm{z_{\eps}^{\lambda'}}_{\dot{H}^{2}}^{2}}
            {(\Delta^{\lambda,\lambda'})^{2\alpha}}
        \right)^{1/2}
        \left(
            \norm{z_{\eps}^{\lambda'}}_{\dot{H}^{2}}^{2}
            \frac{\ell^{\lambda}\norm{z_{\eps}^{\lambda}}_{\dot{H}^{2}}^{2}}
            {(\Delta^{\lambda,\lambda'})^{2\alpha}}
        \right)^{1/2}
        = C_{\alpha}\norm{z_{\eps}^{\lambda}}_{\dot{H}^{2}}^{2}
        \frac{\ell^{\lambda'}\norm{z_{\eps}^{\lambda'}}_{\dot{H}^{2}}^{2} 
        (\ell^{\lambda})^{1/2}}
        {(\ell^{\lambda'})^{1/2}(\Delta^{\lambda,\lambda'})^{2\alpha}}.
    \end{align*}
    Therefore
    \begin{align*}
        \abs{G_{2}} \leq C_{\alpha}R_{\tht}(\Phi_{\eps*}\Omega)
        Q(\Phi_{\eps*}\Omega)\norm{z_{\eps}^{\lambda}}_{\dot{H}^{2}}^{2}.
    \end{align*}

    \textbf{Estimate for $G_{3}$.} We can assume $R_{\tht}(\Phi_{\eps*}\Omega)<\infty$,
    in which case $\Delta(z_{\eps}^{\lambda},z_{\eps}^{\lambda'}) > 0$
    for $|\theta|$-almost all $\lambda'$.  Thus we can apply \cite[Lemma~A.4]{JeoZlaTouching}
    to conclude that
    \begin{align*}
        G_{3} &\leq C_{\alpha}\overline{\int_{\mathcal{L}}}
        \int_{\ell^{\lambda}\bbT\times \ell^{\lambda'}\bbT}
        \frac{\abs{\kappa^{\lambda}(s)}\left(
            \mathcal{M}\kappa^{\lambda}(s)
            + \mathcal{M}\kappa^{\lambda'}(s')
        \right)}
        {\abs{z_{\eps}^{\lambda}(s) - z_{\eps}^{\lambda'}(s')}^{1+2\alpha}}
        \,ds'\,ds\,d|\theta|(\lambda'),
    \end{align*}
    where $\mathcal{M}$ is the maximal operator from (A.2) in \cite{JeoZlaTouching}.
    Let us split the integrand into the sum of two terms with numerators
    $\abs{\kappa^{\lambda}(s)}\mathcal{M}\kappa^{\lambda}(s)$ and
    $\abs{\kappa^{\lambda}(s)}\mathcal{M}\kappa^{\lambda'}(s')$, respectively.  Then
     the maximal inequality (see (A.3) in \cite{JeoZlaTouching}) shows that  the same argument
    as in the estimate for $G_{2}$ bounds the integral of the second term by
    $C_{\alpha}R_{\tht}(\Phi_{\eps*}\Omega)Q(\Phi_{\eps*}\Omega)
    \norm{z_{\eps}^{\lambda}}_{\dot{H}^{2}}$.
    As for the first term, Lemma~\ref{L3.5} with $\beta=\frac 12$, Schwarz inequality, and
    the maximal inequality show that
    \begin{align*}
        &\overline{\int_{\mathcal{L}}}
        \int_{\ell^{\lambda}\bbT\times \ell^{\lambda'}\bbT}
        \frac{\abs{\kappa^{\lambda}(s)}\mathcal{M}\kappa^{\lambda}(s)}
        {\abs{z_{\eps}^{\lambda}(s) - z_{\eps}^{\lambda'}(s')}^{1+2\alpha}}
        \,ds'\,ds\,d|\theta|(\lambda')
        \\&\quad\quad \leq
        \overline{\int_{\mathcal{L}}}
        \int_{\ell^{\lambda}\bbT}\abs{\kappa^{\lambda}(s)}\mathcal{M}\kappa^{\lambda}(s)
        \frac{C_{\alpha}\ell^{\lambda'}\norm{z_{\eps}^{\lambda'}}_{\dot{H}^{2}}^{2}}
        {d(z_{\eps}^{\lambda}(s), \operatorname{im}(z_{\eps}^{\lambda'}))^{2\alpha}}
        \,ds\,d|\theta|(\lambda')
        \\&\quad\quad \leq
        C_{\alpha}Q(\Phi_{\eps*}\Omega)
        \int_{\mathcal{L}}
        \int_{\ell^{\lambda}\bbT}
        \frac{\abs{\kappa^{\lambda}(s)}\mathcal{M}\kappa^{\lambda}(s)}
        {d(z_{\eps}^{\lambda}(s), \operatorname{im}(z_{\eps}^{\lambda'}))^{2\alpha}}
        \,ds\,d|\theta|(\lambda')
        \\&\quad\quad \leq
        C_{\alpha}L_{\tht}(\Phi_{\eps*}\Omega)Q(\Phi_{\eps*}\Omega)
        \int_{\ell^{\lambda}\bbT}
        \abs{\kappa^{\lambda}(s)}\mathcal{M}\kappa^{\lambda}(s)\,ds
        \\&\quad\quad \leq C_{\alpha}L_{\tht}(\Phi_{\eps*}\Omega)Q(\Phi_{\eps*}\Omega)
        \norm{z_{\eps}^{\lambda}}_{\dot{H}^{2}}^{2},
    \end{align*}
    where  the regular (rather than upper) integral can be taken after the second inequality  because
    the integrand is jointly measurable in $(s,\lambda')$.
    Aggregating the estimates for $G_{2}$ and $G_{3}$ now yields the desired conclusion.
\end{proof}

Lemmas \ref{L3.1} and \ref{L3.7} suggest that we may have an $\eps$-independent estimate
\begin{equation}\label{3.13}
    \max\set{\partial_{t}^{+}Q(\Phi_{\eps*}^{t}\Omega),
    -\partial_{t-}Q(\Phi_{\eps*}^{t}\Omega)}
    \leq C_{\alpha}(L_{\tht}(\Phi_{\eps*}^{t}\Omega) + R_{\tht}(\Phi_{\eps*}^{t}\Omega))
    Q(\Phi_{\eps*}^{t}\Omega)^{2}
\end{equation}
from which we can run a Gr\"{o}nwall-type argument. However,
because of the factor $Q(\Phi_{\eps*}^{t}\Omega)$ in the right-hand side of \eqref{3.11},
\eqref{3.13} follows from these Lemmas only if we know a priori that
$Q(\Phi_{\eps*}^{t}\Omega)$ is upper semicontinuous (or locally bounded,
which implies upper semicontinuity via \eqref{3.11}).

One way to derive upper semicontinuity (or local boundedness) would be
again through estimating the right-hand side of \eqref{3.2}
in terms of $\norm{z_{\eps}^{t,\lambda}}_{\dot{H}^{2}}^{2}$.
In contrast to \eqref{3.11}, such an estimate can depend on $\eps$
while it must not have the factor $Q(\Phi_{\eps*}^{t}\Omega)$.
However, the second integral of \eqref{3.12} contains only one factor of $\kappa^{t,\lambda}(s)$,
so the trivial bound we obtain by estimating $D^{2}(u_{\eps}(\omega_{\eps}^{t}))$
by its $L^{\infty}$ norm will contain the $L^{1}$ norm of $\kappa^{t,\lambda}$,
and in order to replace it by the $L^{2}$ norm we have to introduce
an additional factor of $(\ell^{t,\lambda})^{1/2}$.
This means that unless we impose an upper on $\ell^{t,\lambda}$, the resulting estimate
will not be enough for concluding upper semicontinuity (nor local boundedness),
because it does not rule out the possibility that
over an arbitrarily small time interval and for an arbitrarily large constant $M$,
some $\ell^{t,\lambda}\norm{z_{\eps}^{t,\lambda}}_{\dot{H}^{2}}^{2}$ with very large
$\ell^{t,\lambda}$ grows larger than $M$.
Another way would be to estimate the $\dot{H}^{2}$ norm of
$\Phi_{\eps}^{t}\circ\tilde{z}^{0,\lambda}$ where
$\tilde{z}^{0,\lambda}\colon\bbT\to\bbR^{2}$ is a constant-speed parametrization
of $z^{0,\lambda}$, but it runs into a similar issue.

To resolve this, we now introduce a sequence of approximations of $\omega_{\eps}^{t}$.
Take an increasing sequence $\set{\mathcal{L}_{N}'}_{N=1}^{\infty}$ of
measurable subsets of $\mathcal{L}$ such that $|\theta|(\mathcal{L}_{N}')<\infty$
for each $N\in\bbN$ and $\mathcal{L} = \bigcup_{N=1}^{\infty}\mathcal{L}_{N}'$.
For each $N\in\bbN$, let
\[
    \mathcal{L}_{N}\coloneqq \set{\lambda\in\mathcal{L}_{N}'\colon
    \ell(z^{0,\lambda}) \leq N},\quad
    \Omega_{N}\coloneqq \Omega\cap(\bbR^{2}\times\mathcal{L}_{N}),\quad
    \omega_{N}^{0}(x)\coloneqq
    \int_{\mathcal{L}_{N}}\mathbbm{1}_{\Omega^{\lambda}}(x)\,d\theta(\lambda).
\]
Note that $\omega_{N}^{0} \in L^{1}(\bbR^{2})\cap L^{\infty}(\bbR^{2})$ because
$\norm{\omega_{N}^{0}}_{L^{\infty}} \leq |\theta|(\mathcal{L}_{N}') <\infty$ and
also $\norm{\omega_{N}^{0}}_{L^{1}} \leq \frac{N^{2}}{4\pi}|\theta|(\mathcal{L}_{N}') < \infty$
by the isoperimetric inequality. Clearly, $L_{\tht}(\Omega_{N}) \leq L_{\tht}(\Omega) < \infty$.

Let $\Phi_{\eps,N}\in C\left(\bbR;C(\bbR^{2};\bbR^{2})\right)$ be the corresponding
$\eps$-mollified flow map, i.e., the identity map plus the solution to \eqref{2.1}
with $\omega_{N}^{0}$ in place of $\omega^{0}$.
Let $z_{\eps,N}^{t,\lambda}\coloneqq \Phi_{\eps,N}^{t}\circ z^{0,\lambda}$
and $\omega_{\eps,N}^{t} \coloneqq \omega_{N}^{0} \circ (\Phi_{\eps,N}^{t})^{-1}$.

Then \eqref{3.12}, Cauchy-Schwarz inequality, the inequality
$\ell(\gamma)\norm{\gamma}_{\dot{H}^{2}}^{2} \geq 4$
for any $\gamma\in\operatorname{CC}(\bbR^{2})$
(which follows by \cite[Lemma~A.1]{JeoZlaTouching}) and \eqref{1.8} show
\begin{align*}
    \abs{\partial_{t}\norm{z_{\eps,N}^{t,\lambda}}_{\dot{H}^{2}}^{2}}
    &\leq 5\norm{u_{\eps}(\omega_{\eps,N}^{t})}_{\dot{C}^{0,1}}
    \norm{z_{\eps,N}^{t,\lambda}}_{\dot{H}^{2}}^{2}
    + 2\norm{D^{2}(u_{\eps}(\omega_{\eps,N}^{t}))}_{L^{\infty}}
    \ell(z_{\eps,N}^{t,\lambda})^{1/2}\norm{z_{\eps,N}^{t,\lambda}}_{\dot{H}^{2}} \\
    &\leq \left(
        5\norm{D(\nabla^{\perp}K_{\eps})}_{L^{\infty}}
        + \norm{D^{2}(\nabla^{\perp}K_{\eps})}_{L^{\infty}}N
    \right)\norm{\omega_{N}^{0}}_{L^{1}}
    \norm{z_{\eps,N}^{t,\lambda}}_{\dot{H}^{2}}^{2},
\end{align*}
which implies
\[
    \norm{z_{\eps,N}^{t+h,\lambda}}_{\dot{H}^{2}}^{2}
    \leq e^{C\abs{h}}\norm{z_{\eps,N}^{t,\lambda}}_{\dot{H}^{2}}^{2}
\]
for each $(t,h,\lambda)\in\bbR^{2}\times\mathcal{L}_{N}$, for some constant
$C$ that depends on ${\alpha,\eps,N,\norm{\omega_{N}^{0}}_{L^{1}}}$, but not on
$\lambda\in\mathcal{L}_{N}$. Together with the inequality
$\ell(z_{\eps,N}^{t+h,\lambda}) \leq e^{C\abs{h}}\ell(z_{\eps,N}^{t,\lambda})$
for another $C$ that depends on ${\alpha,\eps,\norm{\omega_{N}^{0}}_{L^{1}}}$,
which follows from \eqref{3.3} (with $(\omega_{\eps,N},z_{\eps,N})$
in place of $(\omega_{\eps},z_{\eps})$) and \eqref{1.8}, this shows
\[
    Q(\Phi_{\eps,N*}^{t+h}\Omega) \leq e^{C\abs{h}}Q(\Phi_{\eps,N*}^{t}\Omega),
\]
from which we conclude upper semicontinuity of $Q(\Phi_{\eps,N*}^{t}\Omega)$ in $t$.

From Lemmas \ref{L3.1}, \ref{L3.4}, \ref{L3.7}, Corollary~\ref{C2.7}, and \eqref{3.3} with
$(\omega_{\eps,N},\Omega_{N},z_{\eps,N},\Phi_{\eps,N})$ in place of
$(\omega_{\eps},\Omega,z_{\eps}, \Phi_{\eps})$,
and the inequalities $L_{\tht}(\Omega_{N})\leq L_{\tht}(\Omega)$,
$R_{\tht}(\Omega_{N})\leq R_{\tht}(\Omega)$, and $Q(\Phi_{\eps,N*}^{t}\Omega) \geq 4$, we see that
\begin{align*}
    \ell(z_{\eps,N}^{t+h,\lambda})\norm{z_{\eps,N}^{t+h,\lambda}}_{\dot{H}^{2}}^{2}
    - Q(\Phi_{\eps,N*}^{t}\Omega)
    &\leq \ell(z_{\eps,N}^{t+h,\lambda})\norm{z_{\eps,N}^{t+h,\lambda}}_{\dot{H}^{2}}^{2}
    - \ell(z_{\eps,N}^{t,\lambda})\norm{z_{\eps,N}^{t,\lambda}}_{\dot{H}^{2}}^{2} \\
    &\leq C_{\alpha}\abs{\int_{t}^{t+h}
    (L_{\tht}(\Omega) + R_{\tht}(\Omega))Q(\Phi_{\eps,N*}^{\tau}\Omega)^{2}\,d\tau}
\end{align*}
holds for each $(t,t+h,\lambda)\in[-T_{0},T_{0}]^{2}\times\mathcal{L}_{N}$,
for some $C_{\alpha}$ that depends only on $\alpha$.
By upper semicontinuity of $Q(\Phi_{\eps,N*}^{t}\Omega)$,
taking supremum over $\lambda\in\mathcal{L}_{N}$,
dividing by $\abs{h}$ and then letting $h\to 0$ yields
\[
    \max\set{
        \partial_{t}^{+}Q(\Phi_{\eps,N*}^{t}\Omega),
        -\partial_{t-}Q(\Phi_{\eps,N*}^{t}\Omega)
    } \leq C_{\alpha}(L_{\tht}(\Omega) + R_{\tht}(\Omega))Q(\Phi_{\eps,N*}^{t}\Omega)^{2},
\]
which we then use with a Gr\"{o}nwall-type argument to conclude
\begin{equation}\label{3.14}
    Q(\Phi_{\eps,N*}^{t}\Omega) \leq \frac{Q(\Omega_{N})}
    {1 - C_{\alpha}(L_{\tht}(\Omega) + R_{\tht}(\Omega))Q(\Omega_{N})\abs{t}}
    \leq \frac{Q(\Omega)}
    {1 - C_{\alpha}(L_{\tht}(\Omega) + R_{\tht}(\Omega))Q(\Omega)\abs{t}}
\end{equation}
for any $t$ with $\abs{t} < \frac{1}{C_{\alpha}(L_{\tht}(\Omega) + R_{\tht}(\Omega))Q(\Omega)}$,
where the second inequality is because $Q(\Omega_{N}) \leq Q(\Omega)$.

In order to turn \eqref{3.14} into a bound on $Q(\Phi_{*}^{t}\Omega)$,
we need to know if $z_{\eps,N}$ converges to $z_{\eps}$ when $N\to\infty$.
This indeed happens if $\omega_{N}^{0} \to \omega^{0}$ in $L^{1}$
(which also implies the convergence in $L^{\infty}$ since
$\norm{\omega_{N}^{0}}_{\dot{C}^{0,2\alpha}}$ is uniformly bounded),
but our assumptions on $\Omega$ alone do not seem to imply the $L^{1}$ convergence.
Note that if $\Omega^{\lambda}$'s for $\lambda$'s in the positive part of $\theta$
are disjoint from those for $\lambda$'s in the negative part of $\theta$,
in particular if $\Omega^{\lambda}$'s are the super-level sets
of $(\omega^{0})^{+}$ and $(\omega^{0})^{-}$,
then $L^{1}$ convergence does follow. However, we are allowing
these ``positive super-level sets'' and ``negative super-level sets'' to sit on top of each other,
possibly creating ``ripples'' of the graph of $\omega^{0}$.

Even though we do not have the $L^{1}$ convergence, we can still show that
the difference between $z_{\eps,N}$ and $z_{\eps}$ is almost constant,
although that constant may vary as $N$ varies, as the following lemma shows.

\begin{lemma}\label{L3.8}
    Let $T_{0}$ be as in Proposition~\ref{P2.8}.
    Then for any $R\geq 0$ and $x_{0}\in\bbR^{2}$,
    \[
        \lim_{N\to\infty}\sup_{t\in[-T_{0},T_{0}]}\norm{\left.\left(
            \Phi_{\eps,N}^{t} - \Phi_{\eps}^{t}
            - \Phi_{\eps,N}^{t}(x_{0}) + \Phi_{\eps}^{t}(x_{0})
        \right)\right|_{B_{R}(x_{0})}}_{L^{\infty}} = 0.
    \]
\end{lemma}

Once this is shown, for each $\lambda\in\mathcal{L}$,
we pick any constant-speed parametrization $\tilde{z}^{0,\lambda}\colon\bbT\to\bbR^{2}$
of $z^{0,\lambda}$, then
\begin{align*}
    &\norm{\Phi_{\eps,N}^{t}\circ\tilde{z}^{0,\lambda}
    - \Phi_{\eps}^{t}\circ\tilde{z}^{0,\lambda}
    - \Phi_{\eps,N}^{t}\circ\tilde{z}^{0,\lambda}(0)
    + \Phi_{\eps}^{t}\circ\tilde{z}^{0,\lambda}(0)}_{L^{\infty}}
    \\&\quad\quad
    \leq \norm{\left.\left(
        \Phi_{\eps,N}^{t} - \Phi_{\eps}^{t}
        - \Phi_{\eps,N}^{t}(\tilde{z}^{0,\lambda}(0))
        + \Phi_{\eps}^{t}(\tilde{z}^{0,\lambda}(0))
    \right)\right|_{B_{\ell(z^{0,\lambda})/2}(\tilde{z}^{0,\lambda}(0))}}_{L^{\infty}} \to 0
\end{align*}
as $N\to\infty$, for each $t\in\bbR$. This shows that
$w_{\eps,N}^{t,\lambda}\coloneqq z_{\eps,N}^{t,\lambda}
- \Phi_{\eps,N}^{t}\circ\tilde{z}^{0,\lambda}(0)
+ \Phi_{\eps}^{t}\circ\tilde{z}^{0,\lambda}(0) \to z_{\eps}^{t,\lambda}$
in $\operatorname{CC}(\bbR^{2})$ as $N\to\infty$.
Since $w_{\eps,N}^{t,\lambda}$ is just a spatial translation of $z_{\eps,N}^{t,\lambda}$,
we have $\ell(w_{\eps,N}^{t,\lambda}) = \ell(z_{\eps,N}^{t,\lambda})$
and $\norm{w_{\eps,N}^{t,\lambda}}_{\dot{H}^{2}} = \norm{z_{\eps,N}^{t,\lambda}}_{\dot{H}^{2}}$,
so the lower semicontinuity of the functional
$\gamma\mapsto\ell(\gamma)\norm{\gamma}_{\dot{H}^{2}}^{2}$
on $\operatorname{CC}(\bbR^{2})$ shows
\begin{align*}
    \ell(z_{\eps}^{t,\lambda})
    \norm{z_{\eps}^{t,\lambda}}_{\dot{H}^{2}}^{2}
    &\leq \liminf_{N\to\infty}\ell(w_{\eps,N}^{t,\lambda})
    \norm{w_{\eps,N}^{t,\lambda}}_{\dot{H}^{2}}^{2}
    = \liminf_{N\to\infty}\ell(z_{\eps,N}^{t,\lambda})
    \norm{z_{\eps,N}^{t,\lambda}}_{\dot{H}^{2}}^{2} \\
    &\leq \frac{Q(\Omega)}
    {1 - C_{\alpha}(L_{\tht}(\Omega) + R_{\tht}(\Omega))Q(\Omega)\abs{t}}
\end{align*}
for any $t$ with $\abs{t} < \frac{1}{C_{\alpha}(L_{\tht}(\Omega) + R_{\tht}(\Omega))Q(\Omega)}$.
Then, again by lower semicontinuity of
$\gamma\mapsto\ell(\gamma)\norm{\gamma}_{\dot{H}^{2}}^{2}$,
letting $\eps\to 0^{+}$ yields
\[
    \ell(z^{t,\lambda})
    \norm{z^{t,\lambda}}_{\dot{H}^{2}}^{2} \leq \frac{Q(\Omega)}
    {1 - C_{\alpha}(L_{\tht}(\Omega) + R_{\tht}(\Omega))Q(\Omega)\abs{t}},
\]
so taking supremum over $\lambda$ finally gives
\[
    Q(\Phi_{*}^{t}\Omega) \leq \frac{Q(\Omega)}
    {1 - C_{\alpha}(L_{\tht}(\Omega) + R_{\tht}(\Omega))Q(\Omega)\abs{t}},
\]
from which Theorem~\ref{T1.4} follows.

Now it remains to prove Lemma~\ref{L3.8}.

\begin{proof}[Proof of Lemma~\ref{L3.8}]
    In this proof, all constants written as $C$ with subscripts can change
    from one inequality to another, and they depend only on the indicated variables;
    e.g., $C_{\alpha}$ depends only on $\alpha$ and $C_{\alpha,\eps}$ depends only on
    $\alpha$ and $\eps$.

    First, note that the inequality $L_{\tht}(\Omega_{N})\leq L_{\tht}(\Omega)$
    and Proposition~\ref{P2.8} show
    \begin{equation}\label{3.15}
        \sup_{t\in[-T_{0},T_{0}]}\max\set{
            \norm{\Phi_{\eps}^{t}}_{\dot{C}^{0,1}},
            \norm{(\Phi_{\eps}^{t})^{-1}}_{\dot{C}^{0,1}},
            \norm{\Phi_{\eps,N}^{t}}_{\dot{C}^{0,1}},
            \norm{(\Phi_{\eps,N}^{t})^{-1}}_{\dot{C}^{0,1}}
        } \leq 3.
    \end{equation}

    Let $\delta>0$ be given, then since $\omega^{0}\in L^{1}(\bbR^{2})$,
    we can find $R_{\delta} \geq R$ such that
    \begin{equation}\label{3.16}
        \int_{\bbR^{2}\setminus B_{R_{\delta}}(x_{0})}\abs{\omega^{0}(y)}dy \leq \delta.
    \end{equation}
    For given $t\in[-T_{0},T_{0}]$ and $x\in B_{R_{\delta}}(x_{0})$ we can write
    \begin{align*}
        \partial_{t}\left(
            D\Phi_{\eps,N}^{t}(x) - D\Phi_{\eps}^{t}(x)
        \right)
        &= D(u_{\eps}(\omega_{\eps,N}^{t}))(\Phi_{\eps,N}^{t}(x))D\Phi_{\eps,N}^{t}(x)
        - D(u_{\eps}(\omega_{\eps}^{t}))(\Phi_{\eps}^{t}(x))D\Phi_{\eps}^{t}(x)
    \end{align*}
    as the sum of the terms
    \begin{align*}
        V_{1} &\coloneqq
        D(u_{\eps}(\omega_{\eps,N}^{t}))(\Phi_{\eps,N}^{t}(x))
        \left(D\Phi_{\eps,N}^{t}(x) - D\Phi_{\eps}^{t}(x)\right), \\
        V_{2} &\coloneqq \left[
            D(u_{\eps}(\Phi_{\eps,N*}^{t}\omega_{N}^{0}))
            - D(u_{\eps}(\Phi_{\eps,N*}^{t}\omega^{0}))
        \right](\Phi_{\eps,N}^{t}(x))D\Phi_{\eps}^{t}(x), \\
        V_{3} &\coloneqq \left[
            D(u_{\eps}(\Phi_{\eps,N*}^{t}\omega^{0}))(\Phi_{\eps,N}^{t}(x))
            - D(u_{\eps}(\Phi_{\eps*}^{t}\omega^{0}))(\Phi_{\eps}^{t}(x))
        \right]D\Phi_{\eps}^{t}(x),
    \end{align*}
    which we estimate separately.

    \textbf{Estimate for $V_{1}$.} Clearly, Lemma~\ref{L2.1}, \eqref{1.7},
    \eqref{3.15} and the inequality $L_{\tht}(\Omega_{N}) \leq L_{\tht}(\Omega)$ show
    \begin{align*}
        \abs{V_{1}} &\leq \norm{u_{\eps}(\omega_{\eps,N}^{t})}_{\dot{C}^{0,1}}
        \abs{D\Phi_{\eps,N}^{t}(x) - D\Phi_{\eps}^{t}(x)}
        \leq C_{\alpha}L_{\tht}(\Phi_{\eps,N*}^{t}\Omega_{N})
        \abs{D\Phi_{\eps,N}^{t}(x) - D\Phi_{\eps}^{t}(x)} \\
        &\leq C_{\alpha}L_{\tht}(\Omega)\norm{\left.\left(
            D\Phi_{\eps,N}^{t} - D\Phi_{\eps}^{t}
        \right)\right|_{B_{R_{\delta}}(x_{0})}}_{L^{\infty}}.
    \end{align*}

    \textbf{Estimate for $V_{2}$.} Since
    \[
        (\omega^{0} - \omega_{N}^{0})(x')
        = \int_{\mathcal{L}\setminus\mathcal{L}_{N}}
        \mathbbm{1}_{\Omega^{\lambda}}(x')\,d\theta(\lambda)
    \]
    is a generalized layer cake representation of $\omega^{0} - \omega_{N}^{0}$,
    Lemma~\ref{L2.1}, the last inequality of \eqref{1.7} without the supremum,
    \eqref{3.15} and $L_{\theta}(\Omega)<\infty$ show
    \begin{align*}
        \abs{D\left(u_{\eps}\left(
            \Phi_{\eps,N*}^{t'}(\omega_{N}^{0} - \omega^{0})
        \right)\right)(\Phi_{\eps,N}^{t'}(x'))}
        &\leq C_{\alpha}\int_{\mathcal{L}\setminus\mathcal{L}_{N}}
        \frac{d|\theta|(\lambda)}
        {d(\Phi_{\eps,N}^{t'}(x'), \Phi_{\eps,N}^{t'}(\partial\Omega^{\lambda}))^{2\alpha}} \\
        &\leq C_{\alpha}\norm{(\Phi_{\eps,N}^{t'})^{-1}}_{\dot{C}^{0,1}}^{2\alpha}
        \int_{\mathcal{L}\setminus\mathcal{L}_{N}}
        \frac{d|\theta|(\lambda)}
        {d(x',\partial\Omega^{\lambda})^{2\alpha}} \to 0
    \end{align*}
    uniformly in $t'\in[-T_{0},T_{0}]$ as $N\to\infty$ for each $x'\in\bbR^{2}$.
    (Note that that we do not know if the convergence is uniform in $x'$.)
    On the other hand, Lemma~\ref{L3.6}, \eqref{1.7} and \eqref{3.15} show
    for any $t'\in[-T_{0},T_{0}]$,
    \begin{align*}
        \norm{D\left(D\left(u_{\eps}\left(
            \Phi_{\eps,N*}^{t'}(\omega_{N}^{0} - \omega^{0})
        \right)\right)\circ \Phi_{\eps,N}^{t'}\right)}_{L^{\infty}}
        &\leq \norm{D^{2}\left(u_{\eps}\left(
            \Phi_{\eps,N*}^{t'}(\omega_{N}^{0} - \omega^{0})
        \right)\right)}_{L^{\infty}}\norm{D\Phi_{\eps,N}^{t'}}_{L^{\infty}} \\
        &\leq \frac{C_{\alpha}}{\eps}\norm{\Phi_{\eps,N}^{t'}}_{\dot{C}^{0,1}}
        \sup_{x'\in\bbR^{2}}\int_{\mathcal{L}\setminus\mathcal{L}_{N}}
        \frac{d|\theta|(\lambda)}{d(x',\Phi_{\eps,N}^{t'}(\partial\Omega^{\lambda}))^{2\alpha}} \\
        &\leq \frac{C_{\alpha}}{\eps}\norm{\Phi_{\eps,N}^{t'}}_{\dot{C}^{0,1}}
        \norm{(\Phi_{\eps,N}^{t'})^{-1}}_{\dot{C}^{0,1}}^{2\alpha}L_{\tht}(\Omega) \\
        &\leq \frac{C_{\alpha}}{\eps}L_{\tht}(\Omega).
    \end{align*}
    Therefore, Arzel\`{a}-Ascoli theorem applied to the collection
    \[
        \set{x'\mapsto \left(
            t'\mapsto D\left(u_{\eps}\left(
                \Phi_{\eps,N*}^{t'}(\omega_{N}^{0} - \omega^{0})
            \right)\right)(\Phi_{\eps,N}^{t'}(x'))
        \right)}_{N\in\bbN}
    \]
    shows that there is $N_{0}\in\bbN$ (depending on $\delta$) such that
    \[
        \sup_{x'\in B_{R_{\delta}}(x_{0})}
        \sup_{t'\in[-T_{0},T_{0}]}
        D\left(u_{\eps}\left(
            \Phi_{\eps,N*}^{t'}(\omega_{N}^{0} - \omega^{0})
        \right)\right)(\Phi_{\eps,N}^{t'}(x'))
        \leq \delta
    \]
    whenever $N\geq N_{0}$.
    Then for such $N$, \eqref{3.15} shows
    \begin{align*}
        \abs{V_{2}} &\leq \abs{D\left(u_{\eps}\left(
            \Phi_{\eps,N*}^{t}(\omega_{N}^{0} - \omega^{0})
        \right)\right)(\Phi_{\eps,N}^{t}(x))}
        \abs{D\Phi_{\eps}^{t}(x)}
        \leq 3\delta.
    \end{align*}

    \textbf{Estimate of $V_{3}$.} Expanding the definition of $u_{\eps}$ gives
    \begin{align*}
        V_{3} = \int_{\bbR^{2}}\left(\left(
            D(\nabla^{\perp}K_{\eps})
            (\Phi_{\eps,N}^{t}(x) - \Phi_{\eps,N}^{t}(y))
            - D(\nabla^{\perp}K_{\eps})
            (\Phi_{\eps}^{t}(x) - \Phi_{\eps}^{t}(y))
        \right)D\Phi_{\eps}^{t}(x)\right)
        \omega^{0}(y)\,dy.
    \end{align*}
    By \eqref{3.16}, the integral over $\bbR^{2}\setminus B_{R_{\delta}}(x_{0})$ is bounded by
    $2\norm{D(\nabla^{\perp}K_{\eps})}_{L^{\infty}}\norm{\Phi_{\eps}^{t}}_{\dot{C}^{0,1}}\delta$.
    For $y\in B_{R_{\delta}}(x_{0})$, the fundamental theorem of calculus shows that
    the integrand is bounded by
    \begin{align*}
        &\norm{\omega^{0}}_{L^{\infty}}\norm{\Phi_{\eps}^{t}}_{\dot{C}^{0,1}}
        \int_{0}^{1}\abs{D^{2}(\nabla^{\perp}K_{\eps})
        (\eta(\Phi_{\eps,N}^{t}(x) - \Phi_{\eps,N}^{t}(y))
        + (1-\eta)(\Phi_{\eps}^{t}(x) - \Phi_{\eps}^{t}(y)))}d\eta
        \\&\qquad\qquad\qquad\quad
        \cdot\abs{\int_{0}^{1}\left(D\Phi_{\eps,N}^{t} - D\Phi_{\eps}^{t}\right)
        (\eta x + (1-\eta)y)\,d\eta}\abs{x - y}.
    \end{align*}
    The second integral is bounded by
    $\norm{\left.\left(D\Phi_{\eps,N}^{t}
    - D\Phi_{\eps}^{t}\right)\right|_{B_{R_{\delta}}(x_{0})}}_{L^{\infty}}$, and
    the first integral is bounded by
    \begin{align*}
        &\min\set{
            \norm{D^{2}(\nabla^{\perp}K_{\eps})}_{L^{\infty}},
            \frac{C_{\alpha}}
            {\min\set{
                \abs{\Phi_{\eps,N}^{t}(x) - \Phi_{\eps,N}^{t}(y)},
                \abs{\Phi_{\eps}^{t}(x) - \Phi_{\eps}^{t}(y)}
            }^{3+2\alpha}}
        }
        \\&\quad\quad
        \leq \min\set{
            \norm{D^{2}(\nabla^{\perp}K_{\eps})}_{L^{\infty}},
            \max\set{
                \norm{(\Phi_{\eps,N}^{t})^{-1}}_{\dot{C}^{0,1}},
                \norm{(\Phi_{\eps}^{t})^{-1}}_{\dot{C}^{0,1}}
            }^{3+2\alpha}
            \frac{C_{\alpha}}{\abs{x - y}^{3+2\alpha}}
        }.
    \end{align*}
    Therefore, \eqref{3.15} shows
    \begin{align*}
        \abs{V_{3}} &\leq
        \norm{\omega^{0}}_{L^{\infty}}\norm{\Phi_{\eps}^{t}}_{\dot{C}^{0,1}}
        \norm{\left.\left(D\Phi_{\eps,N}^{t}
        - D\Phi_{\eps}^{t}\right)\right|_{B_{R_{\delta}}(x_{0})}}_{L^{\infty}}
        \\&\quad\quad\quad\cdot
        \bigg(
            \norm{D^{2}(\nabla^{\perp}K_{\eps})}_{L^{\infty}}
            \int_{\abs{x - y}\leq 1}\abs{x - y}\,dy
            \\&\quad\quad\quad\quad\quad
            + C_{\alpha}\max\set{
                \norm{(\Phi_{\eps,N}^{t})^{-1}}_{\dot{C}^{0,1}},
                \norm{(\Phi_{\eps}^{t})^{-1}}_{\dot{C}^{0,1}}
            }^{3+2\alpha}
            \int_{\abs{x - y} > 1}\frac{dy}{\abs{x - y}^{2+2\alpha}}
        \bigg)
        \\&\quad
        + 2\norm{D(\nabla^{\perp}K_{\eps})}_{L^{\infty}}
        \norm{\Phi_{\eps}^{t}}_{\dot{C}^{0,1}}\delta \\
        &\leq C_{\alpha,\eps}\norm{\omega^{0}}_{L^{\infty}}
        \norm{\left.\left(D\Phi_{\eps,N}^{t}
        - D\Phi_{\eps}^{t}\right)\right|_{B_{R_{\delta}}(x_{0})}}_{L^{\infty}}
        + C_{\alpha,\eps}\delta.
    \end{align*}\smallskip

    Aggregating the estimates for $V_{1}$, $V_{2}$, $V_{3}$ now yields
    \begin{align*}
        \abs{\partial_{t}\left(
            D\Phi_{\eps,N}^{t}(x) - D\Phi_{\eps}^{t}(x)
        \right)} &\leq C_{\alpha,\eps,L_{\tht}(\Omega),\norm{\omega^{0}}_{L^{\infty}}}
        \norm{\left.\left(D\Phi_{\eps,N}^{t}
        - D\Phi_{\eps}^{t}\right)\right|_{B_{R_{\delta}}(x_{0})}}_{L^{\infty}}
        + C_{\alpha,\eps}\delta
    \end{align*}
    for all $t\in[-T_{0},T_{0}]$, $x\in B_{R_{\delta}}(x_{0})$ and $N\geq N_{0}$.
    Then for any $t+h\in[-T_{0},T_{0}]$, we have
    \begin{align*}
        \abs{\left(
            D\Phi_{\eps,N}^{t+h}(x) - D\Phi_{\eps}^{t+h}(x)
        \right)}
        &\leq \abs{\left(
            D\Phi_{\eps,N}^{t}(x) - D\Phi_{\eps}^{t}(x)
        \right)}
        \\&\quad\quad
        + C_{\alpha,\eps,L_{\tht}(\Omega),\norm{\omega^{0}}_{L^{\infty}}}\int_{t}^{t+h}
        \norm{\left.\left(D\Phi_{\eps,N}^{\tau}
        - D\Phi_{\eps}^{\tau}\right)\right|_{B_{R_{\delta}}(x_{0})}}_{L^{\infty}}
        d\tau
        \\&\quad\quad
        + C_{\alpha,\eps}\delta\abs{h}.
    \end{align*}
    Since $D\Phi_{\eps}^{t}$ and $D\Phi_{\eps,N}^{t}$ are continuous in $t$,
    taking supremum over $x\in B_{R_{\delta}}(x_{0})$,
    dividing by $\abs{h}$ and sending $h \to 0$ shows
    \begin{align*}
        &\max\set{
            \partial_{t}^{+}\norm{\left.\left(D\Phi_{\eps,N}^{t}
            - D\Phi_{\eps}^{t}\right)\right|_{B_{R_{\delta}}(x_{0})}}_{L^{\infty}},
            -\partial_{t-}\norm{\left.\left(D\Phi_{\eps,N}^{t}
            - D\Phi_{\eps}^{t}\right)\right|_{B_{R_{\delta}}(x_{0})}}_{L^{\infty}}
        }
        \\&\quad\quad\leq
        C_{\alpha,\eps,L_{\tht}(\Omega),\norm{\omega^{0}}_{L^{\infty}}}
        \norm{\left.\left(D\Phi_{\eps,N}^{t}
        - D\Phi_{\eps}^{t}\right)\right|_{B_{R_{\delta}}(x_{0})}}_{L^{\infty}}
        + C_{\alpha,\eps}\delta,
    \end{align*}
    thus a Gr\"{o}nwall-type argument shows
    \[
        \norm{\left.\left(D\Phi_{\eps,N}^{t}
        - D\Phi_{\eps}^{t}\right)\right|_{B_{R_{\delta}}(x_{0})}}_{L^{\infty}}
        \leq \frac{\exp(C_{\alpha,\eps,L_{\tht}(\Omega),\norm{\omega^{0}}_{L^{\infty}}}\abs{t}) - 1}
        {C_{\alpha,\eps,L_{\tht}(\Omega),\norm{\omega^{0}}_{L^{\infty}}}}
        C_{\alpha,\eps}\delta
    \]
    since $D\Phi_{\eps,N}^{0} = D\Phi_{\eps}^{0} = \mathrm{Id}$.

    Therefore, for any $t\in[-T_{0},T_{0}]$, 
    $x\in B_{R}(x_{0}) \subseteq B_{R_{\delta}}(x_{0})$ and $N\geq N_{0}$,
    \begin{align*}
        \abs{\Phi_{\eps,N}^{t}(x) - \Phi_{\eps}^{t}(x)
        - \Phi_{\eps,N}^{t}(x_{0}) + \Phi_{\eps}^{t}(x_{0})}
        &\leq \frac{\exp(C_{\alpha,\eps,L_{\tht}(\Omega),\norm{\omega^{0}}_{L^{\infty}}}T_{0}) - 1}
        {C_{\alpha,\eps,L_{\tht}(\Omega),\norm{\omega^{0}}_{L^{\infty}}}}
        RC_{\alpha,\eps}\delta,
    \end{align*}
    thus
    \[
        \limsup_{N\to\infty}\sup_{t\in[-T_{0},T_{0}]}\norm{\left.\left(
            \Phi_{\eps,N}^{t} - \Phi_{\eps}^{t}
            - \Phi_{\eps,N}^{t}(x_{0}) + \Phi_{\eps}^{t}(x_{0})
        \right)\right|_{B_{R}(x_{0})}}_{L^{\infty}}
        \leq C_{\alpha,\eps,L_{\tht}(\Omega),\norm{\omega^{0}}_{L^{\infty}}}R\delta,
    \]
    and since $\delta>0$ is arbitrary, the claim follows.
\end{proof}


\appendix

\section{Regularity of Functions with $L_{\tht}(\Omega)<\infty$}

{\color{red}Is it ever possible to have a generalized layer cake representation
that is better than the usual super-level sets?}

\begin{proposition}
    For any $\omega\colon\bbR^{2}\to\bbR$ and a generalized layer cake representation
    $(\Omega,\theta)$ of $\omega$,
    \[
        \norm{\omega}_{\dot{C}^{0,2\alpha}} \leq L_{\tht}(\Omega)
    \]
    holds.
\end{proposition}

\begin{proof}
    Let $x,y\in\bbR^{2}$, $x\neq y$ be given. Then
    \begin{align*}
        \abs{\omega(x) - \omega(y)}
        \leq \int_{\mathcal{L}}
        \abs{\mathbbm{1}_{\Omega^{\lambda}}(x) - \mathbbm{1}_{\Omega^{\lambda}}(y)}
        \,d|\theta|(\lambda)
    \end{align*}
    holds. Since the integrand in the right-hand side is nonzero
    only if $\abs{x - y} \geq d(x,\partial\Omega^{\lambda})$, we obtain
    \[
        \abs{\omega(x) - \omega(y)}
        \leq \int_{\mathcal{L}}\frac{\abs{x - y}^{2\alpha}}
        {d(x,\partial\Omega^{\lambda})^{2\alpha}}\,d|\theta|(\lambda)
        \leq L_{\tht}(\Omega)\abs{x - y}^{2\alpha},
    \]
    thus $\norm{\omega}_{\dot{C}^{0,2\alpha}} \leq L_{\tht}(\Omega)$ follows.
\end{proof}

\begin{proposition}
    Let $\omega\colon\bbR^{2}\to\bbR$ be a uniformly continuous bounded function
    whose modulus of continuity $\rho$ satisfies
    $\int_{0}^{1}\frac{\min\set{\rho(\delta),1}}{\delta^{1+2\alpha}}\,d\delta < \infty$.
    Let $\mathcal{L}\coloneqq \left[
        \inf\omega,
        \sup\omega
    \right]$, $\theta$ the signed measure on $\mathcal{L}$ given as
    \[
        \theta(A) \coloneqq \int_{A}\operatorname{sgn}(\lambda)\,d\lambda,
    \]
    and
    \[
        \Omega\coloneqq \set{(x,\lambda)\in
        \bbR^{2}\times\left(0,\sup\omega\right]
        \colon
        \omega(x) > \lambda}
        \cup
        \set{(x,\lambda)\in
        \bbR^{2}\times\left[\inf\omega,0\right)
        \colon
        \omega(x) < \lambda}.
    \]
    Then $(\Omega,\theta)$ is a generalized layer cake representation of $\omega$
    satisfying $L_{\tht}(\Omega) < \infty$.
\end{proposition}

\begin{proof}
    TBD.
\end{proof}

\begin{proposition}
    There exist $\omega\colon\bbR^{2}\to\bbR$ and
    a generalized layer cake representation $(\Omega,\theta)$ of $\omega$
    such that $L_{\tht}(\Omega)<\infty$ and
    \[
        \min\left\{\delta^{2\alpha},1\right\}
        \leq \rho(\delta) \leq
        2\min\left\{\delta^{2\alpha},1\right\}
    \]
    hold for all $\delta\geq 0$ where $\rho$ is the modulus of continuity of $\omega$.
\end{proposition}

\begin{proof}
    TBD.
\end{proof}


%%%%%%%%%%%%%%%%%%%%%%%%%%%%%%%%%%%%%%%%%%%%%%%%%%%%%%%%%%%%%%%%%%%


\begin{thebibliography}{10}

\bibitem{JeoZlaTouching}
J. Jeon and A. Zlato\v{s},
\textit{Well-Posedness and Finite Time Singularity for Touching g-SQG Patches on the Plane},
preprint.


\end{thebibliography}


\end{document}























% Unused

\begin{proposition}
    For any $z\in L_{tb}^{\infty}(\mathcal{L};\operatorname{SC}(\bbR^{2}))$ and
    Lipschitz continuous injections $\Phi_{1},\Phi_{2}\colon\bbR^{2}\to\bbR^{2}$,
    \[
        \norm{\omega(\Phi_{1}\circ z) - \omega(\Phi_{2}\circ z)}_{L^{1}} \leq
        \frac{\norm{\Phi_{1} - \Phi_{2}}_{L^{\infty}}}{2}\int_{\mathcal{L}}\left(
            \ell(\Phi_{1}\circ z^{\lambda})
            + \ell(\Phi_{2}\circ z^{\lambda})
        \right)d\lambda.
    \]
\end{proposition}

\begin{proof}
    Since
    \begin{align*}
        \norm{\omega(\Phi_{1}\circ z) - \omega(\Phi_{2}\circ z)}_{L^{1}}
        &\leq \int_{\mathcal{L}}\int_{\bbR^{2}}
        \abs{\mathbbm{1}_{\Omega(\Phi_{1}\circ z^{\lambda})}(x)
        - \mathbbm{1}_{\Omega(\Phi_{2}\circ z^{\lambda})}(x)}\,dx\,d\lambda \\
        &= \int_{\mathcal{L}}
        \abs{\Omega(\Phi_{1}\circ z^{\lambda})\,\triangle\,\Omega(\Phi_{2}\circ z^{\lambda})}
        d\lambda,
    \end{align*}
    the desired conclusion follows at once by the lemma below.
\end{proof}

\begin{lemma}
    Let $\gamma_{i}\colon\ell_{i}\bbT\to\bbR^{2}$ be rectifiable
    simple closed curves parametrized by arclength for $i=1,2$.
    Then for any Lipschitz continuous orientation-preserving homeomorphism
    $\phi\colon\ell_{1}\bbT\to\ell_{2}\bbT$,
    \begin{align*}
        \abs{\Omega(\gamma_{1})\triangle\Omega(\gamma_{2})}
        \leq \frac{1}{2}\left(
            \norm{\gamma_{1} - \gamma_{2}\circ\phi}_{L^{1}}
            + \norm{\gamma_{1}\circ\phi^{-1} - \gamma_{2}}_{L^{1}}
        \right)
    \end{align*}
    where $\abs{\,\cdot\,}$ denotes the $2$-dimensional Lebesgue measure.
\end{lemma}

\begin{proof}
    Consider the homotopy $H\colon [0,1]\times\ell_{1}\bbT\to\bbR^{2}$
    given as $H\colon (t,s)\mapsto (1-t)\gamma_{1}(s) + t\gamma_{2}(\phi(s))$.
    Then for any $x\in\Omega(\gamma_{1})\triangle\Omega(\gamma_{2})$,
    since the winding number of $\gamma_{1}$ and $\gamma_{2}$ around $x$
    are different, there must exist an intermediate curve
    $H(t,\,\cdot\,)\colon\ell_{1}\bbT\to\bbR^{2}$ that intersects $x$.
    Then since $H$ is Lipschitz continuous, the area formula shows
    \begin{align*}
        \abs{\Omega(\gamma_{1})\triangle\Omega(\gamma_{2})}
        &\leq \abs{H\left([0,1]\times\ell_{1}\bbT\right)} \\
        &\leq \int_{[0,1]\times\ell_{1}\bbT}
        \abs{\partial_{t}H^{\perp}\cdot\partial_{s}H} \\
        &\leq \int_{0}^{1}\int_{\ell_{1}\bbT}
        \abs{\gamma_{1}(s) - \gamma_{2}(\phi(s))}
        \abs{(1-t)\partial_{s}\gamma_{1}(s)
        + t\partial_{s}(\gamma_{2}\circ\phi)(s)}\,ds\,dt \\
        &\leq \int_{0}^{1}(1-t)
        \int_{\ell_{1}\bbT}\abs{\gamma_{1}(s) - \gamma_{2}(\phi(s))}\,ds
        + t\int_{\ell_{2}\bbT}\abs{\gamma_{1}(\phi^{-1}(s)) - \gamma_{2}(s)}\,ds\,dt \\
        &= \frac{1}{2}\left(
            \norm{\gamma_{1} - \gamma_{2}\circ\phi}_{L^{1}}
            + \norm{\gamma_{1}\circ\phi^{-1} - \gamma_{2}}_{L^{1}}
        \right).
    \end{align*}
\end{proof}


\begin{lemma}
    For any $\omega^{0}\in L^{1}(\bbR^{2})\cap L^{\infty}(\bbR^{2})$
    and a generalized layer cake representation
    $(\Omega,\theta)$ of $\omega^{0}$,
    $L_{\tht}(\Omega) \leq 2^{2\alpha}\hat{L}(\Omega)$ holds.
\end{lemma}

\begin{proof}
    Fix $x\in\bbR^{2}$.
    Let $r\coloneqq\ess\inf_{\lambda\in\mathcal{L}}d(x,\partial\Omega^{\lambda})$,
    then for given $\eta>0$, there exists a non-$\abs{\theta}$-null set
    $A\subseteq\mathcal{L}$ such that $d(x,\partial\Omega^{\lambda})\leq r + \eta$
    holds for all $\lambda\in A$. Take any such $\lambda\in A$.
    Then for $\abs{\theta}$-almost every $\lambda'\in\mathcal{L}$,
    \[
        \Delta(z^{0,\lambda},z^{0,\lambda'})
        \leq d(x,\partial\Omega^{\lambda}) + d(x,\partial\Omega^{\lambda'})
        \leq 2d(x,\partial\Omega^{\lambda'}) + \eta,
    \]
    so
    \[
        \int_{\mathcal{L}}\frac{d|\theta|(\lambda')}
        {\Delta(z^{0,\lambda},z^{0,\lambda'})^{2\alpha}}
        \geq \int_{\mathcal{L}}\frac{d|\theta|(\lambda')}
        {(2d(x,\partial\Omega^{\lambda'}) + \eta)^{2\alpha}}
    \]
    holds for all $\lambda\in A$. Hence, the right-hand side is bounded above by
    $\hat{L}(\Omega)$, and since $\eta>0$ and $x\in\bbR^{2}$ are arbitrary,
    we obtain $L_{\tht}(\Omega)\leq 2^{2\alpha}\hat{L}(\Omega)$.
\end{proof}