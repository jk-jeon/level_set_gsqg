\documentclass[reqno,centertags,12pt]{amsart}

\usepackage{amsmath}
\usepackage{amscd}
\usepackage{stackrel}
\usepackage{amssymb}
\usepackage{amsthm}
\usepackage{bbm}
\usepackage{latexsym}
\usepackage{mathrsfs}
\usepackage{verbatim}
\usepackage{tikz-cd}
\usepackage{mathtools}
\usepackage[hidelinks]{hyperref}


%\usepackage{refcheck}

\usepackage[shortlabels]{enumitem}

\usepackage{xcolor}

% For figure
\usepackage{tikz}
\pgfdeclarelayer{nodelayer}
\pgfdeclarelayer{edgelayer}
\pgfsetlayers{edgelayer,nodelayer,main}
\tikzstyle{arrow}=[draw=black,arrows=-latex]

\textheight 21cm \topmargin 0cm \leftmargin 0cm \marginparwidth 0mm
\textwidth 16.6cm \hsize \textwidth \advance \hsize by
-\marginparwidth \oddsidemargin -4mm \evensidemargin \oddsidemargin




%%%%%%%%%%%%% fonts/sets %%%%%%%%%%%%%%%%%%%%%%%

\newtheorem{theorem}{Theorem}[section]
\newtheorem*{t1}{Theorem 1}
\newtheorem{proposition}[theorem]{Proposition}
\newtheorem{lemma}[theorem]{Lemma}
\newtheorem{corollary}[theorem]{Corollary}
\theoremstyle{definition}
\newtheorem{definition}[theorem]{Definition}
\newtheorem{example}[theorem]{Example}
\newtheorem{conjecture}[theorem]{Conjecture}
\newtheorem{xca}[theorem]{Exercise}
%\theoremstyle{remark}
%\newtheorem{remark}[theorem]{Remark}
\newtheorem*{remark}{Remark}

%%%%%%%%%%%%%%  Rowan's unspaced list %%%%%%%%%%%%%%%%

\newcounter{smalllist}
\newenvironment{SL}{\begin{list}{{\rm\roman{smalllist})}}{%
\diffetlength{\topsep}{0mm}\diffetlength{\parsep}{0mm}\diffetlength{\itemsep}{0mm}%
\diffetlength{\labelwidth}{2em}\diffetlength{\leftmargin}{2em}\usecounter{smalllist}%
}}{\end{list}}

%%%%%%%%%%%%%%% operators %%%%%%%%%%%%%%%%%%%%%%

\DeclareMathOperator{\real}{Re} 
\DeclareMathOperator{\ima}{Im}
\DeclareMathOperator{\diam}{diam}
\DeclareMathOperator*{\slim}{s-lim}
\DeclareMathOperator*{\wlim}{w-lim}
\DeclareMathOperator*{\simlim}{\sim}
\DeclareMathOperator*{\eqlim}{=}
\DeclareMathOperator*{\arrow}{\rightarrow}
\DeclareMathOperator*{\dist}{dist} 
\DeclareMathOperator*{\divg}{div}
\DeclareMathOperator*{\Lip}{Lip} 
\DeclareMathOperator*{\sgn}{sgn} 
\DeclareMathOperator*{\ches}{chess}
\DeclareMathOperator*{\ch}{ch}
\allowdisplaybreaks
\numberwithin{equation}{section}

% Absolute value notation
\newcommand{\abs}[1]{\left\lvert#1\right\rvert}

% Norm notation
\newcommand{\norm}[1]{\left\|#1\right\|}

\newcommand{\set}[1]{\left\{ #1 \right\}}
\newcommand{\setbc}[2]{\left\{ #1\colon#2 \right\}}
\newcommand{\seq}[1]{\left( #1 \right)}

%\renewcommand{\qedsymbol}{}

%%%%%%%%%%%%%%%%%%  abbreviations %%%%%%%%%%%%%%%%

\newcommand{\dott}{\,\cdot\,}
\newcommand{\no}{\nonumber}
\newcommand{\lb}{\label}
\newcommand{\f}{\frac}
\newcommand{\ul}{\underline}
\newcommand{\ol}{\overline}
\newcommand{\ti}{\tilde  }
\newcommand{\wti}{\widetilde  }
\newcommand{\bi}{\bibitem}
\newcommand{\hatt}{\widehat}
%\newcommand{%\marginlabel}[1]{\mbox{}%\marginpar{\raggedleft\hspace{0pt}#1}}

\newcommand{\Oh}{O}
\newcommand{\oh}{o}
\newcommand{\tr}{\text{\rm{Tr}}}
\newcommand{\loc}{\text{\rm{loc}}}
\newcommand{\spec}{\text{\rm{spec}}}
\newcommand{\rank}{\text{\rm{rank}}}
\newcommand{\dom}{\text{\rm{dom}}}
\newcommand{\ess}{\text{\rm{ess}}}
\newcommand{\ac}{\text{\rm{ac}}}
\newcommand{\singc}{\text{\rm{sc}}}
\newcommand{\sing}{\text{\rm{sing}}}
\newcommand{\pp}{\text{\rm{pp}}}
\newcommand{\supp}{\text{\rm{supp}}}
\newcommand{\AC}{\text{\rm{AC}}}

\newcommand{\beq}{\begin{equation}}
\newcommand{\eeq}{\end{equation}}
\newcommand{\eq}{equation}
\newcommand{\bal}{\begin{align}}
\newcommand{\eal}{\end{align}}
\newcommand{\bals}{\begin{align*}}
\newcommand{\eals}{\end{align*}}

%%%%%%%%%%%%%% fonts/sets %%%%%%%%%%%%%%%%%%%%%%%

\newcommand{\calA}{{\mathcal A}}
\newcommand{\bbN}{{\mathbb{N}}}
\newcommand{\bbR}{{\mathbb{R}}}
\newcommand{\bbD}{{\mathbb{D}}}
\newcommand{\bbP}{{\mathbb{P}}}
\newcommand{\bbE}{{\mathbb{E}}}
\newcommand{\bbZ}{{\mathbb{Z}}}
\newcommand{\bbC}{{\mathbb{C}}}
\newcommand{\bbQ}{{\mathbb{Q}}}
\newcommand{\bbT}{{\mathbb{T}}}
\newcommand{\bbS}{{\mathbb{S}}}

\newcommand{\calE}{{\mathcal E}}
\newcommand{\calS}{{\mathcal S}}
\newcommand{\calT}{{\mathcal T}}
\newcommand{\calM}{{\mathcal M}}
\newcommand{\calN}{{\mathcal N}}
\newcommand{\calB}{{\mathcal B}}
\newcommand{\calI}{{\mathcal I}}
\newcommand{\calL}{{\mathcal L}}
\newcommand{\calC}{{\mathcal C}}
\newcommand{\calF}{{\mathcal F}}
\newcommand{\calH}{{\mathcal H}}
\newcommand{\calK}{{\mathcal K}}
\newcommand{\calG}{{\mathcal G}}
\newcommand{\calZ}{{\mathcal Z}}
\newcommand{\calU}{{\mathcal U}}

\newcommand{\eps}{\varepsilon}
\newcommand{\del}{\delta}
\newcommand{\tht}{\theta}
\newcommand{\ka}{\kappa}
\newcommand{\al}{\alpha}
\newcommand{\be}{\beta}
\newcommand{\ga}{\gamma}
\newcommand{\laa}{\lambda}
\newcommand{\partt}{\tfrac{\partial}{\partial t}}
\newcommand{\lan}{\langle}
\newcommand{\ran}{\rangle}
\newcommand{\til}{\tilde}
\newcommand{\tilth}{\til\tht}
\newcommand{\tilT}{\til T}
\newcommand{\tildel}{\til\del}
\newcommand{\ffi}{\varphi}

%\newcommand{\ce}{c^*_e}
\newcommand{\diff}{D}
\newcommand{\diffe}{\diff_e}
\newcommand{\Om}{\Omega}
\newcommand{\tilOm}{{\tilde\Omega}}
\newcommand{\aaa }{a}



\newcommand{\izero}{\iota}


%%%%%%%%%%%%%%%%%%%%%%%%%%%%%%%%%%%%%%%%%%%%%
%%%%%%%%%%%%%%%%%%%% end of  definitions %%%%%%%%%%%%%%%
%%%%%%%%%%%%%%%%%%%%%%%%%%%%%%%%%%%%%%%%%%%%%


\begin{document}
\title[TBD]
{TBD}

\author{Junekey Jeon and Andrej Zlato\v{s}}

\address{\noindent Department of Mathematics \\ University of
California San Diego \\ La Jolla, CA 92093 \newline Email: \tt
zlatos@ucsd.edu,
j6jeon@ucsd.edu}


\begin{abstract}
    This is the abstract.
\end{abstract}

\maketitle


%%%%%%%%%%%%%%%%%%%%%%%%%%%%%%%%%%%%%%%%%%%%%%
\section{Introduction} \label{S1}
%%%%%%%%%%%%%%%%%%%%%%%%%%%%%%%%%%%%%%%%%%%%%%

Notations.
\begin{itemize}
    \item A generic constant is denoted as $C$, which may change line by line.
    When we want to stress its dependence on parameters $p_{1},\ \cdots\ ,p_{n}$,
    we denote as $C = C(p_{1},\ \cdots\ ,p_{n})$.

    \item for $x=(x_{1},x_{2})\in\bbR^{2}$, $x^{\perp}\coloneqq (-x_{2},x_{1})$.

    \item $\bbT\coloneqq\bbR/\bbZ$, $\ell\bbT\coloneqq\bbR/\ell\bbZ$ if
    $l>0$ or $\bbR/\bbR=\set{0}$ if $l=0$.

    \item $\norm{f}_{\dot{H}^{k}} \coloneqq \norm{\partial_{\xi}^{k}f}_{L^{2}}$
    if $f$ is weakly differentiable $k$ times, or $\infty$ otherwise.

    \item $\norm{f}_{H^{k}} \coloneqq \sum_{i=0}^{k}\norm{\partial_{\xi}^{i}f}_{L^{2}}$
    if $f$ is weakly differentiable $k$ times, or $\infty$ otherwise.

    \item $\norm{f}_{\dot{C}^{k,\beta}} \coloneqq \sup_{\xi\neq\eta}
    \frac{\abs{\partial_{\xi}^{k}f(\xi) - \partial_{\xi}^{k}f(\eta)}}
    {\abs{\xi - \eta}^{\beta}}$ if $f$ is $k$ times continuously differentiable,
    or $\infty$ otherwise.

    \item $\norm{f}_{C^{k,\beta}} \coloneqq
    \sum_{i=0}^{k}\norm{\partial_{\xi}^{i}f}_{C^{0}}
    + \norm{\partial_{\xi}^{k}f}_{\dot{C}^{\beta}}$
    if $f$ is $k$ times continuously differentiable, or $\infty$ otherwise.

    \item (Currying/uncurrying) For a function $f\colon X_{1}\times X_{2}\to Y$,
    $f(x_{1})$ means the function $x_{2}\mapsto f(x_{1},x_{2})$, and conversely,
    for a function $f\colon X_{1}\to Y^{X_{2}}$,
    $f(x_{1},x_{2})$ means the iterated evaluation $f(x_{1})(x_{2})$.
    We apply the same correspondence for functions of more than two arguments as well.

    \item (Fr\'{e}chet derivatives) For a function $f\colon\bbR^{n}\to\bbR^{m}$,
    $Df\colon \bbR^{n} \to \operatorname{Hom}(\bbR^{n};\bbR^{m})$ is the usual
    Fr\'{e}chet derivative of $f$. Through uncurrying, we identify the iterated linear map
    $D^{2}f(x)\in\operatorname{Hom}(\bbR^{n};\operatorname{Hom}(\bbR^{n};\bbR^{m}))$
    with a bilinear map $\bbR^{n}\times\bbR^{n}\to\bbR^{m}$.

    \item (Dini derivatives) For a real-valued function $f$ on an interval, we denote the
    \emph{upper-right Dini derivative} of $f$ as
    \[
        \partial_{t}^{+}f(t)\coloneqq \limsup_{h\to 0^{+}}\frac{f(t+h) - f(t)}{h}
    \]
    and the \emph{lower-right Dini derivative} of $f$ as
    \[
        \partial_{t+}f(t)\coloneqq \liminf_{h\to 0^{+}}\frac{f(t+h) - f(t)}{h}.
    \]
\end{itemize}


%%%%%%%%%%%%%%%%%%%%%%%%%%%%%%%%%%%%%%%%%%%%%%
\section{The well-posedness result} \label{S2}
%%%%%%%%%%%%%%%%%%%%%%%%%%%%%%%%%%%%%%%%%%%%%%

Let us first define on which exact space we are formulating our
differential equation. Since the object we want to deal with is an
infinite collection of closed curves in $\bbR^{2}$, it is natural to consider
a space of curve-valued functions on a measure space. To state this precisely,
we first define the space $\operatorname{Curve}(\bbR^{2})$ of closed curves in $\bbR^{2}$.

Consider the space $C(\bbT;\bbR^{2})$ of all
continuous closed paths in $\bbR^{2}$ endowed with the uniform norm
$\norm{\,\cdot\,}_{C^{0}}$. On this Banach space, we define a pseudometric
\[
    d_{\mathrm{F}}\colon (\gamma_{1},\gamma_{2})\mapsto
    \inf_{\phi\colon \bbT\to \bbT}
    \norm{\gamma_{1} - \gamma_{2}\circ\phi}_{C^{0}},
\]
where $\phi$ ranges over every \emph{orientation-preserving homeomorphism}
from $\bbT$ onto $\bbT$; i.e., any homeomorphism obtained by projecting a
strictly increasing homeomorphism $\tilde{\phi}\colon \bbR\to \bbR$ such that
$\tilde{\phi}(x+n) = \tilde{\phi}(x)+n$ holds for all $x\in\bbR$ and $n\in\bbZ$.
By identifying paths with the zero distance to each other, we obtain the notion of
\emph{Fr\'{e}chet metric} (references needed).

\begin{definition}\label{D2.1}
    By $\operatorname{Curve}(\bbR^{2})$, we mean the quotient space of
    $C(\bbT;\bbR^{2})$ with respect to the relation $\sim$ defined as
    $\gamma_{1}\sim\gamma_{2}$ if and only if $d_{\mathrm{F}}(\gamma_{1},\gamma_{2}) = 0$.
    An element $\gamma$ in this quotient space is called a \emph{closed curve} in $\bbR^{2}$
    and a representative in $C(\bbT;\bbR^{2})$ of $\gamma$ is called
    a \emph{parameterization} of $\gamma$. The image of any parameterization of $\gamma$
    is called the \emph{image} of $\gamma$ and is denoted as $\operatorname{im}(\gamma)$.
    The metric $d_{\mathrm{F}}$ induced on $\operatorname{Curve}(\bbR^{2})$ is called
    the \emph{Fr\'{e}chet metric}.
\end{definition}

\textbf{Remark.} \begin{enumerate}
    \item The Fr\'{e}chet metric $d_{\mathrm{F}}$ is in some sense a more natural
    metric to give on the space of curves than the \emph{Hausdorff metric} $d_{\mathrm{H}}$.
    For example, $d_{\mathrm{F}}$ is a complete metric on
    $\operatorname{Curve}(\bbR^{2})$ (reference needed; see also
    Appendix~\ref{SA}), while in $d_{\mathrm{H}}$ curves can converge
    to a non-path-connected sets. In general, given two paths
    $\gamma_{1},\gamma_{2}\in C(\bbT;\bbR^{2})$, $d_{\mathrm{F}}(\gamma_{1},\gamma_{2}) = 0$
    does \emph{not} imply the existence of an orientation-preserving homeomorphism
    $\phi\colon\bbT\to \bbT$ with $\gamma_{1} = \gamma_{2}\circ\phi$; for example,
    such $\phi$ cannot exist if $\gamma_{1}$ and $\gamma_{2}$ refer to the same simple
    closed curve and $\gamma_{1}$ is an embedding while $\gamma_{2}$ is constant on a
    subinterval of $\bbT$. However, it can be shown (references needed) that
    $d_{\mathrm{F}}(\gamma_{1},\gamma_{2}) = 0$ is equivalent to the existence of
    a chain of reparameterizations between the two in some appropriate sense.
    Furthermore, it can be shown (references needed) that $d_{\mathrm{F}}$ coincides
    with the quotient pseudometric on $\operatorname{Curve}(\bbR^{2})$ induced by
    the uniform metric on $C(\bbT;\bbR^{2})$. Together with the fact that
    $(\operatorname{Curve}(\bbR^{2}),d_{\mathrm{F}})$ is a complete metric space,
    these suggest that $(\operatorname{Curve}(\bbR^{2}),d_{\mathrm{F}})$ is indeed
    a natural setting for our purpose. See Appendix~\ref{SA} for details.
    Also, in the development of the theory done in this paper,
    it seems apparent that $d_{\mathrm{F}}$ is more convenient to work with than
    $d_{\mathrm{H}}$ in general.
    
    \item Another notable difference between $d_{\mathrm{F}}$ and $d_{\mathrm{H}}$ is that
    $d_{\mathrm{F}}$ respects the orientation of the curve while $d_{\mathrm{H}}$ only
    looks at the image of the curve. Note that we can orient all simple closed curves
    with the counterclockwise orientation, but for general closed curves there is no
    canonical choice of the orientation. Since we a priori does not preclude non-simple
    closed curves, it sounds more appropriate to work with a metric that respects
    the orientation.
    
    \item Clearly, $d_{\mathrm{F}}$ is at least $d_{\mathrm{H}}$,
    but $d_{\mathrm{F}}$ can be arbitrarily large while $d_{\mathrm{H}}$ remains bounded,
    even when orientation is ignored. The reverse inequality
    holds up to a constant for simple closed curves with the same orientation and uniformly
    bounded arc-chord constants ({\color{red}I thought it is true,
    but it actually seems more involved...}), thus for such a case
    there is no difference in working with $d_{\mathrm{F}}$ or $d_{\mathrm{H}}$.
    Even without such an assumption, since many of the sets of curves we consider
    in this paper is compact with respect to $d_{\mathrm{F}}$, on those sets
    $d_{\mathrm{F}}$ and $d_{\mathrm{H}}$ are at least bi-uniformly equivalent
    once we ignore the orientation.
\end{enumerate}

Let $\mathcal{L}$ be a finite measure space throughout this paper.
The $\sigma$-algebra and the measure on $\mathcal{L}$ will not be explicitly spelled out.
Instead, the measure of a measurable subset $A\subseteq\mathcal{L}$ is denoted as
$\abs{A}$. We will formulate our differential equation on a space of
$\operatorname{Curve}(\bbR^{2})$-valued functions on $\mathcal{L}$.

\begin{definition}
    Let $(X,d)$ be a complete metric space. A \emph{totally bounded $X$-valued
    measurable function} on $\mathcal{L}$ is a uniform limit of
    $X$-valued simple functions on $\mathcal{L}$ (i.e., a finitely-valued function
    whose inverse image of each point in the range is measurable).
    The space of all totally bounded $X$-valued measurable functions on $\mathcal{L}$
    quotiented by the almost everywhere equivalence is denoted as
    $L_{tb}^{\infty}(\mathcal{L};X)$. We define a metric $d_{\infty}$ on
    $L_{tb}^{\infty}(\mathcal{L};X)$ as
    \[
        d_{\infty}\colon (z_{1},z_{2})
        \mapsto \norm{\lambda\mapsto d(z_{1}(\lambda), z_{2}(\lambda))}_{L^{\infty}}.
    \]
    Then $d_{\infty}$ is a well-defined complete metric on $L_{tb}^{\infty}(\mathcal{L};X)$.
    See Appendix~\ref{SC} for details.
\end{definition}

\textbf{Remark.}
\begin{enumerate}
    \item The reason for only considering functions with totally bounded ranges is
    purely technical, but this is not a big loss of generality.
    Indeed, as noted in Appendix~\ref{SB},
    the set of all curves contained in a closed ball with uniformly bounded lengths is compact
    in $\operatorname{Curve}(\bbR^{2})$, so for any bounded function with a bounded support,
    the set of all level sets for nonzero levels form a totally bounded subset of
    $\operatorname{Curve}(\bbR^{2})$ whenever all level sets are closed curves of
    bounded lengths.

    \item It is often inconvenient that functions in $L_{tb}^{\infty}(\mathcal{L};X)$
    are defined only up to a.e. equivalence. To recover well-defined pointwise behavior,
    we often prefer to work with the space $C(\bar{\mathcal{L}};X)$ of $X$-valued
    continuous functions on the \emph{Gelfand spectrum} $\bar{\mathcal{L}}$ of the
    $C^{*}$-algebra $L^{\infty}(\mathcal{L};\bbC)$. It can be shown
    (Proposition~\ref{PC.2}) that
    these two spaces are indeed identical, because both can be seen as
    completions of the space of simple functions.
    See Appendix~\ref{SC} for details.
\end{enumerate}

To prove our well-posedness result, we impose a regularity condition by assuming that
a certain geometric functional defined below is finite.

\begin{definition}\label{D2.3}
    For $\gamma,\eta\in\operatorname{Curve}(\bbR^{2})$, we define the followings.
    \begin{enumerate}
        \item $\norm{\gamma}_{C^{0}}$ is the uniform norm of any
        parameterization of $\gamma$.

        \item The \emph{length} $\ell(\gamma)$ is the total variation of
        any parameterization of $\gamma$.

        \item For any $k\in\bbN$ and $\beta\in[0,1]$,
        \[
            \norm{\gamma}_{C^{k,\beta}} \coloneqq \begin{cases}
                \infty & \textrm{if $\ell(\gamma) = \infty$,} \\
                \norm{\tilde{\gamma}}_{C^{k,\beta}} & \textrm{otherwise,}
            \end{cases}
        \]
        where $\tilde{\gamma}$ is any arclength parameterization of $\gamma$.
        We define $\norm{\gamma}_{\dot{C}^{k,\beta}}$, $\norm{\gamma}_{H^{k}}$
        and $\norm{\gamma}_{\dot{H}^{k}}$ in the same way.

        \item The \emph{total squared curvature} $\mathcal{K}_{2}(\gamma)$ is defined as
        \[
            \mathcal{K}_{2}(\gamma) \coloneqq \begin{cases}
                \infty & \textrm{if $\ell(\gamma) = \infty$,} \\
                \norm{\gamma}_{\dot{H}^{2}}^{2}
                & \textrm{otherwise}.
            \end{cases}
        \]

        \item Finally, $\Delta(\gamma,\eta) \coloneqq
        \min_{\xi,\zeta\in\bbT}\abs{\gamma(\xi) - \eta(\zeta)}$ is the minimum
        distance between the images of $\gamma$ and $\eta$.
    \end{enumerate}
\end{definition}

\textbf{Remark.} \begin{enumerate}
    \item Any of these are independent of the choice of parameterization.

    \item When $\mathcal{K}_{2}(\gamma)$ is finite, integration by parts and
    Cauchy-Schwarz inequality shows that $\ell(\gamma)$ is bounded by
    $\norm{\gamma}_{C^{0}}^{2}\mathcal{K}_{2}(\gamma)$. Hence, defining
    $\mathcal{K}_{2}(\gamma)$ to be infinite when $\gamma$ is not rectifiable
    is reasonable, even though the curvature can still be defined
    everywhere for unrectifiable curves.

    \item If $\gamma$ is a nontrivial rectifiable curve and
    $\tilde{\gamma}\colon\bbT\to\bbR^{2}$ is a constant-speed parameterization of $\gamma$,
    then $\mathcal{K}_{2}(\gamma)=
    \frac{1}{\ell(\gamma)^{3}}\norm{\tilde{\gamma}}_{\dot{H}^{2}}^{2}$.
\end{enumerate}

\begin{definition}\label{D2.4}
    For $z\in L_{tb}^{\infty}(\mathcal{L};\operatorname{Curve}(\bbR^{2}))$, we define
    \begin{align*}
        L(z) &\coloneqq \norm{
            \lambda\mapsto \norm{z(\lambda)}_{C^{0}}
        }_{L^{\infty}}
        + \norm{
            \lambda\mapsto \mathcal{K}_{2}(z(\lambda))
        }_{L^{\infty}}
        + \norm{\lambda\mapsto \int_{\mathcal{L}}\frac{d\lambda'}
        {\Delta(z(\lambda),z(\lambda'))^{2\alpha}}}_{L^{\infty}} + 1.
    \end{align*}
\end{definition}

\textbf{Remark.} \begin{enumerate}
    \item The ``$+1$'' at the end has no deep meaning and it
    is just to ensure that $L(z)^{p} \leq L(z)^{q}$ always holds when $p\leq q$.

    \item Every function inside $\norm{\cdot}_{L^{\infty}}$ is measurable,
    so the expression above makes sense.
    Similarly, $\lambda\mapsto \norm{z(\lambda)}_{\dot{C}^{1,\beta}}$ is
    measurable for any $\beta\in[0,1]$.
    See Appendix~\ref{SC} for details.

    \item By the second remark after Definition~\ref{D2.3},
    $L(z)<\infty$ implies $\norm{\lambda\mapsto\ell(z(\lambda))}_{L^{\infty}}<\infty$.
    (Again, the function inside $\norm{\,\cdot\,}_{L^{\infty}}$ is measurable;
    see Appendix~\ref{SC} for details.)
\end{enumerate}

In Appendix~\ref{SC}, it is shown
(Proposition~\ref{PC.9}) that whenever
$\norm{\lambda\mapsto \mathcal{K}_{2}(z(\lambda))}_{L^{\infty}} < \infty$,
we can find a jointly measurable function $\mathcal{L}\times\bbR\to\bbR^{2}$
(which we still denote as $z$) such that the projection onto $\ell(z(\lambda))\bbT$ of
$z(\lambda,\,\cdot\,)$ is an arclength parameterization of the curve $z(\lambda)$,
for almost every $\lambda\in\mathcal{L}$. From now on, we will always implicitly
assume such a joint parameterization is chosen.

Next, we define the velocity field generated by a collection of curves.

\begin{definition}\label{D2.5}
    For $z\in L_{tb}^{\infty}(\mathcal{L};\operatorname{Curve}(\bbR^{2}))$ with
    $L(z) < \infty$, we define a vector field $u(z)\colon \bbR^{2} \to \bbR^{2}$ as
    \begin{align*}
        u(z)\colon x\mapsto
        -\int_{\mathcal{L}}\int_{\ell(z(\lambda))\bbT}
        \frac{\partial_{s}z(\lambda,s)}
        {\abs{x - z(\lambda,s)}^{2\alpha}} \,ds\,d\lambda.
    \end{align*}
\end{definition}

At this point, it is not immediately clear if the above $u(z)$ is well-defined.
In Section~\ref{S4}, we show that
$u(z)$ is well-defined, bounded, and is in $C^{1-2\alpha}$.

Finally, we define the notion of solutions.
For $z\in L_{tb}^{\infty}(\mathcal{L};\operatorname{Curve}(\bbR^{2}))$,
a continuous vector field $v\colon\bbR^{2}\to\bbR^{2}$ and $h\in\bbR$, we define
\[
    X_{v}^{h}[z]\colon \lambda\mapsto \left(
        \xi\mapsto z(\lambda)(\xi) + hv(z(\lambda)(\xi))
    \right),
\]
where we have fixed any parameterization $z(\lambda)\colon\bbT\to\bbR^{2}$
for each $\lambda\in\mathcal{L}$ in the above expression.
Clearly, $X_{u}^{h}[z]$ is a $\operatorname{Curve}(\bbR^{2})$-valued function
which does not depend on the choice of parameterizations of $z(\lambda)$'s.
Then it can be shown (Corollary~\ref{CC.4})
that $X_{u}^{h}[z]$ is a well-defined element of
$L_{tb}^{\infty}(\mathcal{L};\operatorname{Curve}(\bbR^{2}))$ that does not depend
on the choice of a representative of $z$.

\begin{definition}\label{D2.6}
    Let $z_{0}\in L_{tb}^{\infty}(\mathcal{L};\operatorname{Curve}(\bbR^{2}))$ with
    $L(z_{0}) < \infty$ be given. Then for $T\in(0,\infty]$, we call a function
    $z\colon [0,T)\to L_{tb}^{\infty}(\mathcal{L};\operatorname{Curve}(\bbR^{2}))$
    an \emph{$H^{2}$-solution to the g-SQG equation with the initial data $z_{0}$}, if
    $z(0) = z_{0}$,
    \begin{equation}\label{eq:solution}
        \lim_{h\to 0}\frac{d_{\mathrm{F},\infty}\left(
            z(t+h), X_{u(z(t))}^{h}[z(t)]
        \right)}{h} = 0
    \end{equation}
    holds for all $t\in[0,T)$, and $\sup_{t\in[0,T']}L(z(t)) < \infty$ holds
    for all $T'\in[0,T)$.
\end{definition}

\textbf{Remark.} The definition given above is a natural extension of the definition of
patch solutions given in \cite{KisYaoZla}, except that we replace the metric $d_{\mathrm{H}}$
by $d_{\mathrm{F}}$. Similar approaches to differential equations formulated
on metric spaces without linear structures can be found in (references needed).\\

Then our well-posedness result is stated below.

\begin{theorem}\label{T2.7}
    For any $z_{0}\in L_{tb}^{\infty}(\mathcal{L};\operatorname{Curve}(\bbR^{2}))$ with
    $L(z_{0}) < \infty$, there exists a maximal time of existence $T\in(0,\infty]$
    and a unique $H^{2}$-solution $z\colon[0,T) \to
    L_{tb}^{\infty}(\mathcal{L};\operatorname{Curve}(\bbR^{2}))$ to the g-SQG equation
    with the initial data $z_{0}$.
\end{theorem}


%%%%%%%%%%%%%%%%%%%%%%%%%%%%%%%%%%%%%%%%%%%%%%
\section{Geometric lemmas}
%%%%%%%%%%%%%%%%%%%%%%%%%%%%%%%%%%%%%%%%%%%%%%

In this section, we collect some geometric lemmas about curves with $C^{1,\beta}$-regularity.
Let $\beta\in(0,1]$ throughout this section.

\begin{lemma}\label{L3.1}
    Let $\gamma\colon\ell\bbT\to\bbR^{2}$ be a nontrivial $C^{1,\beta}$ closed curve
    parameterized by the arclength and $\mathbf{T}\coloneqq\partial_{s}\gamma$.
    Then for any $s,s'\in\ell\bbT$,
    % \begin{enumerate}
    %     \item $\mathbf{T}(s)\cdot\mathbf{T}(s') \geq
    %     1 - \frac{1}{2}\norm{\mathbf{T}}_{\dot{C}^{\beta}}^{2}\abs{s - s'}^{2\beta}$, and
    %     \item $\abs{\gamma(s) - \gamma(s')} \geq \abs{s - s'} \left(
    %         1 - \frac{1}{2}\norm{\mathbf{T}}_{\dot{C}^{\beta}}^{2}\abs{s - s'}^{2\beta}
    %     \right)^{1/2}$.
    % \end{enumerate}
    $\mathbf{T}(s)\cdot\mathbf{T}(s') \geq
    1 - \frac{1}{2}\norm{\mathbf{T}}_{\dot{C}^{\beta}}^{2}\abs{s - s'}^{2\beta}$.
\end{lemma}

\begin{proof}
    Since $\mathbf{T}$ is $C^{\beta}$, we have
    \begin{align*}
        \norm{\mathbf{T}}_{\dot{C}^{\beta}}^{2}\abs{s - s'}^{2\beta}
        &\geq \abs{\mathbf{T}(s) - \mathbf{T}(s')}^{2}
        = 2 - 2\mathbf{T}(s)\cdot \mathbf{T}(s'),
    \end{align*}
    so rearranging the above gives
    \[
        \mathbf{T}(s)\cdot\mathbf{T}(s')
        \geq 1 - \frac{1}{2}\norm{\mathbf{T}}_{\dot{C}^{\beta}}^{2}\abs{s - s'}^{2\beta}.
    \]
    % \begin{enumerate}
    %     \item Since $\mathbf{T}$ is $C^{\beta}$, we have
    %     \begin{align*}
    %         \norm{\mathbf{T}}_{\dot{C}^{\beta}}^{2}\abs{s - s'}^{2\beta}
    %         &\geq \abs{\mathbf{T}(s) - \mathbf{T}(s')}^{2}
    %         = 2 - 2\mathbf{T}(s)\cdot \mathbf{T}(s'),
    %     \end{align*}
    %     so rearranging the above gives
    %     \[
    %         \mathbf{T}(s)\cdot\mathbf{T}(s')
    %         \geq 1 - \frac{1}{2}\norm{\mathbf{T}}_{\dot{C}^{\beta}}^{2}\abs{s - s'}^{2\beta}.
    %     \]

    %     \item Without loss of generality, let us assume $s\geq s'$.
    %     By the mean value theorem, there exists $\tau$ between $s$ and $s'$ such that
    %     \begin{align*}
    %         \abs{\gamma(s) - \gamma(s')}^{2} &=
    %         (s - s')\mathbf{T}(\tau)\cdot (\gamma(s) - \gamma(s')).
    %     \end{align*}
    %     Applying the mean value theorem again, the right-hand side becomes
    %     \begin{align*}
    %         \abs{s - s'}^{2}\mathbf{T}(\tau)\cdot \mathbf{T}(\tau')
    %         &\geq \abs{s - s'}^{2}\left(
    %             1 - \frac{1}{2}\norm{\mathbf{T}}_{\dot{C}^{\beta}}^{2}\abs{s - s'}^{2\beta}
    %         \right)
    %     \end{align*}
    %     for some $\tau'$ between $s$ and $s'$, and this shows
    %     \[
    %         \abs{\gamma(s) - \gamma(s')} \geq \abs{s - s'} \left(
    %             1 - \frac{1}{2}\norm{\mathbf{T}}_{\dot{C}^{\beta}}^{2}\abs{s - s'}^{2\beta}
    %         \right)^{1/2}.
    %     \]
    % \end{enumerate}
\end{proof}

\begin{lemma}\label{L3.2}
    Let $\gamma\colon\ell\bbT\to\bbR^{2}$ be a nontrivial $C^{1,\beta}$ closed curve
    parameterized by the arclength and $\mathbf{T}\coloneqq\partial_{s}\gamma$.
    Let $x\in\bbR^{2}$ and $d\leq\frac{1}{2\norm{\mathbf{T}}_{\dot{C}^{\beta}}^{1/\beta}}$.
    Then there exist finitely many disjoint compact proper subintervals
    $I_{1},\ \cdots\ ,I_{N}$ of $\ell\bbT$ with $N\leq \frac{\ell}{4d}$ such that
    \begin{enumerate}
        \item $A\coloneqq \setbc{s\in\ell\bbT}{\abs{x - \gamma(s)} \leq d}
        \subseteq\bigcup_{i=1}^{N}I_{i}$ and $\set{A\cap I_{i}}_{i=1}^{N}$
        is precisely the set of connected components of $A$,

        \item for each $i=1,\ \cdots\ ,N$, $\abs{I_{i}}=4d$ and the center
        $s_{i}$ of $I_{i}$ satisfies $\abs{x - \gamma(s_{i})} \leq d$ and
        $(x - \gamma(s_{i}))\cdot\mathbf{T}(s_{i}) = 0$, and the function
        $g_{i}\colon s\mapsto (x - \gamma(s))\cdot\mathbf{T}(s_{i})$ defined on $I_{i}$
        is a homeomorphism onto its image such that $g_{i}'(s) \leq -\frac{1}{2}$ and
        $\abs{g_{i}(s)} \geq \frac{1}{2}\abs{s - s_{i}}$ hold for all $s\in I_{i}$.
    \end{enumerate}
\end{lemma}

\begin{proof}
    Let $J$ be any connected component of $A$. Choose a point $s\in J$ that achieves
    the minimum of $\abs{x - \gamma(s)}$. We claim $(x - \gamma(s))\cdot\mathbf{T}(s) = 0$.
    This is clear if $s$ is in the interior of $J$, so suppose that $s$ is in the boundary
    of $J$. Then by definition of $J$, we should have $\abs{x - \gamma(s)} = d$.
    Since $\partial_{s}\abs{x - \gamma(s)}^{2} = -(x - \gamma(s))\cdot\mathbf{T}(s)$, if
    the right-hand side is not zero, then there exists a small interval of $s'$ having $s$
    as its one of endpoints such that $\abs{x - \gamma(s')}$ is strictly less than $d$
    on that interval. By definition of $J$, this interval must be contained in $J$,
    contradicting to the definition of $s$. Hence, the claim is proved.

    Now, for any $s'\in\ell\bbT$ with $\abs{s - s'} \leq 2d$, we have
    \[
        \partial_{s'}\left((x - \gamma(s'))\cdot\mathbf{T}(s)\right)
        = -\mathbf{T}(s)\cdot\mathbf{T}(s')
        \leq -\left(1 - \frac{1}{2}\norm{\mathbf{T}}_{\dot{C}^{\beta}}^{2}
        \abs{s - s'}^{2\beta}\right) \leq -\frac{1}{2}.
    \]
    Since $(x - \gamma(s))\cdot\mathbf{T}(s) = 0$, let $I$ be the closed interval of length
    $4d$ centered at $s$, then $I$ should be a proper subinterval of $\ell\bbT$, and it
    contains $J$ because $s'\mapsto (x - \gamma(s'))\cdot\mathbf{T}(s)$ passes through
    $\pm d$ at some $s'\in I'$ for any open interval $I'$ containing $I$.

    Next, we show that $s$ is the only point in $I$ such that
    $(x - \gamma(s))\cdot\mathbf{T}(s) = 0$. Suppose for the sake of contradiction that
    there is another point $s'\in I$ such that $(x - \gamma(s'))\cdot\mathbf{T}(s') = 0$.
    Without loss of generality, we can let $s'\geq s$. Then by applying the estimate on
    $\partial_{s'}\left((x - \gamma(s'))\cdot\mathbf{T}(s)\right)$ derived above,
    once directly and once with $s$ and $s'$ interchanged, we obtain
    \begin{align*}
        (\gamma(s) - \gamma(s'))\cdot\mathbf{T}(s)
        &= (x - \gamma(s'))\cdot\mathbf{T}(s) < -\frac{s' - s}{2}, \\
        (\gamma(s') - \gamma(s))\cdot\mathbf{T}(s')
        &= (x - \gamma(s))\cdot\mathbf{T}(s') > \frac{s' - s}{2}.
    \end{align*}
    This shows
    \[
        (s' - s)\abs{\mathbf{T}(s') - \mathbf{T}(s)}
        \geq \abs{\gamma(s') - \gamma(s)}\abs{\mathbf{T}(s') - \mathbf{T}(s)}
        > s' - s,
    \]
    thus $\abs{s' - s} > \frac{1}{\norm{\mathbf{T}}_{\dot{C}^{\beta}}^{1/\beta}} \geq 2d$,
    which is a contradiction.

    Note that this implies $A\cap I = J$; otherwise, there exists another connected
    component of $A$ that intersects with $I$, which implies that if $s'\in I$
    is a minimizer of $\abs{x - \gamma(s')}$ in that component,
    then $(x - \gamma(s'))\cdot\mathbf{T}(s') = 0$ holds, which is a contradiction.

    Finally, it remains to show that the interval $I$'s obtained from different $J$'s
    are disjoint, which also shows that there are at most $N\leq\frac{\ell}{4d}$
    such intervals. Suppose for the sake of contradiction that there are two
    connected components $J_{1},J_{2}$ of $A$ with the unique minimizers
    $s_{1}\in J_{1}$, $s_{2}\in J_{2}$ of $\abs{x - \gamma(s)}$ such that
    $s_{1}\leq s_{2}$ and $[s_{1}, s_{1}+2d]\cap [s_{2}-2d,s_{2}]\neq\emptyset$.
    Then since $\abs{x - \gamma(s_{1})}$ and $\abs{x - \gamma(s_{2})}$ are both at most $d$
    and there exists $s\in[s_{1}, s_{2}]$ with $\abs{x - \gamma(s)} > d$,
    a maximizer of $\abs{x - \gamma(s)}$ on $[s_{1},s_{2}]$ should exist in $(s_{1},s_{2})$.
    Let $s$ be such a maximizer, then we should have either $s\in[s_{1},s_{1}+2d]$
    or $s\in[s_{2}-2d,s_{2}]$, but this is absurd because such $s$ must satisfy
    $(x - \gamma(s))\cdot\mathbf{T}(s) = 0$ where $s_{1},s_{2}$ are the only such points
    in $[s_{1},s_{1}+2d]$, $[s_{2}-2d,s_{2}]$, respectively.
\end{proof}

The next lemma is from \cite{JeonZla21}, but is reproduced here to be more precise about the dependence
on the $C^{1,\beta}$-regularity of the curves.

\begin{lemma}\label{L3.3}
    Let $\gamma_{i}\colon\ell_{i}\bbT\to\bbR^{2}$, $i=1,2$ be nontrivial $C^{1,\beta}$
    closed curves parameterized by the arclength that do not cross each other,
    and $\mathbf{T}_{i}\coloneqq\partial_{s}\gamma_{i}$,
    $\mathbf{N}_{i}\coloneqq-\mathbf{T}_{i}^{\perp}$ for $i=1,2$.
    Then for any $s_{1}\in\ell_{1}\bbT$ and $s_{2}\in\ell_{2}\bbT$, we have
    \begin{align*}
        \abs{\mathbf{T}_{1}(s_{1})\cdot\mathbf{N}_{2}(s_{2})}
        \leq 12\norm{\mathbf{T}}_{\dot{C}^{\beta}}^{\frac{1}{1+\beta}}
        \abs{\gamma_{1}(s_{1}) - \gamma_{2}(s_{2})}^{\frac{\beta}{1+\beta}},
    \end{align*}
    where $\norm{\mathbf{T}}_{\dot{C}^{\beta}}\coloneqq
    \max\left(\norm{\mathbf{T}_{1}}_{\dot{C}^{\beta}},
    \norm{\mathbf{T}_{2}}_{\dot{C}^{\beta}}\right)$.
\end{lemma}

\begin{proof}
    Let $R\coloneqq \frac{1}{4^{1+1/\beta}
    \norm{\mathbf{T}}_{\dot{C}^{\beta}}^{1/\beta}}$. Note that the inequality we want
    to show holds trivially if $r\coloneqq\abs{\gamma_{1}(s_{1}) - \gamma_{2}(s_{2})}>R$,
    so we assume $r\leq R$. Let
    \begin{align*}
        h\coloneqq \frac{r^{\frac{1}{1+\beta}}}
        {4\norm{\mathbf{T}}_{\dot{C}^{\beta}}^{\frac{1}{1+\beta}}},
    \end{align*}
    then for any $s_{1}' \in [s_{1} - h, s_{1} + h]$,
    $\abs{(\gamma_{1}(s_{1}') - \gamma_{2}(s_{2}))\cdot \mathbf{T}_{2}(s_{2})}\leq r + h$
    holds. Note that $r\leq R$ implies
    \[
        r^{\frac{\beta}{1+\beta}} \leq \frac{1}
        {4\norm{\mathbf{T}}_{\dot{C}^{\beta}}^{\frac{1}{1+\beta}}},
    \]
    thus $r\leq h$, so we get
    \[
        \abs{(\gamma_{1}(s_{1}') - \gamma_{2}(s_{2}))\cdot \mathbf{T}_{2}(s_{2})} \leq 2h
    \]
    for all $s_{1}' \in [s_{1} - h, s_{1} + h]$.

    On the other hand, since
    \[
        4h \leq \frac{R^{\frac{1}{1+\beta}}}
        {\norm{\mathbf{T}}_{\dot{C}^{\beta}}^{\frac{1}{1+\beta}}}
        \leq \frac{1}{\norm{\mathbf{T}}_{\dot{C}^{\beta}}^{1/\beta}},
    \]
    Lemma~\ref{L3.1} and mean value theorem show
    \begin{align*}
        (\gamma_{2}(s_{2}+4h) - \gamma_{2}(s_{2}))\cdot \mathbf{T}_{2}(s_{2})
        &\geq 4h\left(
            1 - \frac{1}{2}\norm{\mathbf{T}}_{\dot{C}^{\beta}}^{2}
            (4h)^{2\beta}
        \right)
        \geq 2h,
    \end{align*}
    and similarly
    \[
        (\gamma_{2}(s_{2}-4h) - \gamma_{2}(s_{2}))\cdot \mathbf{T}_{2}(s_{2}) \leq -2h.
    \]
    This shows that the segment $z_{1}|_{[s_{1}-h,s_{1}+h]}$ of $\gamma_{1}$
    is entirely contained in the strip
    \[
        S_{1}\coloneqq \setbc{y\in\bbR^{2}}
        {\gamma_{2}(s_{2}-4h)\cdot \mathbf{T}_{2}(s_{2})
        \leq y\cdot\mathbf{T}_{2}(s_{2})
        \leq \gamma_{2}(s_{2}+4h)\cdot \mathbf{T}_{2}(s_{2})}.
    \]

    Next, we estimate normal components. For any $s_{2}'\in[s_{2}-4h,s_{2}+4h]$,
    the mean value theorem gives some $\tau$ between $s_{2}$ and $s_{2}'$ such that
    \begin{align*}
        \abs{(\gamma_{2}(s_{2}') - \gamma_{2}(s_{2}))\cdot\mathbf{N}_{2}(s_{2})}
        &=\abs{s_{2}' - s_{2}}
        \abs{(\mathbf{T}_{2}(\tau) - \mathbf{T}_{2}(s_{2}))\cdot\mathbf{N}_{2}(s_{2})}
        \leq \norm{\mathbf{T}}_{\dot{C}^{\beta}}(4h)^{1+\beta} = r.
    \end{align*}
    Therefore, the segment $\gamma_{2}|_{[s_{2}-4h,s_{2}+4h]}$
    of $\gamma_{2}$ is entirely contained in the strip
    \[
        S_{2}\coloneqq \setbc{y\in\bbR^{2}}
        {\abs{(y - \gamma_{2}(s_{2}))\cdot\mathbf{N}_{2}(s_{2})} \leq r}.
    \]

    On the other hand, the mean value theorem gives some $\tau_{\pm}$ between
    $s_{1}$ and $s_{1}\pm h$ so that
    \begin{align*}
        &(\gamma_{1}(s_{1}\pm h) - \gamma_{2}(s_{2}))\cdot\mathbf{N}_{2}(s_{2}) \\
        &\quad\quad
        = \pm h\mathbf{T}_{1}(s_{1})\cdot\mathbf{N}_{2}(s_{2})
        + (\gamma_{1}(s_{1}) - \gamma_{2}(s_{2}))\cdot\mathbf{N}_{2}(s_{2})
        \pm h(\mathbf{T}_{1}(\tau_{\pm}) - \mathbf{T}_{1}(s_{1}))\cdot\mathbf{N}_{2}(s_{2})
    \end{align*}
    holds, and we know
    \begin{align*}
        \abs{(\gamma_{1}(s_{1}) - \gamma_{2}(s_{2}))\cdot\mathbf{N}_{2}(s_{2})
        \pm h(\mathbf{T}_{1}(\tau_{\pm}) - \mathbf{T}_{1}(s_{1}))\cdot\mathbf{N}_{2}(s_{2})}
        &\leq r + \norm{\mathbf{T}}_{\dot{C}^{\beta}}h^{1+\beta} \leq 2r.
    \end{align*}

    Therefore, if
    \begin{align*}
        \mathbf{T}_{1}(s_{1})\cdot\mathbf{N}_{2}(s_{2})
        > 12\norm{\mathbf{T}}_{\dot{C}^{\beta}}^{\frac{1}{1+\beta}}
        r^{\frac{\beta}{1+\beta}}
    \end{align*}
    holds, then
    \begin{align*}
        (\gamma_{1}(s_{1}+h) - \gamma_{2}(s_{2}))\cdot\mathbf{N}_{2}(s_{2})
        > 3r - 2r = r
    \end{align*}
    and
    \begin{align*}
        (\gamma_{1}(s_{1}-h) - \gamma_{2}(s_{2}))\cdot\mathbf{N}_{2}(s_{2})
        < -3r + 2r = -r
    \end{align*}
    must hold, but this is absurd because this implies the curve segments
    $\gamma_{1}|_{[s_{1}-h,s_{1}+h]}$ and $\gamma_{2}|_{[s_{2}-4h,s_{2}+4h]}$
    must cross at some point in $S_{1}\cap S_{2}$. The inequality
    \[
        \mathbf{T}_{1}(s_{1})\cdot\mathbf{N}_{2}(s_{2}) <
        -12\norm{\mathbf{T}}_{\dot{C}^{\beta}}^{\frac{1}{1+\beta}}
        r^{\frac{\beta}{1+\beta}}
    \]
    also leads to contradiction in the same way, so we are done.
\end{proof}

If $\gamma_{i}$'s are $H^{2}$ instead of $C^{1,\beta}$, then identical
proof still applies with $\beta=1$ when $\norm{\mathbf{T}_{i}}_{\dot{C}^{\beta}}$ is
replaced by $\mathcal{M}\kappa_{i}(s_{i})$, where
$\kappa_{i}\coloneqq -\partial_{s}\mathbf{T}_{i}\cdot\mathbf{N}_{i}$
is the (signed) curvature and $\mathcal{M}$ is the maximal operator defined as
\[
    \mathcal{M}f\colon s\mapsto \max\left(
        \sup_{h\in\left(0,\frac{\ell}{2}\right]}
        \frac{1}{h}\int_{s}^{s+h}\abs{f(s')}\,ds',
        \sup_{h\in\left(0,\frac{\ell}{2}\right]}
        \frac{1}{h}\int_{s-h}^{s}\abs{f(s')}\,ds'
    \right)
\]
for any locally integrable function $f\colon \ell\bbT\to\bbR$. Therefore, we obtain
the following result:

\begin{lemma}\label{L3.4}
    Let $\gamma_{i}\colon\ell_{i}\bbT\to\bbR^{2}$, $i=1,2$ be nontrivial $H^{2}$
    closed curves parameterized by the arclength that do not cross each other,
    and $\mathbf{T}_{i}\coloneqq\partial_{s}\gamma_{i}$,
    $\mathbf{N}_{i}\coloneqq-\mathbf{T}_{i}^{\perp}$,
    $\kappa_{i}\coloneqq -\partial_{s}\mathbf{T}_{i}\cdot\mathbf{N}_{i}$ for $i=1,2$.
    Then for any $s_{1}\in\ell_{1}\bbT$ and $s_{2}\in\ell_{2}\bbT$, we have
    \begin{align*}
        \abs{\mathbf{T}_{1}(s_{1})\cdot\mathbf{N}_{2}(s_{2})}
        \leq 12\max\left(\mathcal{M}\kappa_{1}(s_{1}),\mathcal{M}\kappa_{2}(s_{2})\right)^{1/2}
        \abs{\gamma_{1}(s_{1}) - \gamma_{2}(s_{2})}^{1/2}.
    \end{align*}
\end{lemma}


%%%%%%%%%%%%%%%%%%%%%%%%%%%%%%%%%%%%%%%%%%%%%%
\section{Some estimates on common integrals}\label{S4}
%%%%%%%%%%%%%%%%%%%%%%%%%%%%%%%%%%%%%%%%%%%%%%

In this section, we prove some basic estimates on the velocity field.

\begin{lemma}\label{L4.1}
    There exists a constant $C=C(\alpha)\geq 0$ depending only on $\alpha$ such that
    for any $x,y\in\bbR^{2}$ and a $C^{1,\beta}$ closed curve
    $\gamma\colon\ell\bbT\to\bbR^{2}$ parameterized by the arclength,
    \[
        \int_{\ell\bbT}\frac{ds}{\abs{x - \gamma(s)}^{2\alpha}}
        \leq C\ell\norm{\gamma}_{\dot{C}^{1,\beta}}^{2\alpha/\beta}
    \]
    and
    \[
        \int_{\ell\bbT}\abs{
            \frac{1}{\abs{x - \gamma(s)}^{2\alpha}}
            - \frac{1}{\abs{y - \gamma(s)}^{2\alpha}}
        }\,ds
        \leq C\ell
        \norm{\gamma}_{\dot{C}^{1,\beta}}^{1/\beta}\abs{x - y}^{1-2\alpha}.
    \]
\end{lemma}

\begin{proof}
    We may assume that $\gamma$ is a nontrivial curve.
    Applying Lemma~\ref{L3.2} with
    $d = \frac{1}{2\norm{\gamma}_{\dot{C}^{1,\beta}}^{1/\beta}}$, we have
    \begin{align*}
        \int_{\ell\bbT}\frac{ds}{\abs{x - \gamma(s)}^{2\alpha}}
        &\leq \sum_{i=1}^{N}\int_{-2d}^{2d}\frac{2^{2\alpha}ds}{\abs{s}^{2\alpha}}
        + \int_{\ell\bbT}\frac{ds}{d^{2\alpha}}
        = \frac{C(\alpha)Nd^{1-2\alpha}}{1-2\alpha}
        + \frac{\ell}{d^{2\alpha}} \\
        &\leq \frac{C(\alpha)}{1 - 2\alpha}\frac{\ell}{d^{2\alpha}}
        = \frac{C(\alpha)}{1-2\alpha}\ell
        \norm{\gamma}_{\dot{C}^{1,\beta}}^{2\alpha/\beta}.
    \end{align*}

    For the second claim, note first that if $\abs{x - y} > 2d$, then
    \begin{align*}
        \int_{\ell\bbT}\abs{\frac{1}{\abs{x - \gamma(s)}^{2\alpha}}
        - \frac{1}{\abs{y - \gamma(s)}^{2\alpha}}}ds
        &\leq \frac{C(\alpha)}{1-2\alpha}\frac{\ell}{d^{2\alpha}}
        = \frac{C(\alpha)}{1 - 2\alpha}\frac{\ell}{d}d^{1-2\alpha} \\
        &\leq \frac{C(\alpha)}{1-2\alpha}
        \ell\norm{\gamma}_{\dot{C}^{1,\beta}}^{1/\beta}\abs{x-y}^{1-2\alpha}
    \end{align*}
    holds, so assume $\abs{x - y} \leq 2d$.
    Let $B\coloneqq\setbc{s\in\ell\bbT}{\abs{x - \gamma(s)}\leq\abs{y - \gamma(s)}}$ and
    $I_{1},\ \cdots\ ,I_{N}$ be the intervals we get from
    Lemma~\ref{L3.2} with
    $d = \frac{1}{2\norm{\gamma}_{\dot{C}^{1,\beta}}^{1/\beta}}$,
    and $s_{i}$ be the center of $I_{i}$ for each $i$. For each $i$, we split the integral
    \begin{align*}
        \int_{I_{i}\cap B}\abs{
            \frac{1}{\abs{x - \gamma(s)}^{2\alpha}}
            - \frac{1}{\abs{y - \gamma(s)}^{2\alpha}}
        }ds
    \end{align*}
    into two regions, over $\setbc{s\in I_{i}\cap B}{\abs{s - s_{i}}\leq \abs{x-y}}$
    and over the complement. The first integral is clearly bounded by
    \begin{align*}
        \int_{\abs{s}\leq\abs{x - y}}\frac{2^{2\alpha}}{\abs{s}^{2\alpha}}\,ds
        &= \frac{C(\alpha)}{1-2\alpha}\abs{x-y}^{1-2\alpha}.
    \end{align*}
    For the second integral, mean value theorem shows
    \begin{align*}
        \abs{
            \frac{1}{\abs{x - \gamma(s)}^{2\alpha}}
            - \frac{1}{\abs{y - \gamma(s)}^{2\alpha}}
        }
        &\leq \frac{2\alpha\abs{x - y}}{\abs{x - \gamma(s)}^{1+2\alpha}},
    \end{align*}
    so the integral is bounded by
    \begin{align*}
        \int_{\abs{s}>\abs{x-y}}\frac{2^{2+2\alpha}\alpha\abs{x-y}}{\abs{s}^{1+2\alpha}}\,ds
        &\leq C(\alpha)\abs{x - y}^{1-2\alpha}.
    \end{align*}
    As a result, we conclude
    \begin{align*}
        \int_{\bigcup_{i=1}^{N}I_{i}\cap B}
        \abs{
            \frac{1}{\abs{x - \gamma(s)}^{2\alpha}}
            - \frac{1}{\abs{y - \gamma(s)}^{2\alpha}}
        }\,ds
        &\leq \frac{\ell}{4d}\cdot \frac{C(\alpha)}{1-2\alpha}\abs{x - y}^{1-2\alpha} \\
        &\leq \frac{C(\alpha)}{1-2\alpha}
        \ell\norm{\gamma}_{\dot{C}^{1,\beta}}^{1/\beta}
        \abs{x - y}^{1-2\alpha}.
    \end{align*}
    Again by the mean value theorem, the integral outside of $\bigcup_{i=1}^{N}I_{i}$
    can be bounded as
    \begin{align*}
        \int_{B\setminus \bigcup_{i=1}^{N}I_{i}}
        \abs{
            \frac{1}{\abs{x - \gamma(s)}^{2\alpha}}
            - \frac{1}{\abs{y - \gamma(s)}^{2\alpha}}
        }\,ds
        &\leq \int_{\ell\bbT}\frac{2\alpha\abs{x-y}}{d^{1+2\alpha}}\,ds \\
        &\leq \frac{C(\alpha)\ell}{d}\abs{x-y}^{1-2\alpha} \\
        &= C(\alpha)\ell\norm{\gamma}_{\dot{C}^{1,\beta}}^{1/\beta}
        \abs{x - y}^{1-2\alpha}
    \end{align*}
    where the second inequality follows from $\abs{x-y}\leq 2d$. Adding the above two
    and then applying the symmetry of the integrand with respect to $x$ and $y$, we conclude
    \[
        \int_{\ell\bbT}
        \abs{
            \frac{1}{\abs{x - \gamma(s)}^{2\alpha}}
            - \frac{1}{\abs{y - \gamma(s)}^{2\alpha}}
        }\,ds
        \leq \frac{C(\alpha)}{1-2\alpha}
        \ell\norm{\gamma}_{\dot{C}^{1,\beta}}^{1/\beta}\abs{x - y}^{1-2\alpha}.
    \]
\end{proof}

\textbf{Remark.} If $\gamma$ is a simple closed curve, then a form of Green's theorem
can be applied to show that the integral
\[
    \int_{\ell\bbT}\frac{\partial_{s}\gamma(s)}{\abs{x - \gamma(s)}^{2\alpha}}\,ds
\]
converges in an appropriate sense. However, it can fail to converge absolutely
if $\gamma$ is merely just $C^{1}$. For example, $\gamma$ may contain infinitely many
semicircles centered at $x$ with radii $n^{-1/(1-2\alpha)}$ for $n=1,2,3,\ \cdots\ $
while the total length $\ell$ is finite, in which case the integral of
$\abs{x-\gamma(s)}^{-2\alpha}$ diverge to infinity. Also, if $\gamma$ is not
a simple closed curve and arbitrary self-intersection is allowed, then it is possible
to align all of those semicircles in the same direction so that 
the integral above even diverges to infinity in any reasonable sense.
Hence, if we do not a priori exclude self-intersecting curves,
then mere $C^{1}$-regularity is not enough to ensure convergence.\\


The lemma above together with the inequalities
$\ell(\gamma) \leq \norm{\gamma}_{C^{0}}^{2}\mathcal{K}_{2}(\gamma)$
and $\norm{\gamma}_{\dot{C}^{1,1/2}}^{2} \leq \mathcal{K}_{2}(\gamma)$
immediately imply the following conclusion:

\begin{proposition}\label{P4.2}
    For $z\in L_{tb}^{\infty}(\mathcal{L};\operatorname{Curve}(\bbR^{2}))$ with
    $\norm{\lambda\mapsto\ell(z(\lambda))}_{L^{\infty}} < \infty$ and
    $\norm{\lambda\mapsto\norm{z(\lambda)}_{\dot{C}^{1,\beta}}}_{L^{\infty}} < \infty$
    for some $\beta\in(0,1]$, the vector field $u(z)\colon\bbR^{2}\to\bbR^{2}$ given in
    Definition~\ref{D2.5} is well-defined, bounded, and is in $C^{1-2\alpha}$.
    More precisely, there exists a constant $C=C(\alpha)\geq 0$
    depending only on $\alpha$ such that
    \[
        \norm{u(z)}_{C^{0}} \leq C\int\ell(z(\lambda))
        \norm{z(\lambda)}_{\dot{C}^{1,\beta}}^{2\alpha/\beta}\,d\lambda
        \quad\textrm{and}\quad
        \norm{u(z)}_{\dot{C}^{1-2\alpha}}
        \leq C\int\ell(z(\lambda))
        \norm{z(\lambda)}_{\dot{C}^{1,\beta}}^{1/\beta}\,d\lambda.
    \]
    In particular,
    \[
        \norm{u(z)}_{C^{0}}, \norm{u(z)}_{\dot{C}^{1-2\alpha}}
        \leq C\abs{\mathcal{L}}
        \norm{\lambda\mapsto\norm{z(\lambda)}_{C^{0}}}_{L^{\infty}}^{2}
        \left(
            \norm{\lambda\mapsto\mathcal{K}_{2}(z(\lambda))}_{L^{\infty}}^{2} + 1
        \right).
    \]
\end{proposition}

\begin{lemma}\label{L4.3}
    There exists a constant $C=C(\alpha)\geq 0$ depending only on $\alpha$ such that
    for any $x\in\bbR^{2}$ and a $C^{1,\beta}$ closed curve
    $\gamma\colon\ell\bbT\to\bbR^{2}$ parameterized by the arclength,
    \[
        \int_{\ell\bbT}\frac{ds}{\abs{x - \gamma(s)}^{1+2\alpha}}
        \leq C\frac{\ell\norm{\gamma}_{\dot{C}^{1,\beta}}^{1/\beta}}
        {\Delta^{2\alpha}}
    \]
    where $\Delta \coloneqq \min_{s\in\ell\bbT}\abs{x - \gamma(s)}$.
\end{lemma}

\begin{proof}
    We may assume that $\gamma$ is a nontrivial curve.
    Applying Lemma~\ref{L3.2} with
    $d = \frac{1}{2\norm{\gamma}_{\dot{C}^{1,\beta}}^{1/\beta}}$, we have
    \begin{align*}
        \int_{\ell\bbT}\frac{ds}{\abs{x - \gamma(s)}^{1+2\alpha}}
        &\leq \sum_{i=1}^{N}\left(
            \int_{\abs{s}\leq \Delta}\frac{ds}{\Delta^{1+2\alpha}}
            + \int_{\Delta<\abs{s}\leq 2d}\frac{2^{1+2\alpha}}{\abs{s}^{1+2\alpha}}\,ds
        \right)
        + \frac{1}{\Delta^{2\alpha}}\int_{\ell\bbT}\frac{ds}{d} \\
        &\leq \frac{\ell}{2d\Delta^{2\alpha}}
        + \frac{C(\alpha)\ell}{\alpha d\Delta^{2\alpha}}
        + \frac{\ell}{d\Delta^{2\alpha}}
        \leq \frac{C(\alpha)}{\alpha}
        \frac{\ell\norm{\gamma}_{\dot{C}^{1,\beta}}^{1/\beta}}{\Delta^{2\alpha}}.
    \end{align*}
\end{proof}

\textbf{Remark.} When $\alpha=0$, the same estimate holds with $1/\Delta^{2\alpha}$
replaced by $1+\log^{+}(1/\Delta)$.


%%%%%%%%%%%%%%%%%%%%%%%%%%%%%%%%%%%%%%%%%%%%%%
\section{Proof of uniqueness}
%%%%%%%%%%%%%%%%%%%%%%%%%%%%%%%%%%%%%%%%%%%%%%

We first prove that the solution as defined in Definition~\ref{D2.6},
if it exists, is unique for any initial data. In fact, the uniqueness is an immediate
consequence of the Lipschitz dependence of the velocity field $u(z)$ on $z$,
which follows easily from the following estimate.

\begin{lemma}\label{L5.1}
    There exists a constant $C=C(\alpha)\geq 0$ depending only on $\alpha$ such that
    for any $x_{1},x_{2}\in\bbR^{2}$, $\beta\in(0,1]$ and $C^{1,\beta}$ closed curves
    $\gamma_{i}\colon\ell_{i}\bbT\to\bbR^{2}$, $i=1,2$ parameterized by the arclength,
    \begin{align*}
        &\abs{\int_{\ell_{1}\bbT}\frac{\partial_{s}\gamma_{1}(s_{1})}
        {\abs{x_{1} - \gamma_{1}(s_{1})}^{2\alpha}}\,ds_{1}
        - \int_{\ell_{2}\bbT}\frac{\partial_{s}\gamma_{2}(s_{2})}
        {\abs{x_{2} - \gamma_{2}(s_{2})}^{2\alpha}}\,ds_{2}} \\
        &\quad\quad\quad\quad\leq
        C\left(
            \frac{\ell_{1}\norm{\gamma_{1}}_{\dot{C}^{1,\beta}}^{1/\beta}}
            {\Delta_{1}^{2\alpha}}
            + \frac{\ell_{2}\norm{\gamma_{2}}_{\dot{C}^{1,\beta}}^{1/\beta}}
            {\Delta_{2}^{2\alpha}}
        \right)
        \left(\abs{x_{1}-x_{2}} + d_{\mathrm{F}}(\gamma_{1},\gamma_{2})\right)
    \end{align*}
    where $\Delta_{i}\coloneqq \min_{s\in\ell_{i}\bbT}\abs{x_{i} - \gamma_{i}(s)}$
    for $i=1,2$.
\end{lemma}

\begin{proof}
    For each $i=1,2$, let $\tilde{\gamma}_{i}\colon\bbT\to\bbR^{2}$ be the constant-speed
    parameterizations of $\gamma_{i}$ given as
    $\tilde{\gamma}_{i}\colon\xi\mapsto \gamma_{i}(\ell_{i}\xi)$.
    For any given $\epsilon>0$, Lemma~\ref{LA.12}
    shows that there exists an orientation-preserving diffeomorphism
    $\phi\colon\bbT\to\bbT$ such that
    $\norm{\tilde{\gamma}_{1}\circ\phi - \tilde{\gamma}_{2}}_{C^{0}}
    \leq d_{\mathrm{F}}(\gamma_{1},\gamma_{2}) + \epsilon$.
    Define $B\coloneqq\setbc{\xi\in\bbT}{\abs{x_{1} - \tilde{\gamma}_{1}\circ\phi(\xi)}
    < \abs{x_{2} - \tilde{\gamma}_{2}(\xi)}}$, then we have
    \begin{align*}
        &\abs{\int_{\ell_{1}\bbT}\frac{\partial_{s}\gamma_{1}(s_{1})}
        {\abs{x_{1} - \gamma_{1}(s_{1})}^{2\alpha}}\,ds_{1}
        - \int_{\ell_{2}\bbT}\frac{\partial_{s}\gamma_{2}(s_{2})}
        {\abs{x_{2} - \gamma_{2}(s_{2})}^{2\alpha}}\,ds_{2}}
        \\&\quad\quad\leq
        \abs{\int_{\bbT}
        \frac{\partial_{\xi}(\tilde{\gamma}_{1}\circ\phi(\xi) - \tilde{\gamma}_{2}(\xi))}
        {\abs{x_{1} - \tilde{\gamma}_{1}\circ\phi(\xi)}^{2\alpha}}\,d\xi}
        \\&\quad\quad\quad\quad\quad
        + \ell_{2}\int_{\bbT\setminus B}
        \abs{\frac{1}{\abs{x_{1} - \tilde{\gamma}_{1}\circ\phi(\xi)}^{2\alpha}}
        - \frac{1}{\abs{x_{2} - \tilde{\gamma}_{2}(\xi)}^{2\alpha}}}d\xi
        \\&\quad\quad\quad\quad\quad
        + \ell_{1}\int_{\phi(B)}
        \abs{\frac{1}{\abs{x_{1} - \tilde{\gamma}_{1}(\xi)}^{2\alpha}}
        - \frac{1}{\abs{x_{2} - \tilde{\gamma}_{2}\circ\phi^{-1}(\xi)}^{2\alpha}}}d\xi
        \\&\quad\quad\quad\quad\quad
        + \abs{\int_{B}\left(
            \frac{1}{\abs{x_{1} - \tilde{\gamma}_{1}\circ\phi(\xi)}^{2\alpha}}
            - \frac{1}{\abs{x_{2} - \tilde{\gamma}_{2}(\xi)}^{2\alpha}}
        \right)
        \partial_{\xi}(\tilde{\gamma}_{1}\circ\phi(\xi) - \tilde{\gamma}_{2}(\xi))\,d\xi}.
    \end{align*}
    Let us call each term in the right-hand side as
    $G_{1}$, $G_{2}$, $G_{3}$, and $G_{4}$, respectively.\\

    \textbf{Estimate for $G_{1}$.} By integration by parts, we get
    \begin{align*}
        G_{1} &= \abs{
            \int_{\bbT}
            \frac{2\alpha(\tilde{\gamma}_{1}\circ\phi(\xi) - \tilde{\gamma}_{2}(\xi))}
            {\abs{x_{1} - \tilde{\gamma}_{1}\circ\phi(\xi)}^{2+2\alpha}}
            (x_{1} - \tilde{\gamma}_{1}\circ\phi(\xi))\cdot
            \partial_{\xi}(\tilde{\gamma}_{1}\circ\phi)(\xi)\,d\xi
        } \\
        &\leq 2\alpha\norm{\tilde{\gamma}_{1}\circ\phi - \tilde{\gamma}_{2}}_{C^{0}}
        \int_{\ell_{1}\bbT}\frac{ds_{1}}{\abs{x_{1} - \gamma_{1}(s_{1})}^{1+2\alpha}}.
    \end{align*}
    Therefore, Lemma~\ref{L4.3} shows
    \[
        G_{1} \leq C(\alpha)
        \frac{\ell_{1}\norm{\gamma_{1}}_{\dot{C}^{1,\beta}}^{1/\beta}}
        {\Delta_{1}^{2\alpha}}
        \left(d_{\mathrm{F}}(\gamma_{1},\gamma_{2}) + \epsilon\right).
    \]

    \textbf{Estimate for $G_{2}$.} We may assume that $\gamma_{2}$ is nontrivial.
    By mean value theorem, we know that
    \begin{align*}
        \abs{
            \frac{1}{\abs{x_{1} - \tilde{\gamma}_{1}\circ\phi(\xi)}^{2\alpha}}
            - \frac{1}{\abs{x_{2} - \tilde{\gamma}_{2}(\xi)}^{2\alpha}}
        }
        \leq \frac{2\alpha\left(
            \abs{x_{1} - x_{2}}
            + \norm{\tilde{\gamma}_{1}\circ\phi - \tilde{\gamma}_{2}}_{C^{0}}
        \right)}
        {\abs{x_{2} - \tilde{\gamma}_{2}(\xi)}^{1+2\alpha}}
    \end{align*}
    holds for all $\xi\in \bbT\setminus B$.
    Let $I_{1},\ \cdots\ ,I_{N}$ be the intervals
    we get from Lemma~\ref{L3.2} applied to $\gamma_{2}$ with
    $d = \frac{1}{2\norm{\partial_{s}\gamma_{2}}_{\dot{C}^{\beta}}^{1/\beta}}$, and
    $s_{2i}$ be the center of $I_{i}$. For each $i$, we split the integral
    \[
        \int_{I_{i}\setminus \ell_{2}B}\abs{
            \frac{1}{\abs{x_{1} - \tilde{\gamma}_{1}\circ\phi(s_{2}/\ell_{2})}^{2\alpha}}
            - \frac{1}{\abs{x_{2} - \gamma_{2}(s_{2})}^{2\alpha}}
        }\,ds_{2}
    \]
    into two regions, over $\setbc{s_{2}\in I_{i}\setminus \ell_{2}B}
    {\abs{s_{2} - s_{2i}}\leq \Delta_{2}}$ and over the complement. Then the first integral is
    bounded by
    \begin{align*}
        \int_{\abs{s_{2}}\leq \Delta_{2}}
        \frac{2\alpha\left(
            \abs{x_{1} - x_{2}}
            + \norm{\tilde{\gamma}_{1}\circ\phi - \tilde{\gamma}_{2}}_{C^{0}}
        \right)}
        {\Delta_{2}^{1+2\alpha}}\,ds_{2}
        = \frac{4\alpha\left(
            \abs{x_{1} - x_{2}}
            + \norm{\tilde{\gamma}_{1}\circ\phi - \tilde{\gamma}_{2}}_{C^{0}}
        \right)}
        {\Delta_{2}^{2\alpha}}
    \end{align*}
    and the second integral is bounded by
    \begin{align*}
        \int_{\abs{s_{2}}>\Delta_{2}}
        \frac{2^{2+2\alpha}\alpha
        \left(
            \abs{x_{1} - x_{2}}
            + \norm{\tilde{\gamma}_{1}\circ\phi - \tilde{\gamma}_{2}}_{C^{0}}
        \right)}
        {\abs{s_{2}}^{1+2\alpha}}\,ds_{2}
        = \frac{2^{2+2\alpha}\left(
            \abs{x_{1} - x_{2}}
            + \norm{\tilde{\gamma}_{1}\circ\phi - \tilde{\gamma}_{2}}_{C^{0}}
        \right)}
        {\Delta_{2}^{2\alpha}},
    \end{align*}
    thus we conclude
    \begin{align*}
        &\int_{\bigcup_{i=1}^{N}I_{i}\setminus \ell_{2}B}\abs{
            \frac{1}{\abs{x_{1} - \tilde{\gamma}_{1}\circ\phi(s_{2}/\ell_{2})}^{2\alpha}}
            - \frac{1}{\abs{x_{2} - \gamma_{2}(s_{2})}^{2\alpha}}
        }\,ds_{2}
        \\&\quad\quad\quad\leq
        \frac{\ell_{2}}{4d}\cdot
        \frac{C(\alpha)\left(
            \abs{x_{1} - x_{2}}
            + \norm{\tilde{\gamma}_{1}\circ\phi - \tilde{\gamma}_{2}}_{C^{0}}
        \right)}
        {\Delta_{2}^{2\alpha}}
        \\&\quad\quad\quad=
        C(\alpha)\frac{\ell_{2}\norm{\gamma_{2}}_{\dot{C}^{1,\beta}}^{1/\beta}}
        {\Delta_{2}^{2\alpha}}
        \left(
            \abs{x_{1} - x_{2}}
            + \norm{\tilde{\gamma}_{1}\circ\phi - \tilde{\gamma}_{2}}_{C^{0}}
        \right).
    \end{align*}
    On the other hand, the integral outside of $\bigcup_{i=1}^{N}I_{i}$ can be bounded by
    \begin{align*}
        \ell_{2}\int_{\bbT}
        \frac{2\alpha\left(
            \abs{x_{1} - x_{2}}
            + \norm{\tilde{\gamma}_{1}\circ\phi - \tilde{\gamma}_{2}}_{C^{0}}
        \right)}
        {d\Delta_{2}^{2\alpha}}\,d\xi
        = \frac{4\alpha\ell_{2}\norm{\gamma_{2}}_{\dot{C}^{1,\beta}}^{1/\beta}}
        {\Delta_{2}^{2\alpha}}
        \left(
            \abs{x_{1} - x_{2}}
            + \norm{\tilde{\gamma}_{1}\circ\phi - \tilde{\gamma}_{2}}_{C^{0}}
        \right),
    \end{align*}
    thus we conclude
    \[
        G_{2} \leq C(\alpha)
        \frac{\ell_{2}\norm{\gamma_{2}}_{\dot{C}^{1,\beta}}^{1/\beta}}
        {\Delta_{2}^{2\alpha}}
        \left(\abs{x_{1}-x_{2}} + d_{\mathrm{F}}(\gamma_{1},\gamma_{2}) + \epsilon\right).
    \]

    \textbf{Estimate for $G_{3}$.} In the same way as $G_{2}$, we conclude
    \[
        G_{3} \leq C(\alpha)
        \frac{\ell_{1}\norm{\gamma_{1}}_{\dot{C}^{1,\beta}}^{1/\beta}}
        {\Delta_{1}^{2\alpha}}
        \left(\abs{x_{1}-x_{2}} + d_{\mathrm{F}}(\gamma_{1},\gamma_{2}) + \epsilon\right).
    \]

    \textbf{Estimate for $G_{4}$.} Since $B$ is an open subset of $\bbT$,
    either $B=\bbT$ or we can write $B$ as a countable disjoint union of open intervals
    $\bigcup_{i}(a_{i},b_{i})$. Either case, the integral can be written to be over
    a countable union of the form $\bigcup_{i}(a_{i},b_{i})$. By definition of $B$, we have
    \[
        \abs{x_{1} - \tilde{\gamma}_{1}\circ\phi(a_{i})}
        = \abs{x_{2} - \tilde{\gamma}_{2}(a_{i})}
        \quad\textrm{and}\quad
        \abs{x_{1} - \tilde{\gamma}_{1}\circ\phi(b_{i})}
        = \abs{x_{2} - \tilde{\gamma}_{2}(b_{i})}.
    \]
    Therefore, by integration by parts, we can write
    \begin{align*}
        G_{4} &= \abs{
            \sum_{i}\int_{a_{i}}^{b_{i}}
            \partial_{\xi}\left(
                \frac{1}{\abs{x_{1} - \tilde{\gamma}_{1}\circ\phi(\xi)}^{2\alpha}}
            - \frac{1}{\abs{x_{2} - \tilde{\gamma}_{2}(\xi)}^{2\alpha}}
            \right)
            (\tilde{\gamma}_{1}\circ\phi(\xi) - \tilde{\gamma}_{2}(\xi))\,d\xi
        } \\
        &\leq \norm{\tilde{\gamma}_{1}\circ\phi - \tilde{\gamma}_{2}}_{C^{0}}
        \int_{\bbT}\abs{\partial_{\xi}\left(
            \frac{1}{\abs{x_{1} - \tilde{\gamma}_{1}\circ\phi(\xi)}^{2\alpha}}
        - \frac{1}{\abs{x_{2} - \tilde{\gamma}_{2}(\xi)}^{2\alpha}}
        \right)}d\xi \\
        &\leq 2\alpha\norm{\tilde{\gamma}_{1}\circ\phi - \tilde{\gamma}_{2}}_{C^{0}}
        \int_{\bbT}\frac{\abs{\partial_{\xi}(\tilde{\gamma}_{1}\circ\phi)(\xi)}}
        {\abs{x_{1} - \tilde{\gamma}_{1}\circ\phi(\xi)}^{1+2\alpha}}
        + \frac{\abs{\partial_{\xi}\tilde{\gamma}_{2}(\xi)}}
        {\abs{x_{2} - \tilde{\gamma}_{2}(\xi)}^{1+2\alpha}}\,d\xi \\
        &= 2\alpha\norm{\tilde{\gamma}_{1}\circ\phi - \tilde{\gamma}_{2}}_{C^{0}}
        \left(
            \int_{\ell_{1}\bbT}\frac{ds_{1}}{\abs{x_{1} - \gamma_{1}(s_{1})}^{1+2\alpha}}
            + \int_{\ell_{2}\bbT}\frac{ds_{2}}{\abs{x_{2} - \gamma_{2}(s_{2})}^{1+2\alpha}}
        \right),
    \end{align*}
    thus Lemma~\ref{L4.3} shows
    \begin{align*}
        G_{4} \leq C(\alpha)\left(
            \frac{\ell_{1}\norm{\gamma_{1}}_{\dot{C}^{1,\beta}}^{1/\beta}}
            {\Delta_{1}^{2\alpha}}
            + \frac{\ell_{2}\norm{\gamma_{2}}_{\dot{C}^{1,\beta}}^{1/\beta}}
            {\Delta_{2}^{2\alpha}}
        \right)
        \left(d_{\mathrm{F}}(\gamma_{1},\gamma_{2}) + \epsilon\right).
    \end{align*}

    Since $\epsilon>0$ is arbitrary, we get the desired conclusion.
\end{proof}

\begin{lemma}\label{L5.2}
    For any $M\geq 0$, there exists a constant $C=C(\alpha,M)\geq 0$
    depending only on $\alpha$ and $M$ such that for any
    $z_{1},z_{2}\in L_{tb}^{\infty}(\mathcal{L};\operatorname{Curve}(\bbR^{2}))$
    with $L(z_{1}),L(z_{2}) \leq M$,
    \[
        \norm{u(z_{1})\circ\tilde{z}_{1}(\lambda)
        - u(z_{2})\circ\tilde{z}_{2}(\lambda)}_{C^{0}}
        \leq C
        \left(
            \norm{\tilde{z}_{1}(\lambda) - \tilde{z}_{2}(\lambda)}_{C^{0}}
            + d_{\mathrm{F},\infty}(z_{1},z_{2})
        \right)
    \]
    holds for all $\lambda\in\mathcal{L}$ and any parameterizations
    $\tilde{z}_{i}(\lambda)\colon\bbT\to\bbR^{2}$ of the curves
    $z_{i}(\lambda)$, $i=1,2$. Furthermore, for such $z_{1},z_{2}$,
    \[
        d_{\mathrm{F},\infty}\left(
            X_{u(z_{1})}^{h}[z_{1}],
            X_{u(z_{2})}^{h}[z_{2}]
        \right) \leq (1+2C\abs{h})d_{\mathrm{F},\infty}(z_{1},z_{2})
    \]
    holds for any $h\in\bbR$.
\end{lemma}

\begin{proof}
    Fix $\lambda\in\mathcal{L}$ and any parameterizations
    $\tilde{z}_{i}(\lambda)\colon\bbT\to\bbR^{2}$ of $z_{i}(\lambda)$, $i=1,2$.
    Lemma~\ref{L5.1}
    shows that for any $\xi\in\bbT$ and $\lambda'\in\mathcal{L}$,
    \begin{align*}
        &\abs{
            \int_{\ell(z_{1}(\lambda'))\bbT}
            \frac{\partial_{s}z_{1}(\lambda',s_{1}')}
            {\abs{\tilde{z}_{1}(\lambda)(\xi) - z_{1}(\lambda',s_{1}')}^{2\alpha}}
            \,ds_{1}'
            - \int_{\ell(z_{2}(\lambda'))\bbT}
            \frac{\partial_{s}z_{2}(\lambda',s_{2}')}
            {\abs{\tilde{z}_{2}(\lambda)(\xi) - z_{2}(\lambda',s_{2}')}^{2\alpha}}
            \,ds_{2}'
        } \\
        &\quad\quad \leq
        C(\alpha)\left(
            \frac{\ell(z_{1}(\lambda'))
            \norm{z_{1}(\lambda')}_{\dot{C}^{1,1/2}}^{2}}
            {\Delta(z_{1}(\lambda),z_{1}(\lambda'))^{2\alpha}}
            + \frac{\ell(z_{2}(\lambda'))
            \norm{z_{2}(\lambda')}_{\dot{C}^{1,1/2}}^{2}}
            {\Delta(z_{2}(\lambda),z_{2}(\lambda'))^{2\alpha}}
        \right) \\
        &\quad\quad\quad\quad\quad \cdot
        \left(
            \norm{\tilde{z}_{1}(\lambda)-\tilde{z}_{2}(\lambda)}_{C^{0}}
            + d_{\mathrm{F}}(z_{1}(\lambda'),z_{2}(\lambda'))
        \right),
    \end{align*}
    thus applying the inequalities
    $\ell(\gamma) \leq \norm{\gamma}_{C^{0}}^{2}\mathcal{K}_{2}(\gamma)$
    and $\norm{\gamma}_{\dot{C}^{1,1/2}} \leq \mathcal{K}_{2}(\gamma)^{1/2}$,
    and then integrating with respect to $\lambda'$ gives
    \begin{align*}
        &\abs{u(z_{1})(\tilde{z}_{1}(\lambda)(\xi))
        - u(z_{2})(\tilde{z}_{2}(\lambda)(\xi))}
        \\&\quad\quad\quad \leq
        C(\alpha)M^{5} \left(
            \norm{\tilde{z}_{1}(\lambda)-\tilde{z}_{2}(\lambda)}_{C^{0}}
            + d_{\mathrm{F},\infty}(z_{1},z_{2})
        \right).
    \end{align*}
    Finally, taking the supremum over $\xi\in\bbT$ gives the first claim.
    
    For the second claim, fix $h\in\bbR$ and $\lambda\in\mathcal{L}$, then for any
    parameterizations $\tilde{z}_{i}(\lambda)\colon\bbT\to\bbR^{2}$
    of $z_{i}(\lambda)$, $i=1,2$,
    \begin{align*}
        & d_{\mathrm{F}}\left(
            X_{u(z_{1})}^{h}[z_{1}](\lambda),
            X_{u(z_{2})}^{h}[z_{2}](\lambda)
        \right)
        \\&\quad\quad\quad \leq
        \norm{\tilde{z}_{1}(\lambda) - \tilde{z}_{2}(\lambda)}_{C^{0}}
        + \abs{h}\norm{u(z_{1})\circ\tilde{z}_{1}(\lambda)
        - u(z_{2})\circ\tilde{z}_{2}(\lambda)}_{C^{0}}
        \\&\quad\quad\quad \leq
        \left(1 + C\abs{h}\right)
        \norm{\tilde{z}_{1}(\lambda) - \tilde{z}_{2}(\lambda)}_{C^{0}}
        + C\abs{h}d_{\mathrm{F},\infty}(z_{1}, z_{2}).
    \end{align*}
    Hence, taking the infimum over all parameterizations and then taking the
    essential supremum over $\lambda\in\mathcal{L}$ shows the second claim.
\end{proof}

Now we prove the uniqueness part of Theorem~\ref{T2.7}.

\begin{proof}[Proof of Theorem~\ref{T2.7}, uniqueness.]
    Let $z_{1},z_{2}\colon[0,T)\to
    L_{tb}^{\infty}(\mathcal{L};\operatorname{Curve}(\bbR^{2}))$ be both
    solutions to the g-SQG equation with the initial data $z_{0}$.
    Let $M\coloneqq \max_{i=1,2}\sup_{t\in [0,T']}L(z_{i}(t))$
    for any given $T'\in[0,T)$. Since $d_{\mathrm{F},\infty}(z_{1}(t),z_{2}(t))$
    is clearly a continuous function of $t$, by a Gr\"{o}nwall argument
    (do we need a reference?), it is enough to show that
    \[
        \partial_{t}^{+}d_{\mathrm{F},\infty}(z_{1}(t),z_{2}(t))
        \leq C(\alpha,M)d_{\mathrm{F},\infty}(z_{1}(t), z_{2}(t))
    \]
    holds for all $t\in[0,T')$.

    To do so, let $\epsilon>0$ be given, then for any small enough $h$, we have
    \[
        d_{\mathrm{F},\infty}(z_{1}(t+h), z_{2}(t+h))
        \leq d_{\mathrm{F},\infty}\left(
            X_{u(z_{1}(t))}^{h}[z_{1}(t)],
            X_{u(z_{2}(t))}^{h}[z_{2}(t)]
        \right) + \epsilon h.
    \]
    Fix such $h$, then Lemma~\ref{L5.2} shows
    \begin{align*}
        & d_{\mathrm{F},\infty}\left(
            X_{u(z_{1}(t))}^{h}[z_{1}(t)],
            X_{u(z_{2}(t))}^{h}[z_{2}(t)]
        \right)
        \leq \left(1 + C(\alpha,M)h\right)
        d_{\mathrm{F},\infty}(z_{1}(t), z_{2}(t)).
    \end{align*}
    Therefore, we get
    \begin{align*}
        \frac{d_{\mathrm{F},\infty}(z_{1}(t+h), z_{2}(t+h))
        - d_{\mathrm{F},\infty}(z_{1}(t), z_{2}(t))}{h}
        \leq 2Cd_{\mathrm{F},\infty}(z_{1}(t), z_{2}(t))
        + \epsilon,
    \end{align*}
    so taking the limit supremum over $h$ and then sending $\epsilon \to 0$
    gives the desired conclusion.
\end{proof}


%%%%%%%%%%%%%%%%%%%%%%%%%%%%%%%%%%%%%%%%%%%%%%
\section{Proof of existence} \label{S6}
%%%%%%%%%%%%%%%%%%%%%%%%%%%%%%%%%%%%%%%%%%%%%%

\subsection{Mollified contour equations}

We will construct our solution by solving an analogue of
the \emph{contour equation} as done in (references),
that is, by tracking the evolution of specific parameterizations of the curves,
assuming $H^{2}$-regularity on the parameterizations.
Recall that the space of functions we want to construct our solution on is the set
\[
    X\coloneqq
    \setbc{z\in L_{tb}^{\infty}(\mathcal{L};\operatorname{Curve}(\bbR^{2}))}{L(z)<\infty}, 
\]
thus naturally, we need to derive an a priori estimate for $L$.

However, note that the $H^{2}$-norm of a parameterization of a curve $\gamma$
does not need to be bounded by $\mathcal{K}_{2}(\gamma)$ if the parameterization is
not of constant-speed.
Since the contour equation we will consider does not enforce the solution to retain
this condition along its evolution, the control on the functional $L$ does not give us
the control on the $H^{2}$-norms of the parameterizations. This means that, when we attempt to
solve the contour equation by solving a sequence of mollified equations,
the a priori estimate for $L$ does not guarantee
a uniform bound on the $H^{2}$-norms of the solutions to these equations,
so in principle the existence times may go to zero as the mollification parameter
goes to zero. Nevertheless, we can always continue the solution in
$X$ as long as $L$ remains bounded, by first reparameterizing the curves
and then solving the equation again with those reparameterizations as new initial data.
The total existence time of the solution in $X$ obtained in this way depends only
on the evolution of $L$, thus the a priori estimate
we will obtain provides a lower bound to this existence time.
Therefore, we can send the mollification parameter to zero and obtain a solution in $X$
even if ``the non-mollified contour equation'' may not admit a solution.

To make this idea precise, we formulate the mollified contour equation on
the Banach space $E$ constructed in Appendix~\ref{SC}.
Consider the Hilbert space $H^{2}(\bbT;\bbR^{2})$,
and let $\mathscr{T}$ be the locally convex topology on $H^{2}(\bbT;\bbR^{2})$ given by
$\norm{\,\cdot\,}_{C^{1}}$ together with the family of seminorms
\[
    f\mapsto \abs{\int_{\bbT}g\cdot \partial_{\xi}^{2}f}
\]
for each $g\in L^{2}(\bbT;\bbR^{n})$.
Let $\bar{\mathcal{L}}$ be the Gelfand spectrum of $L^{\infty}(\mathcal{L};\bbC)$, then
$E$ is defined to be the space
$C\left(\bar{\mathcal{L}};(H^{2}(\bbT;\bbR^{2}),\mathscr{T})\right)$
of functions from $\bar{\mathcal{L}}$ into $H^{2}(\bbT;\bbR^{2})$ that are continuous
with respect to this locally convex topology $\mathscr{T}$.
(Note that an element in $E$ may not be continuous with respect to the norm topology.)
Then as noted in Appendix~\ref{SC}, this $E$
becomes a Banach space by the norm
\[
    \norm{\tilde{z}}_{E} \coloneqq \sup_{\bar{\lambda}\in\bar{\mathcal{L}}}
    \norm{\tilde{z}(\bar{\lambda})}_{H^{2}}.
\]
By composing with the chain of natural maps
$(H^{2}(\bbT;\bbR^{2}),\mathscr{T})\to C(\bbT;\bbR^{2})\to \operatorname{Curve}(\bbR^{2})$,
we get a map
\[
    \bar{\pi}\colon E\to C(\bar{\mathcal{L}};\operatorname{Curve}(\bbR^{2})),
\]
and further composing with the isometry
$L_{tb}^{\infty}(\mathcal{L};\operatorname{Curve}(\bbR^{2}))
\cong C(\bar{\mathcal{L}};\operatorname{Curve}(\bbR^{2}))$ gives a map
\[
    \pi\colon E\to L_{tb}^{\infty}(\mathcal{L};\operatorname{Curve}(\bbR^{2})).
\]
In this way, each $\tilde{z}\in E$ can be considered as a representative of
the element $\pi(\tilde{z})$ in $L_{tb}^{\infty}(\mathcal{L};\operatorname{Curve}(\bbR^{2}))$.
More specifically, $\tilde{z}$ is a \emph{joint parameterization} of $z$.
Conversely, Proposition~\ref{PC.9} says that whenever
$z\in L_{tb}^{\infty}(\mathcal{L};\operatorname{Curve}(\bbR^{2}))$ satisfies
$\norm{\lambda\mapsto\mathcal{K}_{2}(z)}_{L^{\infty}}<\infty$,
there exists a joint parameterization $\tilde{z}\in E$ of $z$ such that
each $\tilde{z}(\bar{\lambda})$ is a constant-speed parameterization.

Let $\psi\colon\bbR\to\bbR$ be a smooth even function such that $0\leq\psi\leq 1$,
$\psi\equiv 1$ outside $[-1,1]$, and $\lim_{r\to 0}\frac{\psi(r)}{r^{N}} = 0$
for all $N\in\bbZ_{\geq 0}$. Define $K\colon x\mapsto \frac{1}{\abs{x}^{2\alpha}}$,
and for each $\epsilon>0$, define
$K_{\epsilon}\colon x\mapsto \psi(\abs{x}^{2}/\epsilon^{2})K(x)$.
Then for $z\in L_{tb}^{\infty}(\mathcal{L};\operatorname{Curve}(\bbR^{2}))$
with $\norm{\lambda\mapsto\ell(z(\lambda))}_{L^{\infty}} < \infty$,
we consider the mollified velocity field
\[
    u_{\epsilon}(z) \colon x \mapsto
    - \int_{\mathcal{L}}\int_{\ell(z(\lambda))\bbT}
    K_{\epsilon}(x - z(\lambda,s))
    \partial_{s}z(\lambda,s)
    \,ds\,d\lambda.
\]
Clearly, $u_{\epsilon}(z)$ is infinitely differentiable and all of its derivatives
are bounded. If $v\colon\bbR^{2}\to\bbR^{2}$ is infinitely differentiable and all of its
derivatives are bounded, then it can be easily seen that the map
$\gamma\mapsto v\circ\gamma$ is a continuous self-map on
$(H^{2}(\bbT;\bbR^{2}),\mathscr{T})$. Therefore, for given $\tilde{z}\in E$,
let $F_{\epsilon}(\tilde{z})$ be the function
\[
    F_{\epsilon}(\tilde{z})\colon \bar{\lambda} \mapsto
    u_{\epsilon}(\tilde{z}) \circ \tilde{z}(\bar{\lambda}),
\]
then $F_{\epsilon}(\tilde{z})\in E$ since $F_{\epsilon}(\tilde{z})$
is the composition of $\tilde{z}$ with the continuous function
$\gamma\mapsto u_{\epsilon}(\tilde{z})\circ\gamma$.
Furthermore, $F_{\epsilon}\colon E\to E$ is Lipschitz continuous on
a ball of radius $M$ where the Lipschitz constant depends on $M$.
Therefore, for any $\tilde{z}_{0}\in E$, the differential equation
\begin{equation}\label{6.1}
    \tilde{z}_{\epsilon}'(t) = F_{\epsilon}(\tilde{z}_{\epsilon}(t))
\end{equation}
with the initial condition $\tilde{z}_{\epsilon}(0) = \tilde{z}_{0}$
always has a unique maximal solution in $E$.

Next, we show that, for any initial data
$z_{0}\in L_{tb}^{\infty}(\mathcal{L};\operatorname{Curve}(\bbR^{2}))$,
any solution to \eqref{6.1} whose initial data
is a joint parameterization of $z_{0}$ gives the same function when projected down to
$L_{tb}^{\infty}(\mathcal{L};\operatorname{Curve}(\bbR^{2}))$.

\begin{lemma}\label{L6.1}
    For any $x_{1},x_{2}\in\bbR^{2}$ and
    $C^{1}$-curves $\gamma_{i}\colon\bbT\to\bbR^{2}$, $i=1,2$,
    \begin{align*}
        &\abs{\int_{\bbT}K_{\epsilon}(x_{1} - \gamma_{1}(\xi_{1}))
        \partial_{\xi}\gamma_{1}(\xi_{1})\,d\xi
        - \int_{\bbT}K_{\epsilon}(x_{2} - \gamma_{2}(\xi_{2}))
        \partial_{\xi}\gamma_{2}(\xi_{2})\,d\xi}
        \\&\quad\quad\quad
        \leq \norm{DK_{\epsilon}}_{C^{0}}
        \left(\ell(\gamma_{1}) + \ell(\gamma_{2})\right)
        \left(
            \abs{x_{1} - x_{2}}
            + d_{\mathrm{F}}(\gamma_{1},\gamma_{2})
        \right).
    \end{align*}
\end{lemma}

\begin{proof}
    For any orientation-preserving diffeomorphism $\phi\colon\bbT\to\bbT$,
    integration by parts gives
    \begin{align*}
        &\abs{\int_{\bbT}K_{\epsilon}(x_{1} - \gamma_{1}(\xi))
        \partial_{\xi}\gamma_{1}(\xi)\,d\xi
        - \int_{\bbT}K_{\epsilon}(x_{2} - \gamma_{2}(\xi))
        \partial_{\xi}\gamma_{2}(\xi)\,d\xi}
        \\&\quad\quad\quad\leq
        \abs{\int_{\bbT}
        \left(
            DK_{\epsilon}(x_{1} - \gamma_{1}\circ\phi(\xi))
            \cdot \partial_{\xi}(\gamma_{1}\circ\phi)(\xi)
        \right)(\gamma_{1}\circ\phi(\xi) - \gamma_{2}(\xi))\,d\xi}
        \\&\quad\quad\quad\quad\quad\quad+
        \abs{\int_{\bbT}\left(
            K_{\epsilon}(x_{1} - \gamma_{1}\circ\phi(\xi))
            - K_{\epsilon}(x_{2} - \gamma_{2}(\xi))
        \right)\partial_{\xi}\gamma_{2}(\xi)\,d\xi}
        \\&\quad\quad\quad\leq
        \norm{DK_{\epsilon}}_{C^{0}}
        \norm{\gamma_{1}\circ\phi - \gamma_{2}}_{C^{0}}
        \ell(\gamma_{1})
        + \norm{DK_{\epsilon}}_{C^{0}}\left(
            \abs{x_{1} - x_{2}}
            + \norm{\gamma_{1}\circ\phi - \gamma_{2}}_{C^{0}}
        \right)\ell(\gamma_{2}),
    \end{align*}
    thus taking the infimum over $\phi$ and applying
    Lemma~\ref{LA.12} gives the desired conclusion.
\end{proof}

\begin{lemma}\label{L6.2}
    Let $\tilde{z}_{\epsilon,1}\colon I_{1}\to E$,
    $\tilde{z}_{\epsilon,2}\colon I_{2}\to E$ be unique solutions to
    \eqref{6.1} with initial data
    $\tilde{z}_{0,1}\in E$ and $\tilde{z}_{0,2}\in E$, respectively.
    Then there exists a constant $C=C(\epsilon,\tilde{z}_{0,1},\tilde{z}_{0,2})>0$ such that
    \[
        d_{\mathrm{F},\infty}(
            \pi(\tilde{z}_{\epsilon,1}(t)),
            \pi(\tilde{z}_{\epsilon,2}(t))
        )
        \leq e^{C\abs{t}}
        d_{\mathrm{F},\infty}(\pi(\tilde{z}_{0,1}), \pi(\tilde{z}_{0,2}))
    \]
    holds for all $\abs{t}<C^{-1}$.
\end{lemma}

\begin{proof}
    Let us denote $z_{\epsilon,i}(t)\coloneqq\pi(\tilde{z}_{\epsilon,i}(t))$
    for each $t\in I_{i}$, $i=1,2$. Since $\tilde{z}_{\epsilon,i}\colon I_{i}\to E$ is
    continuous, there exists a constant $M=M(\epsilon,\tilde{z}_{0,1},\tilde{z}_{0,2})\geq 0$
    such that
    \[
        \ell(\tilde{z}_{\epsilon,1}(t,\bar{\lambda}))
        + \ell(\tilde{z}_{\epsilon,2}(t,\bar{\lambda}))
        \leq \norm{\tilde{z}_{\epsilon,1}(t)}_{E}
        + \norm{\tilde{z}_{\epsilon,2}(t)}_{E}
        < M
    \]
    holds for all $\bar{\lambda}\in\bar{\mathcal{L}}$ and $\abs{t}<M^{-1}$.
    Fix such $t$ and any $\bar{\lambda}\in\bar{\mathcal{L}}$.
    Since the evaluation at $\bar{\lambda}$ is a continuous map $E\to H^{2}(\bbT;\bbR^{2})$,
    \eqref{6.1} gives
    \[
        \tilde{z}_{\epsilon,i}(t+h,\bar{\lambda},\xi)
        = \tilde{z}_{\epsilon,i}(t,\bar{\lambda},\xi)
        + \int_{t}^{t+h}u_{\epsilon}(\tilde{z}_{\epsilon,i}(\tau))
        \left(\tilde{z}_{\epsilon,i}(\tau,\bar{\lambda},\xi)\right)d\tau
    \]
    for all $i=1,2$, $\xi\in\bbT$, and small enough $h$.
    Then Lemma~\ref{L6.1} shows that
    there exists a constant $C=C(\alpha,\abs{\mathcal{L}},\epsilon)\geq 0$
    depending only on $\alpha$, $\abs{\mathcal{L}}$ and $\epsilon$ such that
    for any orientation-preserving homeomorphism $\phi\colon\bbT\to\bbT$ and
    small enough $h>0$,
    \begin{align*}
        &d_{\mathrm{F}}\left(z_{\epsilon,1}(t+h,\bar{\lambda}),
        z_{\epsilon,2}(t+h,\bar{\lambda})\right) \\
        &\quad\quad\quad\quad \leq
        \norm{\tilde{z}_{\epsilon,1}(t,\bar{\lambda})
        - \tilde{z}_{\epsilon,2}(t,\bar{\lambda})\circ\phi}_{C^{0}}
        \\&\quad\quad\quad\quad\quad
        + MC\int_{t}^{t+h}
        \left(
            \norm{\tilde{z}_{\epsilon,1}(\tau,\bar{\lambda})
            - \tilde{z}_{\epsilon,2}(\tau,\bar{\lambda})\circ\phi}_{C^{0}}
            + d_{\mathrm{F},\infty}\left(z_{\epsilon,1}(\tau),
            z_{\epsilon,2}(\tau)\right)
        \right)
        d\tau \\
        &\quad\quad\quad\quad \leq
        (1 + hMC)
        \norm{\tilde{z}_{\epsilon,1}(t,\bar{\lambda})
        - \tilde{z}_{\epsilon,2}(t,\bar{\lambda})\circ\phi}_{C^{0}}
        \\&\quad\quad\quad\quad\quad
        + MC\int_{t}^{t+h}
        d_{\mathrm{F},\infty}(z_{\epsilon,1}(\tau),z_{\epsilon,2}(\tau))
        + \sum_{i=1}^{2}\norm{\tilde{z}_{\epsilon,i}(\tau,\bar{\lambda})
        - \tilde{z}_{\epsilon,i}(t,\bar{\lambda})}_{C^{0}}d\tau
    \end{align*}
    holds. Hence, taking the infimum over $\phi$ and then
    supremum over $\bar{\lambda}$ shows
    \begin{align*}
        &d_{\mathrm{F},\infty}\left(z_{\epsilon,1}(t+h),z_{\epsilon,2}(t+h)\right)
        \\&\quad\quad\quad
        \leq (1+hMC)
        d_{\mathrm{F},\infty}\left(z_{\epsilon,1}(t),z_{\epsilon,2}(t)\right)
        \\&\quad\quad\quad\quad\quad\quad
        + MC\int_{t}^{t+h}
        d_{F,\infty}(z_{\epsilon,1}(\tau),z_{\epsilon,2}(\tau))
        + \sum_{i=1}^{2}\norm{\tilde{z}_{\epsilon,i}(\tau)
        - \tilde{z}_{\epsilon,i}(t)}_{E}d\tau.
    \end{align*}
    Since the integrand on the right-hand side is continuous in $\tau$, we get
    \[
        \partial_{t}^{+}d_{\mathrm{F},\infty}\left(
            z_{\epsilon,1}(t),z_{\epsilon,2}(t)
        \right)
        \leq 2MC
        d_{\mathrm{F},\infty}\left(
            z_{\epsilon,1}(t),z_{\epsilon,2}(t)
        \right).
    \]
    By repeating a similar argument for $h<0$, we obtain the desired conclusion.
\end{proof}

\begin{corollary}\label{C6.3}
    Let $\tilde{z}_{\epsilon,1}\colon I_{1}\to E$,
    $\tilde{z}_{\epsilon,2}\colon I_{2}\to E$ be unique solutions to
    \eqref{6.1} with initial data
    $\tilde{z}_{0,1}\in E$ and $\tilde{z}_{0,2}\in E$, respectively,
    such that $\pi(\tilde{z}_{0,1}) = \pi(\tilde{z}_{0,2})$.
    Then $\pi(\tilde{z}_{\epsilon,1}(t)) = \pi(\tilde{z}_{\epsilon,2}(t))$
    holds for all $t\in I_{1}\cap I_{2}$.
\end{corollary}

\begin{proof}
    Follows directly from Lemma~\ref{L6.2}
    and continuity of $d_{\mathrm{F},\infty}(\pi(\tilde{z}_{\epsilon,1}(t)),
    \pi(\tilde{z}_{\epsilon,2}(t)))$.
\end{proof}


\subsection{The a priori estimate}

Now, let an initial data $z_{0}\in L_{tb}^{\infty}(\mathcal{L};\operatorname{Curve}(\bbR^{2}))$
with $L(z_{0}) < \infty$ be given, and let
$\bar{z}_{0}\in C(\bar{\mathcal{L}};\operatorname{Curve}(\bbR^{2}))$
be the corresponding function in $C(\bar{\mathcal{L}};\operatorname{Curve}(\bbR^{2}))$. Then
using Proposition~\ref{PC.9}, we can find a representative $\tilde{z}_{0}\in E$
of $z_{0}$ where each $\tilde{z}_{0}(\bar{\lambda})$ is a constant-speed parameterization
of $\bar{z}_{0}(\bar{\lambda})$. Let $\tilde{z}_{\epsilon}\colon I\to E$ be the
unique maximal solution to \eqref{6.1} with $\tilde{z}_{0}$
as its initial data. Let $z_{\epsilon}(t)$, $\bar{z}_{\epsilon}(t)$ be the images of
$\tilde{z}_{\epsilon}(t)$ by the natural maps $\pi$, $\bar{\pi}$, respectively.
We now derive an a priori estimate on $L(z_{\epsilon}(t))$ independent of $\epsilon$.

By Corollary~\ref{CC.8}, we have
\begin{align*}
    L(z_{\epsilon}(t))
    &= \sup_{\bar{\lambda}\in\bar{\mathcal{L}}}
    \norm{\bar{z}_{\epsilon}(t,\bar{\lambda})}_{C^{0}}
    + \sup_{\bar{\lambda}\in\bar{\mathcal{L}}}
    \mathcal{K}_{2}\left(\bar{z}_{\epsilon}(t,\bar{\lambda})\right)
    + \sup_{\bar{\lambda}\in\bar{\mathcal{L}}}
    \int_{\bar{\mathcal{L}}}\frac{d\bar{\lambda}'}
    {\Delta(\bar{z}_{\epsilon}(t,\bar{\lambda}),
    \bar{z}_{\epsilon}(t,\bar{\lambda}'))^{2\alpha}} + 1.
\end{align*}
Note that each term in the above is of the form
$\sup_{\bar{\lambda}\in\bar{\mathcal{L}}}g(t,\bar{\lambda})$.
We will first estimate $\partial_{t}^{+}g(t,\bar{\lambda})$ for a fixed $\bar{\lambda}$ by
$G(L(z_{\epsilon}(t)))$ for some continuous increasing function $G$.
Since we know that each $g(t,\bar{\lambda})$ and $L(z_{\epsilon}(t))$ are
lower semicontinuous in $t$ (Corollary~\ref{CB.4},
Proposition~\ref{PC.7} and
Corollary~\ref{CC.8}), this gives the bound
\[
    g(t+h,\bar{\lambda}) \leq g(t,\bar{\lambda}) + \int_{t}^{t+h}
    G(L(z_{\epsilon}(\tau))) d\tau
\]
for each $\bar{\lambda}$. Hence, we get
\[
    \sup_{\bar{\lambda}\in\bar{\mathcal{L}}}g(t+h,\bar{\lambda})
    \leq \sup_{\bar{\lambda}\in\bar{\mathcal{L}}}g(t,\bar{\lambda})
    + \int_{t}^{t+h} G(L(z_{\epsilon}(\tau))) d\tau,
\]
so adding all $g$'s gives us a sufficient bound to run a Gr\"{o}nwall argument.

\begin{proposition}\label{P6.4}
    There exists a constant $C=C(\alpha,\abs{\mathcal{L}})\geq 0$ depending only on
    $\alpha$ and $\abs{\mathcal{L}}$ such that
    \[
        \partial_{t}^{+}\norm{\bar{z}_{\epsilon}(t,\bar{\lambda})}_{C^{0}}
        \leq \norm{u_{\epsilon}(z_{\epsilon}(t))}_{C^{0}}
        \leq CL(z_{\epsilon}(t))^{4}
    \]
    holds for any $\bar{\lambda}\in\bar{\mathcal{L}}$ and $t\in I_{\epsilon}$.
\end{proposition}

\begin{proof}
    Since the evaluation at $\bar{\lambda}\in\bar{\mathcal{L}}$ is a continuous
    map $E\to H^{2}(\bbT;\bbR^{2})$, for $t,t+h\in I_{\epsilon}$, we have
    \[
        \tilde{z}_{\epsilon}(t+h,\bar{\lambda})
        = \tilde{z}_{\epsilon}(t,\bar{\lambda})
        + \int_{t}^{t+h}F_{\epsilon}(\tilde{z}_{\epsilon}(\tau))(\bar{\lambda})\,d\tau
    \]
    where the integral is with respect to the $H^{2}$-norm.
    Then by continuity of $F_{\epsilon}\colon E\to E$,
    $\tilde{z}_{\epsilon}\colon I_{\epsilon}\to E$, and
    $\norm{\,\cdot\,}_{C^{0}}\colon H^{2}(\bbT;\bbR^{2})\to[0,\infty)$, we obtain
    \[
        \partial_{t}^{+}\norm{\tilde{z}_{\epsilon}(t,\bar{\lambda})}_{C^{0}}
        \leq \norm{F_{\epsilon}(\tilde{z}_{\epsilon}(t))(\bar{\lambda})}_{C^{0}}
        \leq \norm{u_{\epsilon}(z_{\epsilon}(t))}_{C^{0}}.
    \]
    Then the result follows by Lemma~\ref{P4.2}
    (with $K$ replaced by $K_{\epsilon}$).
\end{proof}

Therefore, we obtain:

\begin{corollary}\label{C6.5}
    There exists a constant $C=C(\alpha,\abs{\mathcal{L}})\geq 0$
    depending only on $\alpha$ and $\abs{\mathcal{L}}$ such that
    \[
        \sup_{\bar{\lambda}\in\bar{\mathcal{L}}}
        \norm{\bar{z}_{\epsilon}(t+h,\bar{\lambda})}_{C^{0}}
        \leq
        \sup_{\bar{\lambda}\in\bar{\mathcal{L}}}
        \norm{\bar{z}_{\epsilon}(t+h,\bar{\lambda})}_{C^{0}}
        + C\int_{t}^{t+h}L(z_{\epsilon}(\tau))^{4}\,d\tau
    \]
    holds for each $t,t+h\in I_{\epsilon}$, $h>0$.
\end{corollary}

Next, we bound $\sup_{\bar{\lambda}\in\bar{\mathcal{L}}}
\mathcal{K}_{2}\left(\bar{z}_{\epsilon}(t,\bar{\lambda})\right)$.

\begin{lemma}\label{L6.6}
    There exists a constant $C=C(\epsilon,\tilde{z}_{0})>0$ such that
    \begin{gather*}
        e^{-C\abs{t}}
        \ell(\bar{z}_{0}(\bar{\lambda}))
        \leq \abs{\partial_{\xi}\tilde{z}_{\epsilon}(t,\bar{\lambda},\xi)}
        \leq e^{C\abs{t}}
        \ell(\bar{z}_{0}(\bar{\lambda})), \\
        e^{-C\abs{t}}
        \norm{\tilde{z}_{0}(\bar{\lambda})}_{\dot{H}^{2}}
        \leq \norm{\tilde{z}_{\epsilon}(t,\bar{\lambda})}_{\dot{H}^{2}}
        \leq e^{C\abs{t}}
        \norm{\tilde{z}_{0}(\bar{\lambda})}_{\dot{H}^{2}}
    \end{gather*}
    hold for all $t$ with $\abs{t}<C^{-1}$,
    $\bar{\lambda}\in\bar{\mathcal{L}}$, and $\xi\in\bbT$.
\end{lemma}

\begin{proof}
    By continuity of the involved operations, we have
    \[
        \partial_{\xi}\tilde{z}_{\epsilon}(t+h,\bar{\lambda},\xi)
        = \partial_{\xi}\tilde{z}_{\epsilon}(t,\bar{\lambda},\xi)
        + \int_{t}^{t+h}D(u_{\epsilon}(z_{\epsilon}(\tau)))
        (\tilde{z}_{\epsilon}(\tau,\bar{\lambda},\xi))
        \cdot\partial_{\xi}\tilde{z}_{\epsilon}(\tau,\bar{\lambda},\xi)\,d\tau
    \]
    for all $t,t+h\in I_{\epsilon}$, $\bar{\lambda}\in\bar{\mathcal{L}}$, and
    $\xi\in\bbT$. The integrand is a continuous function of $\tau$, so
    \[
        \partial_{t}\partial_{\xi}\tilde{z}_{\epsilon}(t,\bar{\lambda},\xi)
        = D(u_{\epsilon}(z_{\epsilon}(t)))
        (\tilde{z}(t,\bar{\lambda},\xi))
        \cdot \partial_{\xi}\tilde{z}_{\epsilon}(t,\bar{\lambda},\xi).
    \]
    Clearly,
    \[
        \norm{D(u_{\epsilon}(z_{\epsilon}(t)))}_{C^{0}}
        \leq \norm{DK_{\epsilon}}_{C^{0}}
        \int_{\mathcal{L}}\ell(z_{\epsilon}(t,\lambda))\,d\lambda
        \leq \norm{DK_{\epsilon}}_{C^{0}}
        \abs{\mathcal{L}}
        \norm{\tilde{z}_{\epsilon}(t)}_{E}.
    \]
    Since $\norm{\tilde{z}_{\epsilon}(t)}_{E}$ varies continuously,
    there exists $M\geq 0$ such that $\norm{\tilde{z}_{\epsilon}(t)}_{E}\leq M$
    for all $t$ close enough to $0$. Then the claim about
    $\abs{\partial_{\xi}\tilde{z}_{\epsilon}(t,\bar{\lambda},\xi)}$ follows.

    Similarly, we have
    \begin{align*}
        &\abs{\norm{\partial_{\xi}^{2}\tilde{z}_{\epsilon}(t+h,\bar{\lambda})}_{L^{2}}
        - \norm{\partial_{\xi}^{2}\tilde{z}_{\epsilon}(t,\bar{\lambda})}_{L^{2}}}
        \\&\quad\quad
        \leq \abs{\int _{t}^{t+h}\norm{D^{2}(u_{\epsilon}(z_{\epsilon}(\tau)))}_{C^{0}}
        \norm{\partial_{\xi}\tilde{z}_{\epsilon}(\tau,\bar{\lambda})}_{L^{4}}^{2}
        + \norm{D(u_{\epsilon}(z_{\epsilon}(\tau)))}_{C^{0}}
        \norm{\partial_{\xi}^{2}\tilde{z}_{\epsilon}(\tau,\bar{\lambda})}_{L^{2}}\,d\tau}.
    \end{align*}
    Using integration by parts and H\"{o}lder's inequality, one can show
    \[
        \norm{\partial_{\xi}\tilde{z}_{\epsilon}(\tau,\bar{\lambda})}_{L^{4}}^{2}
        \leq \sqrt{3}\norm{\tilde{z}_{\epsilon}(\tau,\bar{\lambda})}_{C^{0}}
        \norm{\partial_{\xi}^{2}\tilde{z}_{\epsilon}(\tau,\bar{\lambda})}_{L^{2}}
        \leq \sqrt{3}M\norm{\partial_{\xi}^{2}\tilde{z}_{\epsilon}(\tau,\bar{\lambda})}_{L^{2}}
    \]
    for $\tau$ close enough to $0$. Hence,
    \[
        \limsup_{h\to 0}\abs{
            \frac{\norm{\partial_{\xi}^{2}\tilde{z}_{\epsilon}(t+h,\bar{\lambda})}_{L^{2}}
            - \norm{\partial_{\xi}^{2}\tilde{z}_{\epsilon}(t,\bar{\lambda})}_{L^{2}}}{h}
        }
        \leq C\norm{\partial_{\xi}^{2}\tilde{z}_{\epsilon}(t,\bar{\lambda})}_{L^{2}}
    \]
    for some constant $C\geq 0$ depending on
    $\norm{DK_{\epsilon}}_{C^{0}}$, $\norm{D^{2}K_{\epsilon}}_{C^{0}}$,
    $\abs{\mathcal{L}}$, and $M$. Hence, the claim about
    $\norm{\tilde{z}_{\epsilon}(t,\bar{\lambda})}_{\dot{H}^{2}}$ follows.
\end{proof}

For notational simplicity, for each $\bar{\lambda},\bar{\lambda}'\in\bar{\mathcal{L}}$ and
$s\in\bbR$, let us write
\begin{itemize}
    \item $u_{\epsilon}(\bar{\lambda},s)
    \coloneqq u_{\epsilon}(z_{0})(\bar{z}_{0}(\bar{\lambda},s))$,
    \item $\ell(\bar{\lambda})
    \coloneqq \ell(\bar{z}_{0}(\bar{\lambda}))$,
    \item $\mathcal{K}_{2}(\bar{\lambda})
    \coloneqq \mathcal{K}_{2}(\bar{z}_{0}(\bar{\lambda}))$,
    \item $\mathbf{T}(\bar{\lambda},s)
    \coloneqq \partial_{s}\bar{z}_{0}(\bar{\lambda},s)$,
    \item $\mathbf{N}(\bar{\lambda},s)
    \coloneqq -\mathbf{T}(\bar{\lambda},s)^{\perp}$,
    \item $\kappa(\bar{\lambda},s)
    \coloneqq -\partial_{s}^{2}\bar{z}_{0}(\bar{\lambda},s)
    \cdot \mathbf{N}(\bar{\lambda},s)$, and
    \item $\Delta(\bar{\lambda},\bar{\lambda}')
    \coloneqq \Delta(\bar{z}_{0}(\bar{\lambda}), \bar{z}_{0}(\bar{\lambda}'))$,
\end{itemize}
where $\bar{z}_{0}(\,\cdot\,,\,\cdot\,)$ is a jointly continuous
arclength parameterization of $z_{0}$ we obtain from
Proposition~\ref{PC.9}.
Note that in this notation, we have
\[
    \partial_{s}\mathbf{T}(\bar{\lambda},s)
    = -\kappa(\bar{\lambda},s)\mathbf{N}(\bar{\lambda},s)
    \quad\textrm{and}\quad
    \partial_{s}\mathbf{N}(\bar{\lambda},s)
    = \kappa(\bar{\lambda},s)\mathbf{T}(\bar{\lambda},s).
\]

\begin{proposition}\label{P6.7}
    For any $\bar{\lambda}\in\bar{\mathcal{L}}$,
    \begin{equation}\label{6.2}
    \begin{aligned}
        \left.\partial_{t}\mathcal{K}_{2}(\bar{z}_{\epsilon}(t,\bar{\lambda}))
        \right|_{t=0} &=
        -3\int_{\ell(\bar{\lambda})\bbT}
        \kappa(\bar{\lambda},s)^{2}
        \left(\partial_{s}u_{\epsilon}(\bar{\lambda},s)\cdot
        \mathbf{T}(\bar{\lambda},s)\right)ds
        \\&\quad\quad\quad\quad
        - 2\int_{\ell(\bar{\lambda})\bbT}
        \kappa(\bar{\lambda},s)
        \left(\partial_{s}^{2}u_{\epsilon}(\bar{\lambda},s)\cdot
        \mathbf{N}(\bar{\lambda},s)\right)ds.
    \end{aligned}
    \end{equation}
    Furthermore, there exists a constant $\delta=\delta(\epsilon,\tilde{z}_{0})$
    such that $(-\delta,\delta)\subseteq I_{\epsilon}$ and
    \[
        \sup_{t\in (-\delta,\delta)}\sup_{\bar{\lambda}\in\bar{\mathcal{L}}}
        \mathcal{K}_{2}(\bar{z}_{\epsilon}(t,\bar{\lambda})) < \infty,
    \]
    and \eqref{6.2} continues to hold at any
    $t\in(-\delta,\delta)$ with $z_{0}$ replaced by $z_{\epsilon}(t)$.
\end{proposition}

\begin{proof}
    We first derive the formula for $t=0$. Fix $\bar{\lambda}\in\bar{\mathcal{L}}$.
    If $\ell(\bar{z}_{0}(\bar{\lambda})) = 0$, then
    Lemma~\ref{L6.6} shows
    $\mathcal{K}_{2}(\bar{z}_{\epsilon}(t,\bar{\lambda})) = 0$ for all small enough $t$,
    so suppose otherwise. Then again by
    Lemma~\ref{L6.6}, for small enough $t$,
    $\abs{\partial_{\xi}\tilde{z}_{\epsilon}(t,\bar{\lambda},\xi)}$
    is bounded below by a positive constant. Hence, the $L^{2}$-convergence
    $\frac{1}{t}\left(\partial_{\xi}^{2}\tilde{z}_{\epsilon}(t,\bar{\lambda})
    -\partial_{\xi}^{2}\tilde{z}_{0}(\bar{\lambda})\right)
    \to \partial_{\xi}^{2}F_{\epsilon}(\tilde{z}_{\epsilon}(t))(\bar{\lambda})$
    together with the uniform convergence
    $\frac{1}{t}\left(\partial_{\xi}\tilde{z}_{\epsilon}(t,\bar{\lambda})
    -\partial_{\xi}\tilde{z}_{0}(\bar{\lambda})\right)
    \to \partial_{\xi}F_{\epsilon}(\tilde{z}_{\epsilon}(t))(\bar{\lambda})$ show
    \begin{align*}
        \left.\partial_{t}\mathcal{K}_{2}(\bar{z}_{\epsilon}(t,\bar{\lambda}))\right|_{t=0}
        &=
        \partial_{t}\left(\int_{\bbT}
        \frac{\abs{
            \partial_{\xi}\tilde{z}_{\epsilon}(t,\bar{\lambda},\xi)
            \cdot\partial_{\xi}^{2}\tilde{z}_{\epsilon}(t,\bar{\lambda},\xi)^{\perp}
        }^{2}}
        {\abs{\partial_{\xi}\tilde{z}_{\epsilon}(t,\bar{\lambda},\xi)}^{5}}
        d\xi\right)_{t=0} \\
        &=
        \int_{\bbT}
        \frac{2\left(
            \partial_{\xi}F_{\epsilon}(\tilde{z}_{0})(\bar{\lambda},\xi)
            \cdot\partial_{\xi}^{2}\tilde{z}_{0}(\bar{\lambda},\xi)^{\perp}
        \right)
        \left(
            \partial_{\xi}\tilde{z}_{0}(\bar{\lambda},\xi)
            \cdot\partial_{\xi}^{2}\tilde{z}_{0}(\bar{\lambda},\xi)^{\perp}
        \right)}
        {\abs{\partial_{\xi}\tilde{z}_{0}(\bar{\lambda},\xi)}^{5}}
        d\xi \\
        &\quad\quad\quad\quad
        + \int_{\bbT}
        \frac{2\left(
            \partial_{\xi}\tilde{z}_{0}(\bar{\lambda},\xi)
            \cdot\partial_{\xi}^{2}F(\tilde{z}_{0})(\bar{\lambda},\xi)^{\perp}
        \right)
        \left(
            \partial_{\xi}\tilde{z}_{0}(\bar{\lambda},\xi)
            \cdot\partial_{\xi}^{2}\tilde{z}_{0}(\bar{\lambda},\xi)^{\perp}
        \right)}
        {\abs{\partial_{\xi}\tilde{z}_{0}(\bar{\lambda},\xi)}^{5}}
        d\xi \\
        &\quad\quad\quad\quad
        - \int_{\bbT}
        \frac{5\abs{\partial_{\xi}\tilde{z}_{0}(\bar{\lambda},\xi)\cdot
        \partial_{\xi}^{2}\tilde{z}_{0}(\bar{\lambda},\xi)^{\perp}}^{2}
        \left(
            \partial_{\xi}\tilde{z}_{0}(\bar{\lambda},\xi)
            \cdot\partial_{\xi}F_{\epsilon}(\tilde{z}_{0})(\bar{\lambda},\xi)
        \right)}
        {\abs{\partial_{\xi}\tilde{z}_{0}(\bar{\lambda},\xi)}^{7}}
        d\xi \\
        &=
        -3\int_{\ell(\bar{\lambda})\bbT}
        \kappa(\bar{\lambda},s)^{2}
        \left(\partial_{s}u_{\epsilon}(\bar{\lambda},s)\cdot
        \mathbf{T}(\bar{\lambda},s)\right)ds \\
        &\quad\quad\quad\quad
        - 2\int_{\ell(\bar{\lambda})\bbT}
        \kappa(\bar{\lambda},s)
        \left(\partial_{s}^{2}u_{\epsilon}(\bar{\lambda},s)\cdot
        \mathbf{N}(\bar{\lambda},s)\right)ds,
    \end{align*}
    where the last equality follows by reparameterizing all three integrals by
    the arclength, and then merging the first and the third integrals.

    To see that the formula continues to hold for all small enough $t$,
    note that Lemma~\ref{L6.6} shows
    \[
        \mathcal{K}_{2}(\tilde{z}_{\epsilon}(t,\bar{\lambda}))
        \leq \frac{\norm{\tilde{z}_{\epsilon}(t,\bar{\lambda})}_{\dot{H}^{2}}^{2}}
        {\min_{\xi\in\bbT}\abs{\partial_{\xi}\tilde{z}_{\epsilon}(t,\bar{\lambda},\xi)}^{3}}
        \leq e^{C(\epsilon,\tilde{z}_{0})\abs{t}}
        \frac{\norm{\tilde{z}_{0}(\bar{\lambda})}_{\dot{H}^{2}}}
        {\ell(\bar{z}_{0}(\bar{\lambda}))^{3}}
        = e^{C(\epsilon,\tilde{z}_{0})\abs{t}}
        \mathcal{K}_{2}(\bar{z}_{0}(\bar{\lambda}))
    \]
    whenever $\abs{t}<C(\epsilon,\tilde{z}_{0})^{-1}$, so
    $\sup_{\bar{\lambda}\in\bar{\mathcal{L}}}
    \mathcal{K}_{2}(\bar{z}_{\epsilon}(t,\bar{\lambda})) < \infty$ holds for such $t$.
    Fix such small $t_{0}$. Then we can apply Proposition~\ref{PC.9} to find
    a new constant-speed joint parameterization in $E$ of $z_{\epsilon}(t)$
    which we can use as a new initial data to \eqref{6.1}.
    By Corollary~\ref{C6.3},
    we know that the projection of the resulting solution
    onto $L_{tb}^{\infty}(\mathcal{L};\operatorname{Curve}(\bbR^{2}))$
    coincides with $z_{\epsilon}(t)$ on their common domain.
    Since the formula we derived for
    $\left.\partial_{t}\mathcal{K}_{2}(\bar{z}_{\epsilon}(t,\bar{\lambda}))\right|_{t=0}$
    only involves quantities associated to $z_{0}$
    but not depending on the choice of $\tilde{z}_{0}\in E$,
    the same argument applies so the formula still holds whenever
    $\abs{t}<C(\epsilon,\tilde{z}_{0})^{-1}$ with $z_{0}$ replaced by $z_{\epsilon}(t)$.
\end{proof}

Now we bound each term in \eqref{6.2}.

\begin{proposition}\label{P6.8}
    There exists a constant $C=C(\alpha)\geq 0$ depending only on $\alpha$ such that
    \begin{align*}
        -\int_{\ell(\bar{\lambda})\bbT}
        \kappa(\bar{\lambda},s)^{2}
        \left(
            \partial_{s}u_{\epsilon}(\bar{\lambda},s)
            \cdot \mathbf{T}(\bar{\lambda},s)
        \right)
        ds
        &\leq
        C\mathcal{K}_{2}(\bar{\lambda})
        \int_{\bar{\mathcal{L}}}
        \frac{\ell(\bar{\lambda}')
        \norm{\bar{z}_{0}(\bar{\lambda}')}_{\dot{C}^{1,\beta}}^{1/\beta}}
        {\Delta(\bar{\lambda},\bar{\lambda}')^{2\alpha}}
        \,d\bar{\lambda}'
        \leq CL(z_{0})^{5}
    \end{align*}
    holds for any $\bar{\lambda}\in\bar{\mathcal{L}}$ and $\beta\in(0,1]$.
\end{proposition}

\begin{proof}
    Since
    \begin{align*}
        \partial_{s}u_{\epsilon}(\bar{\lambda},s)
        &= D(u_{\epsilon}(z_{0}))(\bar{z}_{0}(\bar{\lambda},s))
        (\mathbf{T}(\bar{\lambda},s)) \\
        &= -\int_{\bar{\mathcal{L}}}\int_{\ell(\bar{\lambda}')\bbT}
        DK_{\epsilon}(\bar{z}_{0}(\bar{\lambda},s) - \bar{z}_{0}(\bar{\lambda}',s'))
        (\mathbf{T}(\bar{\lambda},s))
        \mathbf{T}(\bar{\lambda}',s')\,ds'\,d\bar{\lambda}'
    \end{align*}
    and $DK_{\epsilon}(x)(h)$ is equal to $\frac{x\cdot h}{\abs{x}^{2+2\alpha}}$
    times a smooth function which is bounded by a uniform constant independent to $\epsilon$,
    Lemma~\ref{L4.3} shows that
    \begin{align*}
        \abs{\partial_{s}u_{\epsilon}(\bar{\lambda},s)}
        &\leq C(\alpha)\int_{\bar{\mathcal{L}}}
        \frac{\ell(\bar{\lambda}')
        \norm{\bar{z}_{0}(\bar{\lambda}')}_{\dot{C}^{1,\beta}}^{1/\beta}}
        {\Delta(\bar{\lambda},\bar{\lambda}')^{2\alpha}}
        \,d\bar{\lambda}'
    \end{align*}
    holds for all $s\in\ell(\bar{\lambda})\bbT$ and $\beta\in(0,1]$, thus
    letting $\beta=1/2$ and applying the inequalities
    $\norm{\bar{z}_{0}(\bar{\lambda}')}_{\dot{C}^{1,1/2}}
    \leq \norm{\bar{z}_{0}(\bar{\lambda}')}_{\dot{H}^{2}}$ and
    $\ell(\bar{\lambda}') \leq \norm{\bar{z}_{0}(\bar{\lambda}')}_{C^{0}}^{2}
    \mathcal{K}_{2}(\bar{\lambda}')$ gives the claimed estimate.
\end{proof}

\begin{proposition}
    There exist constants $C=C(\alpha)\geq 0$ depending only on $\alpha$ such that
    \begin{align*}
        &-\int_{\ell(\bar{\lambda})\bbT}
        \kappa(\bar{\lambda},s)
        \left(
            \partial_{s}^{2}u_{\epsilon}(\bar{\lambda},s)
            \cdot \mathbf{N}(\bar{\lambda},s)
        \right)
        ds
        \\&\quad\quad \leq
        C\left(
            \mathcal{K}_{2}(\bar{\lambda})
            \int_{\bar{\mathcal{L}}}
            \frac{\ell(\bar{\lambda}')\mathcal{K}_{2}(\bar{\lambda}')}
            {\Delta(\bar{\lambda},\bar{\lambda}')^{2\alpha}}
            \,d\bar{\lambda}'
            + \ell(\bar{\lambda})^{1/2}\mathcal{K}_{2}(\bar{\lambda})
            \int_{\bar{\mathcal{L}}}
            \frac{\ell(\bar{\lambda}')^{1/2}\mathcal{K}_{2}(\bar{\lambda}')}
            {\Delta(\bar{\lambda},\bar{\lambda}')^{2\alpha}}
            \,d\bar{\lambda}'
        \right) \\
        &\quad\quad\leq CL(z_{0})^{6}
    \end{align*}
    holds for any $\bar{\lambda}\in\bar{\mathcal{L}}$.
\end{proposition}

\begin{proof}
    Note that
    \begin{align*}
        \partial_{s}^{2}u_{\epsilon}(\bar{\lambda},s)
        &= D^{2}(u_{\epsilon}(z_{0}))(\bar{z}_{0}(\bar{\lambda},s))
        (\mathbf{T}(\bar{\lambda},s),\mathbf{T}(\bar{\lambda},s))
        - \kappa(\bar{\lambda},s)
        D(u_{\epsilon}(z_{0}))(\bar{z}_{0}(\bar{\lambda},s))(\mathbf{N}(\bar{\lambda},s)) \\
        &= -\int_{\bar{\mathcal{L}}}\int_{\ell(\bar{\lambda}')\bbT}
        D^{2}K_{\epsilon}(\bar{z}_{0}(\bar{\lambda},s) - \bar{z}_{0}(\bar{\lambda}',s'))
        (\mathbf{T}(\bar{\lambda},s),\mathbf{T}(\bar{\lambda},s))
        \mathbf{T}(\bar{\lambda}',s')\,ds'
        \\&\quad\quad\quad
        + \kappa(\bar{\lambda},s)\int_{\bar{\mathcal{L}}}\int_{\ell(\bar{\lambda}')\bbT}
        DK_{\epsilon}(\bar{z}_{0}(\bar{\lambda},s) - \bar{z}_{0}(\bar{\lambda}',s'))
        (\mathbf{N}(\bar{\lambda},s))
        \mathbf{T}(\bar{\lambda},s')\,ds'.
    \end{align*}
    Hence, the integral we want to estimate is the sum of two terms:
    \begin{align*}
        G_{1} &\coloneqq
        \int_{\ell(\bar{\lambda})\bbT}\kappa(\bar{\lambda},s)
        \int_{\bar{\mathcal{L}}}\int_{\ell(\bar{\lambda}')\bbT}
        D^{2}K_{\epsilon}(\bar{z}_{0}(\bar{\lambda},s) - \bar{z}_{0}(\bar{\lambda}',s'))
        (\mathbf{T}(\bar{\lambda},s),\mathbf{T}(\bar{\lambda},s))
        \\&\quad\quad\quad\quad\quad\quad\quad\quad\quad\quad\quad\quad
        \cdot(\mathbf{T}(\bar{\lambda}',s')\cdot\mathbf{N}(\bar{\lambda},s))
        \,ds'\,d\bar{\lambda}'\,ds, \\
        G_{2} &\coloneqq
        -\int_{\ell(\bar{\lambda})\bbT}\kappa(\bar{\lambda},s)^{2}
        \int_{\bar{\mathcal{L}}}\int_{\ell(\bar{\lambda}')\bbT}
        DK_{\epsilon}(\bar{z}_{0}(\bar{\lambda},s) - \bar{z}_{0}(\bar{\lambda}',s'))
        (\mathbf{N}(\bar{\lambda},s))
        \\&\quad\quad\quad\quad\quad\quad\quad\quad\quad\quad\quad\quad
        \cdot(\mathbf{T}(\bar{\lambda}',s')\cdot\mathbf{N}(\bar{\lambda},s))
        \,ds'\,d\bar{\lambda}'\,ds.
    \end{align*}
    By the same argument as in Proposition~\ref{P6.8},
    we know
    \[
        G_{2} \leq C(\alpha)\mathcal{K}_{2}(\bar{\lambda})
        \int_{\bar{\mathcal{L}}}
        \frac{\ell(\bar{\lambda}')\mathcal{K}_{2}(\bar{\lambda}')}
        {\Delta(\bar{\lambda},\bar{\lambda}')^{2\alpha}}
        \,d\bar{\lambda}'.
    \]
    To estimate $G_{1}$, we decompose $\mathbf{T}(\bar{\lambda},s)$ into its
    $\mathbf{T}(\bar{\lambda}',s')$-component and
    $\mathbf{N}(\bar{\lambda}',s')$-component, so that $G_{1}$ is the sum of the
    following two terms:
    \begin{align*}
        G_{3} &\coloneqq
        \int_{\ell(\bar{\lambda})\bbT}\kappa(\bar{\lambda},s)
        \int_{\bar{\mathcal{L}}}\int_{\ell(\bar{\lambda}')\bbT}
        D^{2}K_{\epsilon}(\bar{z}_{0}(\bar{\lambda},s) - \bar{z}_{0}(\bar{\lambda}',s'))
        (\mathbf{T}(\bar{\lambda},s),\mathbf{T}(\bar{\lambda}',s'))
        \\&\quad\quad\quad\quad\quad\quad\quad\quad\quad\quad\quad\quad
        \cdot(\mathbf{T}(\bar{\lambda}',s')\cdot\mathbf{N}(\bar{\lambda},s))
        (\mathbf{T}(\bar{\lambda}',s')\cdot\mathbf{T}(\bar{\lambda},s))
        \,ds'\,d\bar{\lambda}'\,ds, \\
        G_{4} &\coloneqq
        -\int_{\ell(\bar{\lambda})\bbT}\kappa(\bar{\lambda},s)
        \int_{\bar{\mathcal{L}}}\int_{\ell(\bar{\lambda}')\bbT}
        D^{2}K_{\epsilon}(\bar{z}_{0}(\bar{\lambda},s) - \bar{z}_{0}(\bar{\lambda}',s'))
        (\mathbf{T}(\bar{\lambda},s),\mathbf{N}(\bar{\lambda}',s'))
        \\&\quad\quad\quad\quad\quad\quad\quad\quad\quad\quad\quad\quad
        \cdot(\mathbf{T}(\bar{\lambda}',s')\cdot\mathbf{N}(\bar{\lambda},s))^{2}
        \,ds'\,d\bar{\lambda}'\,ds.
    \end{align*}

    We estimate $G_{4}$ first. Note that there exists a universal constant $C\geq 0$ such that
    \[
        \abs{D^{2}K_{\epsilon}(\bar{z}_{0}(\bar{\lambda},s) - \bar{z}_{0}(\bar{\lambda}',s'))}
        \leq \frac{C}
        {\abs{\bar{z}_{0}(\bar{\lambda},s) - \bar{z}_{0}(\bar{\lambda}',s')}^{2+2\alpha}}.
    \]
    Hence, by Lemma~\ref{L3.4}, $G_{4}$ is bounded by a constant times the sum of the
    following two terms:
    \begin{align*}
        G_{5} &\coloneqq
        \int_{\bar{\mathcal{L}}}\int_{\ell(\bar{\lambda})\bbT}
        \abs{\kappa(\bar{\lambda},s)}\mathcal{M}\kappa(\bar{\lambda})(s)
        \int_{\ell(\bar{\lambda}')\bbT}
        \frac{ds'}
        {\abs{\bar{z}_{0}(\bar{\lambda},s) - \bar{z}_{0}(\bar{\lambda}',s')}^{1+2\alpha}}
        \,ds\,d\bar{\lambda}',\\
        G_{6} &\coloneqq
        \int_{\bar{\mathcal{L}}}\int_{\ell(\bar{\lambda})\bbT}\int_{\ell(\bar{\lambda}')\bbT}
        \frac{\abs{\kappa(\bar{\lambda},s)}\mathcal{M}\kappa(\bar{\lambda'})(s')}
        {\abs{\bar{z}_{0}(\bar{\lambda},s) - \bar{z}_{0}(\bar{\lambda}',s')}^{1+2\alpha}}
        ds'\,\,ds\,d\bar{\lambda}'.
    \end{align*}
    By maximal inequality (reference) and Lemma~\ref{L4.3}, we get
    \[
        G_{5} \leq C(\alpha)\mathcal{K}_{2}(\bar{\lambda})
        \int_{\bar{\mathcal{L}}}
        \frac{\ell(\bar{\lambda}')\mathcal{K}_{2}(\bar{\lambda}')}
        {\Delta(\bar{\lambda},\bar{\lambda}')^{2\alpha}}
        \,d\bar{\lambda}'.
    \]
    By H\"{o}lder's inequality, $G_{6}$ is bounded by
    the integral over $\bar{\lambda}'$ of the product of
    \begin{align*}
        \left(
            \int_{\ell(\bar{\lambda})\bbT}
            \kappa(\bar{\lambda},s)^{2}
            \int_{\ell(\bar{\lambda'})\bbT}
            \frac{ds'}
            {\abs{\bar{z}_{0}(\bar{\lambda},s) - \bar{z}_{0}(\bar{\lambda}',s')}^{1+2\alpha}}
            \,ds
        \right)^{1/2}
    \end{align*}
    and
    \begin{align*}
        \left(
            \int_{\ell(\bar{\lambda}')\bbT}
            \mathcal{M}\kappa(\bar{\lambda}')(s')^{2}
            \int_{\ell(\bar{\lambda})\bbT}
            \frac{ds}
            {\abs{\bar{z}_{0}(\bar{\lambda},s) - \bar{z}_{0}(\bar{\lambda}',s')}^{1+2\alpha}}
            \,ds'
        \right)^{1/2}.
    \end{align*}
    By Lemma~\ref{L4.3}, the inner integrals
    of the above are bounded by
    \[
        C(\alpha)\frac{\ell(\bar{\lambda}')\mathcal{K}_{2}(\bar{\lambda}')}
        {\Delta(\bar{\lambda},\bar{\lambda}')^{2\alpha}}
        \quad\textrm{and}\quad
        C(\alpha)\frac{\ell(\bar{\lambda})\mathcal{K}_{2}(\bar{\lambda})}
        {\Delta(\bar{\lambda},\bar{\lambda}')^{2\alpha}},
    \]
    respectively, thus we get
    \begin{align*}
        G_{6} \leq C(\alpha)
        \ell(\bar{\lambda})^{1/2}\mathcal{K}_{2}(\bar{\lambda})
        \int_{\bar{\mathcal{L}}}
        \frac{\ell(\bar{\lambda}')^{1/2}\mathcal{K}_{2}(\bar{\lambda}')}
        {\Delta(\bar{\lambda},\bar{\lambda}')^{2\alpha}}
        \,d\bar{\lambda}'.
    \end{align*}

    For $G_{3}$, since
    \begin{align*}
        &\partial_{s'}
        \left(
            DK_{\epsilon}(\bar{z}_{0}(\bar{\lambda},s) - \bar{z}_{0}(\bar{\lambda}',s'))
            (\mathbf{T}(\bar{\lambda},s))
            \cdot(\mathbf{T}(\bar{\lambda}',s')\cdot\mathbf{N}(\bar{\lambda},s))
            (\mathbf{T}(\bar{\lambda}',s')\cdot\mathbf{T}(\bar{\lambda},s))
        \right)
        \\&\quad\quad
        = -D^{2}K_{\epsilon}(\bar{z}_{0}(\bar{\lambda},s) - \bar{z}_{0}(\bar{\lambda}',s'))
        (\mathbf{T}(\bar{\lambda},s),\mathbf{T}(\bar{\lambda}',s'))
        \\&\quad\quad\quad\quad\quad\quad\quad
        \cdot(\mathbf{T}(\bar{\lambda}',s')\cdot\mathbf{N}(\bar{\lambda},s))
        (\mathbf{T}(\bar{\lambda}',s')\cdot\mathbf{T}(\bar{\lambda},s))
        \\&\quad\quad\quad\quad
        +\kappa(\bar{\lambda}',s')
        DK_{\epsilon}(\bar{z}_{0}(\bar{\lambda},s) - \bar{z}_{0}(\bar{\lambda}',s'))
        (\mathbf{T}(\bar{\lambda},s))
        \\&\quad\quad\quad\quad\quad\quad\quad
        \cdot
        \left(
            (\mathbf{T}(\bar{\lambda}',s')\cdot\mathbf{N}(\bar{\lambda},s))^{2}
            - (\mathbf{T}(\bar{\lambda}',s')\cdot\mathbf{T}(\bar{\lambda},s))^{2}
        \right),
    \end{align*}
    integration by parts shows
    \begin{align*}
        G_{3} &= \int_{\bar{\mathcal{L}}}
        \int_{\ell(\bar{\lambda})\bbT}\int_{\ell(\bar{\lambda}')\bbT}
        \kappa(\bar{\lambda},s)\kappa(\bar{\lambda}',s')
        DK_{\epsilon}(\bar{z}_{0}(\bar{\lambda},s) - \bar{z}_{0}(\bar{\lambda}',s'))
        (\mathbf{T}(\bar{\lambda},s))
        \\&\quad\quad\quad\quad\quad\quad\quad
        \cdot
        \left(
            (\mathbf{T}(\bar{\lambda}',s')\cdot\mathbf{N}(\bar{\lambda},s))^{2}
            - (\mathbf{T}(\bar{\lambda}',s')\cdot\mathbf{T}(\bar{\lambda},s))^{2}
        \right)
        ds'\,ds\,d\bar{\lambda}'.
    \end{align*}
    Then by the same way we estimated $G_{6}$, we obtain
    \[
        G_{3} \leq C(\alpha)
        \ell(\bar{\lambda})^{1/2}\mathcal{K}_{2}(\bar{\lambda})
        \int_{\bar{\mathcal{L}}}
        \frac{\ell(\bar{\lambda}')^{1/2}\mathcal{K}_{2}(\bar{\lambda}')}
        {\Delta(\bar{\lambda},\bar{\lambda}')^{2\alpha}}
        \,d\bar{\lambda}'.
    \]

    Aggregating the bounds for $G_{2}$, $G_{3}$, $G_{5}$, and $G_{6}$
    and then applying the inequality
    $\ell(\bar{\lambda}) \leq \norm{\bar{z}_{0}(\bar{\lambda})}_{C^{0}}^{2}
    \mathcal{K}_{2}(\bar{\lambda})$ gives the estimate we wanted to show.
\end{proof}

Therefore, we obtain:

\begin{corollary}\label{C6.10}
    There exist constants $C(\alpha)\geq 0$ depending only on $\alpha$
    and $\delta=\delta(\epsilon,\tilde{z}_{0})>0$ such that
    $(-\delta,\delta)\subseteq I_{\epsilon}$ and
    \[
        \sup_{\bar{\lambda}\in\bar{\mathcal{L}}}
        \mathcal{K}_{2}(\bar{z}_{\epsilon}(t+h,\bar{\lambda}))
        \leq
        \sup_{\bar{\lambda}\in\bar{\mathcal{L}}}
        \mathcal{K}_{2}(\bar{z}_{\epsilon}(t,\bar{\lambda}))
        + C(\alpha)
        \int_{t}^{t+h}L(z_{\epsilon}(\tau))^{6}\,d\tau
    \]
    holds for any $t\in I_{\epsilon}$ and $h>0$ with $t,t+h\in(-\delta,\delta)$.
\end{corollary}

Finally, we bound $\sup_{\bar{\lambda}\in\bar{\mathcal{L}}}
\int_{\bar{\mathcal{L}}}\frac{d\bar{\lambda}'}
{\Delta(\bar{z}_{\epsilon}(t,\bar{\lambda}),
\bar{z}_{\epsilon}(t,\bar{\lambda}'))^{2\alpha}}$.

\begin{lemma}\label{L6.11}
    For any $\bar{\lambda},\bar{\lambda}'\in\bar{\mathcal{L}}$,
    $\Delta(\bar{z}_{\epsilon}(t,\bar{\lambda}),\bar{z}_{\epsilon}(t,\bar{\lambda}'))$
    is a locally Lipschitz function of $t$, and for all $t\in I_{\epsilon}$ with
    $\Delta(\bar{z}_{\epsilon}(t,\bar{\lambda}),
    \bar{z}_{\epsilon}(t,\bar{\lambda}')) \neq 0$
    there exist $\xi\in\bbT$ and $\xi'\in\bbT'$ such that
    $\Delta(\bar{z}_{\epsilon}(t,\bar{\lambda}),
    \bar{z}_{\epsilon}(t,\bar{\lambda}'))
    = \abs{\tilde{z}_{\epsilon}(t,\bar{\lambda},\xi)
    - \tilde{z}_{\epsilon}(t,\bar{\lambda}',\xi')}$ and
    \begin{equation}\label{6.3}
    \begin{aligned}
        &\partial_{t+}
        \Delta(\bar{z}_{\epsilon}(t,\bar{\lambda}),\bar{z}_{\epsilon}(t,\bar{\lambda}'))
        \\&\quad\quad\quad\geq
        \frac{\tilde{z}_{\epsilon}(t,\bar{\lambda},\xi)
        - \tilde{z}_{\epsilon}(t,\bar{\lambda}',\xi')}
        {\abs{\tilde{z}_{\epsilon}(t,\bar{\lambda},\xi)
        - \tilde{z}_{\epsilon}(t,\bar{\lambda}',\xi')}}
        \cdot \left(
            u_{\epsilon}(z_{\epsilon}(t))
            \left(\tilde{z}_{\epsilon}(t,\bar{\lambda},\xi)\right)
            - u_{\epsilon}(z_{\epsilon}(t))
            \left(\tilde{z}_{\epsilon}(t,\bar{\lambda}',\xi')\right)
        \right).
    \end{aligned}
    \end{equation}
    Furthermore, there exist constants $C=C(\alpha)\geq 0$ depending only on
    $\alpha$ such that
    \begin{equation}\label{6.4}
    \begin{aligned}
        &\partial_{t+}
        \Delta(\bar{z}_{\epsilon}(t,\bar{\lambda}),\bar{z}_{\epsilon}(t,\bar{\lambda}'))
        \\&\quad\quad\quad\geq
        -C\Delta(\bar{z}_{\epsilon}(t,\bar{\lambda}),\bar{z}_{\epsilon}(t,\bar{\lambda}'))
        \\&\quad\quad\quad\quad\cdot
        \int_{\bar{\mathcal{L}}}
        \frac{\ell(\bar{z}_{\epsilon}(t,\bar{\lambda}''))
        \norm{\bar{z}_{\epsilon}(t,\bar{\lambda}'')}_{\dot{C}^{1,\beta}}^{1/\beta}}
        {\Delta(\bar{z}_{\epsilon}(t,\bar{\lambda}),
        \bar{z}_{\epsilon}(t,\bar{\lambda}''))^{2\alpha}}
        + \frac{\ell(\bar{z}_{\epsilon}(t,\bar{\lambda}''))
        \norm{\bar{z}_{\epsilon}(t,\bar{\lambda}'')}_{\dot{C}^{1,\beta}}^{1/\beta}}
        {\Delta(\bar{z}_{\epsilon}(t,\bar{\lambda}'),
        \bar{z}_{\epsilon}(t,\bar{\lambda}''))^{2\alpha}}
        \,d\bar{\lambda}''
        \\&\quad\quad\quad\geq
        -CL(z_{\epsilon}(t))^{5}
        \Delta(\bar{z}_{\epsilon}(t,\bar{\lambda}),\bar{z}_{\epsilon}(t,\bar{\lambda}'))
    \end{aligned}
    \end{equation}
    holds for any $\bar{\lambda},\bar{\lambda}'\in\bar{\mathcal{L}}$,
    $t\in I_{\epsilon}$, and $\beta\in(0,1]$.
\end{lemma}

\begin{proof}
    By repeating the argument given in \cite{ConsEsch}, one can easily see that
    $\Delta(\bar{z}_{\epsilon}(t,\bar{\lambda}),\bar{z}_{\epsilon}(t,\bar{\lambda}'))$
    is locally Lipschitz in $t$. For each $t$, take any
    $(\xi_{t},\xi_{t}')\in\bbT\times\bbT$ achieving
    $\Delta(\bar{z}_{\epsilon}(t,\bar{\lambda}), \bar{z}_{\epsilon}(t,\bar{\lambda}'))
    = \abs{\tilde{z}_{\epsilon}(t,\bar{\lambda},\xi_{t})
    - \tilde{z}_{\epsilon}(t,\bar{\lambda}',\xi_{t}')}$.
    For given $t\in I_{\epsilon}$ with
    $\Delta(\bar{z}_{\epsilon}(t,\bar{\lambda}), \bar{z}_{\epsilon}(t,\bar{\lambda}')) \neq 0$,
    take any decreasing sequence
    $\seq{t_{n}}_{n=1}^{\infty}$ in $I_{\epsilon}$ converging to $t$ such that
    \[
        \partial_{t+}\Delta(\bar{z}_{\epsilon}(t,\bar{\lambda}),
        \bar{z}_{\epsilon}(t,\bar{\lambda}'))
        = \lim_{n\to\infty}
        \frac{
            \Delta(\bar{z}_{\epsilon}(t_{n},\bar{\lambda}),
            \bar{z}_{\epsilon}(t_{n},\bar{\lambda}'))
            - \Delta(\bar{z}_{\epsilon}(t,\bar{\lambda}),
            \bar{z}_{\epsilon}(t,\bar{\lambda}'))
        }{t_{n}-t}.
    \]
    By passing to a subsequence if needed, we can assume that
    $\seq{(\xi_{t_{n}},\xi_{t_{n}}')}_{n=1}^{\infty}$ converges to some
    $(\xi,\xi')\in\bbT\times\bbT$. Then since
    $\Delta(\bar{z}_{\epsilon}(\,\cdot\,,\bar{\lambda}),
    \bar{z}_{\epsilon}(\,\cdot\,,\bar{\lambda}'))$ is continuous and
    $\tilde{z}_{\epsilon}(\,\cdot\,,\bar{\lambda},\,\cdot\,)
    - \tilde{z}_{\epsilon}(\,\cdot\,,\bar{\lambda}',\,\cdot\,)$ is jointly continuous,
    we conclude
    \[
        \Delta(\bar{z}_{\epsilon}(t,\bar{\lambda}), \bar{z}_{\epsilon}(t,\bar{\lambda}'))
        = \abs{\tilde{z}_{\epsilon}(t,\bar{\lambda},\xi)
        - \tilde{z}_{\epsilon}(t,\bar{\lambda}',\xi')}.
    \]
    Then we can assume $\abs{\tilde{z}_{\epsilon}(t,\bar{\lambda},\xi_{t_{n}})
    - \tilde{z}_{\epsilon}(t,\bar{\lambda}',\xi_{t_{n}}')} \neq 0$
    for all $n$. Then, for each $n$, we have
    \begin{align*}
        &\frac{
            \Delta(\bar{z}_{\epsilon}(t_{n},\bar{\lambda}),
            \bar{z}_{\epsilon}(t_{n},\bar{\lambda}'))
            - \Delta(\bar{z}_{\epsilon}(t,\bar{\lambda}),
            \bar{z}_{\epsilon}(t,\bar{\lambda}'))
        }{t_{n}-t} \\
        &\quad\geq
        \frac{\tilde{z}_{\epsilon}(t,\bar{\lambda},\xi_{t_{n}})
        - \tilde{z}_{\epsilon}(t,\bar{\lambda}',\xi_{t_{n}}')}
        {\abs{\tilde{z}_{\epsilon}(t,\bar{\lambda},\xi_{t_{n}})
        - \tilde{z}_{\epsilon}(t,\bar{\lambda}',\xi_{t_{n}}')}}
        \cdot
        \frac{(\tilde{z}_{\epsilon}(t_{n},\bar{\lambda},\xi_{t_{n}})
        - \tilde{z}_{\epsilon}(t_{n},\bar{\lambda}',\xi_{t_{n}}'))
        - (\tilde{z}_{\epsilon}(t,\bar{\lambda},\xi_{t_{n}})
        - \tilde{z}_{\epsilon}(t,\bar{\lambda}',\xi_{t_{n}}'))}{t_{n} - t} \\
        &\quad=
        \frac{\tilde{z}_{\epsilon}(t,\bar{\lambda},\xi_{t_{n}})
        - \tilde{z}_{\epsilon}(t,\bar{\lambda}',\xi_{t_{n}}')}
        {\abs{\tilde{z}_{\epsilon}(t,\bar{\lambda},\xi_{t_{n}})
        - \tilde{z}_{\epsilon}(t,\bar{\lambda}',\xi_{t_{n}}')}}
        \cdot
        \frac{1}{t_{n}-t}\int_{t}^{t_{n}}
        u_{\epsilon}(z_{\epsilon}(\tau))\left(
            \tilde{z}_{\epsilon}(\tau,\bar{\lambda},\xi_{t_{n}})
        \right)
        - u_{\epsilon}(z_{\epsilon}(\tau))\left(
            \tilde{z}_{\epsilon}(\tau,\bar{\lambda}',\xi_{t_{n}}')
        \right)
        \,d\tau.
    \end{align*}
    By continuity, taking the limit $n\to\infty$ gives
    \eqref{6.3}. Then by Proposition~\ref{PC.10},
    integrating the estimate from Lemma~\ref{L5.1}
    (with $K$ replaced by $K_{\epsilon}$) gives the first inequality in
    \eqref{6.4}. The second inequality than follows from
    $\ell(\bar{z}_{\epsilon}(t,\bar{\lambda}''))
    \leq \norm{\bar{z}_{\epsilon}(t,\bar{\lambda}'')}_{C^{0}}^{2}
    \mathcal{K}_{2}(\bar{z}_{\epsilon}(t,\bar{\lambda}''))$ and
    $\norm{\bar{z}_{\epsilon}(t,\bar{\lambda}'')}_{\dot{C}^{1,1/2}}
    \leq \norm{\bar{z}_{\epsilon}(t,\bar{\lambda}'')}_{\dot{H}^{2}}$.
\end{proof}

\begin{corollary}\label{C6.12}
    There exists a constant $C=C(\alpha)\geq 0$ depending only on $\alpha$
    such that
    \begin{align*}
        &\sup_{\bar{\lambda}\in\bar{\mathcal{L}}}\int_{\bar{\mathcal{L}}}
        \frac{d\bar{\lambda}'}
        {\Delta(\bar{z}_{\epsilon}(t+h,\bar{\lambda}),
        \bar{z}_{\epsilon}(t+h,\bar{\lambda}'))^{2\alpha}} \\
        &\quad\quad\quad \leq
        \sup_{\bar{\lambda}\in\bar{\mathcal{L}}}\int_{\bar{\mathcal{L}}}
        \frac{d\bar{\lambda}'}
        {\Delta(\bar{z}_{\epsilon}(t,\bar{\lambda}),
        \bar{z}_{\epsilon}(t,\bar{\lambda}'))^{2\alpha}}
        + L(z_{\epsilon}(t))
        \left(
            \exp\left(
                C\int_{t}^{t+h}
                L(z_{\epsilon}(\tau))^{5}\,d\tau
            \right) - 1
        \right)
    \end{align*}
    holds for any $\bar{\lambda}\in\bar{\mathcal{L}}$,
    $t\in I_{\epsilon}$ and $h>0$ with $t+h\in I_{\epsilon}$.
\end{corollary}

\begin{proof}
    By Lemma~\ref{L6.11}, there exists a constant
    $C=C(\alpha)\geq 0$ such that for any
    $\bar{\lambda},\bar{\lambda}'\in\bar{\mathcal{L}}$ and $t\in I_{\epsilon}$ with
    $\Delta(\bar{z}_{\epsilon}(t,\bar{\lambda}), \bar{z}_{\epsilon}(t,\bar{\lambda}'))\neq 0$,
    we have
    \[
        \partial_{t}^{+}\left(\frac{1}
        {\Delta(\bar{z}_{\epsilon}(t,\bar{\lambda}),
        \bar{z}_{\epsilon}(t,\bar{\lambda}'))^{2\alpha}}
        \right)
        \leq \frac{CL(z_{\epsilon}(t))^{5}}
        {\Delta(\bar{z}_{\epsilon}(t,\bar{\lambda}),
        \bar{z}_{\epsilon}(t,\bar{\lambda}'))^{2\alpha}}
    \]
    since $\Delta(\bar{z}_{\epsilon}(t,\bar{\lambda}), \bar{z}_{\epsilon}(t,\bar{\lambda}'))$
    is continuous in $t$. Hence, a Gr\"{o}nwall argument shows that
    \[
        \frac{1}
        {\Delta(\bar{z}_{\epsilon}(t+h,\bar{\lambda}),
        \bar{z}_{\epsilon}(t+h,\bar{\lambda}'))^{2\alpha}}
        \leq
        \frac{1}
        {\Delta(\bar{z}_{\epsilon}(t,\bar{\lambda}),
        \bar{z}_{\epsilon}(t,\bar{\lambda}'))^{2\alpha}}
        \exp\left(
            C\int_{t}^{t+h}
            L(z_{\epsilon}(\tau))^{5}\,d\tau
        \right)
    \]
    holds for any $t,t+h\in I_{\epsilon}$ and
    $\bar{\lambda},\bar{\lambda}'\in\bar{\mathcal{L}}$, thus
    \begin{align*}
        &\int_{\bar{\mathcal{L}}}
        \frac{d\bar{\lambda}'}
        {\Delta(\bar{z}_{\epsilon}(t+h,\bar{\lambda}),
        \bar{z}_{\epsilon}(t+h,\bar{\lambda}'))^{2\alpha}} \\
        &\quad\quad\quad \leq
        \int_{\bar{\mathcal{L}}}
        \frac{d\bar{\lambda}'}
        {\Delta(\bar{z}_{\epsilon}(t,\bar{\lambda}),
        \bar{z}_{\epsilon}(t,\bar{\lambda}'))^{2\alpha}}
        + L(z_{\epsilon}(t))
        \left(
            \exp\left(
                C\int_{t}^{t+h}
                L(z_{\epsilon}(\tau))^{5}\,d\tau
            \right) - 1
        \right)
    \end{align*}
    holds for any $t\in I_{\epsilon}$ and $\bar{\lambda}\in\bar{\mathcal{L}}$.
    Taking the supremum over $\bar{\lambda}$ then shows the claim.
\end{proof}

\begin{corollary}\label{C6.13}
    There exists a constant $C=C(\alpha,\abs{\mathcal{L}})\geq 0$ depending only on
    $\alpha$ and $\abs{\mathcal{L}}$ such that
    \begin{equation}\label{6.5}
        \partial_{t}^{+}L(z_{\epsilon}(t))
        \leq CL(z_{\epsilon}(t))^{6}
    \end{equation}
    and
    \begin{equation}\label{6.6}
        L(z_{\epsilon}(t))
        \leq \frac{L(z_{0})}{(1 - 5CL(z_{0})^{5}t)^{1/5}}
    \end{equation}
    hold for all $t\in I_{\epsilon}\cap \left[0, \frac{1}{5CL(z_{0})^{5}}\right)$.
\end{corollary}

\begin{proof}
    By adding Corollary~\ref{C6.5},
    Corollary~\ref{C6.10}, and
    Corollary~\ref{C6.12}, we obtain
    \begin{equation}\label{6.7}
        L(z_{\epsilon}(t+h)) \leq
        C(\alpha,\abs{\mathcal{L}})\int_{t}^{t+h}L(z_{\epsilon}(\tau))^{6}\,d\tau
        + L(z_{\epsilon}(t))
        \exp\left(C(\alpha)\int_{t}^{t+h}L(z_{\epsilon}(\tau))^{5}\,d\tau\right)
    \end{equation}
    for any $t\in I_{\epsilon}$ and $h>0$ with
    $t,t+h\in (-\delta,\delta)$ for some $\delta=\delta(\epsilon,\tilde{z}_{0})$.

    Note that by Lemma~\ref{L6.1}
    and \eqref{6.3}, there exists a constant
    $M=M(\epsilon,\tilde{z}_{0})>0$ such that
    \[
        \Delta(\bar{z}_{\epsilon}(t,\bar{\lambda}),\bar{z}_{\epsilon}(t,\bar{\lambda}'))
        \geq e^{-Mt}
        \Delta(\bar{z}_{0}(\bar{\lambda}),\bar{z}_{0}(\bar{\lambda}'))
    \]
    holds whenever $\abs{t}<M^{-1}$. This, together with
    the continuity of $\sup_{\bar{\lambda}\in\bar{\mathcal{L}}}
    \norm{\bar{z}_{\epsilon}(t,\bar{\lambda})}_{C^{0}}$ and
    Proposition~\ref{P6.7} show that
    $L(z_{\epsilon}(t))$ is uniformly bounded by a finite constant
    if $t$ is small enough. Therefore, for such $t$, by sending $h\to 0^{+}$,
    \eqref{6.7}, we obtain
    \[
        \limsup_{h\to 0^{+}}L(z_{\epsilon}(t+h)) \leq L(z_{\epsilon}(t)),
    \]
    which in turn shows that
    \[
        \partial_{t}^{+}L(z_{\epsilon}(t))
        \leq C(\alpha,\abs{\mathcal{L}})L(z_{\epsilon}(t))^{6}
    \]
    holds for all small enough $t$. By a Gr\"{o}nwall argument, this implies that
    \eqref{6.6} holds for all such $t$.
    By Proposition~\ref{PC.9}, for any $t_{0}\in I_{\epsilon}$ with
    $L(z_{\epsilon}(t_{0})) < \infty$, we can find a joint parameterization
    $\tilde{z}_{0}'\in E$ of $z_{\epsilon}(t_{0})$, and
    Corollary~\ref{C6.3} shows that
    the unique solution we get by solving \eqref{6.1} with
    $\tilde{z}_{0}'$ as the initial data should coincide with $\bar{z}_{\epsilon}$
    on their common domain when projected down to
    $L_{tb}^{\infty}(\mathcal{L};\operatorname{Curve}(\bbR^{2}))$. Therefore,
    repeating the same argument as above shows that the same estimate
    \eqref{6.5} continues to hold around $t_{0}$.
    Since $L(z_{\epsilon}(t))$ is lower semicontinuous in $t$,
    a typical connectivity argument then shows that
    $L(z_{\epsilon}(t)) < \infty$, \eqref{6.5} and \eqref{6.6} in fact hold for all
    $t\in I_{\epsilon}\cap \left[0, \frac{1}{5CL(z_{0})^{5}}\right)$.
\end{proof}


\subsection{Construction of the solution}

\begin{proposition}\label{P6.14}
    Let $T_{0} \coloneqq \frac{1}{5CL(z_{0})^{5}}$
    where $C$ is the constant from Corollary~\ref{C6.13}.
    Then for any initial data
    $z_{0}\in L_{tb}^{\infty}(\mathcal{L};\operatorname{Curve}(\bbR^{2}))$
    with $L(z_{0}) < \infty$, there uniquely exists a continuous function
    $z_{\epsilon}\colon \left[0, T_{0}\right)
    \to L_{tb}^{\infty}(\mathcal{L};\operatorname{Curve}(\bbR^{2}))$
    such that for any $t\in \left[0, T_{0}\right)$,
    there exists a continuous function $\tilde{z}_{\epsilon}\colon I\to E$
    defined on an open neighborhood of $t$ solving
    \eqref{6.1} whose projection onto
    $L_{tb}^{\infty}(\mathcal{L};\operatorname{Curve}(\bbR^{2}))$
    coincides with $z_{\epsilon}$ on their common domain.
    Furthermore, this $z_{\epsilon}$ satisfies \eqref{6.5}
    and \eqref{6.6}
    for all $t\in\left[0,T_{0}\right)$.
\end{proposition}

\begin{proof}
    By definition, the set $A$ of $t\in\left[0,T_{0}\right]$
    such that such a function $z_{\epsilon}$ uniquely exists on $[0,t)$
    is closed and is nonempty since it contains $0$.
    Let $T\coloneqq\max A$ and take the unique function
    $z_{\epsilon}\colon[0,T)\to L_{tb}^{\infty}(\mathcal{L};\operatorname{Curve}(\bbR^{2}))$.
    Clearly, the assumption on $z_{\epsilon}$ implies that
    \eqref{6.5} and
    \eqref{6.6} hold for all $t\in[0,T)$.

    Suppose $T<T_{0}$ for sake of contradiction,
    then $M\coloneqq \sup_{t\in[0,T)}L(z_{\epsilon}(t)) < \infty$ holds by
    \eqref{6.6}. For given $t\in[0,T)$,
    find a $\tilde{z}_{\epsilon}\colon I\to E$ defined on an open neighborhood
    $I\subseteq [0,T)$ of $t$ given by the assumption on $z_{\epsilon}$.
    Then for any $t'\in I$, we have
    \begin{align*}
        d_{\mathrm{F},\infty}(z_{\epsilon}(t),z_{\epsilon}(t'))
        \leq \sup_{\bar{\lambda}}
        \norm{\int_{t}^{t'}u_{\epsilon}(z_{\epsilon}(\tau))
        \circ \tilde{z}_{\epsilon}(\tau,\bar{\lambda})\,d\tau}_{C^{0}}
        \leq \abs{t - t'}\sup_{\tau\in [0,T)}\norm{u_{\epsilon}(z_{\epsilon}(\tau))}_{C^{0}}.
    \end{align*}
    As noted in Proposition~\ref{P6.4}, we know
    $\norm{u_{\epsilon}(z_{\epsilon}(\tau))} \leq
    C(\alpha,\abs{\mathcal{L}})L(z_{\epsilon}(\tau))^{4} \leq C(\alpha,\abs{\mathcal{L}})M^{4}$
    for any $\tau\in[0,T)$, thus $\seq{z_{\epsilon}(t)}_{t<T}$ is a Cauchy net.
    Since $L_{tb}^{\infty}(\mathcal{L};\operatorname{Curve}(\bbR^{2}))$
    is a complete metric space (Theorem~\ref{TA.11} and
    Proposition~\ref{PC.1}), $z_{\epsilon}(t)$ converges to some
    $z_{\epsilon}(T)\in L_{tb}^{\infty}(\mathcal{L};\operatorname{Curve}(\bbR^{2}))$
    as $t\to T$. By lower semicontinuity of $L$, we know
    $L(z_{\epsilon}(T)) \leq M$, thus there exists a continuous function
    $\tilde{w}_{\epsilon}\colon I\to E$ defined on an open neighborhood
    $I\subseteq\left[0,T_{0}\right)$ of $T$ that
    solves \eqref{6.1} and satisfies
    $\pi(\tilde{w}_{\epsilon}(T)) = z_{\epsilon}(T)$.
    Define $w_{\epsilon}\colon I\to L_{tb}^{\infty}(\mathcal{L};\operatorname{Curve}(\bbR^{2}))$
    as $w_{\epsilon} = \pi\circ \tilde{w}_{\epsilon}$.

    By Lemma~\ref{L6.6},
    there exists a constant $\delta=\delta(\epsilon,\tilde{w}_{\epsilon}(T))>0$ such that
    $(T-\delta,T+\delta)\subseteq I$ and
    \[
        M_{1}\coloneqq \sup_{t\in(T-\delta,T]}
        \norm{\lambda\mapsto
        \ell(w_{\epsilon}(t,\lambda)) + \ell(z_{\epsilon}(t,\lambda))}_{L^{\infty}}
        <\infty.
    \]
    Then, by following the proof of
    Lemma~\ref{L6.2}
    with a fixed $t_{0}\in (T-\delta,T)$, one can find a constant
    $C=C(\alpha,\abs{\mathcal{L}},\epsilon,M_{1})>0$ such that
    \[
        d_{\mathrm{F},\infty}(z_{\epsilon}(t),w_{\epsilon}(t))
        \leq e^{C\abs{t-t_{0}}}d_{\mathrm{F},\infty}
        (z_{\epsilon}(t_{0}),w_{\epsilon}(t_{0}))
    \]
    holds for any $t\in(T-\delta,T)$ sufficiently close to $t_{0}$.
    Since the constant $C$ does not depend on $t_{0}$, in fact this holds for
    all $t\in (T-\delta,T)$. Hence, by sending $t_{0}\uparrow T$, we conclude that
    $z_{\epsilon}(t) = w_{\epsilon}(t)$ holds for all $t\in(T-\delta,T]$,
    which shows that $z_{\epsilon}$ can be extended onto $[0,T]\cup I$.
    This is a contradiction, so we must have
    $T = T_{0}$ as claimed.
\end{proof}

Since we now have a nontrivial interval independent of $\epsilon>0$ where all of
$z_{\epsilon}$'s are defined, the next process is to send $\epsilon\to 0$ and claim
the resulting function is a solution to the original equation. From now on,
for each $\epsilon>0$ let $z_{\epsilon}\colon \left[0,T_{0}\right) \to
L_{tb}^{\infty}(\mathcal{L};\operatorname{Curve}(\bbR^{2}))$ be the unique continuous function
given by Proposition~\ref{P6.14}, for a given
initial condition $z_{0}\in L_{tb}^{\infty}(\mathcal{L};\operatorname{Curve}(\bbR^{2}))$
with $L(z_{0}) < \infty$. Also, let $\bar{z}_{\epsilon}\colon \left[0,T_{0}\right) \to
C(\bar{\mathcal{L}};\operatorname{Curve}(\bbR^{2}))$ be the corresponding
$C(\bar{\mathcal{L}};\operatorname{Curve}(\bbR^{2}))$-valued function.

\begin{lemma}\label{L6.15}
    There exists a constant $C=C(\alpha)\geq 0$ depending only on $\alpha$ such that
    for any $\epsilon>0$, $x\in\bbR^{2}$, $\beta\in(0,1]$ and a $C^{1,\beta}$ closed curve
    $\gamma\colon\ell\bbT\to\bbR^{2}$ parameterized by the arclength,
    \[
        \int_{\abs{x - \gamma(s)}\leq\epsilon}
        \frac{ds}{\abs{x - \gamma(s)}^{2\alpha}}
        \leq C(\alpha)\ell\norm{\gamma}_{\dot{C}^{1,\beta}}^{1/\beta}
        \epsilon^{1-2\alpha}.
    \]
\end{lemma}

\begin{proof}
    We may assume that $\gamma$ is a nontrivial curve.
    Applying Lemma~\ref{L3.2} with
    $d = \frac{1}{2\norm{\gamma}_{\dot{C}^{1,\beta}}^{1/\beta}}$, we have
    \begin{align*}
        \int_{\abs{x - \gamma(s)}\leq\epsilon}\frac{ds}{\abs{x - \gamma(s)}^{2\alpha}}
        &\leq \sum_{i=1}^{N}\int_{\abs{s}\leq \min(\epsilon,2d)}
        \frac{2^{2\alpha}ds}{\abs{s}^{2\alpha}}
        + \int_{d\leq \abs{x - \gamma(s)}\leq \epsilon}\frac{ds}{d^{2\alpha}} \\
        &\leq C(\alpha)\cdot \frac{\ell}{d}\cdot \frac{1}{1-2\alpha}
        \min(\epsilon,2d)^{1-2\alpha}
        + \frac{\ell}{d^{2\alpha}}\mathbbm{1}_{d<\epsilon} \\
        &\leq \frac{C(\alpha)}{1-2\alpha}\frac{\ell\min(\epsilon,d)^{1-2\alpha}}{d}
        + \frac{\ell}{d}\epsilon^{1-2\alpha} \\
        &\leq \frac{C(\alpha)}{1-2\alpha}
        \ell\norm{\gamma}_{\dot{C}^{1,\beta}}^{1/\beta}\epsilon^{1-2\alpha}.
    \end{align*}
\end{proof}

\begin{proposition}
    The net $\seq{z_{\epsilon}}_{\epsilon>0}$ is uniformly Cauchy
    on any compact subinterval of $[0,T_{0})$.
\end{proposition}

\begin{proof}
    Let $T\in(0,T_{0})$ and
    $M\coloneqq \sup_{\epsilon>0}\sup_{t\in[0,T]}L(z_{\epsilon}(t)) < \infty$.
    For fixed $\epsilon_{1}>\epsilon_{2}>0$ and $t\in[0,T)$, we estimate
    $\partial_{t}^{+}d_{\mathrm{F},\infty}(z_{\epsilon_{1}}(t),z_{\epsilon_{2}}(t))$.
    Find any solutions $\tilde{z}_{\epsilon_{i}}\colon I\to E$, $i=1,2$
    to \eqref{6.1} with $\epsilon=\epsilon_{i}$
    which project onto $z_{\epsilon_{i}}$, defined on a common interval $I\subseteq [0,T)$
    around $t$. Fix $h>0$ with $t+h\in I$, $\bar{\lambda}\in\bar{\mathcal{L}}$,
    and any orientation-preserving homeomorphism $\phi\colon\bbT\to\bbT$, then
    \begin{align*}
        &\norm{\tilde{z}_{\epsilon_{1}}(t+h,\bar{\lambda})
        - \tilde{z}_{\epsilon_{2}}(t+h,\bar{\lambda})\circ\phi}_{C^{0}}
        \\&\quad\quad\quad
        \leq \norm{\tilde{z}_{\epsilon_{1}}(t,\bar{\lambda})
        - \tilde{z}_{\epsilon_{2}}(t,\bar{\lambda})\circ\phi}_{C^{0}}
        \\&\quad\quad\quad\quad\quad\quad
        + \int_{t}^{t+h}
        \norm{u_{\epsilon_{1}}(z_{\epsilon_{1}}(\tau))
        \circ \tilde{z}_{\epsilon_{1}}(\tau,\bar{\lambda})
        - u_{\epsilon_{2}}(z_{\epsilon_{2}}(\tau))
        \circ \tilde{z}_{\epsilon_{2}}(\tau,\bar{\lambda})\circ\phi}_{C^{0}}\,d\tau
    \end{align*}
    holds. We bound the integrand by the sum of the following terms:
    \begin{align*}
        G_{1} &\coloneqq
        \norm{u(z_{\epsilon_{1}}(t))
        \circ \tilde{z}_{\epsilon_{1}}(t,\bar{\lambda})
        - u(z_{\epsilon_{2}}(t))
        \circ \tilde{z}_{\epsilon_{2}}(t,\bar{\lambda})\circ\phi}_{C^{0}}, \\
        G_{2} &\coloneqq
        \norm{u(z_{\epsilon_{1}}(\tau))
        \circ \tilde{z}_{\epsilon_{1}}(\tau,\bar{\lambda})
        - u(z_{\epsilon_{1}}(t))
        \circ \tilde{z}_{\epsilon_{1}}(t,\bar{\lambda})}_{C^{0}}, \\
        G_{3} &\coloneqq
        \norm{u(z_{\epsilon_{2}}(\tau))
        \circ \tilde{z}_{\epsilon_{2}}(\tau,\bar{\lambda})
        - u(z_{\epsilon_{2}}(t))
        \circ \tilde{z}_{\epsilon_{2}}(t,\bar{\lambda})}_{C^{0}}, \\
        G_{2} &\coloneqq
        \norm{u(z_{\epsilon_{1}}(\tau))
        \circ \tilde{z}_{\epsilon_{1}}(\tau,\bar{\lambda})
        - u_{\epsilon_{1}}(z_{\epsilon_{1}}(\tau))
        \circ \tilde{z}_{\epsilon_{1}}(\tau,\bar{\lambda})}_{C^{0}}, \\
        G_{3} &\coloneqq
        \norm{u(z_{\epsilon_{2}}(\tau))
        \circ \tilde{z}_{\epsilon_{2}}(\tau,\bar{\lambda})
        - u_{\epsilon_{2}}(z_{\epsilon_{2}}(\tau))
        \circ \tilde{z}_{\epsilon_{2}}(\tau,\bar{\lambda})}_{C^{0}}.
    \end{align*}
    Firstly, Lemma~\ref{L5.2} shows
    \begin{align*}
        G_{1} &\leq C(\alpha,M)\left(
            \norm{\tilde{z}_{\epsilon_{1}}(t,\bar{\lambda})
            - \tilde{z}_{\epsilon_{2}}(t,\bar{\lambda})\circ\phi}_{C^{0}}
            + d_{\mathrm{F},\infty}(z_{\epsilon_{1}}(t), z_{\epsilon_{2}}(t))
        \right), \\
        G_{2} &\leq C(\alpha,M)\left(
            \norm{\tilde{z}_{\epsilon_{1}}(\tau,\bar{\lambda})
            - \tilde{z}_{\epsilon_{1}}(t,\bar{\lambda})}_{C^{0}}
            + d_{\mathrm{F},\infty}(z_{\epsilon_{1}}(\tau), z_{\epsilon_{1}}(t))
        \right), \textrm{and}\\
        G_{3} &\leq C(\alpha,M)\left(
            \norm{\tilde{z}_{\epsilon_{2}}(\tau,\bar{\lambda})
            - \tilde{z}_{\epsilon_{2}}(t,\bar{\lambda})}_{C^{0}}
            + d_{\mathrm{F},\infty}(z_{\epsilon_{2}}(\tau), z_{\epsilon_{2}}(t))
        \right).
    \end{align*}

    For $G_{4}$, fix a jointly continuous arclength parameterization
    $\bar{z}_{\epsilon_{1}}(\tau)\colon \bar{\mathcal{L}}\times\bbR\to \bbR^{2}$
    of $\bar{z}_{\epsilon_{1}}(\tau)$ (Proposition~\ref{PC.9}), then
    Lemma~\ref{L6.15} shows
    \begin{align*}
        G_{4} &\leq \sup_{s\in\ell(\bar{z}_{\epsilon_{1}}(\tau,\bar{\lambda}))\bbT}
        \int_{\bar{\mathcal{L}}}
        \int_{\abs{\bar{z}_{\epsilon_{1}}(\tau,\bar{\lambda},s)
        - \bar{z}_{\epsilon_{1}}(\tau,\bar{\lambda}',s')}\leq\epsilon_{1}}
        \frac{1}{\abs{\bar{z}_{\epsilon_{1}}(\tau,\bar{\lambda},s)
        - \bar{z}_{\epsilon_{1}}(\tau,\bar{\lambda}',s')}^{2\alpha}}
        \,ds'\,d\bar{\lambda}' \\
        &\leq C(\alpha)
        \left(\int_{\bar{\mathcal{L}}}\ell(\bar{z}_{\epsilon_{1}}(\tau,\bar{\lambda}'))
        \mathcal{K}_{2}(\bar{z}_{\epsilon_{1}}(\tau,\bar{\lambda}'))
        \,d\bar{\lambda}'\right)\epsilon_{1}^{1-2\alpha} \\
        &\leq C(\alpha)\abs{\mathcal{L}}M^{4}\epsilon_{1}^{1-2\alpha},
    \end{align*}
    and similarly $G_{5}$ satisfies the same bound. Hence, taking the infimum
    over $\phi$ and then supremum over $\bar{\lambda}$ shows
    \begin{align*}
        d_{\mathrm{F},\infty}(z_{\epsilon_{1}}(t+h), z_{\epsilon_{2}}(t+h))
        &\leq 
        d_{\mathrm{F},\infty}(z_{\epsilon_{1}}(t), z_{\epsilon_{2}}(t))
        + hC\left(
            d_{\mathrm{F},\infty}(z_{\epsilon_{1}}(t), z_{\epsilon_{2}}(t))
            + \epsilon_{1}^{1-2\alpha}
        \right)
        \\&\quad\quad
        + C\int_{t}^{t+h}
        \norm{\tilde{z}_{\epsilon_{1}}(\tau) - \tilde{z}_{\epsilon_{1}}(t)}_{E}
        + \norm{\tilde{z}_{\epsilon_{2}}(\tau) - \tilde{z}_{\epsilon_{2}}(t)}_{E}
        \,d\tau,
    \end{align*}
    from which we obtain
    \[
        \partial_{t}^{+}
        d_{\mathrm{F},\infty}(z_{\epsilon_{1}}(t), z_{\epsilon_{2}}(t))
        \leq Cd_{\mathrm{F},\infty}(z_{\epsilon_{1}}(t), z_{\epsilon_{2}}(t))
        +C\epsilon_{1}^{1-2\alpha}.
    \]
    Since this holds for every $t\in[0,T)$, a Gr\"{o}nwall argument shows
    \[
        d_{\mathrm{F},\infty}(z_{\epsilon_{1}}(t), z_{\epsilon_{2}}(t))
        \leq (e^{Ct} - 1)\epsilon_{1}^{1-2\alpha},
    \]
    thus $\seq{z_{\epsilon}}_{\epsilon>0}$ is uniformly Cauchy on
    $[0,T)$.
\end{proof}

Consequently, $\seq{z_{\epsilon}}_{\epsilon>0}$ converges locally uniformly to
a continuous function
$z\colon[0,T_{0})\to L_{tb}^{\infty}(\mathcal{L};\operatorname{Curve}(\bbR^{2}))$.
By lower semicontinuity of $L$, this $z$ satisfies
\eqref{6.6} as well. To show that $z$ is actually a solution,
that is, it satisfies \eqref{eq:solution}, we need a uniform estimate on
\[
    d_{\mathrm{F},\infty}\left(z_{\epsilon}(t+h),
    X_{u(z_{\epsilon}(t))}^{h}[z_{\epsilon}(t)]\right),
\]
which is the final ingredient for the existence proof.
As we have been doing throughout this section, let
$\bar{z}\colon [0,T_{0})\to C(\bar{\mathcal{L}};\operatorname{Curve}(\bbR^{2}))$
be the corresponding $C(\bar{\mathcal{L}};\operatorname{Curve}(\bbR^{2}))$-valued function.

\begin{lemma}\label{L6.17}
    Let $T\in(0,T_{0})$ and
    $M\coloneqq\sup_{\epsilon>0}\sup_{t\in[0,T]}L(z_{\epsilon}(t)) < \infty$.
    Then there exists a constant $C = C(\alpha,\abs{\mathcal{L}},M)\geq 0$
    depending only on $\alpha$, $\abs{\mathcal{L}}$, and $M$ such that
    \[
        d_{\mathrm{F},\infty}\left(
            z_{\epsilon}(t+h),
            X_{u(z_{\epsilon}(t))}^{h}[z_{\epsilon}(t)]
        \right)
        \leq C(h^{2} + \abs{h}\epsilon^{1-2\alpha})
    \]
    holds for all $t\in[0,T)$, $h\in\bbR$ with $t+h\in[0,T)$ and $\epsilon>0$.
\end{lemma}

\begin{proof}
    We only show the case $h>0$. The other case can be done in the same way.
    Fix $t\in[0,T)$, $h>0$ with $t+h<T$ and $\epsilon>0$.
    Then there exists a finite partition
    $t=t_{0}<\ \cdots\ <t_{n}=t+h$ such that each $[t_{i-1},t_{i}]$
    is contained in the domain of a solution to \eqref{6.1}
    which projects to $z_{\epsilon}$. Let $\bar{\lambda}\in\bar{\mathcal{L}}$ and
    $\delta>0$ be given. Take any solution
    $\tilde{z}_{\epsilon}^{1}$ to \eqref{6.1} defined on
    $[t_{0},t_{1}]$ which projects to $z_{\epsilon}$, and let $\phi^{1}\colon\bbT\to\bbT$
    be the identity map. For $i=2,\ \cdots\ ,n$, we inductively find
    a solution $\tilde{z}_{\epsilon}^{i}$ to \eqref{6.1}
    defined on $[t_{i-1},t_{i}]$ which projects to $z_{\epsilon}$
    and an orientation-preserving homeomorphism $\phi^{i}\colon\bbT\to\bbT$ such that
    \[
        \norm{\tilde{z}_{\epsilon}^{i}(t_{i-1},\bar{\lambda})\circ\phi^{i}
        - \tilde{z}_{\epsilon}^{i-1}(t_{i-1},\bar{\lambda})\circ\phi^{i-1}}_{C^{0}}
        \leq \frac{\delta}{n}.
    \]
    Then,
    \begin{align*}
        \tilde{z}_{\epsilon}^{n}(t_{n},\bar{\lambda})\circ\phi^{n}
        &= \tilde{z}_{\epsilon}^{n}(t_{n-1},\bar{\lambda})\circ\phi^{n}
        + \int_{t_{n-1}}^{t_{n}}
        u_{\epsilon}(z_{\epsilon}(\tau))
        \circ \tilde{z}_{\epsilon}^{n}(\tau,\bar{\lambda})
        \circ \phi^{n}
        \,d\tau \\
        &= \tilde{z}_{\epsilon}^{n-1}(t_{n-2},\bar{\lambda})\circ\phi^{n-1}
        + \int_{t_{n-2}}^{t_{n-1}}
        u_{\epsilon}(z_{\epsilon}(\tau))
        \circ \tilde{z}_{\epsilon}^{n-1}(\tau,\bar{\lambda})
        \circ \phi^{n-1}
        \\&\quad\quad\quad
        + \left(\tilde{z}_{\epsilon}^{n}(t_{n-1},\bar{\lambda})\circ\phi^{n}
        - \tilde{z}_{\epsilon}^{n-1}(t_{n-1},\bar{\lambda})\circ\phi^{n-1}\right)
        \\&\quad\quad\quad
        + \int_{t_{n-1}}^{t_{n}}
        u_{\epsilon}(z_{\epsilon}(\tau))
        \circ \tilde{z}_{\epsilon}^{n}(\tau,\bar{\lambda})
        \circ \phi^{n}
        \,d\tau \\
        &=\ \cdots\ \\
        &=\tilde{z}_{\epsilon}^{1}(t_{0},\bar{\lambda})\circ\phi^{1}
        + \sum_{i=1}^{n} \int_{t_{i-1}}^{t_{i}}
        u_{\epsilon}(z_{\epsilon}(\tau))
        \circ \tilde{z}_{\epsilon}^{i}(\tau,\bar{\lambda})
        \circ \phi^{i}
        \,d\tau
        \\&\quad\quad\quad
        + \sum_{i=2}^{n}\left(
            \tilde{z}_{\epsilon}^{i}(t_{i-1},\bar{\lambda})\circ\phi^{i}
            - \tilde{z}_{\epsilon}^{i-1}(t_{i-1},\bar{\lambda})\circ\phi^{i-1}
        \right),
    \end{align*}
    thus
    \begin{align*}
        &d_{\mathrm{F}}\left(\bar{z}_{\epsilon}(t+h,\bar{\lambda}),
        X_{u(z_{\epsilon}(t))}^{h}[\bar{z}_{\epsilon}(t)](\bar{\lambda})\right)
        \\&\quad\quad\quad \leq
        \sum_{i=1}^{n} \int_{t_{i-1}}^{t_{i}}
        \norm{u_{\epsilon}(z_{\epsilon}(\tau))
        \circ \tilde{z}_{\epsilon}^{i}(\tau,\bar{\lambda})
        \circ \phi^{i}
        - u(z_{\epsilon}(t))\circ \tilde{z}_{\epsilon}^{1}(t,\bar{\lambda})}_{C^{0}}
        \,d\tau
        \\&\quad\quad\quad\quad\quad\quad
        + \sum_{i=2}^{n}\norm{
            \tilde{z}_{\epsilon}^{i}(t_{i-1},\bar{\lambda})\circ\phi^{i}
            - \tilde{z}_{\epsilon}^{i-1}(t_{i-1},\bar{\lambda})\circ\phi^{i-1}
        }_{C^{0}}.
    \end{align*}
    The integrand of the first summation is bounded by the sum of the two terms:
    \begin{align*}
        G_{1} &\coloneqq
        \norm{u(z_{\epsilon}(\tau))
        \circ \tilde{z}_{\epsilon}^{i}(\tau,\bar{\lambda})
        \circ \phi^{i}
        - u(z_{\epsilon}(t))
        \circ \tilde{z}_{\epsilon}^{1}(t,\bar{\lambda})}_{C^{0}}, \\
        G_{2} &\coloneqq
        \norm{u_{\epsilon}(z_{\epsilon}(\tau))\circ
        \tilde{z}_{\epsilon}^{i}(\tau,\bar{\lambda})
        - u(z_{\epsilon}(\tau))\circ
        \tilde{z}_{\epsilon}^{i}(\tau,\bar{\lambda})}_{C^{0}}.
    \end{align*}
    For $G_{1}$, Lemma~\ref{L5.2} shows
    \begin{align*}
        G_{1} &\leq C(\alpha,M)
        \left(
            \norm{\tilde{z}_{\epsilon}^{i}(\tau,\bar{\lambda})\circ\phi^{i}
            - \tilde{z}_{\epsilon}^{1}(t,\bar{\lambda})}_{C^{0}}
            + d_{\mathrm{F},\infty}(z_{\epsilon}(\tau), z_{\epsilon}(t))
        \right),
    \end{align*}
    and for $G_{2}$, Lemma~\ref{L6.15} shows
    \begin{align*}
        G_{2} &\leq C(\alpha)
        \left(
            \int_{\bar{\mathcal{L}}}\ell(\bar{z}_{\epsilon}(\tau,\bar{\lambda}'))
            \mathcal{K}_{2}(\bar{z}_{\epsilon}(\tau,\bar{\lambda}'))
            \,d\bar{\lambda}'
        \right)\epsilon^{1-2\alpha}
        \leq C(\alpha)\abs{\mathcal{L}}M^{4}\epsilon^{1-2\alpha}.
    \end{align*}
    Note that for any $j=1,\ \cdots\ ,n$ and $\tau_{1},\tau_{2}\in[t_{j-1},t_{j}]$,
    by Proposition~\ref{P6.4},
    \begin{align*}
        \norm{\tilde{z}_{\epsilon}^{j}(\tau_{1},\bar{\lambda})
        - \tilde{z}_{\epsilon}^{j}(\tau_{2},\bar{\lambda})}_{C^{0}}
        &\leq \abs{\tau_{1} - \tau_{2}}
        \sup_{\tau\in[t_{j-1},t_{j}]}\norm{u_{\epsilon}(z_{\epsilon}(\tau))}_{C^{0}} \\
        &\leq C(\alpha,\abs{\mathcal{L}})M^{4}\abs{\tau_{1} - \tau_{2}}
    \end{align*}
    holds. Therefore,
    \begin{align*}
        \norm{\tilde{z}_{\epsilon}^{i}(\tau,\bar{\lambda})\circ\phi^{i}
        - \tilde{z}_{\epsilon}^{1}(t,\bar{\lambda})}_{C^{0}}
        &\leq \norm{\tilde{z}_{\epsilon}^{i}(\tau,\bar{\lambda})
        - \tilde{z}_{\epsilon}^{i}(t_{i-1},\bar{\lambda})}_{C^{0}}
        \\&\quad\quad\quad+
        \sum_{j=2}^{i}
        \norm{\tilde{z}_{\epsilon}^{j}(t_{j-1},\bar{\lambda}) \circ\phi^{j}
        - \tilde{z}_{\epsilon}^{j-1}(t_{j-1},\bar{\lambda}) \circ\phi^{j-1}}_{C^{0}}
        \\&\quad\quad\quad+
        \sum_{j=1}^{i-1}
        \norm{\tilde{z}_{\epsilon}^{j}(t_{j},\bar{\lambda})
        - \tilde{z}_{\epsilon}^{j}(t_{j-1},\bar{\lambda})}_{C^{0}} \\
        &\leq C(\alpha,\abs{\mathcal{L}},M)\abs{\tau - t} + \delta.
    \end{align*}
    Note that this implies
    \[
        d_{\mathrm{F}}(z_{\epsilon}(\tau,\bar{\lambda}),z_{\epsilon}(t,\bar{\lambda}))
        \leq C(\alpha,\abs{\mathcal{L}},M)\abs{\tau - t} + \delta,
    \]
    and the same bound can be derived for any
    $\bar{\lambda}\in\bar{\mathcal{L}}$ in the same way. Therefore, we conclude
    \[
        G_{1} \leq C(\alpha,\abs{\mathcal{L}},M)
        \left(\abs{\tau - t} + \delta\right),
    \]
    so
    \begin{align*}
        d_{\mathrm{F}}\left(\bar{z}_{\epsilon}(t+h,\bar{\lambda}),
        X_{u(z_{\epsilon}(t))}^{h}[\bar{z}_{\epsilon}(t)](\bar{\lambda})\right)
        &\leq C(\alpha,\abs{\mathcal{L}},M)
        \left(
            \sum_{i=1}^{n}
            \int_{t_{i-1}}^{t_{i}}\abs{\tau-t} + \delta + \epsilon^{1-2\alpha}\,d\tau
            + \delta
        \right) \\
        &\leq C(\alpha,\abs{\mathcal{L}},M)(h^{2} + h\epsilon^{1-2\alpha} + (h+1)\delta).
    \end{align*}
    Since $\delta>0$ and $\bar{\lambda}\in\bar{\mathcal{L}}$ are arbitrary,
    we obtain the claim.
\end{proof}

Now we prove the existence part of Theorem~\ref{T2.7}.

\begin{proof}[Proof of Theorem~\ref{T2.7}, existence.]
    Fix $T\in(0,T_{0})$ and let
    $M\coloneqq\sup_{\epsilon>0}\sup_{t\in[0,T]}L(z_{\epsilon}(t)) < \infty$.
    It only remains to show that \eqref{eq:solution} holds for any $t\in [0,T)$.
    Fix $t\in [0,T)$ and $h\in\bbR$ with $t+h\in[0,T)$.
    By Lemma~\ref{L5.2}, we know that
    \begin{align*}
        d_{\mathrm{F},\infty}\left(
            X_{u(z_{\epsilon}(t))}^{h}[z_{\epsilon}(t)],
            X_{u(z(t))}^{h}[z(t)]
        \right)
        \leq (1+\abs{h}C(\alpha,M))d_{\mathrm{F},\infty}(z_{\epsilon}(t), z(t))
    \end{align*}
    holds for any $\epsilon>0$. Since $\seq{z_{\epsilon}}_{\epsilon>0}$
    converges uniformly to $z$ on $[0,T)$, we can find
    $\epsilon\in\left(0,\abs{h}^{\frac{1}{1-2\alpha}}\right)$ such that
    the above and $d_{\mathrm{F},\infty}(z_{\epsilon}(t+h),z(t+h))$
    are both at most $h^{2}$. Then, Lemma~\ref{L6.17} shows
    \begin{align*}
        d_{\mathrm{F},\infty}(z(t+h), X_{u(z(t))}^{h}[z(t)])
        &\leq d_{\mathrm{F},\infty}(z(t+h), z_{\epsilon}(t+h))
        + d_{\mathrm{F},\infty}(z_{\epsilon}(t+h),
        X_{u(z_{\epsilon}(t))}^{h}[z_{\epsilon}(t)])
        \\&\quad\quad\quad
        + d_{\mathrm{F},\infty}\left(
            X_{u(z_{\epsilon}(t))}^{h}[z_{\epsilon}(t)],
            X_{u(z(t))}^{h}[z(t)]
        \right) \\
        &\leq 2h^{2} + C(\alpha,\abs{\mathcal{L}},M)h^{2},
    \end{align*}
    thus \eqref{eq:solution} holds.
\end{proof}


%%%%%%%%%%%%%%%%%%%%%%%%%%%%%%%%%%%%%%%%%%%%%%%%%%%%%%%%%%%%%%%%%%%

\begin{thebibliography}{10}

    \bibitem{ConsEsch}
    A. Constantin and J. Escher,
    \textit{Wave breaking for nonlinear nonlocal shallow water equations},
    Acta Math. \textbf{181} (1998), no. 2, 229-243.

    \bibitem{JeonZla21}
    J. Jeon and A. Zlato\v{s},
    \textit{An Improved Regularity Criterion and Absence of Splash-like Singularities for g-SQG Patches},
    \rm Anal. PDE (to appear)

    \bibitem{KisYaoZla}
    A. Kiselev, Y. Yao, and A. Zlato\v{s},
    \textit{Local regularity for the modified SQG patch equation},
    Comm. Pure and Appl. Math. \textbf{70} (2017), 1253--1315.
\end{thebibliography}


%%%%%%%%%%%%%%%%%%%%%%%%%%%%%%%%%%%%%%%%%%%%%%%%%%%%%%%%%%%%%%%%%%%
\appendix

%%%%%%%%%%%%%%%%%%%%%%%%%%%%%%%%%%%%%%%%%%%%%%
\section{Fr\'{e}chet metric}\label{SA}
%%%%%%%%%%%%%%%%%%%%%%%%%%%%%%%%%%%%%%%%%%%%%%

In this section, we collect some relevant results about Fr\'{e}chet metric.
Most of the results here can be found in (references needed).

First, we define curves as equivalence classes of paths, but here
the equivalence relation will not be defined in terms of the Fr\'{e}chet metric,
rather in terms of \emph{reparameterizations}. We will show that
these two notions are actually equivalent.

\begin{definition}
	A function $\phi\colon\bbT\to\bbT$ is called a \emph{reparameterization} if
	there exists a non-decreasing continuous function $h\colon\bbR\to\bbR$ such that
	$h(x+n) = h(x) + n$ and $\phi(x+\bbZ) = h(x) + \bbZ$ hold for all $x\in\bbR$ and $n\in\bbZ$.
\end{definition}

Since $\bbR$ is a covering space of $\bbT$, any continuous path
$z\colon[0,1]\to\bbT$ uniquely lifts to a function
$\tilde{z}\colon[0,1]\to\bbR$ with $\tilde{z}(t)+\bbZ = z(t)$ for all $t\in[0,1]$
and $\tilde{z}(0)\in[0,1)$. When $z(0) = z(1)$, we can regard $z$ as a function
defined on $\bbT$, and in this case the lifting $\tilde{z}$ satisfies
$\tilde{z}(1) = \tilde{z}(0)+n$ for some $n\in\bbZ$. When the lifting $\tilde{z}$ is
non-decreasing, this $n$ is precisely the number of windings of $z$.
So reparameterizations are continuous endomorphisms on $\bbT$ which travel
$\bbT$ exactly once, without any backtracking, but possibly with stoppings.

Note that every lifting $h\colon\bbR\to\bbR$ of a reparameterization $\phi\colon\bbT\to\bbT$
is a translation by some $n\in\bbZ$ of each other. In particular, there exists a lifting
that is a homeomorphism from $\bbR$ onto $\bbR$ if and only if all other liftings are
also homeomorphisms. Also, this happens exactly when a lifting $h$ is strictly increasing.

\begin{definition}
	An \emph{orientation-preserving homeomorphism} on $\bbT$ is a reparameterization
	$\phi\colon\bbT\to\bbT$ with a strictly increasing lifting $h\colon\bbR\to\bbR$.
\end{definition}

\begin{definition}
    Let $\gamma,\delta$ be closed paths in a topological space $X$, i.e., continuous functions
	from $\bbT$ into $X$. Then we say $\delta$ is a \emph{reparameterization} of $\gamma$
	if there exists a reparameterization $\phi\colon\bbT\to\bbT$ such that
	$\delta=\gamma\circ\phi$. Also, we say $\gamma$ and $\delta$ are \emph{equivalent}
    and denote as $\gamma\sim \delta$ if there exists a finite sequence
    $\gamma=\gamma_{1},\ \cdots\ ,\gamma_{n}=\delta$ of closed paths in $X$ such that
    for each $i=1,\ \cdots \,n-1$, either $\gamma_{i}$ is a reparameterization of
    $\gamma_{i+1}$ or vice versa. Clearly, this is an equivalence
	relation on the set of all closed paths on $X$, and each equivalence class
	is called a \emph{closed curve} in $X$. A representative of a closed curve
    $\gamma$ in $X$ is called a \emph{parameterization} of $\gamma$.
    We denote the set of all closed curves in $X$ as $\operatorname{Curve}(X)$.
\end{definition}

\textbf{Remark.} A constant path is not equivalent to any nonconstant path.\\

We will show that two paths $\gamma,\delta$ are equivalent if and only if
they are both reparameterizations of another path $\epsilon$.
To do so, we first show that any nonconstant path is a reparameterization of
an \emph{unstopping} path.

\begin{definition}
	A path $\gamma\colon[0,1]\to X$ in a topological space $X$ is said to be
	\emph{unstopping} if there is no nontrivial subinterval of $[0,1]$
    where $\gamma$ is constant.
\end{definition}

\begin{lemma}\label{LA.5}
	Let $\gamma\colon[0,1]\to X$ be any nonconstant path into a topological space $X$.
	Then there exists an unstopping path $\tilde{\gamma}\colon[0,1]\to X$
	and a non-decreasing continuous surjection $\phi\colon[0,1]\to[0,1]$ such that
	$\gamma = \tilde{\gamma}\circ\phi$.
\end{lemma}

\begin{proof}
	Consider the relation $\sim$ defined on $[0,1]$ as $s\sim t$ if and only if
	$\gamma$ is constant on the closed interval between $s$ and $t$. Clearly, $\sim$
	is an equivalence relation and each equivalence class is a (possibly degenerate)
	closed interval. Let $Y\coloneqq[0,1]/\sim$ and define a relation $\leq$ on $Y$ as
	$[s]\leq [t]$ if and only if $\inf[s]\leq\sup[t]$.
	We claim that $\leq$ is a total order and the order topology
	induced on $Y$ is precisely the quotient topology.

	First, we show that $[s]\leq[t]$ is equivalent to $\sup[s]\leq\sup[t]$.
	Indeed, suppose $\inf[s]\leq\sup[t]$. If $\sup[s]>\sup[t]$, then
	$z$ is constant on $\left[\inf[s],\sup[s]\right]\owns\sup[t]$, so
	$[s]=[t]$ follows, but then $\sup[s]>\sup[t]$ cannot be the case.
	Therefore, $[s]\leq[t]$ is equivalent to $\sup[s]\leq\sup[t]$.

	Then clearly $\leq$ is reflexive and transitive. Now, suppose
	$[s]\not\leq[t]$, that is, $\inf[s]>\sup[t]$. Then clearly $[s]\geq [t]$
	and $[s]\neq[t]$ follows, so $\leq$ is indeed a total order on $Y$.

	To show equivalence of topologies, let $[s]\in Y$. Then
	\[
		[[s],\infty) = \setbc{[t]\in Y}{\inf[s]\leq\sup[t]}
		= \setbc{[t]\in Y}{\inf[s]\leq t},
	\]
	because if $t<\inf[s]\leq\sup[t]$ holds, then $\gamma$ is constant on
	$[t,\sup[t]]\owns\inf[s]$, so $[s]=[t]$ should hold, which contradicts to
	$t<\inf[s]$. Therefore, let $\pi\colon [0,1]\to Y$ be the canonical projection, then
	\[
		\pi^{-1}\left[[[s],\infty)\right]=[\inf[s],1].
	\]
	Similarly,
	\[
		(-\infty,[s]] = \setbc{[t]\in Y}{\inf[t]\leq\sup[s]}
		= \setbc{[t]\in Y}{t\leq\sup[s]},
	\]
	so
	\[
		\pi^{-1}\left[(-\infty,[s]]\right]=[0,\sup[s]].
	\]
	This shows that each closed interval in $(Y,\leq)$ is closed in the quotient topology,
	thus the quotient topology is finer than or equal to the order topology
	induced by $\leq$. Since the quotient topology is compact and
	the order topology is Hausdorff, they should coincide to each other.
	Therefore, the claim is shown.

	As a result, the order topology of $(Y,\leq)$ is compact, connected, and separable.
	Since $\gamma$ is not globally constant, $Y$ has at least two points, thus
	$(Y,\leq)$ is order-isomorphic to $[0,1]$. Let $\psi\colon Y\to [0,1]$
	be an order-isomorphism, then it must be a homeomorphism as well.
	Now, by the universal property of the quotient topology, there uniquely exists a path
    $\bar{\gamma}\colon Y\to[0,1]$ such that $\gamma = \bar{\gamma}\circ\pi$. Hence, define
    $\phi\coloneqq\psi\circ\pi$, then $\phi\colon[0,1]\to[0,1]$ is a continuous surjection
	and $\gamma = (\bar{\gamma}\circ\psi^{-1})\circ\phi$. By definition of $\sim$,
    it is clear that $\tilde{\gamma}\coloneqq\bar{\gamma}\circ\psi^{-1}$ is unstopping.
	Also, $\phi$ is non-decreasing since both $\phi$ and $\pi$ are order-preserving.
	Indeed, if $s\leq t$, then $\inf[s]\leq s\leq t\leq \sup[t]$, so $[s]\leq [t]$ holds.
\end{proof}

The next lemma shows that if a sequence of reparameterizations
of an unstopping path $\gamma$ converges to a path,
then the limit path is also a reparameterization of $\gamma$.

\begin{lemma}\label{LA.6}
	Let $X$ be a metric space and $\gamma\colon\bbT\to X$ be an unstopping
	closed path in $X$. Let $\seq{\phi_{\alpha}\colon\bbT\to\bbT}_{\alpha\in D}$ be a net of
	reparameterizations such that the net $\seq{\gamma\circ\phi_{\alpha}}_{\alpha\in D}$
    converges uniformly to a closed path $\delta\colon\bbT\to X$. Then there exists a
	reparameterization $\phi\colon\bbT\to\bbT$ such that $\delta = \gamma\circ\phi$.
	Furthermore, if $\delta$ is unstopping, then $\phi$ can be chosen to be
	an orientation-preserving homeomorphism.
\end{lemma}

\begin{proof}
	For each $\alpha\in D$, consider the lifting $\tilde{\phi}_{\alpha}\colon[0,1]\to\bbR$
	of $\phi_{\alpha}$ such that $\tilde{\phi}_{\alpha}(0) \in [0,1)$. Since the image of
	$\tilde{\phi}_{\alpha}$ is contained in $[0,2]$, by passing to a subnet if necessary,
	we may assume that the net $\seq{\tilde{\phi}_{\alpha}}_{\alpha\in D}$
	converges pointwise to a non-decreasing function $\tilde{\phi}\colon[0,1]\to[0,2]$.
	By the assumption on $\seq{\phi_{\alpha}}_{\alpha\in D}$, we know that
	$\gamma(\tilde{\phi}(t)+\bbZ) = \delta(t+\bbZ)$ holds for all $t\in[0,1]$.
    In particular if $\delta$ is unstopping, then this shows that $\tilde{\phi}$ is
    actually strictly increasing. Also, since
    $\tilde{\phi}_{\alpha}(1) = \tilde{\phi}_{\alpha}(0) + 1$
	for all $\alpha\in D$, we have $\tilde{\phi}(1) = \tilde{\phi}(0) + 1$.
	
	We claim that $\tilde{\phi}$ is continuous. Fix any $t_{0}\in(0,1]$.
	We first show that the left limit $\tilde{\phi}(t_{0}^{-})$ coincides with
	$\tilde{\phi}(t_{0})$. For the sake of contradiction, suppose otherwise, that is,
	$\tilde{\phi}(t_{0}^{-})$ is strictly less than $\tilde{\phi}(t_{0})$.
	Take any $s\in\left(\tilde{\phi}(t_{0}^{-}),\tilde{\phi}(t_{0})\right)$, then
	for any $t<t_{0}$, we have $s\in\left(\tilde{\phi}(t),\tilde{\phi}(t_{0})\right)$,
	so $s\in\left(\tilde{\phi}_{\alpha}(t),\tilde{\phi}_{\alpha}(t_{0})\right)$
	holds for all large enough $\alpha\in D$. Hence,
	\[
		\gamma(s+\bbZ) \in (\gamma\circ\phi_{\alpha})\left((t,t_{0})+\bbZ\right)
	\]
	holds for such $\alpha\in D$, and since $\seq{\gamma\circ\phi_{\alpha}}_{\alpha\in D}$
	converges uniformly to $\delta$, for any given $\epsilon>0$,
	the right-hand side of the above is contained in the $\epsilon$-neighborhood of
	$\delta\left((t,t_{0})+\bbZ\right)$ if $\alpha$ is large enough.
	Since this holds for any $\epsilon>0$, we conclude
	\[
		\gamma(s+\bbZ) \in \overline{\delta\left((t,t_{0})+\bbZ\right)},
	\]
	and since this holds for all $t<t_{0}$ and $\delta$ is continuous,
	we conclude $\gamma(s+\bbZ) = \delta(t_{0}+\bbZ)$. Therefore, it follows that $\gamma$
	is constant on the interval $\left(\tilde{\phi}(t_{0}^{-}),\tilde{\phi}(t_{0})\right)$,
	contradicting to the assumption that $\gamma$ is unstopping. Therefore, we must have
	$\tilde{\phi}(t_{0}^{-}) = \tilde{\phi}(t_{0})$. By the same argument,
	we can see that $\tilde{\phi}(t_{0}^{+}) = \tilde{\phi}(t_{0})$ holds for all
	$t\in[0,1)$, proving the claim that $\tilde{\phi}$ is continuous.
	Therefore, we get the desired reparameterization $\phi\colon\bbT\to\bbT$ given as
	$\phi(t+\bbZ)\coloneqq\tilde{\phi}(t) + \bbZ$ for $t\in[0,1)$.
\end{proof}

\begin{lemma}\label{LA.7}
	Let $\varphi\colon\bbT\to\bbT$ be a reparameterization. Then for any $\epsilon>0$,
	there exists an orientation-preserving homeomorphism $\psi\colon\bbT\to\bbT$
	such that $\max_{t\in\bbT}d(\varphi(t),\psi(t))\leq\epsilon$.
\end{lemma}

\begin{proof}
	Let $\tilde{\varphi}\colon[0,1]\to\bbR$ be a lifting of $\varphi$.
	For each $\delta>0$, define $\tilde{\varphi}_{\delta}\colon[0,1]\to\bbR$ as
	\[
		\tilde{\varphi}_{\delta}\colon t\mapsto
		\tilde{\varphi}(0) +
		\frac{\tilde{\varphi}(t) - \tilde{\varphi}(0) + \delta t}{1+\delta}.
	\]
	Since $\tilde{\varphi}$ is non-decreasing, $\tilde{\varphi}_{\delta}$ is strictly increasing.
	Furthermore, we have $\tilde{\varphi}_{\delta}(0) = \tilde{\varphi}(0)$,
	$\tilde{\varphi}_{\delta}(1) = \tilde{\varphi}(1)$, and
	\[
		\abs{\tilde{\varphi}_{\delta}(t) - \tilde{\varphi}(t)}
		= \frac{\delta}{1+\delta}\abs{\tilde{\varphi}(0) + t - \tilde{\varphi}(t)}
	\]
	for all $t\in[0,1]$. Hence, define $\varphi_{\delta}\colon\bbT\to\bbT$ as
	$\varphi_{\delta}(t+\bbZ)\coloneqq\tilde{\varphi}_{\delta}(t)+\bbZ$ for $t\in[0,1)$, then
	each $\varphi_{\delta}$ is an orientation-preserving homeomorphism and
	$\max_{t\in\bbT}d(\varphi(t),\varphi_{\delta}(t)) \leq \frac{\delta}{1+\delta}$ holds.
\end{proof}

\begin{corollary}\label{CA.8}
	Let $X$ be a metric space. Let $\gamma,\delta\colon\bbT\to X$ be
	unstopping closed paths in $X$ and $\phi,\psi\colon\bbT\to\bbT$ be
	reparameterizations such that $\gamma\circ\phi = \delta\circ\psi$. Then there exists an
	orientation-preserving homeomorphism $\eta\colon\bbT\to\bbT$ such that
    $\delta=\gamma\circ\eta$.
\end{corollary}

\begin{proof}
	By Lemma~\ref{LA.7}, we can find a
	sequence $\seq{\psi_{n}\colon\bbT\to\bbT}_{n=1}^{\infty}$ of
	orientation-preserving homeomorphisms converging uniformly to $\psi$.
	Note that this implies that the sequence $\seq{\psi\circ\psi_{n}^{-1}}_{n=1}^{\infty}$
	uniformly converges to the identity map because
	\[
		\sup_{t\in\bbT}d(\psi\circ\psi_{n}^{-1}(t),t)
		=\sup_{s\in\bbT}d(\psi(s),\psi_{n}(s)) \to 0
	\]
	as $n\to\infty$, since each $\psi_{n}$ is a bijection.
	Since $\delta\colon\bbT\to X$ is uniformly continuous, it follows that the sequence
	\[
		\seq{\gamma\circ\phi\circ\psi_{n}^{-1}}_{n=1}^{\infty}
		= \seq{\delta\circ\psi\circ\psi_{n}^{-1}}_{n=1}^{\infty}
	\]
	converges uniformly to the unstopping closed path $\delta$, thus by
	Lemma~\ref{LA.6}
	we obtain an orientation-preserving homeomorphism $\eta\colon\bbT\to\bbT$
	such that $\delta = \gamma\circ\eta$.
\end{proof}

\begin{theorem}\label{TA.9}
	Let $X$ be a metric space and $\gamma,\delta\colon\bbT\to X$ be
	closed paths in $X$. Then $\gamma$ and $\delta$ are equivalent if and only if
	there exist an unstopping closed path $\epsilon\colon\bbT\to X$ and reparameterizations
	$\phi,\psi\colon\bbT\to\bbT$ such that
	$\gamma = \epsilon\circ\phi$ and $\delta = \epsilon\circ\psi$.
\end{theorem}

\begin{proof}
	The ``if'' direction is trivial, so we only show the converse. Assume $\gamma\sim \delta$,
	so there exists a finite sequence $\gamma=\gamma_{1},\ \cdots\ ,\gamma_{n}=\delta$
	of closed paths in $X$ such that for each $i=1,\ \cdots\ ,n-1$, either $\gamma_{i}$ is a
	reparameterization of $\gamma_{i+1}$ or vice versa. The desired conclusion is clear
	if any of these are constant paths, so we may assume every
	$\gamma_{i}$ is a nonconstant path. We use induction on $n$.

	The base case $n=1$ follows immediately from
	Lemma~\ref{LA.5}, so we only prove
	the induction step. Suppose there exists an unstopping closed path
	$\epsilon\colon\bbT\to X$ and reparameterizations $\phi_{1},\phi_{n-1}\colon\bbT\to\bbT$
	such that $\gamma_{1}=\epsilon\circ\phi_{1}$ and $\gamma_{n-1}=\epsilon\circ\phi_{n-1}$.
	If $\gamma_{n}$ is a reparameterization of $\gamma_{n-1}$, then we are done, so
	assume that $\gamma_{n-1}$ is a reparameterization of $\gamma_{n}$.
	By Lemma~\ref{LA.5}, $\gamma_{n}$ is a
	reparameterization of an unstopping closed path $\zeta\colon\bbT\to X$, so there exists
	a reparameterization $\psi\colon\bbT\to\bbT$ such that
	$\gamma_{n-1} = \zeta\circ\psi$. Hence,
	\[
		\epsilon\circ\phi_{n-1} = \gamma_{n-1} = \zeta\circ\psi,
	\]
	so Corollary~\ref{CA.8} shows that
	there exists an orientation-preserving homeomorphism $\eta\colon\bbT\to\bbT$
	such that $\zeta = \epsilon\circ\eta$. Then it follows that
	$\gamma=\gamma_{1}$ and $\delta=\gamma_{n}$ are both reparameterizations of $\epsilon$.
\end{proof}

Next, we show that the notion of equivalence of paths in terms of reparameterization
can be equivalently characterized in terms of the Fr\'{e}chet metric, as claimed in
the second remark after Definition~\ref{D2.1}.

\begin{definition}
	Let $(X,d)$ be a metric space and $\gamma_{1},\gamma_{2}$ be closed paths in $X$.
	Then the \emph{Fr\'{e}chet distance} between $\gamma_{1}$ and $\gamma_{2}$ is defined as
    \[
        d_{\mathrm{F}}(\gamma_{1},\gamma_{2})\coloneqq
        \inf_{\phi\colon\bbT\to\bbT}d_{\infty}(\gamma_{1}, \gamma_{2}\circ\phi)
    \]
    where $d_{\infty}$ is the uniform metric and $\phi$ ranges over all
    orientation-preserving homeomorphisms.
\end{definition}

Recall that in general on a pseudometric space $(X,d)$ and an equivalence relation $\sim$,
the \emph{quotient pseudometric} is the pseudometric $\tilde{d}$ on $X/\sim$ defined as
\begin{align*}
	\tilde{d}\colon ([x],[y]) \mapsto \inf\sum_{i=1}^{n}d(x_{i},y_{i})
\end{align*}
where $[x],[y]$ denote the equivalence classes which $x,y$ respectively belong to,
and the infimum is taken over all finite sequences
$\seq{x_{i}}_{i=1}^{n}$ and $\seq{y_{i}}_{i=1}^{n}$ such that
$[x] = [x_{1}]$, $[y] = [y_{n}]$, and $[x_{i+1}] = [y_{i}]$ for
all $i=1,\ \cdots\ ,n-1$. Then it turns out that the Fr\'{e}chet metric is precisely
the quotient pseudometric obtained from the uniform metric $d_{\infty}$
and the equivalence of paths we defined in terms of reparameterizations.

\begin{theorem}\label{TA.11}
	Let $(X,d)$ be a metric space. Then $d_{\mathrm{F}}$ coincides with the
	quotient pseudometric on $\operatorname{Curve}(X)$ induced by the uniform metric
	$d_{\infty}$. Furthermore, $d_{\mathrm{F}}$ is in fact a metric on
	$\operatorname{Curve}(X)$. If $(X,d)$ is complete, then so is
	$(\operatorname{Curve}(X),d_{\mathrm{F}})$.
\end{theorem}

\begin{proof}
	Clearly, $d_{\mathrm{F}}$ is a pseudometric on the space of all closed paths in $X$.
	We first show that if $\gamma,\delta$ are equivalent closed paths in $X$, then
	$d_{\mathrm{F}}(\gamma,\delta) = 0$, which in particular shows that $d_{\mathrm{F}}$ is
	a well-defined pseudometric on $\operatorname{Curve}(X)$.
	Since $d_{\mathrm{F}}$ is a pseudometric on the space of all closed paths in $X$,
	it is enough to show that $d_{\mathrm{F}}(\gamma,\delta) = 0$ holds whenever
	$\delta = \gamma\circ\phi$ for some reparameterization $\phi\colon\bbT\to\bbT$.
	This easily follows from Lemma~\ref{LA.7}.

	Let $\gamma,\delta\colon\bbT\to X$ be closed paths, then the quotient pseudometric
	between them is given as
	\[
		d([\gamma],[\delta])\coloneqq \inf \sum_{i=1}^{n}d_{\infty}(\gamma_{i},\delta_{i}),
	\]
	where the infimum is taken over all finite sequences $\seq{\gamma_{i}}_{i=1}^{n}$
	and $\seq{\delta_{i}}_{i=1}^{n}$ such that $[\gamma]=[\gamma_{1}]$,
	$[\delta]=[\delta_{n}]$, and $[\gamma_{i+1}]=[\delta_{i}]$ for all $i=1,\ \cdots\ ,n-1$.
	Then we clearly have $d([\gamma],[\delta])\leq d_{\mathrm{F}}(\gamma,\delta)$.
	For the other direction, pick any finite sequences $\seq{\gamma_{i}}_{i=1}^{n}$ and
	$\seq{\delta_{i}}_{i=1}^{n}$ such that $[\gamma]=[\gamma_{1}]$, $[\delta]=[\delta_{n}]$,
	and $[\gamma_{i+1}]=[\delta_{i}]$ for all $i=1,\ \cdots\ ,n-1$.
	Then since $d_{\mathrm{F}}(\gamma,\gamma_{1}) = 0$
	and $d_{\mathrm{F}}(\delta_{n},\delta) = 0$, for given $\epsilon>0$, we can find
	orientation-preserving homeomorphisms $\phi_{0},\phi_{n}\colon\bbT\to\bbT$ such that
	$d_{\infty}(\gamma,\gamma_{1}\circ\phi_{0})$ and
	$d_{\infty}(\delta_{n},\delta\circ\phi_{n})$ are
	both bounded by $\frac{\epsilon}{n+1}$. Similarly, we can find orientation-preserving
	homeomorphisms $\phi_{1},\ \cdots\ ,\phi_{n-1}$ such that
	$d_{\infty}(\delta_{i},\gamma_{i+1}\circ\phi_{i}) \leq \frac{\epsilon}{n+1}$ holds
	for all $i=1,\ \cdots\ ,n-1$. Then,
	\begin{align*}
		\sum_{i=1}^{n}d_{\infty}(\gamma_{i},\delta_{i}) + \epsilon &\geq
		d_{\infty}(\gamma,\gamma_{1}\circ\phi_{0}) +
		\sum_{i=1}^{n-1}\left(
			d_{\infty}(\gamma_{i},\delta_{i})
			+ d_{\infty}(\delta_{i},\gamma_{i+1}\circ\phi_{i})
		\right)
		\\&\quad\quad\quad
		+ d_{\infty}(\gamma_{n},\delta_{n}) + d_{\infty}(\delta_{n},\delta\circ\phi_{n}) \\
		&\geq d_{\infty}(\gamma,\delta\circ\phi) \geq d_{\mathrm{F}}(\gamma,\delta)
	\end{align*}
	where $\phi\coloneqq\phi_{n}\circ\ \cdots\ \circ\phi_{0}$.
	Therefore, we get $d([\gamma],[\delta])\geq d_{\mathrm{F}}(\gamma,\delta)$, so
	$d_{\mathrm{F}}$ is precisely the quotient pseudometric.

	Next, we show that $d_{\mathrm{F}}$ is in fact a metric on $\operatorname{Curve}(X)$,
	which means that $d_{\mathrm{F}}(\gamma,\delta) = 0$ implies $\gamma\sim \delta$.
	If $\gamma$ is a constant path, then $d_{\mathrm{F}}(\gamma,\delta) = 0$ implies
	$\gamma=\delta$, so suppose that $\gamma$ is not a constant path. Then
	by Lemma~\ref{LA.5}, we can find an
	unstopping closed path $\epsilon\colon\bbT\to X$ such that $\gamma$
	is a reparameterization of $\epsilon$. Then since $\gamma\sim \epsilon$, we have
	$d_{\mathrm{F}}(\epsilon,\delta) = 0$ as well, which means that
	there exists a sequence $\seq{\phi_{n}}_{n=1}^{\infty}$ of orientation-preserving
	homeomorphisms such that $d_{\infty}(\epsilon,\delta\circ\phi_{n})\to 0$ as $n\to\infty$.
	Thus, Lemma~\ref{LA.6}
	shows that $\delta$ is a reparameterization of $\epsilon$, so
	$\gamma\sim \epsilon\sim \delta$.

	Finally, we show that if $(X,d)$ is complete, then so is
	$(\operatorname{Curve}(X),d_{\mathrm{F}})$. Let $\seq{[\gamma_{n}]}_{n=1}^{\infty}$
	be a Cauchy sequence in $(\operatorname{Curve}(X),d_{\mathrm{F}})$.
	Without loss of generality, we may assume that
	$d_{\mathrm{F}}(\gamma_{n},\gamma_{n+1})<\frac{1}{2^{n}}$ for all $n$. Then
	for each $n$, we can find an orientation-preserving homeomorphism
	$\phi_{n}\colon\bbT\to\bbT$ such that
	$d_{\infty}(\gamma_{n},\gamma_{n+1}\circ\phi_{n})<\frac{1}{2^{n}}$.
	Define $\delta_{1}\coloneqq \gamma_{1}$ and
	$\delta_{n}\coloneqq \gamma_{n}\circ\phi_{n-1}\circ\ \cdots\ \circ\phi_{1}$ for $n>1$,
	then it follows that $\seq{\delta_{n}}_{n=1}^{\infty}$ is a Cauchy sequence
	in the space of all continuous functions $\bbT\to X$ equipped with the
	uniform metric. Since $X$ is complete, there exists a closed path
	$\gamma\colon\bbT\to X$ such that $\delta_{n}\to \gamma$ uniformly as $n\to\infty$.
	Since $\delta_{n}\sim \gamma_{n}$, it follows that
	\[
		d_{\mathrm{F}}(\gamma_{n},\gamma) = d_{\mathrm{F}}(\delta_{n},\gamma)
		\leq d_{\infty}(\delta_{n},\gamma) \to 0
	\]
	as $n\to\infty$, thus $\seq{[\gamma_{n}]}_{n=1}^{\infty}$ converges to
	$[\gamma]\in\operatorname{Curve}(X)$.
\end{proof}

\begin{lemma}\label{LA.12}
    Let $(X,d)$ be a metric space. Then for any
    $\gamma_{1},\gamma_{2}\in \operatorname{Curve}(X)$,
    \[
        d_{\mathrm{F}}(\gamma_{1},\gamma_{2}) =
        \inf_{\phi\colon\bbT\to\bbT}\norm{\gamma_{1} - \gamma_{2}\circ\phi}_{C^{0}}
    \]
    holds where $\phi$ ranges over every orientation-preserving
    \emph{diffeomorphism} from $\bbT$ onto $\bbT$.
\end{lemma}

\begin{proof}
    It is enough to show that for any orientation-preserving homeomorphism
    $\phi\colon\bbT\to\bbT$, we can find an orientation-preserving diffeomorphism
    $\psi\colon\bbT\to\bbT$ uniformly approximating $\phi$ arbitrarily well.
    Let $\tilde{\phi}\colon\bbR\to\bbR$ be the lifting of $\phi$, then we can set
    $\psi$ to be the projection of the function $\tilde{\psi}\colon\bbR\to\bbR$ given as
    \[
        \tilde{\psi}\colon x\mapsto
        \tilde{\phi}*\eta(0) +
        \frac{\tilde{\phi}*\eta(x) - \tilde{\phi}*\eta(0) + \delta x}{1+\delta},
    \]
    where $\eta\colon\bbR\to[0,\infty)$ is a smooth bump function with $\int\eta = 1$
    sufficiently close to the Dirac measure and $\delta>0$ is sufficiently small.
    Details are omitted.
\end{proof}


%%%%%%%%%%%%%%%%%%%%%%%%%%%%%%%%%%%%%%%%%%%%%%
\section{Compact subsets of $\operatorname{Curve}(\bbR^{n})$} \label{SB}
%%%%%%%%%%%%%%%%%%%%%%%%%%%%%%%%%%%%%%%%%%%%%%

In this section, we collect some results related to compactness
in $\operatorname{Curve}(\bbR^{2})$ we used in the paper.
Since there is nothing specific about the dimension in these results,
we state them over $\bbR^{n}$. All the functionals on curves we defined for $\bbR^{2}$
trivially extends in this setting.

\begin{proposition}\label{PB.1}
    For any $R_{0},R_{1}\in[0,\infty)$, the set
    \[
        A\coloneqq\setbc{\gamma\in\operatorname{Curve}(\bbR^{n})}
        {\norm{\gamma}_{C^{0}} \leq R_{0},\ 
        \ell(\gamma) \leq R_{1}}
    \]
    is compact with respect to the metric $d_{\mathrm{F}}$.
\end{proposition}

\begin{proof}
    For each $\gamma\in A$, choose any constant-speed parameterization
    $\tilde{\gamma}\in C(\bbT;\bbR^{n})$, and let $\tilde{A}$ be the resulting
    subset of $C(\bbT;\bbR^{n})$. By the definition of constant-speed parameterizations,
    it can be easily seen that any $\tilde{\gamma}\in\tilde{A}$ is Lipschitz continuous
    with the Lipschitz constant precisely equal to $\ell(\gamma)$.
    Therefore, $\tilde{A}$ is a family of functions with bounded ranges and
    bounded Lipschitz constants, thus is relatively compact in $C(\bbT;\bbR^{n})$
    by Arzel\`{a}-Ascoli theorem. Hence, let $\seq{\gamma_{n}}_{n=1}^{\infty}$
    be any sequence in $A$, then there exists a subsequence
    $\seq{\gamma_{n_{k}}}_{k=1}^{\infty}$ so that the corresponding sequence
    $\seq{\tilde{\gamma}_{n_{k}}}_{k=1}^{\infty}$ is convergent to some
    $\tilde{\gamma} \in C(\bbT;\bbR^{n})$.
    Since the $\norm{\,\cdot\,}_{C^{0}}$-norms and the Lipschitz constants of
    $\tilde{\gamma}_{n}$'s are uniformly bounded by $R_{0}$ and $R_{1}$, respectively,
    $\tilde{\gamma}$ should satisfy the same bound. Hence, let $\gamma$ be the curve
    corresponding to $\tilde{\gamma}$, then we must have $\gamma\in A$.
    Since $\tilde{\gamma}_{n_{k}}\to\tilde{\gamma}$, we must have
    $\gamma_{n_{k}}\to\gamma$, thus $A$ is compact.
\end{proof}

\begin{corollary}\label{CB.2}
    The functional $\ell\colon\operatorname{Curve}(\bbR^{n})\to[0,\infty]$
    is lower semicontinuous, thus in particular Borel measurable.
\end{corollary}

\begin{proof}
    Given $R_{1}\in[0,\infty)$, it is enough to show that the set
    $A\coloneqq \setbc{\gamma\in\operatorname{Curve}(\bbR^{n})}{\ell(\gamma)\leq R_{1}}$
    is closed in $\operatorname{Curve}(\bbR^{n})$. Let $\seq{\gamma_{n}}_{n=1}^{\infty}$
    be any sequence in this set convergent to some $\gamma\in\operatorname{Curve}(\bbR^{n})$.
    Then it can be easily seen that there exists $R_{0}\in[0,\infty)$ such that
    $\sup_{n}\norm{\gamma_{n}}_{C^{0}}\leq R_{0}$, thus the sequence is contained in a
    compact subset of $A$, so the limit $\gamma$ must be in $A$ as well.
    Therefore, $A$ is closed.
\end{proof}

Next, we prove similar statements about the functional $\mathcal{K}_{2}$.

\begin{proposition}\label{PB.3}
    For any $R_{0},R_{2}\in[0,\infty)$, the set
    \[
        A\coloneqq\setbc{\gamma\in\operatorname{Curve}(\bbR^{n})}
        {\norm{\gamma}_{C^{0}} \leq R_{0},\ 
        \mathcal{K}_{2}(\gamma) \leq R_{2}}
    \]
    is compact with respect to the metric
    \[
        d\colon(\gamma_{1},\gamma_{2})
        \mapsto \inf_{\tilde{\gamma}_{1}\in\gamma_{1},
        \tilde{\gamma}_{2}\in\gamma_{2}}
        \left(
            \norm{\tilde{\gamma}_{1} - \tilde{\gamma}_{2}}_{C^{1}(\bbT;\bbR^{n})}
            + \sum_{i=1}^{\infty}\frac{1}{2^{i}}
            \abs{\int_{\bbT}g_{i}\cdot\partial_{\xi}^{2}
            (\tilde{\gamma}_{1} - \tilde{\gamma}_{2})}
        \right)
    \]
    where $\set{g_{i}}_{i=1}^{\infty}$ is any dense subset of the unit ball of
    $L^{2}(\bbT;\bbR^{n})$ and the infimum is taken over all \emph{constant-speed parameterizations}
    of $\gamma_{1}$ and $\gamma_{2}$. Therefore, $d$ and $d_{\mathrm{F}}$ are
    bi-uniformly equivalent on $A$. In particular, $\ell$ is uniformly continuous
    with respect to $d_{\mathrm{F}}$ on $A$.
\end{proposition}

\begin{proof}
    For each $\gamma\in A$, choose any constant-speed parameterization
    $\tilde{\gamma}\in C(\bbT;\bbR^{n})$, and let $\tilde{A}$ be the resulting
    subset of $C(\bbT;\bbR^{n})$. Note that any constant-speed parameterization
    $\tilde{\gamma}$ of $\gamma\in A$ should be in $H^{2}(\bbT;\bbR^{2})$,
    and since $\ell(\gamma) \leq \norm{\gamma}_{C^{0}}^{2}\mathcal{K}_{2}(\gamma)$ holds,
    any $\gamma\in A$ satisfies $\norm{\tilde{\gamma}}_{C^{0}}\leq R_{0}$ and
    $\norm{\partial_{\xi}^{2}\tilde{\gamma}}_{L^{2}}^{2}
    = \ell(\gamma)^{3}\mathcal{K}_{2}(\gamma) \leq
    \norm{\gamma}_{C^{0}}^{6}\mathcal{K}_{2}(\gamma)^{4}
    \leq R_{0}^{6}R_{2}^{4}$, so $d$ is well-defined and bounded on $A$.
    Since elements in $\tilde{A}$ have a uniform $C^{1,1/2}$-norm bound,
    Arzel\`{a}-Ascoli theorem shows that $\tilde{A}$ is relatively compact in
    $C^{1}(\bbT;\bbR^{n})$. Hence, let $\seq{\gamma_{n}}_{n=1}^{\infty}$ be any
    sequence in $A$, then there exists a subsequence $\seq{\gamma_{n_{k}}}_{k=1}^{\infty}$
    such that $\seq{\tilde{\gamma}_{n_{k}}}_{k=1}^{\infty}$ converges to some
    $\tilde{\gamma}\in C^{1}(\bbT;\bbR^{n})$ in the $C^{1}$-norm.
    Let $\gamma\in\operatorname{Curve}(\bbR^{n})$ be the curve corresponding to
    $\tilde{\gamma}$, then because of the $C^{1}$-convergence, $\tilde{\gamma}$
    should be a constant-speed parameterization of $\gamma$. Also, since
    $\norm{\partial_{\xi}^{2}\tilde{\gamma}_{n_{k}}}_{L^{2}}$ is uniformly bounded,
    we can assume by passing to a further subsequence that
    $\seq{\partial_{\xi}^{2}\tilde{\gamma}_{n_{k}}}_{k=1}^{\infty}$
    is weakly convergent to some function $f\in L^{2}(\bbT;\bbR^{2})$,
    then it immediately follows that $f$ is the weak derivative of
    $\partial_{\xi}\tilde{\gamma}$. Finally, since we have
    $\ell(\gamma_{n_{k}}) \to \ell(\gamma)$ by the $C^{1}$-convergence,
    \begin{align*}
        \norm{f}_{L^{2}} &\leq
        \liminf_{k\to\infty}\norm{\partial_{\xi}^{2}\tilde{\gamma}_{n_{k}}}_{L^{2}}
        = \liminf_{k\to\infty}\ell(\gamma_{n_{k}})^{3}
        \mathcal{K}_{2}(\gamma_{n_{k}})
        = \ell(\gamma)^{3}\liminf_{k\to\infty}\mathcal{K}_{2}(\gamma_{n_{k}}),
    \end{align*}
    thus we conclude
    \[
        \mathcal{K}_{2}(\gamma) \leq
        \liminf_{k\to\infty}\mathcal{K}_{2}(\gamma_{n_{k}}) \leq R_{2}.
    \]
    Therefore, we deduce $\gamma\in A$, and clearly
    $d(\gamma_{n_{k}},\gamma) \to 0$ since $\tilde{\gamma}_{n_{k}} \to \tilde{\gamma}$
    in $C^{1}$ and $\partial_{\xi}^{2}\tilde{\gamma}_{n_{k}} \to \partial_{\xi}^{2}\gamma$
    weakly in $L^{2}$. Therefore, compactness of $A$ follows.
    The bi-uniform equivalence of the metrics immediately follows,
    because $d_{\mathrm{F}}\leq d$ is obvious and $d_{\mathrm{F}}$ gives a
    Hausdorff topology. Uniform continuity of $\ell$ with respect to $d_{\mathrm{F}}$
    is also clear, because $\ell$ is obviously uniformly continuous
    with respect to $d$.
\end{proof}

\begin{corollary}\label{CB.4}
    The functional $\mathcal{K}_{2}\colon\operatorname{Curve}(\bbR^{n})\to[0,\infty]$
    is lower semicontinuous, thus in particular Borel measurable.
\end{corollary}

\begin{proof}
    Repeat the same argument as in
    Corollary~\ref{CB.2}.
\end{proof}

We can prove a similar statement for H\"{o}lder norms.

\begin{proposition}\label{PB.5}
    For any $R_{0},R_{1},R_{2}\in[0,\infty)$, the set
    \[
        A\coloneqq\setbc{\gamma\in\operatorname{Curve}(\bbR^{n})}
        {\norm{\gamma}_{C^{0}} \leq R_{0},\ 
        \ell(\gamma) \leq R_{1},\ 
        \norm{\gamma}_{\dot{C}^{1,\beta}} \leq R_{2}}
    \]
    is compact with respect to the metric
    \[
        d\colon(\gamma_{1},\gamma_{2})
        \mapsto \inf_{\tilde{\gamma}_{1}\in\gamma_{1},
        \tilde{\gamma}_{2}\in\gamma_{2}}
        \norm{\tilde{\gamma}_{1} - \tilde{\gamma}_{2}}_{C^{1}(\bbT;\bbR^{n})}
    \]
    where the infimum is taken over all \emph{constant-speed parameterizations}
    of $\gamma_{1}$ and $\gamma_{2}$. Therefore, $d$ and $d_{\mathrm{F}}$ are
    bi-uniformly equivalent on $A$. In particular, $\ell$ is uniformly continuous
    with respect to $d_{\mathrm{F}}$ on $A$.
\end{proposition}

\begin{proof}
    Given a rectifiable curve $\gamma$ parameterized by the arclength,
    if $\tilde{\gamma}\in C(\bbT;\bbR^{n})$ is a constant-speed parameterization of $\gamma$,
    then $\norm{\partial_{\xi}\tilde{\gamma}}_{\dot{C}^{\beta}}
    = \ell(\gamma)^{1+\beta}\norm{\partial_{s}\gamma}_{\dot{C}^{\beta}}$ holds.
    Hence, if we let $\tilde{A}$ be any subset of $C^{1}(\bbT;\bbR^{n})$
    obtained by finding a constant-speed parameterization of each
    $\gamma\in A$, then $\tilde{A}$ has a uniformly bounded
    $C^{1,\beta}$-norm. Therefore, for any sequence
    $\seq{\gamma_{n}}_{n=1}^{\infty}$ in $A$, there exists a subsequence
    $\seq{\gamma_{n_{k}}}_{k=1}^{\infty}$ so that the corresponding sequence
    $\seq{\tilde{\gamma}_{n_{k}}}_{k=1}^{\infty}$ in $\tilde{A}$ is convergent
    to some $\tilde{\gamma}\in C^{1}(\bbT;\bbR^{n})$ with respect to the $C^{1}$-norm.
    By the $C^{1}$-convergence, $\tilde{\gamma}$ should be a constant-speed parameterization
    of a curve $\gamma$. Also, note that
    \[
        \norm{\partial_{\xi}\tilde{\gamma}}_{\dot{C}^{\beta}}
        \leq \limsup_{k\to\infty}\norm{\partial_{\xi}\tilde{\gamma}_{n_{k}}}_{\dot{C}^{\beta}}.
    \]
    Since we know $\ell(\gamma_{n_{k}}) \to \ell(\gamma)$ from the $C^{1}$-convergence,
    this implies
    \begin{align*}
        \ell(\gamma)^{1+\beta}\norm{\gamma}_{\dot{C}^{1,\beta}}
        \leq \limsup_{k\to\infty}\ell(\gamma_{n_{k}})^{1+\gamma}
        \norm{\gamma_{n_{k}}}_{\dot{C}^{1,\beta}}
        = \ell(\gamma)^{1+\beta}
        \limsup_{k\to\infty}\norm{\gamma_{n_{k}}}_{\dot{C}^{1,\beta}},
    \end{align*}
    thus $\norm{\gamma}_{\dot{C}^{1,\beta}} \leq R_{2}$.
    This shows $\gamma\in A$, and clearly $d(\gamma_{n_{k}},\gamma)
    \leq \norm{\tilde{\gamma}_{n_{k}} - \tilde{\gamma}}_{C^{1}} \to 0$,
    thus $A$ is compact.
\end{proof}

Since a bound on $\norm{\,\cdot\,}_{\dot{C}^{1,\beta}}$ and $\norm{\,\cdot\,}_{C^{0}}$
does not give a bound on $\ell$, we prove the following lemma to show
lower semicontinuity of $\norm{\,\cdot\,}_{\dot{C}^{1,\beta}}$.

\begin{lemma}\label{LB.6}
    For any $\beta\in(0,1]$ and $M\in[0,\infty)$, there exists
    $\delta(\beta,M)>0$ such that any pair of rectifiable curves
    $\gamma_{0},\gamma\in\operatorname{Curve}(\bbR^{n})$ with
    with $\norm{\gamma}_{\dot{C}^{1,\beta}} \leq M$ and
    $d_{\mathrm{F}}(\gamma,\gamma_{0}) < \delta(\beta,M)$ satisfies
    $\ell(\gamma) \leq \frac{4\ell(\gamma_{0}) + 1}{\beta}$.
\end{lemma}

\begin{proof}
    Set
    \[
        \delta \coloneqq \frac{\beta}{2(1+\beta)}\left(\frac{1}{(1+\beta)M}\right)^{1/\beta}.
    \]
    Take any rectifiable curves $\gamma_{0},\gamma\in\operatorname{Curve}(\bbR^{n})$ with
    $\norm{\gamma}_{\dot{C}^{1,\beta}} \leq M$ and
    $d_{\mathrm{F}}(\gamma,\gamma_{0}) < \delta$.
    Let $\gamma_{0}\colon\ell(\gamma_{0})\to\bbR^{n}$, $\gamma\colon\ell(\gamma)\bbT\to\bbR^{n}$
    be arclength parameterizations of $\gamma_{0}$, $\gamma$, respectively,
    then there exists an orientation-preserving homeomorphism
    $\phi\colon\bbT\to\bbT$ such that
    \[
        \abs{\gamma(\ell(\gamma)\xi) - \gamma_{0}(\ell(\gamma_{0})\phi(\xi))} < \delta
    \]
    holds for all $\xi\in\bbT$. Since $\gamma_{0}$ is parameterized by the arclength,
    there exists proper compact subintervals $I_{1},\ \cdots\ ,I_{N}$ of $\bbT$
    covering $\bbT$ with $N\leq \frac{\ell(\gamma_{0}) + 1/4}{\delta}$ such that
    \[
        \abs{\gamma_{0}(\xi) - \gamma_{0}(\xi')} < \delta
    \]
    holds whenever $\xi,\xi' \in I_{k}$ for some $k=1,\ \cdots\ ,N$.

    Now, take any $s,s' \in \ell(\gamma)\phi^{-1}(I_{k})$ for some
    $k=1,\ \cdots\ ,N$, then we have
    \[
        \abs{\gamma(s) - \gamma(s')} < 2\delta.
    \]
    On the other hand,
    \begin{align*}
        \abs{\gamma(s) - \gamma(s')}
        &= \abs{\int_{s'}^{s}\partial_{s}\gamma(\tau)\,d\tau} \\
        &\geq \abs{s - s'}
        - \abs{\int_{s'}^{s}\abs{\partial_{s}\gamma(\tau) - \partial_{s}\gamma(s)}d\tau} \\
        &\geq \abs{s - s'} - M\abs{s - s'}^{1+\beta}
    \end{align*}
    holds, thus we must have
    \[
        \abs{s - s'} - M\abs{s - s'}^{1+\beta} < 2\delta.
    \]
    We claim that $\abs{s - s'} < ((1+\beta)M)^{-1/\beta}$ should hold
    for all $s,s'\in \ell(\gamma)\phi^{-1}(I_{k})$.
    Indeed, suppose otherwise, then we can choose $s,s'\in \ell(\gamma)\phi^{-1}(I_{k})$
    with $\abs{s - s'} = ((1+\beta)M)^{-1/\beta}$, in which case we get
    \begin{align*}
        \left(\frac{1}{(1+\beta)M}\right)^{1/\beta}
        - M\left(\frac{1}{(1+\beta)M}\right)^{1+1/\beta}
        &= \frac{\beta}{1+\beta}\left(\frac{1}{(1+\beta)M}\right)^{1/\beta}
        = 2\delta,
    \end{align*}
    which is a contradiction. Therefore, we conclude
    \begin{align*}
        \ell(\gamma) \leq N\left(\frac{1}{(1+\beta)M}\right)^{1/\beta}
        \leq \frac{2(1+\beta)}{\beta}(\ell(\gamma_{0}) + 1/4)
        \leq \frac{4\ell(\gamma_{0})+1}{\beta}.
    \end{align*}
\end{proof}

\begin{corollary}\label{CB.7}
    For any $\beta\in[0,1]$, the functional $\norm{\,\cdot\,}_{\dot{C}^{1,\beta}}\colon
    \operatorname{Curve}(\bbR^{n})\to[0,\infty]$ is lower semicontinuous,
    thus is in particular Borel measurable.
\end{corollary}

\begin{proof}
    We may assume $\beta\in(0,1]$ since the case $\beta=0$ is trivial.
    Let $R_{2}\in[0,\infty)$ be given. We are to show that the set
    \[
        A\coloneqq\setbc{\gamma\in\operatorname{Curve}(\bbR^{n})}
        {\norm{\gamma}_{\dot{C}^{1,\beta}} \leq R_{2}}   
    \]
    is closed in $\operatorname{Curve}(\bbR^{n})$. Let $\seq{\gamma_{n}}_{n=1}^{\infty}$
    be any sequence in this set convergent to some $\gamma\in\operatorname{Curve}(\bbR^{n})$.
    Clearly, there exists $R_{0}\in[0,\infty)$ such that
    $\sup_{n}\norm{\gamma_{n}}_{C^{0}} \leq R_{0}$.
    By Lemma~\ref{LB.6}, we know that
    there exists $\delta=\delta(\beta,R_{2})>0$ such that
    $d_{\mathrm{F}}(\gamma_{n},\gamma_{m})<\delta$
    implies $\ell(\gamma_{n}) \leq \frac{4\ell(\gamma_{m}) + 1}{\beta}$.
    Since $\seq{\gamma_{n}}_{n=1}^{\infty}$ is a Cauchy sequence,
    there exists $m$ such that $\ell(\gamma_{n}) \leq \frac{4\ell(\gamma_{m}) + 1}{\beta}$
    holds for every $n\geq m$. Therefore, there exists $R_{1}\in[0,\infty)$
    such that $\sup_{n}\ell(\gamma_{n}) \leq R_{1}$.
    Hence, Proposition~\ref{PB.5} shows that
    the sequence $\seq{\gamma_{n}}_{n=1}^{\infty}$ is contained in a
    compact subset of $A$, thus the limit $\gamma$ must be in $A$ as well.
    Therefore, $A$ is closed.
\end{proof}


%%%%%%%%%%%%%%%%%%%%%%%%%%%%%%%%%%%%%%%%%%%%%%
\section{$\operatorname{Curve}(\bbR^{n})$-valued measurable functions} \label{SC}
%%%%%%%%%%%%%%%%%%%%%%%%%%%%%%%%%%%%%%%%%%%%%%

In this section, we prove claims about curve-valued measurable functions
we stated in the paper. For a set $X$, by an \emph{$X$-valued simple function on $\mathcal{L}$}
we mean a function $f\colon\mathcal{L}\to X$ with a finite range such that each
$f^{-1}(x)$ is a measurable subset of $\mathcal{L}$.

\begin{proposition}\label{PC.1}
    Let $(X,d)$ be a complete metric space and $L_{tb}^{\infty}(\mathcal{L};X)$ the set of
    all a.e.-equivalence classes of uniform limits of
    $X$-valued simple functions on $\mathcal{L}$. Then for any such uniform limits
    $z_{1},z_{2}\colon\mathcal{L}\to X$, the function
    $\lambda\mapsto d(z_{1}(\lambda),z_{2}(\lambda))$ is measurable, thus
    \[
        d_{\infty}\colon (z_{1},z_{2})\mapsto
        \norm{\lambda\mapsto d(z_{1}(\lambda),z_{2}(\lambda))}_{L^{\infty}}
    \]
    is a well-defined complete metric on $L_{tb}^{\infty}(\mathcal{L};X)$.
\end{proposition}

\begin{proof}
    Since the $z_{1},z_{2}$ are uniform limits of simple functions,
    their ranges are totally bounded. In particular, the union $A$ of their ranges
    is separable, thus
    \[
        \lambda\mapsto d(z_{1}(\lambda),z_{2}(\lambda))
        = \inf_{x\in S}\left(d(z_{1}(\lambda),x) + d(z_{2}(\lambda),x)\right)
    \]
    is measurable, where $S$ is any countable dense subset of $A$.
    Then it is clear that $d_{\infty}$ is a metric on $L_{tb}^{\infty}(\mathcal{L};X)$.

    To show completeness, let $\seq{z_{n}}_{n=1}^{\infty}$
    be a Cauchy sequence in $L_{tb}^{\infty}(\mathcal{L};X)$,
    then we can take a representative of each $z_{n}$ so that the resulting sequence
    converges uniformly to a function
    $z\colon\mathcal{L}\to X$. Clearly, $z$ is measurable
    (in the sense that the inverse images of Borel sets in $X$ are measurable)
    and has a totally bounded range. Therefore, there exists a sequence
    of $X$-valued simple functions converging uniformly to $z$,
    so $z\in L_{tb}^{\infty}(\mathcal{L};X)$.
\end{proof}

Next, we see that the space $L_{tb}^{\infty}(\mathcal{L};X)$ can be realized as a
space of continuous functions. As in Section~\ref{S6},
let $\bar{\mathcal{L}}$ be the Gelfand spectrum of the $C^{*}$-algebra
$L^{\infty}(\mathcal{L};\bbC)$. Then the Boolean algebra of measurable sets in $\mathcal{L}$
modulo null sets is isomorphic to the Boolean algebra of clopen subsets of $\bar{\mathcal{L}}$.
Hence, for a set $Y$, this correspondence of Boolean algebras gives a correspondence
from a simple function $f\colon\mathcal{L}\to Y$ into a \emph{clopen-simple function},
i.e., a function $\bar{f}\colon\bar{\mathcal{L}}\to Y$ with finite range such that
each $\bar{f}^{-1}(y)$ is a clopen subset of $\bar{\mathcal{L}}$.
Clearly, such $\bar{f}$ is a continuous function if $Y$ is a topological space.

\begin{proposition}\label{PC.2}
    Let $(X,d)$ be a complete metric space. Then there uniquely exists an
    isometric bijection $\Phi\colon L_{tb}^{\infty}(\mathcal{L};X) \to C(\bar{\mathcal{L}};X)$
    extending the correspondence of $X$-valued simple functions described above,
    where $C(\bar{\mathcal{L}};X)$ is endowed with the uniform metric.
\end{proposition}

\begin{proof}
    It can be easily shown that clopen-simple functions form a dense subset of
    $C(\bar{\mathcal{L}};X)$. Since both $L_{tb}^{\infty}(\mathcal{L};X)$ and
    $C(\bar{\mathcal{L}};X)$ are complete metric spaces, it is enough to see that the
    correspondence of simple functions is an isometry, which is obvious.
\end{proof}

The next proposition and corollary show that the $\operatorname{Curve}(\bbR^{n})$-valued
function $X_{v}^{h}[z]$ we used in Definition~\ref{D2.6}
is a well-defined element of $L_{tb}^{\infty}(\mathcal{L};\operatorname{Curve}(\bbR^{n}))$.

\begin{proposition}
    Let $X,Y$ be complete metric spaces and $f\colon X\to Y$ a continuous function.
    If $z\in L_{tb}^{\infty}(\mathcal{L};X)$,
    then $f\circ z$ is an element in $L_{tb}^{\infty}(\mathcal{L};Y)$
    which does not depend on the choice of representative of $z$.
    Furthermore, the map $z\mapsto f\circ z$ from
    $L_{tb}^{\infty}(\mathcal{L};X)$ into $L_{tb}^{\infty}(\mathcal{L};Y)$ is continuous.
\end{proposition}

\begin{proof}
    Consider the map $f_{*}\colon\bar{z}\mapsto f\circ\bar{z}$ from
    $C(\bar{\mathcal{L}};X)$ into $C(\bar{\mathcal{L}};Y)$.
    Then for any $z\in L_{tb}^{\infty}(\mathcal{L};X)$ with the corresponding element
    $\bar{z}\in C(\bar{\mathcal{L}};X)$, by considering a sequence
    of simple functions converging uniformly to $z$ and comparing it with
    the corresponding sequence in $C(\bar{\mathcal{L}};X)$, it can be easily seen that
    $f\circ z\in L_{tb}^{\infty}(\mathcal{L};Y)$ and is precisely the element corresponding
    to $f\circ\bar{z}$, thus in particular is independent of the choice of
    representative.

    For the second claim, it is enough to show that
    $f_{*}\colon C(\bar{\mathcal{L}};X)\to C(\bar{\mathcal{L}};Y)$ is continuous.
    This can be easily proven by using compactness of $\bar{\mathcal{L}}$.
    Details are omitted.
\end{proof}

\begin{corollary}\label{CC.4}
    Let $z\in L_{tb}^{\infty}(\mathcal{L};\operatorname{Curve}(\bbR^{n}))$ and
    $u\colon\bbR^{n}\to\bbR^{n}$ be a continuous vector field.
    Define a new $\operatorname{Curve}(\bbR^{n})$-valued function $w$ as
    \[
        w\colon \lambda\mapsto \left(
            \xi \mapsto u(z(\lambda)(\xi))
        \right)
    \]
    where we choose any parameterization $z(\lambda)\colon\bbT\to\bbR^{n}$
    for each $\lambda\in\mathcal{L}$. Then $w$ belongs to
    $L_{tb}^{\infty}(\mathcal{L};\operatorname{Curve}(\bbR^{n}))$
    and is independent of the choice of parameterizations.
\end{corollary}

\begin{proof}
    By the previous proposition, it is enough to show that
    \[
        F\colon \gamma \mapsto \left(\xi \mapsto u(\gamma(\xi))\right)
    \]
    is a continuous map from $\operatorname{Curve}(\bbR^{n})$ into itself.
    Fix $\gamma\in\operatorname{Curve}(\bbR^{n})$, then there exists a compact set
    $K\subseteq\bbR^{n}$ such that the image of
    any $\delta\in\operatorname{Curve}(\bbR^{n})$ with
    $d_{\mathrm{F}}(\gamma,\delta) \leq 1$ is contained $K$.
    Since $u$ is uniformly continuous on $K$, for given $\epsilon>0$
    there exists $\eta\in(0,1)$ such that $\abs{x - y}\leq\eta$ implies
    $\abs{u(x) - u(y)}\leq\epsilon$ for any $x,y\in K$.

    If $\delta\in\operatorname{Curve}(\bbR^{n})$ is such that
    $d_{\mathrm{F}}(\gamma,\delta) \leq \eta/2$, then we can find
    parameterizations of $\gamma$, $\delta$ so that $\norm{\gamma - \delta}_{C^{0}} \leq \eta$.
    Hence, for any $\xi\in\bbT$, we have
    \[
        \abs{F(\delta)(\xi) - F(\gamma)(\xi)}
        = \abs{u(\delta(\xi)) - u(\gamma(\xi))} \leq \epsilon,
    \]
    so $d_{\mathrm{F}}(F(\delta),F(\gamma)) \leq \epsilon$ follows.
    Therefore, $F$ is continuous.
\end{proof}

Next, we see how quantities involved in the functional $L$ given in
Definition~\ref{D2.4} can be translated by the correspondence
$L_{tb}^{\infty}(\mathcal{L};X) \cong C(\bar{\mathcal{L}};X)$.

Recall that the \emph{essential image} of a measurable function
$g\colon\mathcal{L}\to X$ into a complete metric space $(X,d)$ is the set of
$x\in X$ such that $g^{-1}(U)$ is not a null set for any open neighborhood $U$ of $x$.

\begin{proposition}
    Let $(X,d)$ be a complete metric space and $z\in L_{tb}^{\infty}(\mathcal{L};X)$
    with the corresponding $\bar{z}\in C(\bar{\mathcal{L}};X)$.
    Then the essential image of $z$ is precisely the image of $\bar{z}$,
    and $z(\lambda)$ is in the essential image of $z$ for almost every $\lambda\in\mathcal{L}$.
\end{proposition}

\begin{proof}
    Let $\seq{\bar{\varphi}_{n}}_{n=1}^{\infty}$ be a sequence of $X$-valued
    clopen-simple functions on $\bar{\mathcal{L}}$ uniformly convergent to $\bar{z}$.
    Without loss of generality, we can assume that the image of each
    $\bar{\varphi}_{n}$ is contained in the image of $\bar{z}$.
    Take any sequence $\seq{\varphi_{n}}_{n=1}^{\infty}$ of $X$-valued simple functions
    on $\mathcal{L}$ corresponding to $\seq{\bar{\varphi}_{n}}_{n=1}^{\infty}$,
    then by inductively modifying $\varphi_{n}$ on a null set if necessary,
    we can assume that $\seq{\varphi_{n}}_{n=1}^{\infty}$ is uniformly Cauchy,
    thus is convergent to some $w\colon\mathcal{L}\to X$.
    Then any point in the image of $w$ is a limit of points in images of
    $\bar{\varphi}_{n}$'s, which are all contained in the compact
    set $\bar{z}(\bar{\mathcal{L}})$. Since $z=w$ almost everywhere,
    there exists a null set $N\subseteq\mathcal{L}$ such that
    $z(\mathcal{L}\setminus N) \subseteq \bar{z}(\bar{\mathcal{L}})$.
    This in particular shows that the essential image of $z$
    is contained in $\bar{z}(\bar{\mathcal{L}})$.

    Now, suppose $x\in \bar{z}(\bar{\mathcal{L}})$ and take an arbitrary $\epsilon>0$.
    Then one can find a clopen-simple function $\bar{\varphi}\colon\bar{\mathcal{L}}\to X$
    containing $x$ in its range such that the uniform distance between $\bar{z}$ and
    $\bar{\varphi}$ is within $\epsilon$. Let $\varphi\colon\mathcal{L}\to X$
    be any simple function corresponding to $\bar{\varphi}$, then
    $\norm{\lambda\mapsto d(\varphi(\lambda),z(\lambda))}_{L^{\infty}} \leq \epsilon$
    and $\varphi^{-1}(x)$ is a non-null set in $\mathcal{L}$.
    Then $d(z(\lambda),x)\leq\epsilon$ holds for almost every
    $\lambda\in\varphi^{-1}(x)$, and since $\varphi^{-1}(x)$ is a non-null set,
    we conclude that $x$ is in the essential range of $z$.
\end{proof}

\begin{corollary}\label{CC.6}
    Let $(X,d)$ be a complete metric space, $f\colon X\to[0,\infty]$
    a lower semicontinuous function, and $z\in L_{tb}^{\infty}(\mathcal{L};X)$
    with the corresponding $\bar{z}\in C(\bar{\mathcal{L}};X)$. Then
    \[
        \norm{\lambda\mapsto f\circ z(\lambda)}_{L^{\infty}}
        = \sup_{\bar{\lambda}\in\bar{\mathcal{L}}}f\circ\bar{z}(\bar{\lambda}).
    \]
\end{corollary}

\begin{proof}
    By the previous proposition, it is enough to show
    $\norm{\lambda\mapsto f\circ z(\lambda)}_{L^{\infty}}$ is the supremum
    of $f(x)$ where $x$ ranges in the essential image of $z$.
    Since $z(\lambda)$ is in the essential image for almost every $\lambda$,
    this supremum is clearly at least $\norm{\lambda\mapsto f\circ z(\lambda)}_{L^{\infty}}$.
    For the other direction, let $x\in X$ be such that
    $f(x) > \norm{\lambda\mapsto f\circ z(\lambda)}_{L^{\infty}}$, then by
    definition of the essential supremum, there exists $r<f(x)$ such that
    $(f\circ z)^{-1}\left((r,\infty]\right)
    = z^{-1}\left(f^{-1}\left((r,\infty]\right)\right)$ is a null set.
    Since $f$ is lower semicontinuous and $r<f(x)$,
    it follows that $f^{-1}\left(r,\infty\right]$
    is an open neighborhood of $x$, thus $x$ is not in the essential image of $z$.
    Therefore, we obtain the desired equality.
\end{proof}


Next, we show some basic properties of the functional $L$ given in
Definition~\ref{D2.4}.

\begin{proposition}\label{PC.7}
    Consider the function $G\colon\operatorname{Curve}(\bbR^{n})
    \times L_{tb}^{\infty}(\mathcal{L};\operatorname{Curve}(\bbR^{n}))\to[0,\infty]$ given as
    \[
        G\colon (\gamma,z)\mapsto \int_{\mathcal{L}}
        \frac{d\lambda}{\Delta(\gamma,z(\lambda))^{2\alpha}}.
    \]
    Then $G$ is well-defined and lower semicontinuous. Furthermore,
    \[
        G(\gamma,z) = \int_{\bar{\mathcal{L}}}
        \frac{d\bar{\lambda}}{\Delta(\gamma,\bar{z}(\bar{\lambda}))^{2\alpha}}
    \]
    holds for any $\gamma\in\operatorname{Curve}(\bbR^{n})$
    and $z\in L_{tb}^{\infty}(\mathcal{L};\operatorname{Curve}(\bbR^{n}))$,
    where $\bar{z}\in C(\bar{\mathcal{L}};\operatorname{Curve}(\bbR^{n}))$
    is the function corresponding to $z$.
\end{proposition}

\begin{proof}
    Clearly, $\Delta$ is a continuous function on
    $\operatorname{Curve}(\bbR^{n})\times\operatorname{Curve}(\bbR^{n})$, thus
    for any $(\gamma,z)\in \operatorname{Curve}(\bbR^{n})
    \times L_{tb}^{\infty}(\mathcal{L};\operatorname{Curve}(\bbR^{n}))$,
    $\lambda\mapsto\frac{1}{\Delta(\gamma,z(\lambda))^{2\alpha}}$
    is a measurable function from $\mathcal{L}$ into $[0,\infty]$, so
    $G$ is well-defined. Also, lower semicontinuity of $G$
    follows directly from Fatou's lemma.

    To show the second claim, let $(\gamma,z)\in \operatorname{Curve}(\bbR^{n})
    \times L_{tb}^{\infty}(\mathcal{L};\operatorname{Curve}(\bbR^{n}))$
    and $\epsilon>0$ be given and consider the integral
    \[
        \int_{\mathcal{L}}\frac{d\lambda}
        {\Delta(\gamma,z(\lambda))^{2\alpha} + \epsilon}.
    \]
    Then by the dominated convergence theorem, the above integral is equal to
    \[
        \lim_{n\to\infty}\int_{\mathcal{L}}\frac{d\lambda}
        {\Delta(\gamma,\varphi_{n}(\lambda))^{2\alpha} + \epsilon}
    \]
    for a sequence $\seq{\varphi_{n}}_{n=1}^{\infty}$ of $\operatorname{Curve}(\bbR^{n})$-valued
    simple functions on $\mathcal{L}$ converging uniformly to $z$. Let
    $\seq{\bar{\varphi}_{n}}_{n=1}^{\infty}$ be the corresponding sequence of
    clopen-simple functions on $\bar{\mathcal{L}}$, then it can be easily seen that
    the integral in the above is equal to the corresponding integral on $\bar{\mathcal{L}}$.
    Therefore, again the dominated convergence theorem shows
    \[
        \int_{\mathcal{L}}\frac{d\lambda}
        {\Delta(\gamma,z(\lambda))^{2\alpha} + \epsilon}
        = \lim_{n\to\infty}\int_{\bar{\mathcal{L}}}\frac{d\bar{\lambda}}
        {\Delta(\gamma,\bar{\varphi}_{n}(\bar{\lambda}))^{2\alpha} + \epsilon}
        = \int_{\bar{\mathcal{L}}}\frac{d\bar{\lambda}}
        {\Delta(\gamma,\bar{z}(\bar{\lambda}))^{2\alpha} + \epsilon}.
    \]
    Hence, sending $\epsilon\to 0$ and applying the monotone convergence theorem
    shows the claim.
\end{proof}

\begin{corollary}\label{CC.8}
    For $z\in L_{tb}^{\infty}(\mathcal{L};\operatorname{Curve}(\bbR^{n}))$,
    let $\bar{z}$ denote the corresponding function in
    $C(\bar{\mathcal{L}};\operatorname{Curve}(\bbR^{n}))$.
    \begin{enumerate}
        \item The functional $L_{1}\colon z\mapsto
        \norm{\lambda\mapsto\norm{z(\lambda)}_{C^{0}}}_{L^{\infty}}$
        is well-defined, is continuous, and satisfies
        \[
            L_{1}(z) = \sup_{\bar{\lambda}\in\bar{\mathcal{L}}}
            \norm{\bar{z}(\bar{\lambda})}_{C^{0}}
        \]
        for all $z\in L_{tb}^{\infty}(\mathcal{L};\operatorname{Curve}(\bbR^{n}))$.

        \item The functional $L_{2}\colon z\mapsto
        \norm{\lambda\mapsto\ell(z(\lambda))}_{L^{\infty}}$
        is well-defined, is lower semicontinuous, and satisfies
        \[
            L_{2}(z) = \sup_{\bar{\lambda}\in\bar{\mathcal{L}}}
            \ell(\bar{z}(\bar{\lambda}))
        \]
        for all $z\in L_{tb}^{\infty}(\mathcal{L};\operatorname{Curve}(\bbR^{n}))$.

        \item The functional $L_{3}\colon z\mapsto
        \norm{\lambda\mapsto\mathcal{K}_{2}(z(\lambda))}_{L^{\infty}}$
        is well-defined, is lower semicontinuous, and satisfies
        \[
            L_{3}(z) = \sup_{\bar{\lambda}\in\bar{\mathcal{L}}}
            \mathcal{K}_{2}(\bar{z}(\bar{\lambda}))
        \]
        for all $z\in L_{tb}^{\infty}(\mathcal{L};\operatorname{Curve}(\bbR^{n}))$.

        \item The functional $L_{4}\colon z\mapsto
        \norm{\lambda\mapsto\norm{z(\lambda)}_{\dot{C}^{1,\beta}}}_{L^{\infty}}$
        for any $\beta\in[0,1]$ is well-defined, is lower semicontinuous, and satisfies
        \[
            L_{4}(z) = \sup_{\bar{\lambda}\in\bar{\mathcal{L}}}
            \norm{\bar{z}(\bar{\lambda})}_{\dot{C}^{1,\beta}}
        \]
        for all $z\in L_{tb}^{\infty}(\mathcal{L};\operatorname{Curve}(\bbR^{n}))$.

        \item The functional $L_{5}\colon z\mapsto
        \norm{\lambda\mapsto\int_{\mathcal{L}}
        \frac{d\lambda'}{\Delta(z(\lambda),z(\lambda'))^{2\alpha}}}_{L^{\infty}}$
        is well-defined, is lower semicontinuous, and satisfies
        \[
            L_{5}(z) = \sup_{\bar{\lambda}\in\bar{\mathcal{L}}}
            \int_{\bar{\mathcal{L}}}\frac{d\bar{\lambda}'}
            {\Delta(\bar{z}(\bar{\lambda}),\bar{z}(\bar{\lambda}'))^{2\alpha}}
        \]
        for all $z\in L_{tb}^{\infty}(\mathcal{L};\operatorname{Curve}(\bbR^{n}))$.
    \end{enumerate}
\end{corollary}

\begin{proof}
    The first claim is a direct consequence of continuity of
    $\norm{\,\cdot\,}_{C^{0}}$, and similarly the second, the third and the fourth claims
    follow directly from Corollary~\ref{CB.2},
    Corollary~\ref{CB.4},
    Corollary~\ref{CB.7},
    and Corollary~\ref{CC.6}.
    The last claim also follows directly from
    Proposition~\ref{PC.7}
    and Corollary~\ref{CC.6}
    by considering the function $w\colon\mathcal{L}\to\operatorname{Curve}(\bbR^{n})
    \times L_{tb}^{\infty}(\mathcal{L};\operatorname{Curve}(\bbR^{n}))$
    defined as $w\colon\lambda\mapsto (z(\lambda),z)$ for given
    $z\in L_{tb}^{\infty}(\mathcal{L};\operatorname{Curve}(\bbR^{n}))$.
\end{proof}

Next, we show the existence of a simultaneous parameterization satisfying a certain
regularity condition. Consider the Hilbert space $H^{2}(\mathbb{T};\bbR^{n})$ and let
$\mathscr{T}$ be the locally convex topology on $H^{2}(\mathbb{T};\bbR^{n})$ given by
$\norm{\,\cdot\,}_{C^{1}}$ together with the family of seminorms
\[
    f\mapsto \abs{\int_{\bbT}g\cdot \partial_{\xi}^{2}f}
\]
for each $g\in L^{2}(\bbT;\bbR^{n})$.
Denote by $E$ the space $C\left(\bar{\mathcal{L}};(H^{2}(\bbT;\bbR^{n}),\mathscr{T})\right)$
of functions from $\bar{\mathcal{L}}$ into $H^{2}(\bbT;\bbR^{n})$ that are continuous
with respect to $\mathscr{T}$. For any $\tilde{z}\in E$, the image of $\tilde{z}$
is a compact subset of $(H^{2}(\bbT;\bbR^{n}),\mathscr{T})$, which must be
bounded with respect to the $H^{2}$-norm by the uniform boundedness principle.
Therefore, we can endow $E$ with the norm
\[
    \norm{\tilde{z}}_{E} \coloneqq \sup_{\bar{\lambda}\in\bar{\mathcal{L}}}
    \norm{\tilde{z}(\bar{\lambda})}_{H^{2}}.
\]
Then since $\mathscr{T}$ is coarser than the $H^{2}$-norm topology,
$(E,\norm{\,\cdot\,}_{E})$ is a Banach space. Indeed, if $\seq{\tilde{z}_{n}}_{n=1}^{\infty}$
is a Cauchy sequence in $E$, then it converges uniformly to some function
$\tilde{z}\colon\bar{\mathcal{L}}\to H^{2}(\bbT;\bbR^{n})$ with respect to the
$H^{2}$-norm. Then clearly the convergence
$\tilde{z}_{n} \to \tilde{z}$ is also uniform with respect to $\mathscr{T}$,
thus $\tilde{z}$ is continuous with respect to $\mathscr{T}$, so is in $E$.
Note that the function $\bar{\lambda} \mapsto \norm{\tilde{z}(\bar{\lambda})}_{H^{2}}$
is only lower semicontinuous but not continuous in general.

Note that by composing with the chain of natural maps
\[
    (H^{2}(\bbT;\bbR^{n}),\mathscr{T})\to C(\bbT;\bbR^{n})\to\operatorname{Curve}(\bbR^{n}),
\]
we obtain a map $E\to C(\bar{\mathcal{L}};\operatorname{Curve}(\bbR^{n}))
\cong L_{tb}^{\infty}(\mathcal{L};\operatorname{Curve}(\bbR^{n}))$.

\begin{proposition}\label{PC.9}
    Let $z\in L_{tb}^{\infty}(\mathcal{L};\operatorname{Curve}(\bbR^{n}))$ and assume
    $\norm{\lambda\mapsto \mathcal{K}_{2}(z(\lambda))}_{L^{\infty}} < \infty$.
    \begin{enumerate}
        \item There exists $\tilde{z}\in E$ mapped into $z$ by the map
        $E\to L_{tb}^{\infty}(\mathcal{L};\operatorname{Curve}(\bbR^{n}))$
        such that each $\tilde{z}(\bar{\lambda})$ is a constant-speed parameterization.

        \item There exists a jointly measurable function
        $\tilde{z}\colon\mathcal{L}\times\bbT\to \bbR^{n}$ such that
        each $\tilde{z}(\lambda,\,\cdot\,)$ is a constant-speed parameterization
        of $z(\lambda)$ for almost every $\lambda\in\mathcal{L}$.

        \item There exists a jointly measurable function
        $z\colon\mathcal{L}\times\bbR\to\bbR^{n}$ such that
        each $z(\lambda,\,\cdot\,)$ is an $\ell(z(\lambda))$-periodic function
        whose projection onto $\ell(z(\lambda))\bbT$ is an arclength parameterization
        of $z(\lambda)$ for almost every $\lambda\in\mathcal{L}$.

        \item There exists a jointly continuous function
        $\bar{z}\colon\bar{\mathcal{L}}\times\bbR\to\bbR^{n}$ such that
        each $\bar{z}(\bar{\lambda},\,\cdot\,)$ is an $\ell(\bar{z}(\bar{\lambda}))$-periodic
        function whose projection onto $\ell(\bar{z}(\bar{\lambda}))\bbT$ is an
        arclength parameterization of $\bar{z}(\bar{\lambda})$
        for all $\bar{\lambda}\in\bar{\mathcal{L}}$.
    \end{enumerate}
\end{proposition}

\begin{proof}
    \begin{enumerate}
        \item Let $\bar{z}\in C(\bar{\lambda};\operatorname{Curve}(\bbR^{n}))$ be the function
        corresponding to $z$. Then since the image of $\bar{z}$ is compact in
        $\operatorname{Curve}(\bbR^{n})$, it is bounded. Therefore,
        Corollary~\ref{CB.4} and
        Corollary~\ref{CC.6} show that
        there exist $R_{0},R_{2}\in[0,\infty)$ such that the image of $\bar{z}$
        is contained in the set
        \[
            A\coloneqq\setbc{\gamma\in\operatorname{Curve}(\bbR^{n})}
            {\norm{\gamma}_{C^{0}}\leq R_{0},\ \mathcal{K}_{2}(\gamma)\leq R_{2}}.
        \]
        Then by Proposition~\ref{PB.3},
        $\bar{z}\colon\bar{\mathcal{L}}\to A$ is a continuous function when $A$
        is endowed with the metric $d$ described in Proposition~\ref{PB.3}.
        Therefore, we can find a sequence
        $\seq{\bar{\varphi}_{n}}_{n=1}^{\infty}$ of clopen-simple functions
        converging uniformly to $\bar{z}$ with respect to $d$.

        Let $\mathcal{P}_{n}$ be a partition of $\bar{\mathcal{L}}$ by clopen sets
        such that $\bar{\varphi}_{n}$ is constantly $\gamma_{n,P}$ for some
        $\gamma_{n,P}\in A$ for each $P\in\mathcal{P}_{n}$,
        and $\mathcal{P}_{n+1}$ is a refinement of $\mathcal{P}_{n}$ for each $n$.
        Then inductively, we can find a constant-speed parameterization
        $\tilde{\gamma}_{n,P}$ for each $\gamma_{n,P}$ such that the resulting
        sequence $\seq{\tilde{\varphi}_{n}}_{n=1}^{\infty}$
        of $H^{2}(\bbT;\bbR^{n})$-valued clopen-simple functions converges uniformly
        with respect to the metric
        \[
            \tilde{d}\colon (\tilde{\gamma}_{1}, \tilde{\gamma}_{2})\mapsto
            \norm{\tilde{\gamma}_{1} - \tilde{\gamma}_{2}}_{C^{1}(\bbT;\bbR^{n})}
            + \sum_{i=1}^{\infty}\frac{1}{2^{i}}
            \abs{\int_{\bbT}g_{i}\cdot\partial_{\xi}^{2}
            (\tilde{\gamma}_{1} - \tilde{\gamma}_{2})}
        \]
        where $\set{g_{i}}_{i=1}^{\infty}$ is the dense subset of the unit ball in
        $L^{2}(\bbT;\bbR^{n})$ used in the definition of $d$.
        Let $\tilde{z}\colon\bar{\mathcal{L}}\to C^{1}(\bbT;\bbR^{n})$ be the limit of
        $\seq{\tilde{\varphi}_{n}}_{n=1}^{\infty}$, then by the $C^{1}$-convergence,
        it immediately follows that each $\tilde{z}(\bar{\lambda})$
        is a constant-speed parameterization of $\bar{z}(\bar{\lambda})$.
        Note that the ranges of all of $\tilde{\varphi}_{n}$'s are contained in
        a large enough ball in $H^{2}(\bbT;\bbR^{n})$, so the same reasoning
        as in the proof of Proposition~\ref{PB.3}
        shows that each $\tilde{z}(\bar{\lambda})$ is in $H^{2}(\bbT;\bbR^{n})$.
        Also, since $\mathscr{T}$ coincides with the topology induced by $\tilde{d}$
        on any ball in $H^{2}(\bbT;\bbR^{n})$, it follows that
        $\tilde{z}$ is continuous with respect to $\mathscr{T}$, in other words,
        $\tilde{z}\in E$. \\

        \item By following a similar construction as in the first part,
        we obtain a uniform Cauchy sequence $\seq{\tilde{\varphi}_{n}}_{n=1}^{\infty}$
        of $C^{1}(\bbT;\bbR^{n})$-valued simple functions on $\mathcal{L}$.
        Let $\tilde{z}\in L_{tb}^{\infty}(\mathcal{L};C^{1}(\bbT;\bbR^{n}))$ be the limit,
        then the natural image of $\tilde{z}$ in
        $L_{tb}^{\infty}(\mathcal{L};\operatorname{Curve}(\bbR^{n}))$
        coincides with $z$ almost everywhere. Clearly, for each $n$, the function
        $(\lambda,\xi)\mapsto\tilde{\varphi}_{n}(\lambda)(\xi)\in\bbR^{n}$ is jointly measurable,
        thus its pointwise limit $(\lambda,\xi)\mapsto \tilde{z}(\lambda)(\xi)$ is
        also jointly measurable, which is a function satisfying all the desired
        properties. \\

        \item Pick $\tilde{z}\colon\mathcal{L}\times\bbT\to\bbR^{n}$ from the second part,
        and define $z\colon\mathcal{L}\times\bbR\to\bbR^{n}$ as
        \[
            z\colon(\lambda,s)\mapsto
            \begin{cases}
                \tilde{z}(\lambda,s/\ell(z(\lambda)) + \bbZ)
                &\textrm{if $\ell(z(\lambda)) \neq 0$}, \\
                \tilde{z}(\lambda,\bbZ)
                &\textrm{if $\ell(z(\lambda)) = 0$.}
            \end{cases}
        \]
        Then since $\ell\colon\operatorname{Curve}(\bbR^{n})\to[0,\infty]$
        is measurable, $z$ satisfies all the desired properties. \\

        \item Use the same construction as above, but with
        the $\tilde{z}$ we get from the first part.
        Note that joint continuity is because
        $\ell(\tilde{z}(\bar{\lambda})) = \norm{\tilde{z}(\bar{\lambda})}_{\dot{C}^{1}}$
        is a continuous function of $\bar{\lambda}$.
    \end{enumerate}
\end{proof}

\begin{proposition}\label{PC.10}
    Let $\tilde{z}\in C(\bar{\mathcal{L}};C^{1}(\bbT;\bbR^{2}))$ and $z$ the image of
    $\tilde{z}$ under the map $C(\bar{\mathcal{L}};C^{1}(\bbT;\bbR^{2}))\to
    L_{tb}^{\infty}(\mathcal{L};\operatorname{Curve}(\bbR^{2}))$.
    Then for any $\epsilon>0$
    \[
        u_{\epsilon}(z)(x) = -\int_{\bar{\mathcal{L}}} \int_{\bbT}
        K_{\epsilon}\left(x - \tilde{z}(\bar{\lambda},\xi)\right)
        \partial_{\xi}\tilde{z}(\bar{\lambda},\xi)\,d\xi\,d\bar{\lambda}
    \]
    holds for all $x\in\bbR^{2}$, where $K_{\epsilon}\colon\bbR^{2}\to\bbR$
    is the function defined in Section~\ref{S6}. Furthermore, if
    $\norm{\lambda\mapsto\norm{z(\lambda)}_{\dot{C}^{1,\beta}}}_{L^{\infty}}<\infty$
    for any $\beta\in(0,1]$, then
    \[
        u(z)(x) = -\int_{\bar{\mathcal{L}}} \int_{\bbT}
        \frac{\partial_{\xi}\tilde{z}(\bar{\lambda},\xi)}
        {\abs{x - \tilde{z}(\bar{\lambda},\xi)}^{2\alpha}}
        \,d\xi\,d\bar{\lambda}
    \]
    holds for all $x\in\bbR^{n}$.
\end{proposition}

\begin{proof}
    The claim about $u_{\epsilon}(z)$ can be shown by considering a sequence
    $\seq{\tilde{\varphi}_{n}}_{n=1}^{\infty}$ of $C^{1}(\bbT;\bbR^{n})$-valued
    clopen-simple functions with uniformly bounded $C^{1}$-norm
    convergent uniformly to $\tilde{z}$, finding a corresponding sequence
    $\seq{\varphi_{n}}_{n=1}^{\infty}$ in $L_{tb}^{\infty}(\mathcal{L};C^{1}(\bbT;\bbR^{n}))$,
    and then applying the dominated convergence theorem.
    The claim about $u(z)$ can then be shown by applying
    Lemma~\ref{L4.1} and sending $\epsilon\to 0$.
    Details are omitted.
\end{proof}


%%%%%%%%%%%%%%%%%%%%%%%%%%%%%%%%%%%%%%%%%%%%%%
\section{Unused stuffs}
%%%%%%%%%%%%%%%%%%%%%%%%%%%%%%%%%%%%%%%%%%%%%%

The next lemma is also essentially from \cite{JeonZla21}, but with more careful tracking of
the dependence of constants on the $C^{1,\beta}$-regularity of the curves:

\begin{lemma}\label{LD.1}
    Let $\gamma_{i}\colon\ell_{i}\bbT\to\bbR^{2}$, $i=1,2$ be nontrivial $C^{1,\beta}$
    closed curves parameterized by the arclength that do not cross each other,
    and $\mathbf{T}_{i}\coloneqq\partial_{s}\gamma_{i}$,
    $\mathbf{N}_{i}\coloneqq\mathbf{T}_{i}^{\perp}$ for $i=1,2$.
    Let $s_{1}\in\ell_{1}\bbT$ and
    $d\leq\left(10000\norm{\mathbf{T}}_{\dot{C}^{\beta}}\right)^{-1/\beta}$
    where $\norm{\mathbf{T}}_{\dot{C}^{\beta}}\coloneqq
    \max\left(\norm{\mathbf{T}_{1}}_{\dot{C}^{\beta}},
    \norm{\mathbf{T}_{2}}_{\dot{C}^{\beta}}\right)$.
    Then there exist finitely many disjoint compact proper subintervals
    $I_{1},\ \cdots\ ,I_{N}$ of $\ell_{2}\bbT$ with $N\leq\frac{\ell_{2}}{4d}$ such that
    \begin{enumerate}
        \item $A\coloneqq
        \setbc{s_{2}\in\ell_{2}\bbT}{\abs{\gamma_{1}(s_{1}) - \gamma_{2}(s_{2})} \leq d}
        \subseteq\bigcup_{i=1}^{N}I_{i}$ and $\set{A\cap I_{i}}_{i=1}^{N}$
        is precisely the set of connected components of $A$,

        \item for each $i=1,\ \cdots\ ,N$, the function
        $g\colon s_{2}\mapsto(\gamma_{1}(s_{1})-\gamma_{2}(s_{2}))\cdot\mathbf{T}_{1}(s_{1})$
        restricted to $I_{i}$ is a homeomorphism onto $[-d,d]$ with
        $\abs{g|_{I_{i}}'}\geq\frac{1}{2}$,

        \item for each $i=1,\ \cdots\ ,N$,
        define $f_{i}\colon[-d,d]\to\bbR$ as
        \[
            f_{i}\colon h\mapsto
            \left(\gamma_{1}(s_{1}) - \gamma_{2}(g|_{I_{i}}^{-1}(h))\right)
            \cdot\mathbf{N}_{1}(s_{1}),
        \]
        then
        \[
            \abs{f_{i}'(h)}\leq
            12\sqrt{2}\norm{\mathbf{T}}_{\dot{C}^{\beta}}^{\frac{1}{1+\beta}}
            (h^{2}+f_{i}(h)^{2})^{\frac{\beta}{2(1+\beta)}}
            \leq \frac{1}{2}
        \]
        and $\abs{f_{i}(h)}\leq 2d$ hold for all $h\in[-d,d]$,
    \end{enumerate}
\end{lemma}

\begin{proof}
    Let $J_{1},\ \cdots\ ,J_{N}$ be the intervals we obtain by applying
    Lemma~\ref{L3.2} with $x\leftarrow \gamma_{1}(s_{1})$.
    Fix $i=1,\ \cdots\ ,N$ and let $s_{2}$ be the center of $J_{i}$.
    Then for any $s_{2}'\in\ell_{2}\bbT$ with $\abs{s_{2}-s_{2}'}\leq 4d$, we have
    \[
        \abs{\gamma_{1}(s_{1}) - \gamma_{2}(s_{2}')}
        \leq \abs{\gamma_{1}(s_{1}) - \gamma_{2}(s_{2})}
        + \abs{\gamma_{2}(s_{2}') - \gamma_{2}(s_{2}')} \leq 5d,
    \]
    so Lemma~\ref{L3.3} shows
    \begin{align*}
        \abs{g'(s_{2})} &= \abs{\mathbf{T}_{2}(s_{2})\cdot\mathbf{T}_{1}(s_{1})}
        = \left(1 - \abs{\mathbf{T}_{1}(s_{1})\cdot\mathbf{N}_{2}(s_{2})}^{2}\right)^{1/2} \\
        &\geq \left(1 - 72\norm{\mathbf{T}}_{\dot{C}^{\beta}}^{2/(1+\beta)}
        (5d)^{2\beta/(1+\beta)}\right)^{1/2} \geq \frac{1}{2}.
    \end{align*}
    On the other hand, by mean value theorem, we know
    \[
        g(s_{2} \pm 4d) = g(s_{2}) \pm 4dg'(s_{2}^{\pm})
    \]
    for some $s_{2}^{\pm}$ in between $s_{2}$ and $s_{2}\pm 4d$. Since $\abs{g(s_{2})}\leq d$,
    it follows that $\abs{g(s_{2}+4d)}\geq d$ holds while $g(s_{2}+4d)$ and
    $g(s_{2}-4d)$ are of different sign. Therefore, there exists a subinterval
    $I_{i}$ of $J_{i}$ such that $g|_{I_{i}}$ is a homeomorphism onto $[-d,d]$.
    Clearly, $I_{i}$ contains the connected component of $A$ containing $s_{2}$,
    which is precisely $A\cap J_{i}$, which means $A\cap I_{i} = A\cap J_{i}$.

    Next, consider the function $f_{i}\colon [-d,d]\to\bbR$. For given $h\in[-d,d]$,
    let $s_{2}\coloneqq (g|_{I_{i}})^{-1}(h)$, then
    \[
        f_{i}'(h) = -\mathbf{T}_{2}(s_{2})\cdot\mathbf{N}_{1}(s_{1})
        \left(g'(s_{2})\right)^{-1}
        = \frac{\mathbf{T}_{2}(s_{2})\cdot\mathbf{N}_{1}(s_{1})}
        {\mathbf{T}_{2}(s_{2})\cdot\mathbf{T}_{1}(s_{1})}.
    \]
    Since $\abs{\mathbf{T}_{2}(s_{2})\cdot\mathbf{T_{1}}(s_{1})}\geq\frac{1}{2}$,
    Lemma~\ref{L3.3} gives the bound
    \[
        \abs{f_{i}'(h)} \leq
        12\sqrt{2}\norm{\mathbf{T}}_{\dot{C}^{\beta}}^{\frac{1}{1+\beta}}
        (h^{2}+f_{i}(h)^{2})^{\frac{\beta}{2(1+\beta)}}.
    \]
    Also, since $\abs{f_{i}'(h)}\leq 2$ and $\abs{f(h_{0})}\leq d$ for some $h_{0}\in[-d,d]$,
    we have $\abs{f_{i}(h)}\leq 5d$ for all $h\in[-d,d]$, so
    \[
        12\sqrt{2}\norm{\mathbf{T}}_{\dot{C}^{\beta}}^{\frac{1}{1+\beta}}
        (h^{2}+f_{i}(h)^{2})^{\frac{\beta}{2(1+\beta)}}
        \leq 12\sqrt{2}\norm{\mathbf{T}}_{\dot{C}^{\beta}}^{\frac{1}{1+\beta}}
        \left(\sqrt{26}d\right)^{\frac{\beta}{1+\beta}}\leq\frac{1}{2},
    \]
    which then shows $\abs{f_{i}(h)}\leq 2d$.

    Finally, to show $I_{i}\cap I_{j} = \emptyset$ for $i\neq j$, note that
    $g$ maps both $I_{i}$ and $I_{j}$ homeomorphically onto $[-d,d]$
    Therefore, $I_{i}\cap I_{j}\neq\emptyset$ can happen only when they intersect at
    one of the endpoints. However, in that case the sign of
    $g'(s_{2}) = -\mathbf{T}_{2}(s_{2})\cdot\mathbf{T}_{1}(s_{1})$ should change
    abruptly when $s_{2}$ moves from $I_{i}$ into $I_{j}$ through their common endpoints,
    which is absurd.
\end{proof}

Now, let us show that the velocity field $u$
defined as above is $C^{1}$ when restricted to a curve $z(\lambda,\,\cdot\,)$.

\begin{proposition}
    For almost every $\lambda\in\mathcal{L}$, the function
    $s\mapsto u(z(\lambda,s))$ is $C^{1}$ and satisfies
    \[
        \norm{\partial_{s}u(z(\lambda,\,\cdot\,))}_{L^{\infty}}
        \lesssim
        \int_{\mathcal{L}}
        \norm{z(\lambda',\,\cdot\,)}_{C^{1,1/2}}^{2+4\alpha}\ell(\lambda')\,d\lambda'
        + \norm{z(\lambda,\,\cdot\,)}_{C^{1,1/2}}^{2+4\alpha}
        \int_{\mathcal{L}}\ell(\lambda')\,d\lambda'.
    \]
\end{proposition}

\begin{proof}
    Let $\varphi_{\epsilon}\colon\bbR\to\bbR$ be a smooth even function
    such that $0\leq\varphi\leq 1$, $\varphi\equiv 1$ outside $[-1,1]$
    and $\lim_{r\to 0}\frac{\varphi(r)}{r^{N}} = 0$ for all $N\in\bbZ_{\geq0}$.
    Define $\bar{K}\colon r\mapsto\frac{1}{r^{\alpha}}$,
    $K\colon x\mapsto \bar{K}\left(\abs{x}^{2}\right)$, and
    for each $\epsilon>0$, define
    $\bar{K}_{\epsilon}\colon r\mapsto \bar{K}(r)\varphi(r/\epsilon)$,
    $K_{\epsilon}\colon x\mapsto \bar{K}_{\epsilon}\left(\abs{x}^{2}\right)$, and
    \[
        u_{\epsilon}\colon x\mapsto -\int_{\mathcal{L}}\int_{0}^{\ell(\lambda')}
        K_{\epsilon}(x-z(\lambda',s'))\mathbf{T}(\lambda',s')
        \,ds'\,d\lambda'.
    \]
    Assuming that $z(\lambda,\,\cdot\,)$ is $C^{1}$, the derivative of the function
    $s\mapsto K_{\epsilon}(z(\lambda,s) - z(\lambda',s'))$ is given as
    \begin{align*}
        2\bar{K}_{\epsilon}'\left(\abs{z(\lambda,s) - z(\lambda',s')}^{2}\right)
        \left((z(\lambda,s) - z(\lambda',s'))\cdot\mathbf{T}(\lambda,s)\right)
    \end{align*}
    where $\bar{K}_{\epsilon}'(r) = \bar{K}'(r)\varphi(r/\epsilon)
    + \frac{1}{\epsilon}\bar{K}(r)\varphi'(r/\epsilon)$. Note that the integral of
    \[
        \frac{1}{\epsilon}
        \bar{K}\left(\abs{z(\lambda,s)-z(\lambda',s')}^{2}\right)
        \varphi'\left(\frac{\abs{z(\lambda,s)-z(\lambda',s')}^{2}}{\epsilon}\right)
        \left((z(\lambda,s) - z(\lambda',s'))\cdot\mathbf{T}(\lambda,s)\right)
    \]
    with respect to $(\lambda',s')$ converges to zero as $\epsilon\to 0$ uniformly in $s$.
    Thus, we are to show that the net of functions
    \[
        \int_{\mathcal{L}}\int_{0}^{\ell(\lambda')}
        \frac{(z(\lambda,s)-z(\lambda',s'))\cdot\mathbf{T}(\lambda,s)}
        {\abs{z(\lambda,s)-z(\lambda',s')}^{2+2\alpha}}
        \varphi\left(\frac{\abs{z(\lambda,s)-z(\lambda',s')}^{2}}{\epsilon}\right)
        \mathbf{T}(\lambda',s')\,ds'\,d\lambda'
    \]
    is uniformly Cauchy as $\epsilon\downarrow 0$.
    We do this by separating $\mathbf{T}(\lambda,s)$-component and
    $\mathbf{N}(\lambda,s)$-component. Let $\epsilon_{1}>\epsilon_{2}>0$ be given.

    For the $\mathbf{T}(\lambda,s)$-component, let $d$ be the constant appearing in
    Lemma~\ref{LD.1}, and apply the lemma with
    $z(\lambda,\,\cdot\,)$ and $z(\lambda',\,\cdot\,)$ to
    find disjoint intervals $I_{1},\ \cdots\ ,I_{N}$ and corresponding functions
    $f_{1},\ \cdots\ ,f_{N}$. We first estimate the integral over each $I_{i}$.
    Since $dh = -\mathbf{T}(\lambda,s)\cdot\mathbf{T}(\lambda',s')\,ds'$,
    the integral we want to estimate is
    \[
        \int_{-\min(d,\epsilon_{1})}^{\min(d,\epsilon_{1})}
        \frac{h}{(h^{2}+f_{i}(h)^{2})^{1+\alpha}}
        \left(
            \varphi\left(\frac{h^{2}+f_{i}(h)^{2}}{\epsilon_{1}}\right)
            - \varphi\left(\frac{h^{2}+f_{i}(h)^{2}}{\epsilon_{2}}\right)
        \right)dh.
    \]
    Let $P\colon r\mapsto \frac{1}{r^{1+\alpha}}
    \left(\varphi(r/\epsilon_{1}) - \varphi(r/\epsilon_{2})\right)$ then it can be
    easily seen that $\abs{P'(r)}\leq\frac{C}{r^{2+\alpha}}$ for some constant $C$.
    Then after merging the integral over the interval $[0,\min(d,\epsilon_{1})]$
    and $[-\min(d,\epsilon_{1}),0]$ and then applying the mean value theorem, one can see that
    the integrand is bounded by
    \[
        h\cdot \frac{\abs{f_{i}(h) + f_{i}(-h)}\abs{f_{i}(h) - f_{i}(-h)}}
        {(h^{2}+\min(f_{i}(h),f_{i}(-h))^{2})^{2+\alpha}}
    \]
    times a constant. Consider the integral of the above over
    $[0,\min(\abs{f_{i}(0)},d,\epsilon_{1})]$. For $h$ in that interval, we know
    \[
        \abs{f(h)} \leq \abs{f_{i}(0)} + \frac{h}{2}
        \leq \frac{3}{2}\abs{f_{i}(0)},
    \]
    so the integrand is bounded by
    \[
        h\cdot\frac{\abs{f_{i}(0)}\cdot
        hM^{2/3}\abs{f_{i}(0)}^{1/3}}
        {\abs{f_{i}(0)}^{4+2\alpha}}
    \]
    up to a constant where $M\coloneqq\max\left(
        \norm{\partial_{s}z(\lambda,\,\cdot\,)}_{C^{1/2}},
        \norm{\partial_{s}z(\lambda',\,\cdot\,)}_{C^{1/2}}
    \right)$, thus the integral is bounded by
    \[
        M^{2/3}\abs{f_{i}(0)}^{-\frac{8}{3}-2\alpha}
        \min(\abs{f_{i}(0)},d,\epsilon_{1})^{3}
    \]
    up to a constant. Similarly, when $h\in\left[\min(\abs{f_{i}(0)},d,\epsilon_{1}),
    \min(d,\epsilon_{1})\right]$, the integrand is bounded by
    \[
        h\cdot \frac{h^{2}M^{2/3}h^{1/3}}
        {h^{4+2\alpha}}
    \]
    up to a constant, and since $\alpha<\frac{1}{6}$, overall the integral over $I_{i}$
    is bounded by a constant times
    $M^{2/3}\min(d,\epsilon_{1})^{\frac{1}{3}-2\alpha}$.
    Since the number of intervals $N$ is at most $\frac{\ell(\lambda')}{2d}$,
    the sum of the integrals over $I_{i}$'s is then of the order
    \[
        M^{2/3}
        \frac{\min(d,\epsilon_{1})^{\frac{1}{3}-2\alpha}}{d}\ell(\lambda')
        \sim M^{8/3}
        \min(d,\epsilon_{1})^{\frac{1}{3}-2\alpha}\ell(\lambda').
    \]
    On the other hand, one can easily see that the integral outside of $I_{i}$'s
    is of the order
    \[
        \begin{cases}
            \frac{\ell(\lambda')}{d^{1+2\alpha}}
            &\textrm{if $d\leq\epsilon_{1}$}, \\
            0 &\textrm{if $d>\epsilon_{1}$}.
        \end{cases}
    \]
    Therefore, we conclude that $\mathbf{T}(\lambda,s)$-component is bounded by
    \[
        \begin{cases}
            M^{8/3}d^{\frac{1}{3}-2\alpha}\ell(\lambda')
            = M^{2+4\alpha}\ell(\lambda')
            &\textrm{if $d\leq \epsilon_{1}$,} \\
            M^{8/3}\ell(\lambda')
            \epsilon_{1}^{\frac{1}{3}-2\alpha}
            &\textrm{if $d>\epsilon_{1}$}
        \end{cases}
    \]
    up to a constant. Since $M\leq \max\left(
        \norm{z(\lambda,\,\cdot\,)}_{\dot{H}^{2}},
        \norm{z(\lambda',\,\cdot\,)}_{\dot{H}^{2}}
    \right)$ and
    \[
        \int_{\mathcal{L}}
        \norm{z(\lambda',\,\cdot\,)}_{\dot{H}^{2}}^{2+4\alpha}\ell(\lambda')\,d\lambda'
        + \norm{z(\lambda,\,\cdot\,)}_{\dot{H}^{2}}^{2+4\alpha}
        \int_{\mathcal{L}}\ell(\lambda')\,d\lambda'
    \]
    is finite by the assumption on $z$, the dominated convergence theorem shows
    the desired uniform convergence and the bound.

    Similarly, for $\mathbf{N}(\lambda,s)$-component, we apply the same bound for
    the integral outside $I_{i}$'s, and on each $I_{i}$, it is enough to estimate
    \[
        \int_{-\min(d,\epsilon_{1})}^{\min(d,\epsilon_{1})}
        \frac{\abs{hf_{i}'(h)}}{(h^{2}+f_{i}(h)^{2})^{1+\alpha}}\,dh.
    \]
    Here we applied the formula
    $f_{i}'(h) = \frac{\mathbf{T}(\lambda',s')\cdot\mathbf{N}(\lambda,s)}
    {\mathbf{T}(\lambda',s')\cdot\mathbf{T}(\lambda,s)}$.
    Then again we estimate the integral by splitting the interval into
    $[-\min(\abs{f_{i}(0)},d,\epsilon_{1}),\min(\abs{f_{i}(0)},d,\epsilon_{1})]$
    and the outside. For the former, the integrand is bounded by
    \[
        \frac{\abs{h}\cdot M^{2/3}\abs{f_{i}(0)}^{1/3}}
        {\abs{f_{i}(0)}^{2+2\alpha}}
    \]
    up to a constant, so the integral is of the order
    \[
        M^{2/3}\abs{f_{i}(0)}^{-\frac{5}{3}-2\alpha}
        \min(\abs{f_{i}(0)},d,\epsilon_{1})^{2}.
    \]
    For the latter, the integrand is bounded by
    \[
        \frac{\abs{h}\cdot M^{2/3}\abs{h}^{1/3}}
        {\abs{h}^{2+2\alpha}}
    \]
    up to a constant, so it follows that the integral over $I_{i}$ is at most
    \[
        M^{2/3}\min(d,\epsilon_{1})^{\frac{1}{3}-2\alpha}
    \]
    up to a constant, which is the same result as we got for
    $\mathbf{T}(\lambda,s)$-component. Hence, the same analysis applies.
\end{proof}

\begin{proposition}
    Estimate $\kappa(\partial_{s}^{2}u\cdot\mathbf{N})$.
\end{proposition}

\begin{proof}
    As in the previous proposition, take $K_{\epsilon}$ and define
    \[
        u_{\epsilon}\colon x\mapsto -\int_{\mathcal{L}}\int_{0}^{\ell(\lambda')}
        K_{\epsilon}(x-z(\lambda',s'))\mathbf{T}(\lambda',s')\,ds'\,d\lambda'.
    \]
    Then $\partial_{s}^{2}u_{\epsilon}(z(\lambda,s))\cdot\mathbf{N}(\lambda,s)$
    is given as
    \begin{align*}
        &-\int_{\mathcal{L}}\int_{0}^{\ell(\lambda')}
        4\bar{K}_{\epsilon}''\left(\abs{z(\lambda,s) - z(\lambda',s')}^{2}\right)
        \left((z(\lambda,s) - z(\lambda',s'))\cdot\mathbf{T}(\lambda,s)\right)^{2}
        \\&\quad\quad\quad\quad\quad\quad\quad\quad\quad\quad\quad\quad\quad
        \left(\mathbf{T}(\lambda',s')\cdot\mathbf{N}(\lambda,s)\right)\,ds'\,d\lambda' \\
        &\quad\quad\quad
        - \int_{\mathcal{L}}\int_{0}^{\ell(\lambda')}
        2\bar{K}_{\epsilon}'\left(\abs{z(\lambda,s) - z(\lambda',s')}^{2}\right)
        \left(\mathbf{T}(\lambda',s')\cdot\mathbf{N}(\lambda,s)\right)\,ds'\,d\lambda' \\
        &\quad\quad\quad
        + \int_{\mathcal{L}}\int_{0}^{\ell(\lambda')}
        2\bar{K}_{\epsilon}'\left(\abs{z(\lambda,s) - z(\lambda',s')}^{2}\right)
        \left((z(\lambda,s) - z(\lambda',s'))\cdot\mathbf{N}(\lambda,s)\right)
        \\&\quad\quad\quad\quad\quad\quad\quad\quad\quad\quad\quad\quad\quad
        \cdot\kappa(\lambda,s)
        \left(\mathbf{T}(\lambda',s')\cdot\mathbf{N}(\lambda,s)\right)\,ds'\,d\lambda'.
    \end{align*}
    Let us call the first term $G_{1}$, the second term $G_{2}$ and the third term
    $\kappa(\lambda,s)G_{3}$. We bound $G_{3}$ first.

    \textbf{Estimate for $G_{3}$.} As done in the previous proposition,
    we estimate the integral
    \[
        \int_{-d}^{d}\frac{f(h)f'(h)}{\left(h^{2}+f(h)^{2}\right)^{1+\alpha}}\,dh.
    \]
    Again, we split the interval into $[-\min(d,f(0)),\min(d,f(0))]$ and the outside.
    For the former, we bound the integrand by
    \[
        \frac{\abs{f(0)}\cdot M^{2/3}\abs{f(0)}^{1/3}}
        {\abs{f(0)}^{2+2\alpha}}
    \]
    which gives
    \[
        M^{2/3}
        \abs{f(0)}^{-\frac{2}{3}-2\alpha}\min(d,f(0)),
    \]
    where again we define $M\coloneqq\max\left(
        \norm{\partial_{s}z(\lambda,\,\cdot\,)}_{C^{1/2}},
        \norm{\partial_{s}z(\lambda',\,\cdot\,)}_{C^{1/2}}
    \right)$. For the latter, we bound the integrand by
    \[
        \frac{\abs{h}\cdot M^{2/3}\abs{h}^{1/3}}
        {\abs{h}^{2+2\alpha}},
    \]
    and we conclude that the integral is of the order
    \[
        M^{2/3}d^{\frac{1}{3}-2\alpha}.
    \]
    Then in the same way we did in the previous proposition, we conclude that
    $G_{3}$ is of the order
    \[
        \int_{\mathcal{L}}
        \norm{z(\lambda',\,\cdot\,)}_{\dot{H}^{2}}^{2+4\alpha}\ell(\lambda')\,d\lambda'
        + \norm{z(\lambda,\,\cdot\,)}_{\dot{H}^{2}}^{2+4\alpha}
        \int_{\mathcal{L}}\ell(\lambda')\,d\lambda'
    \]
    uniformly in $s$ and $\epsilon$.

    \textbf{Estimate for $G_{1}+G_{2}$.} By integration by parts, we can rewrite $G_{1}$
    as the sum of four terms:
    \begin{align*}
        G_{11}&= -\int_{\mathcal{L}}\int_{0}^{\ell(\lambda')}
        4\bar{K}_{\epsilon}''\left(\abs{z(\lambda,s) - z(\lambda',s')}^{2}\right)\cdot
        \\&\quad\quad\quad\quad\quad\quad
        \left((z(\lambda,s) - z(\lambda',s'))
        \cdot(\mathbf{T}(\lambda,s)+\mathbf{T}(\lambda',s'))\right)\cdot
        \\&\quad\quad\quad\quad\quad\quad
        \left((z(\lambda,s) - z(\lambda',s'))
        \cdot(\mathbf{T}(\lambda,s)-\mathbf{T}(\lambda',s'))\right)\cdot
        \\&\quad\quad\quad\quad\quad\quad
        \left(\mathbf{T}(\lambda',s')\cdot\mathbf{N}(\lambda,s)\right)\,ds'\,d\lambda',\\
        G_{12}&= \int_{\mathcal{L}}\int_{0}^{\ell(\lambda')}
        2\bar{K}_{\epsilon}'\left(\abs{z(\lambda,s) - z(\lambda',s')}^{2}\right)
        \left(\mathbf{T}(\lambda',s')\cdot\mathbf{N}(\lambda,s)\right)\,ds'\,d\lambda',\\
        G_{13}&= \int_{\mathcal{L}}\int_{0}^{\ell(\lambda')}
        2\bar{K}_{\epsilon}'\left(\abs{z(\lambda,s) - z(\lambda',s')}^{2}\right)
        \left((z(\lambda,s) - z(\lambda',s'))\cdot\mathbf{N}(\lambda',s')\right)
        \\&\quad\quad\quad\quad\quad\quad\quad\quad\quad\quad\quad\quad\quad
        \cdot\kappa(\lambda',s')
        \left(\mathbf{T}(\lambda',s')\cdot\mathbf{N}(\lambda,s)\right)\,ds'\,d\lambda',\\
        G_{14}&= \int_{\mathcal{L}}\int_{0}^{\ell(\lambda')}
        2\bar{K}_{\epsilon}'\left(\abs{z(\lambda,s) - z(\lambda',s')}^{2}\right)
        \left((z(\lambda,s) - z(\lambda',s'))\cdot\mathbf{T}(\lambda',s')\right)
        \\&\quad\quad\quad\quad\quad\quad\quad\quad\quad\quad\quad\quad\quad
        \cdot\kappa(\lambda',s')
        \left(\mathbf{T}(\lambda',s')\cdot\mathbf{T}(\lambda,s)\right)\,ds'\,d\lambda'.
    \end{align*}
    Note that $G_{12}$ cancels with $G_{2}$, so we only need to estimate
    $G_{11}$, $G_{13}$, and $G_{14}$. For $G_{13}$, we claim that the function
    $s\mapsto \kappa(\lambda,s)G_{13}$ admits an $L^{1}$-bound uniform in $\epsilon>0$.
    Indeed, by Cauchy-Schwarz inequality, we obtain that the $L^{1}$-norm of
    $s\mapsto\kappa(\lambda,s)G_{13}$ is bounded by the product of the square-roots of
    \begin{align*}
        &\int_{0}^{\ell(\lambda)}\kappa(\lambda,s)^{2}
        \int_{\mathcal{L}}\int_{0}^{\ell(\lambda')}
        \abs{2\bar{K}_{\epsilon}'\left(\abs{z(\lambda,s) - z(\lambda',s')}^{2}\right)}
        \abs{(z(\lambda,s) - z(\lambda',s'))\cdot\mathbf{N}(\lambda',s')}
        \\&\quad\quad\quad\quad\quad\quad\quad\quad\quad\quad\quad\quad\quad
        \cdot\abs{\mathbf{T}(\lambda',s')\cdot\mathbf{N}(\lambda,s)}
        \,ds'\,d\lambda'\,ds
    \end{align*}
    and
    \begin{align*}
        &\int_{\mathcal{L}}\int_{0}^{\ell(\lambda')}\kappa(\lambda',s')^{2}
        \int_{0}^{\ell(\lambda)}
        \abs{2\bar{K}_{\epsilon}'\left(\abs{z(\lambda,s) - z(\lambda',s')}^{2}\right)}
        \abs{(z(\lambda,s) - z(\lambda',s'))\cdot\mathbf{N}(\lambda',s')}
        \\&\quad\quad\quad\quad\quad\quad\quad\quad\quad\quad\quad\quad\quad
        \cdot\abs{\mathbf{T}(\lambda',s')\cdot\mathbf{N}(\lambda,s)}
        \,ds\,ds'\,d\lambda'.
    \end{align*}
    From the estimate of $G_{3}$, we know that the first factor is bounded by
    \[
        \norm{z(\lambda,\,\cdot\,)}_{\dot{H}^{2}}^{2}\int_{\mathcal{L}}
        \norm{z(\lambda',\,\cdot\,)}_{\dot{H}^{2}}^{2+4\alpha}\ell(\lambda')\,d\lambda'
        + \norm{z(\lambda,\,\cdot\,)}_{\dot{H}^{2}}^{4+4\alpha}
        \int_{\mathcal{L}}\ell(\lambda')\,d\lambda'
    \]
    up to a constant. For the second factor, note that by switching the role
    of $s$ and $s'$, the inner most integral can be bounded by
    \[
        \norm{z(\lambda,\,\cdot\,)}_{\dot{H}^{2}}^{2+4\alpha}\ell(\lambda)
        + \norm{z(\lambda',\,\cdot\,)}_{\dot{H}^{2}}^{2+4\alpha}\ell(\lambda)
    \]
    times a constant, thus the second factor can be bounded by
    \[
        \norm{z(\lambda,\,\cdot\,)}_{\dot{H}^{2}}^{2+4\alpha}\ell(\lambda)
        \int_{\mathcal{L}}\norm{z(\lambda',\,\cdot\,)}_{\dot{H}^{2}}^{2}\,d\lambda'
        + \ell(\lambda)\int_{\mathcal{L}}
        \norm{z(\lambda',\,\cdot\,)}_{\dot{H}^{2}}^{4+4\alpha}\,d\lambda'.
    \]
    Therefore, the $L^{1}$-norm of $s\mapsto\kappa(\lambda,s)G_{13}$ is bounded by
    a constant times the sum of these two bounds.

    \textbf{Estimate of $G_{11}$.} Fix $s'\in\bbR/\ell(\rho')\bbZ$.
    Let $\theta$ be the angle from $\mathbf{T}(\lambda,s)$ to $\mathbf{T}(\lambda',s')$,
    then Lemma~\ref{L3.3} shows
    \[
        \abs{\sin\theta} \leq 6\sqrt{2} M^{2/3}\abs{z(\lambda,s) - z(\lambda',s')}^{1/3}.
    \]
    By decomposing $\mathbf{T}(\lambda,s)\pm\mathbf{T}(\lambda',s')$ into
    $\mathbf{T}(\lambda',s')$-component and $\mathbf{N}(\lambda',s')$-component, we see
    \begin{align*}
        &\left((z(\lambda,s) - z(\lambda',s'))
        \cdot(\mathbf{T}(\lambda,s)+\mathbf{T}(\lambda',s'))\right)
        \left((z(\lambda,s) - z(\lambda',s'))
        \cdot(\mathbf{T}(\lambda,s)-\mathbf{T}(\lambda',s'))\right)
        \\&\quad\quad\quad
        = \left(\abs{(z(\lambda,s)-z(\lambda',s'))\cdot\mathbf{N}(\lambda',s')}^{2}
        -\abs{(z(\lambda,s)-z(\lambda',s'))\cdot\mathbf{T}(\lambda',s')}^{2}\right)
        \sin^{2}\theta
        \\&\quad\quad\quad\quad\quad\quad
        -2\left((z(\lambda,s)-z(\lambda',s'))\cdot\mathbf{T}(\lambda',s')\right)
        \left((z(\lambda,s)-z(\lambda',s'))\cdot\mathbf{N}(\lambda',s')\right)
        \sin\theta\cos\theta.
    \end{align*}
    We use the integration by parts again for the last term. The partial derivative of
    \begin{align*}
        &4\bar{K}_{\epsilon}'\left(\abs{z(\lambda,s) - z(\lambda',s')}^{2}\right)
        ((z(\lambda,s) - z(\lambda',s'))\cdot\mathbf{N}(\lambda',s')) \\
        &\quad\quad
        \cdot (\mathbf{T}(\lambda',s')\cdot\mathbf{N}(\lambda,s))^{2}
        (\mathbf{T}(\lambda,s)\cdot\mathbf{T}(\lambda',s'))
    \end{align*}
    with respect to $s'$ is given as
    \begin{align*}
        &-8\bar{K}_{\epsilon}''\left(\abs{z(\lambda,s) - z(\lambda',s')}^{2}\right)
        ((z(\lambda,s) - z(\lambda',s'))\cdot\mathbf{T}(\lambda',s'))
        ((z(\lambda,s) - z(\lambda',s'))\cdot\mathbf{N}(\lambda',s'))
        \sin^{2}\theta\cos\theta
        \\&\quad\quad
        -8\bar{K}_{\epsilon}'\left(\abs{z(\lambda,s) - z(\lambda',s')}^{2}\right)
        \kappa(\lambda',s')
        ((z(\lambda,s) - z(\lambda',s'))\cdot\mathbf{N}(\lambda',s'))
        \sin\theta\cos^{2}\theta
        \\&\quad\quad
        +4\bar{K}_{\epsilon}'\left(\abs{z(\lambda,s) - z(\lambda',s')}^{2}\right)
        \kappa(\lambda',s')
        ((z(\lambda,s) - z(\lambda',s'))\cdot\mathbf{N}(\lambda',s'))
        \sin^{3}\theta.
    \end{align*}
    The integral of the last two terms can be bounded in the same manner as $G_{13}$,
    so we only need to care about the integral
    \begin{align*}
        &\int_{0}^{\ell(\lambda')}
        -4\bar{K}_{\epsilon}''\left(\abs{z(\lambda,s) - z(\lambda',s')}^{2}\right) \\
        &\quad\quad\quad
        \cdot\left(\abs{(z(\lambda,s)-z(\lambda',s'))\cdot\mathbf{N}(\lambda',s')}^{2}
        -\abs{(z(\lambda,s)-z(\lambda',s'))\cdot\mathbf{T}(\lambda',s')}^{2}\right) \\
        &\quad\quad\quad
        \cdot(\mathbf{T}(\lambda',s')\cdot\mathbf{N}(\lambda,s))^{3}\,ds'.
    \end{align*}
    The absolute value of the integrand is bounded by a constant times
    \[
        \frac{M^{2}}{\abs{z(\lambda,s) - z(\lambda',s')}^{1+2\alpha}}.
    \]
    Again, we take intervals $I_{1},\ \cdots\ ,I_{N}$ and functions
    $f_{1},\ \cdots\ ,f_{N}$, then on each $I_{i}$, dividing the integral into over the set of
    $h$ with $\abs{h}\leq\Delta(\lambda,\lambda')$ and the outside gives the bound
    \[
        \frac{M^{2}}{\Delta(\lambda,\lambda')^{2\alpha}},
    \]
    so aggregating all of the integrals on $I_{1},\ \cdots\ ,I_{N}$ gives the bound
    \[
        \frac{M^{2}}{\Delta(\lambda,\lambda')^{2\alpha}}
        \cdot \frac{\ell(\lambda')}{d}
        \lesssim \frac{\left(
            \norm{z(\lambda,\,\cdot\,)}_{\dot{H}^{2}}^{4}
            + \norm{z(\lambda',\,\cdot\,)}_{\dot{H}^{2}}^{4}
        \right)\ell(\lambda')}
        {\Delta(\lambda,\lambda')^{2\alpha}}.
    \]
    Also, the integral outside of $I_{1},\ \cdots\ ,I_{N}$ is bounded by
    \[
        \frac{M^{2}\ell(\lambda')}{d^{1+2\alpha}}
        \lesssim \left(\norm{z(\lambda,\,\cdot\,)}_{\dot{H}^{2}}^{4+4\alpha}
        + \norm{z(\lambda',\,\cdot\,)}_{\dot{H}^{2}}^{4+4\alpha}\right)\ell(\lambda').
    \]
    Since these bounds are independent of $s$, Cauchy-Schwarz inequality shows that
    the $L^{1}$-norm of the function $s\mapsto\kappa(\lambda,s)G_{11}$
    is bounded by a constant times
    \begin{align*}
        &\norm{z(\lambda,\,\cdot\,)}_{\dot{H}^{2}}^{5}\ell(\lambda)^{1/2}
        \int_{\mathcal{L}}\frac{\ell(\lambda')}{\Delta(\lambda,\lambda')^{2\alpha}}\,d\lambda'
        + \norm{z(\lambda,\,\cdot\,)}_{\dot{H}^{2}}\ell(\lambda)^{1/2}
        \int_{\mathcal{L}}\frac{\norm{z(\lambda',\,\cdot\,)}_{\dot{H}^{2}}^{4}\ell(\lambda')}
        {\Delta(\lambda,\lambda')^{2\alpha}}\,d\lambda' \\
        &\quad\quad\quad
        + \norm{z(\lambda,\,\cdot\,)}_{\dot{H}^{2}}^{5+4\alpha}\ell(\lambda)^{1/2}
        \int_{\mathcal{L}}\ell(\lambda')\,d\lambda'
        + \norm{z(\lambda,\,\cdot\,)}_{\dot{H}^{2}}\ell(\lambda)^{1/2}
        \int_{\mathcal{L}}\norm{z(\lambda',\,\cdot\,)}_{\dot{H}^{2}}^{4+4\alpha}
        \ell(\lambda')\,d\lambda'
    \end{align*}
    in addition to the bounds we get for $G_{13}$.

    \textbf{Estimate of $G_{14}$.} Clearly, the $L^{1}$-norm of
    $s\mapsto\kappa(\lambda,s)G_{14}$ is bounded by
    \[
        \int_{\mathcal{L}}\int_{0}^{\ell(\lambda)}\int_{0}^{\ell(\lambda')}
        \frac{\abs{\kappa(\lambda,s)\kappa(\lambda',s')}}
        {\abs{z(\lambda,s) - z(\lambda',s')}^{1+2\alpha}}\,ds'\,ds\,d\lambda'
    \]
    up to a constant. By Cauchy-Schwarz inequality, the above is bounded by the product of
    the square-roots of
    \[
        \int_{\mathcal{L}}\int_{0}^{\ell(\lambda)}\kappa(\lambda,s)^{2}
        \int_{0}^{\ell(\lambda')}\frac{1}{\abs{z(\lambda,s) - z(\lambda',s')}^{1+2\alpha}}
        \,ds'\,ds\,d\lambda'
    \]
    and
    \[
        \int_{\mathcal{L}}\int_{0}^{\ell(\lambda')}\kappa(\lambda',s')^{2}
        \int_{0}^{\ell(\lambda)}\frac{1}{\abs{z(\lambda,s) - z(\lambda',s')}^{1+2\alpha}}
        \,ds\,ds'\,d\lambda'.
    \]
    By the same analysis as we did for $G_{11}$, the inner-most integral of the
    first factor is bounded by a constant times
    \[
        \frac{\left(
            \norm{z(\lambda,\,\cdot\,)}_{\dot{H}^{2}}^{2}
            + \norm{z(\lambda',\,\cdot\,)}_{\dot{H}^{2}}^{2}
        \right)\ell(\lambda')}
        {\Delta(\lambda,\lambda')^{2\alpha}}
        +\left(\norm{z(\lambda,\,\cdot\,)}_{\dot{H}^{2}}^{2+4\alpha}
        + \norm{z(\lambda',\,\cdot\,)}_{\dot{H}^{2}}^{2+4\alpha}\right)\ell(\lambda'),
    \]
    so the first factor is bounded by a constant times
    \begin{align*}
        &\norm{z(\lambda,\,\cdot\,)}_{\dot{H}^{2}}^{4}
        \int_{\mathcal{L}}\frac{\ell(\lambda')}{\Delta(\lambda,\lambda')^{2\alpha}}\,d\lambda'
        + \norm{z(\lambda,\,\cdot\,)}_{\dot{H}^{2}}^{2}
        \int_{\mathcal{L}}\frac{\norm{z(\lambda',\,\cdot\,)}_{\dot{H}^{2}}^{2}\ell(\lambda')}
        {\Delta(\lambda,\lambda')^{2\alpha}}\,d\lambda' \\
        &\quad\quad\quad
        + \norm{z(\lambda,\,\cdot\,)}_{\dot{H}^{2}}^{4+4\alpha}
        \int_{\mathcal{L}}\ell(\lambda')\,d\lambda'
        + \norm{z(\lambda,\,\cdot\,)}_{\dot{H}^{2}}^{2}
        \int_{\mathcal{L}}\norm{z(\lambda',\,\cdot\,)}_{\dot{H}^{2}}^{2+4\alpha}
        \ell(\lambda')\,d\lambda',
    \end{align*}
    and similarly the second factor is bounded by a constant times
    \begin{align*}
        &\norm{z(\lambda,\,\cdot\,)}_{\dot{H}^{2}}^{2}\ell(\lambda)
        \int_{\mathcal{L}}\frac{\norm{z(\lambda',\,\cdot\,)}_{\dot{H}^{2}}^{2}}
        {\Delta(\lambda,\lambda')^{2\alpha}}\,d\lambda'
        + \ell(\lambda) \int_{\mathcal{L}}
        \frac{\norm{z(\lambda',\,\cdot\,)}_{\dot{H}^{2}}^{4}}
        {\Delta(\lambda,\lambda')^{2\alpha}}\,d\lambda' \\
        &\quad\quad\quad
        + \norm{z(\lambda,\,\cdot\,)}_{\dot{H}^{2}}^{2+4\alpha}\ell(\lambda)
        \int_{\mathcal{L}}\norm{z(\lambda',\,\cdot\,)}_{\dot{H}^{2}}^{2}\,d\lambda'
        + \ell(\lambda) \int_{\mathcal{L}}
        \norm{z(\lambda',\,\cdot\,)}_{\dot{H}^{2}}^{4+4\alpha}\,d\lambda',
    \end{align*}
    so adding these two gives a bound on the $L^{1}$-norm of
    $s\mapsto\kappa(\lambda,s)G_{14}$.
\end{proof}

The next proposition is also essentially from \cite{JeonZla21}.

\begin{proposition}
    Suppose $s,s'\in\bbR$ are such that
    $\Delta(\lambda,\lambda') = \abs{z(\lambda,s) - z(\lambda',s')}$. Then
    \begin{align*}
        &\frac{z(\lambda,s) - z(\lambda',s')}{\abs{z(\lambda,s) - z(\lambda',s')}}
        \cdot(u(z(\lambda,s)) - u(z(\lambda',s'))) \\
        &\quad\quad\quad
        \lesssim
        \left(
            \norm{z(\lambda,\,\cdot\,)}_{\dot{H}^{2}}^{2+4\alpha}+
            \norm{z(\lambda',\,\cdot\,)}_{\dot{H}^{2}}^{2+4\alpha}
        \right)
        \int_{\mathcal{L}}\ell(\lambda'')\,d\lambda''
        \Delta(\lambda,\lambda') \\
        &\quad\quad\quad\quad\quad\quad\quad
        + \int_{\mathcal{L}}
        \norm{z(\lambda'',\,\cdot\,)}_{\dot{H}^{2}}^{2+4\alpha}\ell(\lambda'')
        \,d\lambda''\Delta(\lambda,\lambda').
    \end{align*}
\end{proposition}

\begin{proof}
    Let $\mathbf{N}_{0}\coloneqq\frac{z(\lambda,s) - z(\lambda',s')}
    {\abs{z(\lambda,s) - z(\lambda',s')}}$, then the integral we want to estimate is
    \begin{align*}
        \int_{\mathcal{L}}\int_{0}^{\ell(\lambda'')}
        \left(
            K\left(\abs{z(\lambda,s) - z(\lambda'',s'')}^{2}\right)
            -K\left(\abs{z(\lambda',s') - z(\lambda'',s'')}^{2}\right)
        \right)
        (\mathbf{T}(\lambda'',s'')\cdot\mathbf{N}_{0})
        \,ds''\,d\lambda''
    \end{align*}
    where $K\colon r\mapsto \frac{1}{r^{\alpha}}$. Let
    \[
        M\coloneqq\max\left(
            \norm{z(\lambda,\,\cdot\,)}_{\dot{H}^{2}},
            \norm{z(\lambda',\,\cdot\,)}_{\dot{H}^{2}},
            \norm{z(\lambda'',\,\cdot\,)}_{\dot{H}^{2}}
        \right)
    \]
    and $d$ be the constant appearing in Lemma~\ref{LD.1}
    where $\norm{\partial_{s}z}_{C^{\gamma}}$ is replaced by $M$.

    We divide the interval $[0,\ell(\lambda'')]$ into three regions:
    when $\abs{z(\lambda,s) - z(\lambda'',s'')}\leq d$, when
    $\abs{z(\lambda',s') - z(\lambda'',s'')}\leq d$, and the outside.
    For the first region, we apply Lemma~\ref{LD.1} to get intervals
    $I_{1},\ \cdots\ ,I_{N}$ and functions $f_{1},\ \cdots\ ,f_{N}$.
    Then the integral on $I_{i}$ can be written as
    \[
        \pm\int_{-d}^{d}
        \left(
            K\left(h^{2}+f_{i}(h)^{2}\right)
            - K\left(h^{2}+(f_{i}(h)-m)^{2}\right)
        \right)
        f_{i}'(h)\,dh
    \]
    where $m\coloneqq\Delta(\lambda,\lambda')$.
    By the mean value theorem, the integrand is equal to
    \[
        2mK'(h^{2}+t^{2})t
    \]
    for some $t\in[f_{i}(h)-m,f_{i}(h)]$. Therefore, the integrand is bounded by
    \begin{align*}
        \abs{2mf_{i}'(h)K'(h^{2}+t^{2})t}
        \leq \frac{2\alpha\abs{f_{i}'(h)}m}{(h^{2}+t^{2})^{\frac{1}{2}+\alpha}}
        \leq \frac{2\alpha\abs{f_{i}'(h)}m}
        {(h^{2}+\min\left(\abs{f_{i}(h)-m},\abs{f_{i}(h)}\right)^{2})^{\frac{1}{2}+\alpha}}.
    \end{align*}
    Let $a\coloneqq\min\left(\abs{f_{i}(0)-m},\abs{f_{i}(0)}\right)$.
    Note that if $\abs{f_{i}(0)-m}$ is smaller than $\abs{f_{i}(0)}$, then in particular
    the point on the curve $z(\lambda'',\,\cdot\,)$ corresponding to $f_{i}(0)$
    is within the distance $d$ from $z(\lambda',s')$, thus the interval $I_{i}$
    and the function $f_{i} - m$ must appear as one of those pairs we get by applying
    Lemma~\ref{LD.1} to $z(\lambda',\,\cdot\,)$ and
    $z(\lambda'',\,\cdot\,)$. Therefore, the inequality
    \[
        \abs{f_{i}'(h)}\leq M^{2/3}(h^{2}+a^{2})^{1/6}
    \]
    holds for all $h\in[-d,d]$.

    Note that for $\abs{h}\leq a$, we have
    \[
        \min\left(\abs{f_{i}(h)-m},\abs{f_{i}(h)}\right)
        \geq a - \frac{\abs{h}}{2} \geq \frac{a}{2}
    \]
    since $\abs{f_{i}'(h)}\leq\frac{1}{2}$. Therefore, the integral over the region
    $\abs{h}\leq a$ can be bounded by a constant times
    \[
        \int_{-a}^{a}\frac{M^{2/3}a^{1/3}m}{a^{1+2\alpha}}\,dh
        \sim M^{2/3}a^{\frac{1}{3}-2\alpha}m.
    \]

    On the other hand, the integral over the region $\abs{h}>a$ can be bounded by
    a constant times
    \[
        \int_{\abs{h}>a}\frac{M^{2/3}h^{1/3}m}{h^{1+2\alpha}}\,dh
        \lesssim M^{2/3}m\left(d^{\frac{1}{3}-2\alpha} - a^{\frac{1}{3}-2\alpha}\right),
    \]
    thus the integral over all of $I_{i}$'s is bounded by
    \[
        M^{3/2}d^{\frac{1}{3}-2\alpha}m
        \cdot \frac{\ell(\lambda'')}{d}
        \sim M^{2+4\alpha}\ell(\lambda'')m
    \]
    up to a constant.
    
    By symmetry, we get the same bound on the region
    $\abs{z(\lambda',s') - z(\lambda'',s'')}\leq d$. If both of
    $\abs{z(\lambda',s') - z(\lambda'',s'')}$ and $\abs{z(\lambda',s') - z(\lambda'',s'')}$
    are greater than $d$, then the mean value theorem shows that for some
    $t$ between $\abs{z(\lambda,s) - z(\lambda'',s'')}$ and
    $\abs{z(\lambda',s') - z(\lambda'',s'')}$, we have
    \begin{align*}
        &\abs{\frac{\mathbf{T}(\lambda'',s'')\cdot\mathbf{N}_{0}}
        {\abs{z(\lambda,s) - z(\lambda'',s'')}^{2\alpha}}
        - \frac{\mathbf{T}(\lambda'',s'')\cdot\mathbf{N}_{0}}
        {\abs{z(\lambda',s') - z(\lambda'',s'')}^{2\alpha}}} \\
        &\quad\quad\quad
        \leq \frac{2\alpha\abs{z(\lambda,s) - z(\lambda',s')}}
        {\abs{z(\lambda,s) - z(\lambda'',s'')}^{2\alpha}
        \abs{z(\lambda',s') - z(\lambda'',s'')}^{2\alpha}
        t^{1-2\alpha}}
        \leq \frac{2\alpha m}{d^{1+2\alpha}},
    \end{align*}
    thus the integral over the set of such $s''$ gives the same bound
    $M^{2+4\alpha}\ell(\lambda'')m$.
\end{proof}


\begin{enumerate}
    \item Possible H\"{o}lder regularity of $\partial_{s}u$ and resulting
    well-posedness in $C^{1,\gamma}$?
    
    \item Better control of the number of folds, possibly involving the arc-chord constant?

    \item More flexible trade off between $\norm{z(\lambda,\,\cdot\,)}_{\dot{H}^{2}}$
    and $\Delta$?

    \item Better bound for $G_{11}$?

    \item Enough to assume $L^{p}$-bound for $p<\infty$ for
    $\norm{z(\lambda,\,\cdot\,)}_{\dot{H}^{2}}$?

    \item How do things exactly lead to $-2\kappa^{2}(\partial_{s}u\cdot\mathbf{T})
    -\kappa(\partial_{s}^{2}u\cdot\mathbf{N})$ for
    $\norm{z(\lambda,\,\cdot\,)}_{\dot{H}^{2}}$,
    $\int(\partial_{s}u\cdot\mathbf{T})\,ds$ for $\ell$, and
    $\frac{z(\lambda,s) - z(\lambda',s')}{\abs{z(\lambda,s) - z(\lambda',s')}}
    \cdot(u(z(\lambda,s)) - u(z(\lambda',s')))$ for $\Delta$,
    i.e., locally well-posed in what exact sense?

    \item How to translate from/to the level set setting?

    \item Another condition that can naturally handle touching curves?

    \item What about the $H^{3}$-case?
\end{enumerate}

\begin{proposition}
    There exist constants $C=C(\alpha)\geq 0$ depending only on $\alpha$ such that
    \begin{align*}
        &-\int_{\ell(\bar{\lambda})\bbT}
        \kappa(\bar{\lambda},s)
        \left(
            \partial_{s}^{2}u_{\epsilon}(\bar{\lambda},s)
            \cdot \mathbf{N}(\bar{\lambda},s)
        \right)
        ds
        \\&\quad\quad \leq
        C\bigg[
            \mathcal{K}_{2}(\bar{\lambda})
            \int_{\bar{\mathcal{L}}}
            \frac{\ell(\bar{\lambda}')\mathcal{K}_{2}(\bar{\lambda}')}
            {\Delta(\bar{\lambda},\bar{\lambda}')^{2\alpha}}
            \,d\bar{\lambda}'
            + \ell(\bar{\lambda})^{1/2}\mathcal{K}_{2}(\bar{\lambda})
            \int_{\bar{\mathcal{L}}}
            \frac{\ell(\bar{\lambda}')^{1/2}\mathcal{K}_{2}(\bar{\lambda}')}
            {\Delta(\bar{\lambda},\bar{\lambda}')^{2\alpha}}
            \,d\bar{\lambda}'
            \\&\quad\quad\quad\quad\quad\quad\quad\quad
            + \ell(\bar{\lambda})^{1/2}\mathcal{K}_{2}(\bar{\lambda})^{1/2}
            \int_{\bar{\mathcal{L}}}
            \frac{\ell(\bar{\lambda}') \mathcal{K}_{2}(\bar{\lambda}')
            \max\left(\mathcal{K}_{2}(\bar{\lambda}), \mathcal{K}_{2}(\bar{\lambda}')\right)}
            {\Delta(\bar{\lambda},\bar{\lambda}')^{2\alpha}}
            \,d\bar{\lambda}'
        \bigg] \\
        &\quad\quad\leq CL(z_{0})^{8}
    \end{align*}
    holds for any $\bar{\lambda}\in\bar{\mathcal{L}}$.
\end{proposition}

\begin{proof}
    Note that
    \begin{align*}
        \partial_{s}^{2}u_{\epsilon}(\bar{\lambda},s)
        &= D^{2}(u_{\epsilon}(z_{0}))(\bar{z}_{0}(\bar{\lambda},s))
        (\mathbf{T}(\bar{\lambda},s),\mathbf{T}(\bar{\lambda},s))
        - \kappa(\bar{\lambda},s)
        D(u_{\epsilon}(z_{0}))(\bar{z}_{0}(\bar{\lambda},s))(\mathbf{N}(\bar{\lambda},s)) \\
        &= -\int_{\bar{\mathcal{L}}}\int_{\ell(\bar{\lambda}')\bbT}
        D^{2}K_{\epsilon}(\bar{z}_{0}(\bar{\lambda},s) - \bar{z}_{0}(\bar{\lambda}',s'))
        (\mathbf{T}(\bar{\lambda},s),\mathbf{T}(\bar{\lambda},s))
        \mathbf{T}(\bar{\lambda}',s')\,ds'
        \\&\quad\quad\quad
        + \kappa(\bar{\lambda},s)\int_{\bar{\mathcal{L}}}\int_{\ell(\bar{\lambda}')\bbT}
        DK_{\epsilon}(\bar{z}_{0}(\bar{\lambda},s) - \bar{z}_{0}(\bar{\lambda}',s'))
        (\mathbf{N}(\bar{\lambda},s))
        \mathbf{T}(\bar{\lambda},s')\,ds'.
    \end{align*}
    Hence, the integral we want to estimate is the sum of two terms:
    \begin{align*}
        G_{1} &\coloneqq
        \int_{\ell(\bar{\lambda})\bbT}\kappa(\bar{\lambda},s)
        \int_{\bar{\mathcal{L}}}\int_{\ell(\bar{\lambda}')\bbT}
        D^{2}K_{\epsilon}(\bar{z}_{0}(\bar{\lambda},s) - \bar{z}_{0}(\bar{\lambda}',s'))
        (\mathbf{T}(\bar{\lambda},s),\mathbf{T}(\bar{\lambda},s))
        \\&\quad\quad\quad\quad\quad\quad\quad\quad\quad\quad\quad\quad
        \cdot(\mathbf{T}(\bar{\lambda}',s')\cdot\mathbf{N}(\bar{\lambda},s))
        \,ds'\,d\bar{\lambda}'\,ds, \\
        G_{2} &\coloneqq
        -\int_{\ell(\bar{\lambda})\bbT}\kappa(\bar{\lambda},s)^{2}
        \int_{\bar{\mathcal{L}}}\int_{\ell(\bar{\lambda}')\bbT}
        DK_{\epsilon}(\bar{z}_{0}(\bar{\lambda},s) - \bar{z}_{0}(\bar{\lambda}',s'))
        (\mathbf{N}(\bar{\lambda},s))
        \\&\quad\quad\quad\quad\quad\quad\quad\quad\quad\quad\quad\quad
        \cdot(\mathbf{T}(\bar{\lambda}',s')\cdot\mathbf{N}(\bar{\lambda},s))
        \,ds'\,d\bar{\lambda}'\,ds.
    \end{align*}
    By the same argument as in Proposition~\ref{P6.8},
    we know
    \[
        G_{2} \leq C(\alpha)\mathcal{K}_{2}(\bar{\lambda})
        \int_{\bar{\mathcal{L}}}
        \frac{\ell(\bar{\lambda}')\mathcal{K}_{2}(\bar{\lambda}')}
        {\Delta(\bar{\lambda},\bar{\lambda}')^{2\alpha}}
        \,d\bar{\lambda}'.
    \]
    To estimate $G_{1}$, note that
    \begin{align*}
        \mathbf{T}(\bar{\lambda},s)\odot\mathbf{T}(\bar{\lambda},s)
        &= \left(\mathbf{T}(\bar{\lambda},s) + \mathbf{T}(\bar{\lambda}',s')\right)
        \odot \left(\mathbf{T}(\bar{\lambda},s) - \mathbf{T}(\bar{\lambda}',s')\right)
        + \mathbf{T}(\bar{\lambda}',s')\odot\mathbf{T}(\bar{\lambda}',s')
    \end{align*}
    where $\odot$ denotes the symmetric tensor product. Then we further divide
    $G_{1}$ into the sum of two terms:
    \begin{align*}
        G_{3}&\coloneqq
        \int_{\ell(\bar{\lambda})\bbT}\kappa(\bar{\lambda},s)
        \int_{\bar{\mathcal{L}}}\int_{\ell(\bar{\lambda}')\bbT}
        D^{2}K_{\epsilon}(\bar{z}_{0}(\bar{\lambda},s) - \bar{z}_{0}(\bar{\lambda}',s'))
        (\mathbf{T}(\bar{\lambda},s) + \mathbf{T}(\bar{\lambda}',s'),
        \mathbf{T}(\bar{\lambda},s) - \mathbf{T}(\bar{\lambda}',s'))
        \\&\quad\quad\quad\quad\quad\quad\quad\quad\quad\quad\quad\quad
        \cdot(\mathbf{T}(\bar{\lambda}',s')\cdot\mathbf{N}(\bar{\lambda},s))
        \,ds'\,d\bar{\lambda}'\,ds, \\
        G_{4} &\coloneqq
        \int_{\ell(\bar{\lambda})\bbT}\kappa(\bar{\lambda},s)
        \int_{\bar{\mathcal{L}}}\int_{\ell(\bar{\lambda}')\bbT}
        D^{2}K_{\epsilon}(\bar{z}_{0}(\bar{\lambda},s) - \bar{z}_{0}(\bar{\lambda}',s'))
        (\mathbf{T}(\bar{\lambda}',s'),\mathbf{T}(\bar{\lambda}',s'))
        \\&\quad\quad\quad\quad\quad\quad\quad\quad\quad\quad\quad\quad
        \cdot(\mathbf{T}(\bar{\lambda}',s')\cdot\mathbf{N}(\bar{\lambda},s))
        \,ds'\,d\bar{\lambda}'\,ds.
    \end{align*}
    Since
    \begin{align*}
        &\partial_{s'}\left(
            DK_{\epsilon}(\bar{z}_{0}(\bar{\lambda},s) - \bar{z}_{0}(\bar{\lambda}',s'))
            (\mathbf{T}(\bar{\lambda}',s'))
        \right)
        \\&\quad\quad\quad
        = -D^{2}K_{\epsilon}(\bar{z}_{0}(\bar{\lambda},s) - \bar{z}_{0}(\bar{\lambda}',s'))
        (\mathbf{T}(\bar{\lambda}',s'),\mathbf{T}(\bar{\lambda}',s'))
        \\&\quad\quad\quad\quad\quad\quad
        - \kappa(\bar{\lambda}',s')
        DK_{\epsilon}(\bar{z}_{0}(\bar{\lambda},s) - \bar{z}_{0}(\bar{\lambda}',s'))
        (\mathbf{N}(\bar{\lambda}',s')),
    \end{align*}
    integration by parts shows that $G_{4}$ is equal to the sum of two terms:
    \begin{align*}
        G_{5}&\coloneqq
        -\int_{\bar{\mathcal{L}}}\int_{\ell(\bar{\lambda})\bbT}\int_{\ell(\bar{\lambda'})\bbT}
        \kappa(\bar{\lambda},s)\kappa(\bar{\lambda}',s')
        DK_{\epsilon}(\bar{z}_{0}(\bar{\lambda},s) - \bar{z}_{0}(\bar{\lambda}',s'))
        (\mathbf{N}(\bar{\lambda}',s'))
        \\&\quad\quad\quad\quad\quad\quad\quad\quad\quad\quad\quad\quad
        \cdot(\mathbf{T}(\bar{\lambda}',s')\cdot\mathbf{N}(\bar{\lambda},s))
        \,ds'\,ds\,d\bar{\lambda}',\\
        G_{6}&\coloneqq
        -\int_{\bar{\mathcal{L}}}\int_{\ell(\bar{\lambda})\bbT}\int_{\ell(\bar{\lambda'})\bbT}
        \kappa(\bar{\lambda},s)\kappa(\bar{\lambda}',s')
        DK_{\epsilon}(\bar{z}_{0}(\bar{\lambda},s) - \bar{z}_{0}(\bar{\lambda}',s'))
        (\mathbf{T}(\bar{\lambda}',s'))
        \\&\quad\quad\quad\quad\quad\quad\quad\quad\quad\quad\quad\quad
        \cdot(\mathbf{N}(\bar{\lambda}',s')\cdot\mathbf{N}(\bar{\lambda},s))
        \,ds'\,ds\,d\bar{\lambda}'.
    \end{align*}
    By H\"{o}lder's inequality, both of $G_{5}$ and $G_{6}$ are bounded by
    the integral over $\bar{\lambda}'$ of the product of
    \begin{align*}
        \left(
            \int_{\ell(\bar{\lambda})\bbT}
            \kappa(\bar{\lambda},s)^{2}
            \int_{\ell(\bar{\lambda'})\bbT}
            \abs{DK_{\epsilon}(\bar{z}_{0}(\bar{\lambda},s) - \bar{z}_{0}(\bar{\lambda}',s'))}
            \,ds'\,ds\,d\bar{\lambda}'
        \right)^{1/2}
    \end{align*}
    and
    \begin{align*}
        \left(
            \int_{\ell(\bar{\lambda}')\bbT}
            \kappa(\bar{\lambda}',s')^{2}
            \int_{\ell(\bar{\lambda})\bbT}
            \abs{DK_{\epsilon}(\bar{z}_{0}(\bar{\lambda},s) - \bar{z}_{0}(\bar{\lambda}',s'))}
            \,ds\,ds'\,d\bar{\lambda}'
        \right)^{1/2}.
    \end{align*}
    By Lemma~\ref{L4.3}, the inner integrals
    of the above are bounded by
    \[
        C(\alpha)\frac{\ell(\bar{\lambda}')\mathcal{K}_{2}(\bar{\lambda}')}
        {\Delta(\bar{\lambda},\bar{\lambda}')^{2\alpha}}
        \quad\textrm{and}\quad
        C(\alpha)\frac{\ell(\bar{\lambda})\mathcal{K}_{2}(\bar{\lambda})}
        {\Delta(\bar{\lambda},\bar{\lambda}')^{2\alpha}},
    \]
    respectively, thus we get
    \begin{align*}
        G_{5},G_{6} \leq C(\alpha)
        \ell(\bar{\lambda})^{1/2}\mathcal{K}_{2}(\bar{\lambda})
        \int_{\bar{\mathcal{L}}}
        \frac{\ell(\bar{\lambda}')^{1/2}\mathcal{K}_{2}(\bar{\lambda}')}
        {\Delta(\bar{\lambda},\bar{\lambda}')^{2\alpha}}
        \,d\bar{\lambda}'.
    \end{align*}

    Now, it remains to estimate $G_{3}$. Fix $\bar{\lambda}'$, $s$, $s'$ and
    let $\theta$ be the angle between $\mathbf{T}(\bar{\lambda}',s')$ and
    $\mathbf{T}(\bar{\lambda},s)$ so that
    $\sin\theta = \mathbf{T}(\bar{\lambda}',s')\cdot\mathbf{N}(\bar{\lambda},s)$
    and $\cos\theta = \mathbf{T}(\bar{\lambda}',s')\cdot\mathbf{T}(\bar{\lambda},s)$.
    For $x\in\bbR^{2}$, writing
    $x = (x\cdot\mathbf{T}(\bar{\lambda}',s'))\mathbf{T}(\bar{\lambda}',s')
    + (x\cdot\mathbf{N}(\bar{\lambda}',s'))\mathbf{N}(\bar{\lambda}',s')$ shows that
    \begin{align*}
        &\left(x\cdot(\mathbf{T}(\bar{\lambda},s) + \mathbf{T}(\bar{\lambda}',s'))\right)
        \left(x\cdot(\mathbf{T}(\bar{\lambda},s) - \mathbf{T}(\bar{\lambda}',s'))\right)
        \\&\quad\quad
        = \left(
            (x\cdot\mathbf{N}(\bar{\lambda}',s'))^{2}
            - (x\cdot\mathbf{T}(\bar{\lambda}',s'))^{2}
        \right)\sin^{2}\theta
        + 2(x\cdot\mathbf{T}(\bar{\lambda}',s'))(x\cdot\mathbf{N}(\bar{\lambda}',s'))
        \sin\theta\cos\theta.
    \end{align*}
    Since $D^{2}K_{\epsilon}(x)(h_{1},h_{2})$ is the sum of
    $\frac{h_{1}\cdot h_{2}}{\abs{x}^{2+2\alpha}}$ times a smooth function and
    $\frac{(x\cdot h_{1})(x\cdot h_{2})}{\abs{x}^{4+2\alpha}}$ times another smooth function,
    it follows that $G_{3}$ is the sum of the following two terms:
    \begin{align*}
        G_{7} &\coloneqq
        \int_{\ell(\bar{\lambda})\bbT}\kappa(\bar{\lambda},s)
        \int_{\bar{\mathcal{L}}}\int_{\ell(\bar{\lambda}')\bbT}
        \bigg[
            D^{2}K_{\epsilon}(\bar{z}_{0}(\bar{\lambda},s) - \bar{z}_{0}(\bar{\lambda}',s'))
            (\mathbf{N}(\bar{\lambda}',s'),\mathbf{N}(\bar{\lambda}',s'))
            \\&\quad\quad\quad\quad\quad\quad\quad\quad\quad\quad\quad\quad\quad
            - D^{2}K_{\epsilon}(\bar{z}_{0}(\bar{\lambda},s) - \bar{z}_{0}(\bar{\lambda}',s'))
            (\mathbf{T}(\bar{\lambda}',s'),\mathbf{T}(\bar{\lambda}',s'))
        \bigg]
        \\&\quad\quad\quad\quad\quad\quad\quad\quad\quad\quad\quad\quad\quad\quad
        \cdot(\mathbf{T}(\bar{\lambda}',s')\cdot\mathbf{N}(\bar{\lambda},s))^{3}
        \,ds'\,ds\,d\bar{\lambda}',\\
        G_{8} &\coloneqq
        2\int_{\ell(\bar{\lambda})\bbT}\kappa(\bar{\lambda},s)
        \int_{\bar{\mathcal{L}}}\int_{\ell(\bar{\lambda}')\bbT}
        D^{2}K_{\epsilon}(\bar{z}_{0}(\bar{\lambda},s) - \bar{z}_{0}(\bar{\lambda}',s'))
        (\mathbf{T}(\bar{\lambda}',s'),\mathbf{N}(\bar{\lambda}',s'))
        \\&\quad\quad\quad\quad\quad\quad\quad\quad\quad\quad\quad\quad\quad\quad
        \cdot(\mathbf{T}(\bar{\lambda}',s')\cdot\mathbf{N}(\bar{\lambda},s))^{2}
        \cdot(\mathbf{T}(\bar{\lambda}',s')\cdot\mathbf{T}(\bar{\lambda},s))
        \,ds'\,ds\,d\bar{\lambda}'.
    \end{align*}
    Since
    \begin{align*}
        &\partial_{s'}\left(
            DK_{\epsilon}(\bar{z}_{0}(\bar{\lambda},s) - \bar{z}_{0}(\bar{\lambda}',s'))
            (\mathbf{N}(\bar{\lambda}',s'))
        \right)
        \\&\quad\quad\quad
        = -D^{2}K_{\epsilon}(\bar{z}_{0}(\bar{\lambda},s) - \bar{z}_{0}(\bar{\lambda}',s'))
        (\mathbf{T}(\bar{\lambda}',s'),\mathbf{N}(\bar{\lambda}',s'))
        \\&\quad\quad\quad\quad\quad\quad
        + \kappa(\bar{\lambda}',s')
        DK_{\epsilon}(\bar{z}_{0}(\bar{\lambda},s) - \bar{z}_{0}(\bar{\lambda}',s'))
        (\mathbf{T}(\bar{\lambda}',s')),
    \end{align*}
    integration by parts shows that $G_{8}$ is equal to the sum of three terms:
    \begin{align*}
        G_{9} &\coloneqq
        2\int_{\bar{\mathcal{L}}}
        \int_{\ell(\bar{\lambda})\bbT}\int_{\ell(\bar{\lambda}')\bbT}
        \kappa(\bar{\lambda},s)\kappa(\bar{\lambda}',s')
        DK_{\epsilon}(\bar{z}_{0}(\bar{\lambda},s) - \bar{z}_{0}(\bar{\lambda}',s'))
        (\mathbf{T}(\bar{\lambda}',s'))
        \\&\quad\quad\quad\quad\quad\quad\quad\quad\quad\quad\quad\quad
        \cdot(\mathbf{T}(\bar{\lambda}',s')\cdot\mathbf{N}(\bar{\lambda},s))^{2}
        (\mathbf{T}(\bar{\lambda}',s')\cdot\mathbf{T}(\bar{\lambda},s))
        \,ds'\,ds\,d\bar{\lambda}',\\
        G_{10} &\coloneqq
        -4\int_{\bar{\mathcal{L}}}
        \int_{\ell(\bar{\lambda})\bbT}\int_{\ell(\bar{\lambda}')\bbT}
        \kappa(\bar{\lambda},s)\kappa(\bar{\lambda}',s')
        DK_{\epsilon}(\bar{z}_{0}(\bar{\lambda},s) - \bar{z}_{0}(\bar{\lambda}',s'))
        (\mathbf{N}(\bar{\lambda}',s'))
        \\&\quad\quad\quad\quad\quad\quad\quad\quad\quad\quad\quad\quad
        \cdot(\mathbf{T}(\bar{\lambda}',s')\cdot\mathbf{N}(\bar{\lambda},s))
        (\mathbf{T}(\bar{\lambda}',s')\cdot\mathbf{T}(\bar{\lambda},s))^{2}
        \,ds'\,ds\,d\bar{\lambda}',\\
        G_{11} &\coloneqq
        -2\int_{\bar{\mathcal{L}}}
        \int_{\ell(\bar{\lambda})\bbT}\int_{\ell(\bar{\lambda}')\bbT}
        \kappa(\bar{\lambda},s)\kappa(\bar{\lambda}',s')
        DK_{\epsilon}(\bar{z}_{0}(\bar{\lambda},s) - \bar{z}_{0}(\bar{\lambda}',s'))
        (\mathbf{N}(\bar{\lambda}',s'))
        \\&\quad\quad\quad\quad\quad\quad\quad\quad\quad\quad\quad\quad
        \cdot(\mathbf{T}(\bar{\lambda}',s')\cdot\mathbf{N}(\bar{\lambda},s))^{3}
        \,ds'\,ds\,d\bar{\lambda}'.
    \end{align*}
    Note that the same bound we applied to $G_{5}$ and $G_{6}$ can be still applied here:
    \begin{align*}
        G_{9},G_{10},G_{11} \leq C(\alpha)
        \ell(\bar{\lambda})^{1/2}\mathcal{K}_{2}(\bar{\lambda})
        \int_{\bar{\mathcal{L}}}
        \frac{\ell(\bar{\lambda}')^{1/2}\mathcal{K}_{2}(\bar{\lambda}')}
        {\Delta(\bar{\lambda},\bar{\lambda}')^{2\alpha}}
        \,d\bar{\lambda}'.
    \end{align*}

    Finally, to bound $G_{7}$, Lemma~\ref{L3.3} shows that
    \[
        \abs{\mathbf{T}(\bar{\lambda}',s')\cdot\mathbf{N}(\bar{\lambda},s)}^{3}
        \leq (6\sqrt{2})^{3}\max\left(
            \norm{\bar{z}_{0}(\bar{\lambda})}_{\dot{C}^{1,1/2}},
            \norm{\bar{z}_{0}(\bar{\lambda}')}_{\dot{C}^{1,1/2}}
        \right)^{2}
        \abs{\bar{z}_{0}(\bar{\lambda},s) - \bar{z}_{0}(\bar{\lambda}',s')},
    \]
    and there exists a universal constant $C\geq 0$ such that
    \begin{align*}
        \frac{C}
        {\abs{\bar{z}_{0}(\bar{\lambda},s) - \bar{z}_{0}(\bar{\lambda}',s')}^{2+2\alpha}}
        &\geq \bigg|
            D^{2}K_{\epsilon}(\bar{z}_{0}(\bar{\lambda},s) - \bar{z}_{0}(\bar{\lambda}',s'))
            (\mathbf{N}(\bar{\lambda}',s'),\mathbf{N}(\bar{\lambda}',s'))
            \\&\quad\quad\quad\quad\quad\quad
            - D^{2}K_{\epsilon}(\bar{z}_{0}(\bar{\lambda},s) - \bar{z}_{0}(\bar{\lambda}',s'))
            (\mathbf{T}(\bar{\lambda}',s'),\mathbf{T}(\bar{\lambda}',s'))
        \bigg|,
    \end{align*}
    thus Lemma~\ref{L4.3} shows
    \begin{align*}
        G_{7} &\leq
        \int_{\ell(\bar{\lambda})\bbT}\abs{\kappa(\bar{\lambda},s)}
        \int_{\bar{\mathcal{L}}}\int_{\ell(\bar{\lambda}')\bbT}
        \frac{C\max\left(
            \norm{\bar{z}_{0}(\bar{\lambda})}_{\dot{C}^{1,1/2}},
            \norm{\bar{z}_{0}(\bar{\lambda}')}_{\dot{C}^{1,1/2}}
        \right)^{2}}
        {\abs{\bar{z}_{0}(\bar{\lambda},s) - \bar{z}_{0}(\bar{\lambda}',s')}^{1+2\alpha}}
        \,ds'\,d\bar{\lambda}'\,ds \\
        &\leq C(\alpha)\int_{\ell(\bar{\lambda})\bbT}\abs{\kappa(\bar{\lambda},s)}
        \int_{\bar{\mathcal{L}}}
        \frac{\ell(\bar{\lambda}') \mathcal{K}_{2}(\bar{\lambda}')
        \max\left(\mathcal{K}_{2}(\bar{\lambda}), \mathcal{K}_{2}(\bar{\lambda}')\right)}
        {\Delta(\bar{\lambda},\bar{\lambda}')^{2\alpha}}
        \,d\bar{\lambda}'\,ds \\
        &\leq C(\alpha)\ell(\bar{\lambda})^{1/2}\mathcal{K}_{2}(\bar{\lambda})^{1/2}
        \int_{\bar{\mathcal{L}}}
        \frac{\ell(\bar{\lambda}') \mathcal{K}_{2}(\bar{\lambda}')
        \max\left(\mathcal{K}_{2}(\bar{\lambda}), \mathcal{K}_{2}(\bar{\lambda}')\right)}
        {\Delta(\bar{\lambda},\bar{\lambda}')^{2\alpha}}
        \,d\bar{\lambda}'.
    \end{align*}
    Aggregating the bounds for $G_{2}$, $G_{5}$, $G_{6}$, $G_{7}$, $G_{9}$,
    $G_{10}$ and $G_{11}$ and then applying the inequality
    $\ell(\bar{\lambda}') \leq \norm{\bar{z}_{0}(\bar{\lambda}')}_{C^{0}}^{2}
    \mathcal{K}_{2}(\bar{\lambda}')$ gives the estimate we wanted to show.
    
    (I suspect that there may be a version of
    Lemma~\ref{L3.3} where the right-hand side is of the form
    $\mathcal{M}\kappa \cdot \abs{\gamma_{1}(s_{1}) - \gamma_{2}(s_{2})}^{\frac{1}{1+\beta}}$
    where $\mathcal{M}\kappa$ is the maximal function of $\kappa$.
    In this way, we may find a more natural bound for $G_{3}$...)
\end{proof}


\end{document}